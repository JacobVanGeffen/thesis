\section{Optimizations and Design Choices}\label{s:impl}

We implement both the \depsynth algorithm
and the storage systems we study in \cref{sec:eval}
in Rosette~\cite{torlak:rosette},
an extension of Racket~\cite{felleisen:racket}
with support for verification and synthesis.
Using Rosette as our host language gives us symbolic evaluation of the storage system implementation for free,
and simplifies implementing the \crashconsistentalg query in \cref{fig:alg:depsynth}.
The choice of Rosette and Racket is not fundamental;
recent work has shown how to extend the symbolic evaluation approach to languages
such as Python~\cite{sigurbjarnarson:yggdrasil} or C~\cite{nelson:serval}
in which storage systems are more commonly implemented.

\paragraph{Ordering.}
The \depsynthalg algorithm in \cref{fig:alg:depsynth}
is sensitive to the order in which \textsc{NextTest} chooses tests to generate dependency rules for.
Our implementation chooses tests in increasing order of size,
minimizing the number of happens-before graphs for \rulesfortest to explore.
Similarly, \rulesfortest is sensitive to the order it considers writes (\phaseone)
and edges (\phasetwo).
In both cases, we exploit the following observation: 
while an execution that persists writes in program order is not \emph{required} to be crash consistent
(e.g., because storage systems might selectively buffer or coalesce writes),
it is often so in practice.
\rulesfortest therefore prefers to choose writes in \phaseone in program order,
and prefers to remove edges in \phasetwo that contradict program order.

\paragraph{Reducing solver queries.}
Both \phaseone and \phasetwo in \cref{fig:alg:rulesfortests}
have symmetry in their search space:
for a fixed pair of writes $w_1$ and $w_2$,
there are many different branches of \phaseone
that try to order $w_2$ after $w_1$,
and many different branches of \phasetwo
that try to remove the edge $(w_1, w_2)$ from a happens-before graph.
If we can determine ahead of time that such a choice for those writes is always doomed to fail,
we can avoid considering these choices at all
and so save the cost of an SMT solver query by \crashconsistentalg.
Our implementation of \rulesfortest
uses an SMT solver to
pre-compute a set of \emph{necessary} ordering edges---%
edges which \emph{must} be in the happens-before graph---%
and uses that set to short-circuit \crashconsistentalg.

\subsection{Attempting graph search with a \tord solver.}
One bottleneck in \depsynth
comes from the time it takes to search for a minimal
happens-before graph for a given litmus test.
This section describes a strategy that we explored to limit this bottleneck.
Specifically, we encode the happens-before graph search problem
in the theory of ordering consistency ($\tord$) \cite{n:tord}
and execute the search with an existing $\tord$ solver.
Ultimately, we found that this different search strategy did not 
improve the overall runtime, but we find this negative
result worth reporting.

% TODO present results
As described in \cref{sec:alg:onetest}, the \rulesfortest algorithm
finds dependency rules for a litmus test by searching for a total order over writes,
reducing the total order to a minimal happens-before graph,
then generalizing that graph into rules.
The reduction to graph and generalization into rules steps are both straightforward,
but since the space of possible total orders grows exponentially with the number of writes,
finding valid orders over writes can be exceedingly slow for large tests.
While the pruning techniques presenting in \cref{sec:alg}
and the heuristics over prioritizing program order help mitigate the runtime cost of this search,
there remains space for further improvement.

In an attempt to speed up the overall search time for total orders over writes,
we encode the problem in $\tord$ and offload to a $\tord$ solver.
$\tord$ defines predicates for describing partial-ordering constraints
on reads and writes for multi-threaded programs, % TODO
but can also be used to describe orderings of storage system writes.
Specifically, $\tord$ introduces the predicate $\prec$, where
$ w_i \prec w_j $ encodes the fact that write $w_i$ occured before
write $w_j$.
To encode our search in $\tord$,
we describe the existence of each happens-before graph edge
between write events $e_i$ and $e_j$ as the boolean variable $d_{ij}$,
then introduce the following constraint: $ d_{ij} \rightarrow e_i \prec e_j $.
By solving these constraints along with the user-defined constraints specifying
crash consistency for the storage system, we arrive at a total order over all
writes for the given litmus test.

Existing work defines a DPLL(T)-style solver for $\tord$ \cite{n:tord},
which we use for solving the constraints specified above and
determining a total order over writes for single litmus tests.
However, by doing so, we empirically found that the solver
slightly diminished the performance of \depsynth for all three of our evaluated case studies
(in the case of \shardstore, slowing the rules search from
\shardstoresynthesistime to \shardstoresynthesistimetord).
Intuitively, we believe this performance degredation is an artifact of 
bad initial assignments from the SAT solver
and a resulting a large number of iterations in the DPLL(T) loop.
For storage systems with increasingly complex behavior,
\depsynth may require much larger litmus tests,
which may make the benefits of searching for orders with $\tord$ more apparent.
However, for the case studies considered in this paper,
we found our original search algorithm from \cref{sec:alg:onetest}
to outperform the $\tord$-directed search.

\subsection{Faster Consistency Checks for Litmus Tests}
\label{s:improved-search}
As \cref{sec:eval:shardstore} discusses,
a large majority of the litmus tests seen during the \depsynth
algorithm do not result in new dependency rules
(\depsynth generated new rules for only \shardstoretestsused of \shardstoreinputtests
 litmus tests seen for \shardstore).
However, for each litmus test in the input set, \depsynth
must perform a check that the accumulated dependency rules
guarantee the crash consistency of the test.
For synthesizing rules in \shardstore,
these take up \shardstoreverificationminutes (\shardstoreverificationpercent)
of the overall search time.

\paragraph{Decomposing crash consistency queries.}
As \cref{sec:alg:alg} explains, \depsynth offloads a query to an SMT
solver when it first encounters a new litmus test to determine
whether or not the current set of dependency rules satisfy the
crash consistency property for that test.
In practice, this query is encoded by representing the \textit{committed status}
of each write from a litmus test as a boolean variable
and each edge in the happens-before graph as an implication between those variables.
% The value of these variables also affects the disk state,
% which is used
For tests that generate a large happens-before graph
with respect to the current dependency rules,
the query can take tens of seconds.

Experimentally, we observed that for these queries with long runtimes,
decomposing the query by concretizing one or more write-committed variables
drastically speeds up solving time.
(averaging $\splitvariableavgspeedup \times$ and at most $\splitvariablebestspeedup \times$ speedup for a single litmus test).
Specifically, considering the set of write-committed variables $V$,
the happens-before graph $G\subset V\times V$,
and the original encoding for the litmus test crash consistency check $P(G)$,
we choose a variable $v_0\in V$ and issue the following two queries:
\begin{enumerate}
  \item $v_0 \wedge \forall v\in V. (v, v_0)\in G \rightarrow v) \wedge P(G)$
  \item $\neg v_0 \wedge \forall v\in V. (v, v_0)\in G \rightarrow \neg v) \wedge P(G)$
\end{enumerate}
The litmus test is crash consistent with respect to the dependency rules
if and only if both queries succeed.
This concretization can be repeated for any number of variables,
resulting in $2^n$ queries for $n$ concretized variables.
Ultimately, this optimization gives a speedup of $\splitvariableoverallspeedup \times$
for the \shardstore case study discussed in \cref{sec:eval:shardstore},
reducing end-to-end runtime from \shardstoresynthesistime to \splitvariablenewtime.

\paragraph{Choosing Variables to Concretize.}
Intuitively, this decomposition results in faster solving times for two reasons.
First, this optimization can be seen as a pre-processing variable selection step.
By taking advantage of the relationship between nodes in the happens-before graph
(which is abstracted away in the constraints sent to the solver),
we are effectively telling the solver to select a good first variable for branching.
Second, concretizing write-committed variables can also indirectly simplify the
crash consistency predicate for the litmus test.
The crash consistency predicate is a function over disk states,
and the disk state itself depends on the write-committed variables.
Thus, by assigning concrete values to some of those variables,
constraints representing the crash consistency predicate over the disk state
may be simplified or eliminated.

With this intuition in mind, we chose concretization variables
that will most effectively simplify the constraints given to the solver.
We found that for a litmus test with $N$ writes,
choosing $log(N)$ variables to concretize gave the best results.
Moreover, the choice of which variables to concretize is determined by the
happens-before graph over the litmus test writes
(w.r.t. \depsynth's current candidate rules set).
Specifically, we choose the $log(N)$ variables whose corresponding
node in the happens-before graph has the most ancestors and descendants.
% TODO elaborate on this?

\if 0

%\paragraph{Avoiding \multisearch}
%In \autoref{alg:top-level}, we described our top-level algorithm as one that
%first uses an efficient per-litmus-test search before defaulting to a
%slower search for dependency graphs over multiple litmus tests.
%However, even in the presence of rule conflicts, \depsynth can still avoid
%performing the slower search in most cases. The way \depsynth achieves this
%is by searching for new graphs for individual conflicting litmus tests
%that, when extrapolated into rules, do not cause a conflict.
%
%Specifically, if \depsynth encounters a set of litmus tests $\{L_0, L_1, ...\}$
%whose generated rules conflict, \depsynth first tries to generate a new graph and rules
%for $L_0$ such that these new rules jj

\paragraph{Rosette}
\todo{Talk about how rosette is used in \depsynth}

\paragraph{Limiting Queries with Preprocessing}

% TODO explain necessary paths
At each state of the dependency graph search in \sccsearch, \depsynth
performs an SMT query to determine if the current candidate graph
is sufficient. These queries save time by allowing \sccsearch to avoid
searching unnecessary spaces, but the queries themselves are not free.
Though the queries are fairly lightweight, as the free variables are on
the order of 10s of boolean values, they still introduce noticeable overhead.

Generally, the results of sufficiency queries for similar graphs are similar:
one example is that, as previously discussed, subgraphs of insufficient graphs
are always insufficient.
In fact, we noticed that in most cases, certain paths between nodes are
\textit{necessary} in the final output of \sccsearch, meaning that
removing all paths between these nodes from a graph would immediately
cause it to become insufficient.
Intuitively, these necessary paths exist whenever all solution graphs
to our search must contain such a path.
If \depsynth can precompute the set of necessary paths,
we can avoid querying for sufficiency over graphs that lack any.

Formally, we say there is a \textit{necessary path} from $w_0$ to $w_1$, whenever
there \textit{does not} exist a schedule with $\syncbool_0\mapsto\true$
and $\syncbool_1\mapsto\false$ that satisfies the crash consistency property
(i.e. committing $w_0$ before $w_1$ is never crash consistent).
If such a schedule does not exist, it means that for every schedule satisfying
the property, $\syncbool_0\implies\syncbool_1$, and so any solution graph to our
search must contain a path from $w_0$ to $w_1$.
When a graph lacks any path from $w_0$ to $w_1$, it can be pruned
without checking for sufficiency.
For each pair of writes $w_i,w_j$,
we can determine if there is a necessary path from $w_i$ to $w_j$
by querying over all schedules with
$\syncbool_j\mapsto\true$ and $\syncbool_j\mapsto\false$.

% TODO explain how nec paths reduce the number of queries we have to make
Though this precomputation would prevent some sufficiency queries,
it requires checking for necessary paths between all nodes.
Naively, this would take $|\omega|(|\omega|-1)$ queries for set of writes
$\omega$, but we can avoid most of these queries by taking advantage of the fact
that our SMT solver returns a concrete schedule whenever a necessary
path \textit{does not} exist.
Specifically, when query for writes $w_0$ and $w_1$ determines that there is
a schedule that satisfies property $\mathcal{P}$ where $\syncbool_0\mapsto\true$
and $\syncbool_1\mapsto\false$, consider all other pairs $i,j$ in the schedule
where $\syncbool_i\mapsto\true$ and $\syncbool_j\mapsto\false$.
From this schedule, we can also deduce that there is no necessary path
from $w_i$ to $w_j$. Using this optimization, we can compute all necessary paths
on the order of $O(|\omega|)$ queries.

An added benefit of necessary edges is that \sccsearch can start from 
a smaller graph than the complete graph. Specifically, if there exists
a necessary path from $w_0$ to $w_1$, then we can remove edge $(w_1, w_0)$
from the starting graph, as any sufficient graph with both a path from
$w_0$ to $w_1$ and edge $(w_1, w_0)$ must contain a cycle.
In some cases, this means that not all starting nodes share a single SCC.

\paragraph{Prioritizing choices in \sccsearch}
\todo{These few paragraphs could go in the technical section, just not sure}
In \autoref{alg:scc-search}, we describe an algorithm that, at each step,
chooses an arbitrary node to order with respect to other nodes in its
strongly connected component.
In practice, \depsynth prioritizes ordering nodes in two ways:
\begin{enumerate}
  \item Order nodes as they would be ordered with previously generated rules.
  \item Order nodes from newest to oldest.
\end{enumerate}

Since \depsynth keeps the set of rules it has generated so far,
it prefers solutions for new litmus tests to agree with previously generated rules.
This is \textit{not} only because avoiding conflicts saves time.
Intuitively, as \depsynth generates rules for increasingly more tests in the input set,
its confidence in the rules it has generated thus far becomes greater.
This gives further reason for \depsynth to prefer graphs that obey previously generated rules.

To give intuition about our second heuristic, newer writes tend to depend on older writes
because dependencies are "happens before" relationships.
From our experience, data updates are more likely to rely on previous updates than future ones.
As a result, prioritizing our search in this way leads to generating correct rules more often.
Of course, correct dependency graphs may not follow either of these heuristicts,
but since \sccsearch can backtrack on ordering decisions,
these heuristics do not compromise correctness.

\paragraph{Optimizations in \multisearch}
\autoref{alg:multi-search} describes a search over graphs for multiple litmus
tests that, like \autoref{alg:remove-edges}, removes edges one-at-a-time
to arrive at candidates. In practice, \depsynth also the uses SCC-splitting
optimization from \autoref{alg:scc-search} in \multisearch.

% Talk about using input rules as necessary paths?
% Additionally, \depsynth leverages necessary paths to prune the
% search away from graph sets that conflict with input rules.
% Rules output by \multisearch should not conflict with
% the set of input rules. Like necessary paths,

\paragraph{Skipping Tests}
Since \autoref{alg:top-level} gradually builds up a set of rules,
\depsynth will likely encounter litmus tests along the search where
the current candidate set of rules is sufficient to make the litmus
test crash consistent. When this is the case, \depsynth skips \rulessearch
for that test and does not generate any new rules. \depsynth still updates
the mapping from litmus tests to rules with the set of rules used for the test.
This is because the test may be relevant for later conflict resolution.

\paragraph{Reordering Tests}
Depending on the order of the input litmus tests, \depsynth may encounter tests for which
\rulessearch performs slowly. Specifically, when \rulessearch times out
for such a test, \depsynth postpones the search for the test until the end of the loop.
By doing so, \depsynth may be able to synthesize sufficient rules for the test by
finding rules for other, simpler tests. If so, \depsynth can skip the search for the
challenging test at the end.

Executing \rulessearch to a timeout is still expensive, and \depsynth avoids doing so
when possible by ordering the input litmus tests before performing \toplevel.
Specifically, \depsynth orders litmus tests based on the number of disk writes the
test generates, from least to greatest. This allows \depsynth to find rules for simpler
tests before moving on to more complex tests.

\fi
