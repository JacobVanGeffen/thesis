Reliable storage systems must be \emph{crash consistent}---%
guaranteed to recover to a consistent state after a crash. %---%
%to support the availability and durability needs of applications.
Crash consistency is non-trivial
as it requires maintaining complex invariants
about persistent data structures 
in the presence of caching, reordering, and system failures.
Current programming models offer little support for implementing crash consistency,
forcing storage system developers to roll their own consistency mechanisms.
Bugs in these mechanisms can lead to severe data loss
for applications that rely on persistent storage.\tighten

This chapter presents a new \emph{synthesis-aided} programming model
for building crash-consistent storage systems.
In this approach, storage systems can assume
an \emph{angelic crash-consistency} model,
where the underlying storage stack
promises to resolve crashes in favor of consistency whenever possible.
To realize this model,
we introduce a new \emph{labeled writes} interface for developers to identify their writes to disk,
and develop a program synthesis tool, \depsynth,
that generates \emph{dependency rules}
to enforce crash consistency over these labeled writes.
We evaluate our model in a case study
on a production storage system at Amazon Web Services.
We find that \depsynth can automate crash consistency for this complex storage system,
with similar results to existing expert-written code,
and can automatically identify and correct consistency and performance issues.
