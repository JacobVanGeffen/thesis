\section{Related Work}\label{s:related}

\paragraph{Verified storage systems.}
Inspired by successes in other systems verification problems~\cite{leroy:compcert,klein:sel4},
recent work has brought the power of automated and interactive verification
to bear on storage systems as well.
One of the main challenges in verifying storage systems is crash consistency,
as it combines concurrency-like nondeterminism with persistent state.
Yggdrasil~\cite{sigurbjarnarson:yggdrasil} is a verified file system
whose correctness theorem is a \emph{crash refinement}---%
a simulation between a crash-free specification and the nondeterministic, crashing implementation.
This formalization allows clients of Yggdrasil to program against a strong specification free from crashes,
similar to our angelic crash consistency model.
FSCQ~\cite{chen:fscq} is a verified crash-safe file system
with specifications stated in \emph{crash Hoare logic},
which explicitly states the recovery behavior of the system after a crash.
DFSCQ~\cite{chen:dfscq} extends FSCQ and its verification
with support for crash-consistency optimizations such as log-bypass writes
and the metadata-only \texttt{fdatasync} system call.
The \depsynth programming model separates crash consistency of these optimizations from the storage system itself,
and so can simplify their implementation.

Another approach to verified storage systems is at the language level.
Cogent~\cite{amani:cogent} is a language for building storage systems
with a strong type system that precludes some common systems bugs.
A language-level approach like Cogent is complementary to \depsynth:
Cogent provides a high-level language for implementing storage systems,
while \depsynth provides a synthesizer for making those implementations crash consistent.

\paragraph{Crash-consistency bug-finding tools.}
Ferrite~\cite{bornholt:ferrite} is a framework for specifying \emph{crash-consistency models},
which formally define the behavior of a storage system across crashes,
and for automatically finding violations of such models in a storage system implementation.
One way to specify these models is with litmus tests that demonstrate unintuitive behaviors;
\depsynth builds on this approach by automatically synthesizing rules from such litmus tests.
\depsynth also takes inspiration from Ferrite's synthesis tool
for inserting \texttt{fsync} calls into litmus tests to make them crash consistent,
but instead focuses on making the \emph{storage system itself} crash consistent
rather than the user code running on top of it.
CrashMonkey~\cite{mohan:crashmonkey} is a tool for finding crash-consistency bugs in Linux file systems.
Chipmunk~\cite{leblanc:chipmunk} extends the CrashMonkey approach to persistent-memory file systems
by exploring finer-grained crash states to account for the byte-addressable nature of non-volatile memory.
CrashMonkey exhaustively enumerates all litmus tests with a given set of system calls,
runs them against the target file system,
and then tests each possible crash state for consistency.
Connecting CrashMonkey-like litmus test generation with \depsynth
could provide developers with a comprehensive set of litmus tests for their system for free,
lowering the burden of applying \depsynth.
To give stronger coverage guarantees that do not depend on enumerating litmus tests,
FiSC~\cite{yang:fisc} and eXplode~\cite{yang:explode}
use model checking to find bugs in storage systems.

One advantage of bug-finding tools is that they are significantly easier to apply to production systems
than heavyweight verification tools.
Bornholt et al. \cite{bornholt:s3} describe the use of lightweight formal methods
to validate the crash consistency (and other properties)
of ShardStore, the Amazon S3 storage node that we study in Section~\ref{sec:eval:shardstore}.
Their approach applies property-based testing to automatically find and minimize litmus tests
that demonstrate crash-consistency issues.
\depsynth takes this idea one step further by automatically \emph{fixing} such issues once they are found.

\paragraph{Program synthesis for systems code.}
Transit~\cite{udupa:transit} is a tool for automatically inferring distributed protocols
such as those used for cache coherence.
It guides the search using \emph{concolic} snippets~\cite{sen:concolic}---%
effectively litmus tests that can be partially symbolic---%
and finds a protocol that satisfies those snippets for \emph{any} ordering of messages.
MemSynth~\cite{bornholt:memsynth} is a program synthesis tool for
automatically constructing specifications of memory consistency models.
MemSynth takes similar inputs to \depsynth---a set of litmus tests and a target language---%
and its synthesizer generates and checks happens-before graphs for those tests.
Adopting MemSynth's aggressive inference of partial interpretations~\cite{torlak:kodkod}
to shrink the search space of happens-before graphs would be promising future work.\tighten
\vspace*{-1em}

\if 0

Our work is motivated by previous work on crash consistency,
on verifying storage systems, and on program synthesis for systems.

\paragraph{Specification and Verification of Crash Consistency}
Many existing works have explored how to specify crash consistency
for storage systems and how to verify that systems maintain
these properties \todo{other citations}.
DepSynth draws from these works for its own
representation of crash consistency properties.

Featherstitch~ \cite{frost:featherstitch} introduces a framework that generalizes 
multiple techniques for specifying crash consistency into the
"writes-before" dependencies discussed in this chapter.
DepSynth builds on this work by removing the burden of describing
dependencies from the developer.

Several works have aimed to verify properties of file
systems~\cite{amani:cogent,schellhorn:flashix}.
FSCQ~\cite{chen:fscq} is a file system that has been proven crash consistent
with a hand-written Coq proof using crash Hoare logic.
GoJournal~\cite{chajed:go-journal} introduces techniques to limit the complexity of
crash consistency proofs for file systems
and demonstrates with a proof for a journaling file system.
The Yggdrasil~\cite{sigurbjarnarson:yggdrasil} framework enables developers to write crash consistent
file systems, but distinguishes itself by enabling for push-button
verification. Compared to DepSynth, these works have stronger guarantees
about crash consistency.
% TODO explain why?
The trade-off is that the burden on the developer is higher for these tools:
all three require the developer to write a crash consistent file system,
and FSCQ and GoJournal additionally require proofs.
\todo{Is this ok to say?}

\paragraph{Synthesis for Systems}
% TODO structure better
Several existing works in synthesis for ordering constraints in systems
have inspired the techniques used in DepSynth.
One body of such work focuses on distributed protocols for coordination
between concurrent processes~\cite{alur:synth-protocols}.
While DepSynth reasons about the orderings of
writes to disk issued from a single process, these systems typically reason about
transition functions over finite state machines and the order
in which these transition functions may be applied.
For example, Transit~\cite{udupa:transit} completes holes in transition functions
using a CEGIS-style synthesis algorithm.
% TODO explain differences?

Other existing work aims to synthesize memory models for modern architectures.
MemSynth~\cite{bornholt:memsynth} is a tool for reasoning about and synthesizing memory
models that, like DepSynth, takes a set of litmus tests as part of its input.
Using a series of SMT queries, MemSynth gradually fills in holes from an
input memory model sketch to guarantee correctness over a given set of
litmus tests.
The key difference in these problems is the space of behaviors considered:
while memory models must account for more complex orderings,
intermediate states of storage systems are generally more complex than their
architecural counterpart, making checking properties over storage system
states more difficult.

Ferrite~\cite{bornholt:ferrite}, a tool for verifying crash consistency models over litmus tests
that can be extended to synthesize \texttt{fsync} invocations.
While the synthesized \texttt{fsync} calls can guarantee crash consistency properties,
they are far more restrictive on the allowed schedules than
the output rules of DepSynth.
\todo{Correct way to say this without diminishing Ferrite?}

\paragraph{Finding Crash Consistency Bugs}
While verification and synthesis techniques are able to provide
formal guarantees about crash consistency, testing frameworks
allow developers to find bugs without the overhead of modeling
their implementation ~\cite{sun:pfscheck,fu:witcher}.
CrashMonkey and Ace~\cite{mohan:crashmonkey} are a pair of tools that
enable crash consistency testing for file systems
by intelligently enumerating tests and simulating crashes.
\todo{it is actually just exhaustively enumerating tests. I want to cite this +
      something that intelligently enumerates tests}
In our current implementation, DepSynth's random litmus test generator 
is fairly simple, so adapting these techniques to choose better litmus
tests would likely improve overall synthesis time.

\fi
