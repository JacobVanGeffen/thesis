\section{Evaluation}\label{sec:eval}

This section answers three questions to demonstrate the effectiveness of \depsynth:
\begin{enumerate}[left=0pt]
\item Can storage system developers use \depsynth to synthesize dependency rules for a realistic storage system
      rather than implementing their own crash-consistency approach by hand? (\S\ref{sec:eval:shardstore})
\item Can \depsynth help storage system developers avoid crash-consistency bugs? (\S\ref{sec:eval:bugs})
\item Does \depsynth's approach support a variety of storage system designs? (\S\ref{sec:eval:other})
\end{enumerate}

\subsection{\shardstore Case Study}\label{sec:eval:shardstore}

To show that developers can use \depsynth to build realistic storage systems, 
%instead of writing their own manual crash-consistency implementation,
we implemented a key-value store that follows the design of \shardstore%
\footnote{Anonymized for double-blind review.}~%
\cite{bornholt:s3-anon},
a key-value store operating in production for a major commercial cloud object storage service.

\paragraph{Implementation.}
The first step in using \depsynth is to implement the storage system itself.
\shardstore's on-disk representation is a log-structured merge tree (LSM tree)~\cite{oneil:lsm},
but with values stored outside the tree in a collection of extents.
Our \shardstore-like storage system implementation
consists of 1,200 lines of Racket code, including
%includes five operations:
five operations:
the usual \texttt{put}, \texttt{get}, and \texttt{delete} operations on single keys,
as well as a garbage collection \texttt{clean} operation that 
evacuates all live objects in one extent to another extent, % (called \emph{reclamation} by \shard),
and a \texttt{flush} operation that persists the LSM tree memtable to disk.
Our implementation does not handle boundary conditions
such as running out of disk space or objects too large to fit in one extent,
but is otherwise faithful to the published \shardstore design.
As a crash consistency predicate,
we wrote a checker that validates all expected objects are accessible by \texttt{get} after a crash,
and that the on-disk LSM tree contains only valid pointers to objects in extents.

\paragraph{Synthesis.}
With a storage system implementation in hand,
a developer can use \depsynth to synthesize dependency rules that make the system crash consistent.
\depsynth takes as input a set of litmus tests---%
we randomly generated \shardstoreinputtests{} litmus tests for the \shardstore-like system,
ranging in length from 1 to \shardstoremaxops operations.
Executing these tests against the system
led to an average of \shardstoreavgwrites and a maximum of \shardstoremaxwrites disk writes per test.
Given these inputs, \depsynth synthesized a set of \shardstorenumrules dependency rules for \shardstore
in \shardstoresynthesistime.
To find a correct solution for all \shardstoreinputtests litmus tests,
the \depsynth algorithm invoked the \rulesfortest procedure (line~\ref{alg:depsynth:rulesone} in \cref{fig:alg:depsynth})
only \shardstoretestsused times, showing that \depsynth's incremental approach is effective at reducing the search space.\tighten

\paragraph{Comparison to an existing implementation.}
\shardstore is an existing production system and
already supports crash consistency.
Its implementation does not use a dependency-rule language like in \depsynth. 
Instead, it implements a soft-updates approach~\cite{ganger:soft-updates}
by constructing dependency graphs (i.e, happens-before graphs) at run time
and sequencing writes to disk based on those graphs,
similar to patchgroups in Featherstitch~\cite{frost:featherstitch}.
We therefore compare our synthesized rules against \shardstore's dependency graphs
to see how well \depsynth may replace an expert-written crash consistency implementation.

For each of the \shardstoretestsused{} tests that \depsynth used while synthesizing dependency rules for \shardstore,
we used an SMT solver to compute the set of valid crash schedules (\cref{def:valid-schedule})
according to those synthesized dependency rules.
We then executed the same test using the production \shardstore implementation,
collected the run-time dependency graph it generated,
and used an SMT solver to compute the set of valid crash schedules according to that graph.
Given these two sets of crash schedules,
we computed the set intersection and difference to classify them into three groups:
schedules allowed by both implementations (i.e., both implementations agree),
and schedules allowed only by one or the other implementation (i.e., the two implementations disagree).\tighten

\begin{table}
    \centering
    \caption{Valid schedules allowed by the production \shardstore service versus
    the dependency rules we synthesized for our \shardstore-like reimplementation.
    A schedule allowed only by one implementation
    means either that implementation is not crash consistent (it allows a schedule it should forbid)
    or it admits more reordering opportunities (it allows a schedule it should allow).
    ``Fixed'' results are after fixing two issues in \shardstore (one consistency, one performance)
    that we identified by manually inspecting the ``original'' schedules.\tighten}
    {\xsmall\newcommand\shardstorecomparisontotaltests{10}

\begin{tabular}{crrrrrrrr}
\toprule
 & & & \multicolumn{2}{c}{\makecell{Allowed\\by both}} & \multicolumn{2}{c}{\makecell{Allowed only\\by \depsynth}} & \multicolumn{2}{c}{\makecell{Allowed only\\by \shardstore}} \\
\cmidrule(lr){4-5} \cmidrule(lr){6-7} \cmidrule(lr){8-9}
Test & Test Length & Writes & Original & Fixed & Original & Fixed & Original & Fixed \\
\midrule
$T_{1}$ & 1 & 2 & 3 & 3 & 0 & 0 & 0 & 0 \\ 
$T_{2}$ & 2 & 6 & 7 & 14 & 7 & 0 & 3 & 3 \\ 
$T_{3}$ & 5 & 1 & 2 & 2 & 0 & 0 & 0 & 0 \\ 
$T_{4}$ & 5 & 7 & 8 & 15 & 7 & 0 & 3 & 3 \\ 
$T_{5}$ & 4 & 7 & 11 & 29 & 9 & 0 & 9 & 0 \\ 
$T_{6}$ & 5 & 5 & 6 & 12 & 2 & 0 & 4 & 0 \\ 
$T_{7}$ & 7 & 5 & 5 & 11 & 2 & 0 & 5 & 1 \\ 
$T_{8}$ & 10 & 5 & 6 & 12 & 2 & 0 & 4 & 0 \\ 
$T_{9}$ & 16 & 6 & 8 & 22 & 2 & 0 & 12 & 0 \\ 
$T_{10}$ & 13 & 9 & 21 & 41 & 20 & 0 & 9 & 9 \\ 

\bottomrule
\end{tabular}
}
    \label{tab:shardstore-data}
\end{table}

\cref{tab:shardstore-data} shows the results of this classification
across the \shardstoretestsused litmus tests.
Overall, the two implementations agree on the validity of an average of \shardstoreagreementbeforefixes{} of crash schedules.
The remaining crash schedules are in two categories:
%
\begin{enumerate}[left=0pt]
\item Schedules allowed only by \depsynth mean either
\depsynth's rules allow some schedules that are not crash consistent
(a correctness issue in the synthesized rules)
or \shardstore precludes some schedules that are crash consistent
(a performance issue in \shardstore). 
We found that every schedule allowed by \depsynth is crash consistent,
and that \shardstore inserts unnecessary edges in its dependency graphs,
ruling out some reorderings that would be safe.
These edges are not necessary to guarantee crash consistency of the overall storage system,
and so \depsynth is correct to allow them.
However, \shardstore engineers intentionally include these edges as they make
the representation invariant for an on-disk data structure simpler,
even though a more complex invariant that did not require these edges would still be sufficient for consistency.
In other words, \shardstore engineers favored a stronger, simpler invariant in these cases,
where \depsynth is able to identify opportunities for performance improvements.

\item Schedules allowed only by \shardstore mean either
\depsynth's rules preclude some schedules that are crash consistent
(meaning \depsynth's output is not optimal)
or \shardstore allows some schedules that are not crash consistent
(a correctness issue in \shardstore).
\shardstoreresetschedulespct{} of these schedules are incorrectly allowed by \shardstore
due to a rare crash-consistency issue
that was independently discovered concurrently with this work.
We have confirmed with \shardstore engineers that
the issue was an unlikely edge case that could not lead to data loss,
but could lead to ``ghost'' objects---resurrected pointers to deleted objects,
where the object data has been (correctly) deleted,
but the pointer still exists---%
which result in an inconsistent state.
After fixing this issue in \shardstore,
we manually inspected the remaining schedules it allowed
and confirmed they are all cases where \depsynth's rules generate extraneous edges
(i.e., the synthesized rules are not optimal),
and the crash-consistency predicate we wrote for our \shardstore reimplementation
agrees that all the resulting states are consistent.
\end{enumerate}
%
After fixing the two \shardstore issues discussed above,
the synthesized dependency rules agree with \shardstore on the validity of an average of \shardstoreagreementafterfixes{} of crash schedules.
The few remaining schedules are ones that \depsynth's synthesized dependency rules 
conservatively forbid due to the coarse granularity of the dependency rule language.
Overall, this study shows that \depsynth achieves similar results to an expert-written crash consistency implementation,
%allowing most performance optimizations that the real implementation allows,
and can help identify correctness and performance issues in existing storage systems.

\paragraph{Generalization.}
One risk for example-guided synthesis techniques like \depsynth
is that they can overfit to the examples (litmus tests)
and not actually ensure crash consistency on unseen test cases.
\depsynth's design reduces this risk
by using a simple dependency rule language (\cref{def:dependency-rule})
that cannot identify individual write operations.
To test generalization,
we randomly generated an additional \shardstoregeneralizationextratests{} litmus tests
for our \shardstore-like system.
We also allowed these tests to be significantly longer than those used during synthesis---%
up to a maximum of \shardstoregeneralizationmaxwrites{} writes
rather than the \shardstoremaxwrites{} in the input set of litmus tests.
For each new test,
we used the synthesized dependency rules
to compute all valid crash schedules for the test,
and found that every crash schedule resulted in a consistent disk state
according to our crash consistency predicate.
In other words, by limiting the expressivity of our dependency rule language,
the rules we synthesize can generalize well beyond the tests they were generated from.

\subsection{Crash-Consistency Bugs}\label{sec:eval:bugs}

\begin{table}
    \centering
    \caption{Sample crash-consistency bugs in three storage systems reported by two recent papers~\cite{bornholt:s3,mohan:crashmonkey}.
    Each bug includes its identifier (bug number for ShardStore, kernel Git commit for btrfs and f2fs).
    Most of these bugs could have been prevented by using \depsynth to automatically identify missing ordering requirements,
    but some crash-consistency issues are either not ordering related or are unlikely to be detected by \depsynth's litmus-test-driven approach.}
    {\xsmall\begin{tabular}{lll}
\toprule
Storage system & Crash-consistency bug & Preventable by \depsynth? \\
\midrule
ShardStore & Inconsistency in extent allocation (\#6) & Yes \\
ShardStore & Mismatch between soft and hard write pointers (\#7) & Yes \\
ShardStore & Index entries persisted before target data (\#8) & Yes \\
ShardStore & Crash consistency predicate too strong (\#9) & No---specification bug \\
ShardStore & Data loss after UUID collision (\#10) & No---unlikely to detect \\
btrfs & Extents deallocated too early in recovery (\texttt{bf50411}) & Yes \\
btrfs & Inode rename commits out of order (\texttt{d4682ba}) & Yes \\
f2fs & \texttt{fsync} failed after directory rename (\texttt{ade990f}) & Yes \\
f2fs & Wrong file size when zeroing file beyond EOF (\texttt{17cd07a}) & No---not reordering \\
\bottomrule
\end{tabular}
}
    \label{fig:bugs}
\end{table}

To understand how effective \depsynth can be in preventing crash-consistency bugs,
we surveyed all bugs reported by two recent papers~\cite{bornholt:s3,mohan:crashmonkey}
in three production storage systems
for which a known fix is available.
We manually analyze each bug and determine whether \depsynth could discover and prevent them.

\cref{fig:bugs} shows the results of our survey.
In six cases, \depsynth could have prevented the bug
by synthesizing a dependency rule to preclude a problematic reordering optimization.
Each of these bugs had small triggering test cases,
suggesting they would be reachable by a litmus-test-based approach like ours.
In the other three cases, our analysis shows that \depsynth would not prevent the bug.
One bug in ShardStore was a specification bug
in which the crash consistency predicate was too strong.
\depsynth assumes that the crash consistency predicate is correct,
and will miss specification bugs.
%and so unless the predicate was so strong that no solution was possible
%(in which case \depsynth would return \UNSAT),
%it would not detect any issues.
Another bug in ShardStore involved a collision between two randomly generated UUIDs.
While such a bug would be possible to find in principle using litmus tests,
it would be very unlikely,
and without a test that triggers the issue \depsynth cannot preclude it.
One bug in f2fs involved an incorrect file size being computed when zero-filling a file
beyond its existing endpoint.
This bug was a logic issue rather than a reordering one (i.e., occurring even without a crash),
and so no dependency rule would suffice to prevent it.
Overall, our analysis indicates
that \depsynth can prevent a range of ordering-related crash-consistency bugs,
but other bugs would require a different approach.

\subsection{Other Case Studies}\label{sec:eval:other}
% TODO include extra paragraph here, from the ECOOP edits

In addition to the \shardstore case study, % in \cref{sec:eval:shardstore},
we have applied \depsynth to two other storage systems.
The first is a modification of \shardstore with a different underlying disk hardware model.
The second is a simple log-strucutred file system.
\depsynth was effectively able to synthesize rules for both of these
additional cases, demonstrating the generality of our approach.

\paragraph{\shardstore with SMR.}
Part of \depsynth's requirement for input storage system implementations 
is a model of how the underlying disk hardware behaves.
For example, some storage devices automatically enforce
an order over writes or may disallow particular orders of writes
to the same extent on disk.
In the case study of \shardstore discussed above,
we use a disk model that represents a conventional magnetic-recording (CMR) hard disk.
For this second application of \depsynth, we instead use a model of
shingled magnetic recording (SMR) disks and synthesize rules
for \shardstore under this new model.

To understand how this difference in disk models will affect the synthesized rules,
we first give a brief overview of SMR disks.
Like traditional CMR disks, SMR drives store data on platters that contain tracks
read by a magnetic needle.
Where the technologies differ is in the layout of the tracks:
tracks in CMR disks are non-overlapping and lay parallel to each other,
while tracks in SMR lay partially over one another.
This allows SMR disks to store data more densely than their CMR counterparts,
but requires that all writes to a single SMR extent be \textit{appends}.
In order to write data at a location before the append pointer in an SMR extend,
the whole extent must first be wiped.

%This has two implications for our use of \depsynth over \shardstore with
%an SMR disk model.
%First, since SMR drives will force data to be appended to a single extent \textit{in order},
%rules that specify an order over writes to the same extent that \textit{matches the append order} are not necessary.
%Second, since data cannot be written out-of-order to an SMR extent without first wiping the extent,
%rules that specify any order over writes to the same extent \textit{other than the append order}
%will not be enforceable.
% TODO conclude this?

We ran \depsynth to synthesize rules for the \shardstore-like system with an underlying SMR disk model.
To do so, we generated a new set of \shardstoreinputtests litmus tests, each with a maximum
of \shardstoremaxops operations.
For this new system, \depsynth generated \shardstoresmrnumrules
in \shardstoresmrsynthesistime minutes.
During this process, \depsynth invoked the \rulesfortest procedure
for \shardstoresmrtestsused of the total \shardstoreinputtests tests.
%Note that though fewer rules were synthesized in a short amount of time
%compared to the case study in \cref{sec:eval:shardstore},
%this second synthesis task was not simply a sub-task of the first --- 
These results demonstrate that \depsynth is resiliant to changes in hardware used by storage systems.

\paragraph{Log-Structured File System.}
We have used \depsynth
to implement a log-structured file system~\cite{rosenblum:lfs}.
%with support for files and directories.
The file system supports five standard POSIX operations:
\texttt{open}, \texttt{creat}, \texttt{write}, \texttt{close}, and \texttt{mkdir}.
While our implementation is simple (300 lines of Racket code) compared to production file systems,
it has metadata structures for files and directories,
and so has its own subtle crash consistency requirements.
For example, updates to data and inode blocks must reach the
disk before the pointer to the tail of the log is updated.
To synthesize dependency rules for this file system,
we randomly generated \lfsinputtests{} litmus tests
with at most \lfsmaxops{} operations.
\depsynth synthesized a set of \lfsnumrules{} dependency rules
in \lfssynthesistime{} to make the file system crash consistent,
and during the search,
invokes \textsc{RulesForTest} for only \lfstestsused{} tests.
This result shows that \depsynth can automate crash consistency for storage systems other than key-value stores.
