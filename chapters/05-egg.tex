\chapter{\egg: Easy, Extensible, and Efficient \Egraphs}

\label{sec:egg}
\label{sec:impl}
\label{sec:lambda}

We implemented the techniques of rebuilding and \eclass analysis in \egg,
  an easy-to-use, extensible, and efficient \egraph library.
To the best of our knowledge,
  \egg is the first general-purpose, reusable \egraph implementation.
This has allowed focused effort on ease of use and optimization,
  knowing that any benefits will
  be seen across use cases as opposed to a single, ad hoc instance.

This section details \egg's implementation and some of the various
  optimizations and tools it provides to the user.
We use an extended example of a partial evaluator for the lambda calculus\footnote{
  \Egraphs do not have any ``built-in'' support for binding;
  for example, equality modulo alpha renaming is not free.
  The explicit substitution provided in this section is is illustrative but rather high in performance cost.
  Better support for languages with binding is important future work.
},
  for which we provide the complete source code (which few changes for readability)
  in \autoref{fig:lambda-lang} and \autoref{fig:lambda-analysis}.
While contrived, this example is compact and familiar, and it highlights
  (1) how \egg is used and (2) some of its novel features like
  \eclass analyses and dynamic rewrites.
It demonstrates how \egg can tackle binding,
  a perennially tough problem for \egraphs,
  with a simple explicit substitution approach
  powered by \egg's extensibility.
\autoref{sec:case-studies} goes further, providing real-world case studies of
  published projects that have depended on \egg.

\egg is implemented in \textasciitilde{}5000 lines of Rust,\footnote
{
  Rust \cite{rust} is a high-level systems programming language.
  \egg has been integrated into applications written in other
  programming languages using both C FFI and serialization approaches.
}
including code, tests, and documentation.
\egg is open-source, well-documented, and distributed via Rust's package
  management system.\footnote{
  Source: \url{https://github.com/mwillsey/egg}.
  Documentation: \url{https://docs.rs/egg}.
  Package: \url{https://crates.io/crates/egg}.
  \\
  This paper uses version 0.6 of \egg.
}
All of \egg's components are generic over the
  user-provided language, analysis, and cost functions.

\section{Ease of Use}
\label{sec:egg-easy}

\begin{figure}
\begin{subfigure}[t]{0.48\linewidth}
  \begin{lstlisting}[language=Rust, basicstyle=\tiny\ttfamily, numbers=left, escapechar=|,
   xleftmargin=13pt,
   numbersep=7pt,
]
define_language! {
  enum Lambda {
    // enum variants have data or children (eclass Ids)
    // [Id; N] is an array of N `Id`s

    // base type operators
    "+" = Add([Id; 2]), "=" = Eq([Id; 2]),
    "if" = If([Id; 3]),

    // functions and binding
    "app" = App([Id; 2]), "lam" = Lambda([Id; 2]),
    "let" = Let([Id; 3]), "fix" = Fix([Id; 2]),

    // (var x) is a use of `x` as an expression
    "var" = Use(Id),
    // (subst a x b) substitutes a for (var x) in b
    "subst" = Subst([Id; 3]),

    // base types have no children, only data
    Bool(bool), Num(i32), Symbol(String),
  }
}

// example terms and what they simplify to
// pulled directly from the |\egg|test suite

test_fn! { lambda_under, rules(),
  "(lam x (+ 4 (app (lam y (var y)) 4)))"
  => "(lam x 8))",
}

test_fn! { lambda_compose_many, rules(),
  "(let compose (lam f (lam g (lam x
                (app (var f)
                     (app (var g) (var x))))))
   (let add1 (lam y (+ (var y) 1))
   (app (app (var compose) (var add1))
        (app (app (var compose) (var add1))
             (app (app (var compose) (var add1))
                  (app (app (var compose) (var add1))
                       (var add1)))))))"
  => "(lam ?x (+ (var ?x) 5))"
}

test_fn! { lambda_if_elim, rules(),
  "(if (= (var a) (var b))
       (+ (var a) (var a))
       (+ (var a) (var b)))"
  => "(+ (var a) (var b))"
}\end{lstlisting}
\end{subfigure}
\hfill
\begin{subfigure}[t]{0.48\linewidth}
  \begin{lstlisting}[language=Rust, basicstyle=\tiny\ttfamily, escapechar=|, numbers=left, firstnumber=51,
  numbersep=7pt,
]
// Returns a list of rewrite rules
fn rules() -> Vec<Rewrite<Lambda, LambdaAnalysis>> { vec![

 // open term rules
 rw!("if-true";  "(if  true ?then ?else)" => "?then"),
 rw!("if-false"; "(if false ?then ?else)" => "?else"),
 rw!("if-elim";  "(if (= (var ?x) ?e) ?then ?else)" => "?else"
     if ConditionEqual::parse("(let ?x ?e ?then)",
                              "(let ?x ?e ?else)")),
 rw!("add-comm";  "(+ ?a ?b)"        => "(+ ?b ?a)"),
 rw!("add-assoc"; "(+ (+ ?a ?b) ?c)" => "(+ ?a (+ ?b ?c))"),
 rw!("eq-comm";   "(= ?a ?b)"        => "(= ?b ?a)"),

 // substitution introduction
 rw!("fix";     "(fix ?v ?e)" =>
                "(let ?v (fix ?v ?e) ?e)"),
 rw!("beta";    "(app (lam ?v ?body) ?e)" =>
                "(let ?v ?e ?body)"),

 // substitution propagation
 rw!("let-app"; "(let ?v ?e (app ?a ?b))" =>
                "(app (let ?v ?e ?a) (let ?v ?e ?b))"),
 rw!("let-add"; "(let ?v ?e (+   ?a ?b))" =>
                "(+   (let ?v ?e ?a) (let ?v ?e ?b))"),
 rw!("let-eq";  "(let ?v ?e (=   ?a ?b))" =>
                "(=   (let ?v ?e ?a) (let ?v ?e ?b))"),
 rw!("let-if";  "(let ?v ?e (if ?cond ?then ?else))" =>
                "(if (let ?v ?e ?cond)
                     (let ?v ?e ?then)
                     (let ?v ?e ?else))"),

 // substitution elimination
 rw!("let-const";    "(let ?v ?e ?c)" => "?c"
     if is_const(var("?c"))),
 rw!("let-var-same"; "(let ?v1 ?e (var ?v1))" => "?e"),
 rw!("let-var-diff"; "(let ?v1 ?e (var ?v2))" => "(var ?v2)"
     if is_not_same_var(var("?v1"), var("?v2"))),
 rw!("let-lam-same"; "(let ?v1 ?e (lam ?v1 ?body))" =>
                     "(lam ?v1 ?body)"),
 rw!("let-lam-diff"; "(let ?v1 ?e (lam ?v2 ?body))" =>
     ( CaptureAvoid {
        fresh: var("?fresh"), v2: var("?v2"), e: var("?e"),
        if_not_free: "(lam ?v2 (let ?v1 ?e ?body))"
                     .parse().unwrap(),
        if_free: "(lam ?fresh (let ?v1 ?e
                              (let ?v2 (var ?fresh) ?body)))"
                 .parse().unwrap(),
     })
     if is_not_same_var(var("?v1"), var("?v2"))),
]}\end{lstlisting}
\end{subfigure}
\vspace{-0.5em}
\caption[Language and rewrites for the lambda calculus in \egg]{
\egg is generic over user-defined languages;
  here we define a language and rewrite rules for a lambda calculus partial evaluator.
The provided \texttt{define\_language!} macro (lines 1-22) allows the simple definition
  of a language as a Rust \texttt{enum}, automatically deriving parsing and
  pretty printing.
A value of type \texttt{Lambda} is an \enode that holds either data that the
  user can inspect or some number of \eclass children (\eclass \texttt{Id}s).

Rewrite rules can also be defined succinctly (lines 51-100).
Patterns are parsed as s-expressions:
  strings from the \texttt{define\_language!} invocation (ex: \texttt{fix}, \texttt{=}, \texttt{+}) and
  data from the variants (ex: \texttt{false}, \texttt{1}) parse as operators or terms;
  names prefixed by ``\texttt{?}'' parse as pattern variables.

Some of the rewrites made are conditional using the
  ``\texttt{left => right if cond}''
  syntax.
The \texttt{if-elim} rewrite on line 57 uses \egg's provided
  \texttt{ConditionEqual} as a condition, only applying the right-hand side
  if the \egraph can prove the two argument patterns equivalent.
The final rewrite, \texttt{let-lam-diff}, is dynamic to support capture avoidance;
  the right-hand side is a Rust value that
  implements the \texttt{Applier} trait instead of a pattern.
\autoref{fig:lambda-analysis} contains the supporting code for these rewrites.

We also show some of the tests (lines 27-50)
  from \egg's \texttt{lambda} test suite.
The tests proceed by inserting the term on the left-hand side, running
  \egg's equality saturation, and then checking to make sure the right-hand
  pattern can be found in the same \eclass as the initial term.
}
\label{fig:lambda-rules}
\label{fig:lambda-lang}
\label{fig:lambda-examples}
\end{figure}

%%% Local Variables:
%%% TeX-master: "../thesis"
%%% End:


\egg's ease of use comes primarily from its design as a library.
By defining only a language and some rewrite rules,
  a user can quickly
  start developing a synthesis or optimization tool.
Using \egg as a Rust library,
  the user defines the language using the \texttt{define\_language!} macro
  shown in \autoref{fig:lambda-lang}, lines 1-22.
Childless variants in the language may contain data of user-defined types,
  and \eclass analyses or dynamic rewrites may inspect this data.

%Defining a language is the only necessary input for \egg.
%From there, a user may create and manipulate \egraphs that hold expressions from
%  that language.
%If the user wants to perform rewrites and equality saturation, \egg provides
%  facilities for this as well.

The user provides rewrites as shown in
  \autoref{fig:lambda-lang}, lines 51-100.
Each rewrite has a name, a left-hand side, and a right-hand side.
For purely syntactic rewrites, the right-hand is simply a pattern.
More complex rewrites can incorporate conditions or even dynamic right-hand
  sides, both explained in the \autoref{sec:egg-extensible} and \autoref{fig:lambda-applier}.

Equality saturation workflows, regardless of the application domain,
  typically have a similar structure:
add expressions to an empty \egraph, run rewrites until saturation or
  timeout, and extract the best equivalent expressions according to some cost
  function.
This ``outer loop'' of equality saturation involves a significant amount of
  error-prone boilerplate:
\begin{itemize}
  \item Checking for saturation, timeouts, and \egraph size limits.
  \item Orchestrating the read-phase, write-phase, rebuild system
    (\autoref{fig:rebuild-code}) that makes \egg fast.
  \item Recording performance data at each iteration.
  \item Potentially coordinating rule execution so that expansive rules like
    associativity do not dominate the \egraph.
  \item Finally, extracting the best expression(s) according to a
  user-defined cost function.
\end{itemize}

\egg provides these functionalities through its \texttt{Runner} and
  \texttt{Extractor} interfaces.
\texttt{Runner}s automatically detect saturation, and can be configured to stop
  after a time, \egraph size, or iterations limit.
The equality saturation loop provided by \egg calls \texttt{rebuild}, so users
  need not even know about \egg's deferred invariant maintenance.
\texttt{Runner}s record various metrics about each iteration automatically,
  and the user can hook into this to report relevant data.
\texttt{Extractor}s select the optimal term from an \egraph given a
  user-defined, local cost function.\footnote{
    As mentioned in \autoref{sec:tricks-extraction}, extraction can be
    implemented as part of an \eclass analysis.
    The separate \texttt{Extractor} feature is still useful for ergonomic and
    performance reasons.
  }
The two can be combined as well; users commonly record the ``best so far''
  expression by extracting in each iteration.

\autoref{fig:lambda-lang} also shows \egg's \texttt{test\_fn!}
  macro for easily creating tests (lines 27-50).
These tests create an \egraph with the given expression, run equality saturation
  using a \texttt{Runner}, and check to make sure the right-hand pattern can be
  found in the same \eclass as the initial expression.

\section{Extensibility}
\label{sec:egg-extensible}

For simple domains, defining a language and purely syntactic rewrites will
  suffice.
However, our partial evaluator requires interpreted reasoning, so we use some of
  \egg's more advanced features like \eclass analyses and dynamic rewrites.
Importantly, \egg supports these extensibility features as a library:
  the user need not modify the \egraph or \egg's internals.

\begin{figure}
\begin{minipage}[t]{0.49\linewidth}
  \begin{lstlisting}[
   language=Rust, basicstyle=\tiny\ttfamily, numbers=left,
   xleftmargin=11pt,
   numbersep=5pt,
]
type EGraph = egg::EGraph<Lambda, LambdaAnalysis>;
struct LambdaAnalysis;
struct FC {
  free: HashSet<Id>,    // our analysis data stores free vars
  constant: Option<Lambda>, // and the constant value, if any
}

// helper function to make pattern meta-variables
fn var(s: &str) -> Var { s.parse().unwrap() }

impl Analysis<Lambda> for LambdaAnalysis {
  type Data = FC; // attach an FC to each eclass
  // merge implements semilattice join by joining into `to`
  // returning true if the `to` data was modified
  fn merge(&self, to: &mut FC, from: FC) -> bool {
    let before_len = to.free.len();
    // union the free variables
    to.free.extend(from.free.iter().copied());
    if to.constant.is_none() && from.constant.is_some() {
      to.constant = from.constant;
      true
    } else {
      before_len != to.free.len()
    }
  }

  fn make(egraph: &EGraph, enode: &Lambda) -> FC {
    let f = |i: &Id| egraph[*i].data.free.iter().copied();
    let mut free = HashSet::default();
    match enode {
      Use(v) => { free.insert(*v); }
      Let([v, a, b]) => {
        free.extend(f(b)); free.remove(v); free.extend(f(a));
      }
      Lambda([v, b]) | Fix([v, b]) => {
        free.extend(f(b)); free.remove(v);
      }
      _ => enode.for_each_child(
             |c| free.extend(&egraph[c].data.free)),
    }
    FC { free: free, constant: eval(egraph, enode) }
  }

  fn modify(egraph: &mut EGraph, id: Id) {
    if let Some(c) = egraph[id].data.constant.clone() {
      let const_id = egraph.add(c);
      egraph.union(id, const_id);
    }
  }
}\end{lstlisting}
\end{minipage}
\hfill
\begin{minipage}[t]{0.46\linewidth}
  \begin{lstlisting}[language=Rust, basicstyle=\tiny\ttfamily, escapechar=@, numbers=left, firstnumber=51,
   numbersep=7pt,
]
// evaluate an enode if the children have constants
// Rust's `?` extracts an Option, early returning if None
fn eval(eg: &EGraph, enode: &Lambda) -> Option<Lambda> {
  let c = |i: &Id| eg[*i].data.constant.clone();
  match enode {
    Num(_) | Bool(_) => Some(enode.clone()),
    Add([x, y]) => Some(Num(c(x)? + c(y)?)),
    Eq([x, y]) => Some(Bool(c(x)? == c(y)?)),
    _ => None,
  }
}

// Functions of this type can be conditions for rewrites
trait ConditionFn = Fn(&mut EGraph, Id, &Subst) -> bool;

// The following two functions return closures of the
// correct signature to be used as conditions in @\autoref{fig:lambda-rules}@.
fn is_not_same_var(v1: Var, v2: Var) -> impl ConditionFn {
    |eg, _, subst| eg.find(subst[v1]) != eg.find(subst[v2])
}
fn is_const(v: Var) -> impl ConditionFn {
     // check the LambdaAnalysis data
    |eg, _, subst| eg[subst[v]].data.constant.is_some()
}

struct CaptureAvoid {
  fresh: Var, v2: Var, e: Var,
  if_not_free: Pattern<Lambda>, if_free: Pattern<Lambda>,
}

impl Applier<Lambda, LambdaAnalysis> for CaptureAvoid {
  // Given the egraph, the matching eclass id, and the
  // substitution generated by the match, apply the rewrite
  fn apply_one(&self, egraph: &mut EGraph,
               id: Id, subst: &Subst) -> Vec<Id>
  {
    let (v2, e) = (subst[self.v2], subst[self.e]);
    let v2_free_in_e = egraph[e].data.free.contains(&v2);
    if v2_free_in_e {
      let mut subst = subst.clone();
      // make a fresh symbol using the eclass id
      let sym = Lambda::Symbol(format!("_{}", id).into());
      subst.insert(self.fresh, egraph.add(sym));
      // apply the given pattern with the modified subst
      self.if_free.apply_one(egraph, id, &subst)
    } else {
      self.if_not_free.apply_one(egraph, id, &subst)
    }
  }
}\end{lstlisting}
  % \caption{
  %   Some of the rewrites in \autoref{fig:lambda-rules} are conditional,
  %     requiring conditions like \texttt{is\_not\_same\_var} or \texttt{is\_const}.
  %   Others are fully dynamic, using a custom applier like \texttt{CaptureAvoid}
  %     instead of a syntactic right-hand side.
  %   Both conditions and custom appliers can use the computed data from the
  %     \eclass analysis; for example, \texttt{CaptureAvoid} only $\alpha$-renames if
  %     there might be a name collision.
  % }
\end{minipage}
\caption[\Eclass analysis and conditional/dynamic rewrites for the lambda calculus]{
Our partial evaluator example highlights three important features \egg provides
  for extensibility: \eclass analyses, conditional rewrites, and dynamic
  rewrites.
  
The \texttt{LambdaAnalysis} type, which implements the \texttt{Analysis} trait,
  represents the \eclass analysis.
Its associated data (\texttt{FC}) stores
  the constant term from that \eclass (if any) and
  an over-approximation of the free variables used by terms in that \eclass.
The constant term is used to perform constant folding.
The \texttt{merge} operation implements the semilattice join, combining the free
  variable sets and taking a constant if one exists.
In \texttt{make}, the analysis computes the free variable sets based on the
  \enode and the free variables of its children;
  the \texttt{eval} generates the new constants if possible.
The \texttt{modify} hook of \texttt{Analysis} adds the constant to the \egraph.

Some of the conditional rewrites in \autoref{fig:lambda-rules} depend on
  conditions defined here.
Any function with the correct signature may serve as a condition.

The \texttt{CaptureAvoid} type implements the \texttt{Applier} trait, allowing
  it to serve as the right-hand side of a rewrite.
\texttt{CaptureAvoid} takes two patterns and some pattern variables.
It checks the free variable set to determine if a capture-avoiding substitution
  is required, applying the \texttt{if\_free} pattern if so and the
  \texttt{if\_not\_free} pattern otherwise.
}
\label{fig:lambda-applier}
\label{fig:lambda-analysis}
\end{figure}

%%% Local Variables:
%%% TeX-master: "../thesis"
%%% End:


\autoref{fig:lambda-applier} shows the remainder of the code for our lambda
  calculus partial evaluator.
It uses an \eclass analysis (\texttt{LambdaAnalysis})
  to track free variables and constants associated
  with each \eclass.
The implementation of the \eclass analysis is in Lines 11-50.
The \eclass analysis invariant
  guarantees that the analysis data contains an over-approximation of free variables
  from terms represented in that \eclass.
The analysis also does constant folding
  (see the \texttt{make} and \texttt{modify} methods).
The \texttt{let-lam-diff} rewrite (Line 90, \autoref{fig:lambda-rules})
  uses the \texttt{CaptureAvoid} (Lines 81-100, \autoref{fig:lambda-applier})
  dynamic right-hand side to do capture-avoiding
  substitution only when necessary based on the free variable information.
The conditional rewrites from \autoref{fig:lambda-rules} depend on the
  conditions \texttt{is\_not\_same\_var} and
  \texttt{is\_var} (Lines 68-74, \autoref{fig:lambda-applier})
  to ensure correct substitution.

\egg is extensible in other ways as well.
As mentioned above, \texttt{Extractor}s are parameterized by a user-provided
  cost function.
\texttt{Runner}s are also extensible with user-provided rule schedulers that can
  control the behavior of potentially troublesome rewrites.
\label{sec:rule-scheduling}
In typical equality saturation, each rewrite is searched for and applied each
  iteration.
This can cause certain rewrites, commonly associativity or distributivity,
  to dominate others and make the search space less productive.
Applied in moderation, these rewrites can trigger other rewrites and find
  greatly improved expressions,
  but they can also slow the search by
  exploding the \egraph exponentially in size.
By default, \egg uses the built-in backoff scheduler
  that identifies rewrites that are matching in exponentially-growing
  locations and temporarily bans them.
We have observed that this greatly reduced run time (producing the same results)
  in many settings.
\egg can also use a conventional every-rule-every-time scheduler, or the user
  can supply their own.

\section{Efficiency}
\label{sec:egg-efficient}

\egg's novel \textit{rebuilding} algorithm (\autoref{sec:rebuild})
combined with systems programming best practices
  makes \egraphs---and the equality saturation
  use case in particular---more efficient than prior tools.

\egg is implemented in Rust, giving the compiler freedom to
  specialize and inline user-written code.
This is especially important as
  \egg's generic nature leads to tight interaction
  between library code
  (e.g., searching for rewrites) and user code (e.g., comparing operators).
\egg is designed from the ground up to use cache-friendly,
  flat buffers with minimal indirection for most internal data structures.
This is in sharp contrast to traditional representations of \egraphs
  \cite{nelson, simplify} that contains many tree- and linked list-like data
  structures.
\egg additionally compiles patterns to be executed by a small virtual machine
  \cite{ematching}, as opposed to recursively walking the tree-like
  representation of patterns.

Aside from deferred rebuilding, \egg's equality saturation algorithm leads to
  implementation-level performance enhancements.
Searching for rewrite matches, which is the bulk of running time, can be
  parallelized thanks to the phase separation.
Either the rules or \eclasses could be searched in parallel.
Furthermore, the once-per-iteration frequency of rebuilding allows \egg to
  establish other performance-enhancing invariants that hold during the
  read-only search phase.
For example, \egg sorts \enodes within each \eclass to enable binary search, and
  also maintains a cache mapping function symbols to \eclasses that
  contain \enodes with that function symbol.

Many of \egg's extensibility features can also be used to improve performance.
As mentioned above, rule scheduling can lead to great performance improvement in
  the face of ``expansive'' rules that would otherwise dominate the search
  space.
The \texttt{Runner} interface also supports user hooks that can stop
  the equality saturation after some arbitrary condition.
This can be useful when using equality saturation to prove terms equal; once
  they are unified, there is no point in continuing.
\label{sec:egg-batched}
\egg's \texttt{Runner}s also support batch simplification, where multiple terms
  can be added to the initial \egraph before running equality saturation.
If the terms are substantially similar, both rewriting and any \eclass analyses
  will benefit from the \egraph's inherent structural deduplication.
The case study in \autoref{sec:herbie} uses batch simplification to achieve
  a large speedup with simplifying similar expressions.

%%% Local Variables:
%%% TeX-master: "../thesis"
%%% End:
