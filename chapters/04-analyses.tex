\chapter{Extending \Egraphs with \Eclass Analyses}
\label{sec:extensions}

As discussed so far, \egraphs and equality saturation provide an efficient way
  to implement a term rewriting system.
Rebuilding enhances that efficiency, but the approach remains designed for
  purely syntactic rewrites.
However, program analysis and optimization typically require more than just
  syntactic information.
Instead, transformations are \emph{computed} based on the input terms and also semantic facts
  about that input term, e.g., constant value, free variables, nullability,
  numerical sign, size in memory, and so on.
The ``purely syntactic'' restriction has forced existing equality saturation
  applications~\cite{eqsat, eqsat-llvm, herbie} to
  resort to ad hoc passes over the \egraph
  to implement analyses like constant folding.
These ad hoc passes require manually manipulating the \egraph,
  the complexity of which could prevent the implementation of more sophisticated
  analyses.
  % naturally expressed in the more conventional term rewriting setting,
  % but must implemented as complicated \textit{ad hoc} passes over the \egraph.

We present a new technique called \textit{\eclass analysis},
  which allows the concise
  expression of a program analysis over the \egraph.
An \eclass analysis resembles abstract interpretation
  lifted to the \egraph level,
  attaching \textit{analysis data} from a semilattice to each \eclass.
The \egraph maintains and propagates this data as
  \eclasses get merged and new \enodes are added.
Analysis data can be used directly to modify the \egraph, to inform
  how or if rewrites apply their right-hand sides, or to determine the cost of
  terms during the extraction process.

\Eclass analyses provide a general mechanism to replace what previously
  required ad hoc extensions that manually manipulate the \egraph.
\Eclass analyses also fit within the equality saturation workflow,
  so they can naturally cooperate with the equational reasoning provided by
  rewrites.
Moreover, an analysis lifted to the \egraph level automatically benefits from a
  sort of ``partial-order reduction'' for free:
  large numbers of similar programs may be analyzed for little additional cost
  thanks to the \egraph's compact representation.

This section provides a conceptual explanation of \eclass analyses as well
  as dynamic and conditional rewrites that can use the analysis data.
The following sections will provide concrete examples:
  \autoref{sec:impl} discusses the \egg implementation and a complete example of a
  partial evaluator for the lambda calculus;
  \autoref{sec:case-studies} discusses how three published projects have used
  \egg and its unique features (like \eclass analyses).

% An \eclass analysis may,  modify the \egraph itself to inject new based on the computed data,
% In this way, the \eclass analysis can work with the rewrites as opposed

% \Egg offers many convenient ways to interact with the \egraph that are difficult
%   or impossible in other implementations.
% These tools give \egg the flexibility highlighted by the diverse case studies
%   in \autoref{sec:case-studies}
%   and the lambda calculus example in \autoref{sec:lambda}.

% Like the rest of \egg, these extensions are generic over the language and
%   rewrites that the user is working with.
% These are all novel in practice, as \egg is the first (to our knowledge)
%   general-purpose, reusable \egraph library.
% \Eclass analyses appear to be conceptually novel as well.

% \Chandra{this entire section is a bit light in details and doesn't
% always connect back to the goals of egg the right way. For example,
% without a concrete advantage of runners, it's not clear why it is useful.}

\section{E-Class Analyses}
\label{sec:analysis}

An \eclass analysis defines a domain $D$ and associates a value $d_{c} \in D$ to
  each \eclass $c$.
The \eclass $c$ contains the associated data $d_{c}$,
  i.e., given an \eclass $c$, one can get $d_{c}$ easily, but not vice-versa.

The interface of an \eclass analysis is as follows,
  where $G$ refers to the \egraph,
  and $n$ and $c$ refer to \enodes and \eclasses within $G$:

% We will use the following metavariables and syntax in this section:

% \begin{center}
%   $G$: \egraph, $c$: \eclass, $n$: \enode,
%   $d_{c}$: the analysis data associated with \eclass $c$
% \end{center}
\vspace{1em}
\begin{tabular}{lp{0.7\linewidth}}
  $\textsf{make}(n) \to d_{c}$ &
    When a new \enode $n$ is added to $G$ into a new, singleton \eclass $c$,
    construct a new value $d_{c} \in D$ to be associated with $n$'s new \eclass,
    typically by accessing the associated data of $n$'s children.
  \\
  $\textsf{join}(d_{c_1}, d_{c_2}) \to d_{c}$ &
    When \eclasses $c_{1}, c_{2}$ are being merged into $c$,
    join $d_{c_1}, d_{c_2}$ into a new value $d_{c}$ to be associated with the
    new \eclass $c$.
  \\
  $\textsf{modify}(c) \to c'$ &
    Optionally modify the \eclass $c$ based on $d_{c}$, typically by adding an
      \enode to $c$.
    Modify should be idempotent if no other changes occur to the \eclass, i.e.,
      $\textsf{modify}(\textsf{modify}(c)) = \textsf{modify}(c)$
    % Returning $c$ unmodified ($c = c'$) suffices in many cases, and is the
    % default implementation for analyses in \egg.
\end{tabular}
\vspace{1em}

The domain $D$ together with the \textsf{join} operation should form a join-semilattice.
The semilattice perspective is useful for defining the \textit{analysis invariant}
  (where $\vee$ is the \textsf{join} operation):
\[
  \forall c \in G.\quad
  d_{c} = \bigvee_{n \in c} \textsf{make}(n)
  \quad \text{and} \quad
  \textsf{modify}(c) = c
\]

The first part of the analysis invariant states that the data associated with
  each \eclass must be the \textsf{join} of the \textsf{make} for every \enode
  in that \eclass.
Since $D$ is a join-semilattice, this means that
  $\forall c, \forall n \in c, d_{c} \geq \textsf{make}(n) $.
The motivation for the second part is more subtle.
Since the analysis can modify an \eclass through the \textsf{modify} method,
  the analysis invariant asserts that these modifications are driven to a fixed
  point.
When the analysis invariant holds, a client looking at the analysis data can be
  assured that the analysis is ``stable'' in the sense that
  recomputing \textsf{make}, \textsf{join}, and \textsf{modify} will not
  modify the \egraph or any analysis data.

\subsection{Maintaining the Analysis Invariant}


\begin{figure}
  \begin{minipage}[t]{0.47\linewidth}
    \begin{lstlisting}[gobble=4, numbers=left, numberstyle=\color{black}, basicstyle=\scriptsize\ttfamily\color{black!40}, escapechar=|,
   xleftmargin=13pt,
   numbersep=7pt,
]
    def add(enode):
      enode = self.canonicalize(enode)
      if enode in self.hashcons:
        return self.hashcons[enode]
      else:
        eclass = self.new_singleton_eclass(enode)
        for child_eclass in enode.children:
          child_eclass.parents.add(enode, eclass)
        self.hashcons[enode] = eclass
        |\color{black}\label{line:add1} eclass.data = analysis.make(enode)|
        |\color{black}\label{line:add2} analysis.modify(eclass)|
        return eclass

    def merge(eclass1, eclass2)
      union = self.union_find.union(eclass1, eclass2)
      if not union.was_already_unioned:
        |\color{black}\label{line:merge1}d1, d2 = eclass1.data, eclass2.data|
        |\color{black}\label{line:merge2}union.eclass.data = analysis.join(d1, d2)|
        self.worklist.add(union.eclass)
      return union.eclass
    \end{lstlisting}
  \end{minipage}
  \hfill
  \begin{minipage}[t]{0.47\linewidth}
    \begin{lstlisting}[gobble=4, numbers=left, firstnumber=21, numberstyle=\color{black}, basicstyle=\scriptsize\ttfamily\color{black!40}, escapechar=|,
    numbersep=5pt,
]
    def repair(eclass):
      for (p_node, p_eclass) in eclass.parents:
        self.hashcons.remove(p_node)
        p_node = self.canonicalize(p_node)
        self.hashcons[p_node] = self.find(p_eclass)

      new_parents = {}
      for (p_node, p_eclass) in eclass.parents:
        p_node = self.canonicalize(p_node)
        if p_node in new_parents:
          self.union(p_eclass, new_parents[p_node])
        new_parents[p_node] = self.find(p_eclass)
      eclass.parents = new_parents
    \end{lstlisting}
    \vspace{-3mm}
    \begin{lstlisting}[gobble=4, numbers=left, firstnumber=34, basicstyle=\scriptsize\ttfamily, escapechar=|,
    numbersep=5pt,
]

      # any mutations modify makes to eclass
      # will add to the worklist
      |\label{line:repair1}|analysis.modify(eclass)
      for (p_node, p_eclass) in eclass.parents:
        new_data = analysis.join(
          p_eclass.data,
          analysis.make(p_node))
        if new_data != p_eclass.data:
          p_eclass.data = new_data
          |\label{line:repair2}|self.worklist.add(p_eclass)
    \end{lstlisting}
  \end{minipage}
  \caption{
    The pseudocode for maintaining the \eclass analysis invariant is largely
      similar to how rebuilding maintains congruence closure
      (\autoref{sec:rebuilding}).
    Only lines \ref{line:add1}--\ref{line:add2},
      \ref{line:merge1}--\ref{line:merge2},
      and \ref{line:repair1}--\ref{line:repair2} are added.
    Grayed out or missing code is unchanged from \autoref{fig:rebuild-code}.
  }
  \label{fig:rebuild-analysis}
\end{figure}

We extend the rebuilding procedure from \autoref{sec:rebuilding} to restore the
  analysis invariant as well as the congruence invariant.
\autoref{fig:rebuild-analysis} shows the necessary modifications to the
  rebuilding code from \autoref{fig:rebuild-code}.

Adding \enodes and merging \eclasses risk breaking the analysis invariant in
  different ways.
Adding \enodes is the simpler case; lines \ref{line:add1}--\ref{line:add2}
  restore the invariant for the newly created, singleton \eclass that holds the
  new \enode.
When merging \enodes, the first concern is maintaining the semilattice portion of the
  analysis invariant.
Since \textsf{join} forms a semilattice over the domain $D$ of the analysis
  data, the order in which the joins occur does not matter.
Therefore, line \ref{line:merge2} suffices to update the analysis data of the
  merged \eclass.

Since $\textsf{make}(n)$ creates analysis data by looking at the data of $n$'s,
  children, merging \eclasses can violate the analysis invariant in the same way
  it can violate the congruence invariant.
The solution is to use the same worklist mechanism introduced in
  \autoref{sec:rebuilding}.
Lines \ref{line:repair1}--\ref{line:repair2} of the \texttt{repair} method
  (which \texttt{rebuild} on each element of the worklist)
  re-\textsf{make} and \textsf{merge} the analysis data of the parent of any
  recently merged \eclasses.
The new \texttt{repair} method also calls \textsf{modify} once, which suffices
  due to its idempotence.
In the pseudocode, \textsf{modify} is reframed as a mutating method for clarity.

\Egg's implementation of \eclass analyses assumes that the analysis domain $D$
  is indeed a semilattice and that \textsf{modify} is idempotent.
Without these properties, \egg may fail to restore the analysis invariant on
  \texttt{rebuild}, or it may not terminate.

% \subsection{Modifying the \Egraph from an \Eclass Analysis}
\subsection{Example: Constant Folding}

The data produced by \eclass analyses can be
  usefully consumed by other components of an equality saturation system
  (see \autoref{sec:rewrites}),
  but \eclass analyses can be useful on their own thanks to the
  \textsf{modify} hook.
Typical \textsf{modify} hooks will either do nothing, check some invariant about
  the \eclasses being merged, or add an \enode to that \eclass
  (using the regular \texttt{add} and \texttt{merge} methods of the \egraph).

As mentioned above, other equality saturation implementations have implemented
  constant folding as custom, ad hoc passes over the \egraph.
We can formulate constant folding as an \eclass analysis that highlights the
  parallels with abstract interpretation.
Let the domain $D = \texttt{Option<Constant>}$, and let the \texttt{join}
  operation be the ``\texttt{or}'' operation of the \texttt{Option} type:
\begin{figure}[h!]
\begin{lstlisting}[language=Rust, basicstyle=\ttfamily\footnotesize, xleftmargin=35mm]
match (a, b) {
  (None,    None   ) => None,
  (Some(x), None   ) => Some(x),
  (None,    Some(y)) => Some(y),
  (Some(x), Some(y)) => { assert!(x == y); Some(x) }
}
\end{lstlisting}
\end{figure}
Note how \textsf{join} can also aid in debugging by checking properties about
  values that are unified in the \egraph;
  in this case we assert that all terms represented in an \eclass should have
  the same constant value.
The \textsf{make} operation serves as the abstraction function, returning the
  constant value of an \enode if it can be computed from the constant values
  associated with its children \eclasses.
The \textsf{modify} operation serves as a concretization function in this
  setting.
If $d_{c}$ is a constant value, then $\textsf{modify}(c)$ would add
  $\gamma(d_{c}) = n$ to $c$, where $\gamma$ concretizes the constant value into
  a childless \enode.

Constant folding is an admittedly simple analysis, but one that did not formerly
  fit within the equality saturation framework.
\Eclass analyses support more complicated analyses in a general way, as
  discussed in later sections on the \egg implementation and case studies
  (Sections \ref{sec:impl} and \ref{sec:case-studies}).

\section{Conditional and Dynamic Rewrites}
\label{sec:rewrites}

In equality saturation applications, most of the rewrites are purely
  syntactic.
In some cases, additional data may be needed to determine if or how to perform
  the rewrite.
For example, the rewrite $x / x \to 1$ is only valid if $x \neq 0$.
A more complex rewrite may need to compute the right-hand side dynamically based
  on an analysis fact from the left-hand side.

The right-hand side of a rewrite can be generalized to a function
  \textsf{apply} that takes a substitution and an \eclass generated from
  e-matching the left-hand side, and produces a term to be added to the \egraph
  and unified with the matched \eclass.
For a purely syntactic rewrite, the \textsf{apply} function need not inspect the
  matched \eclass in any way; it would simply apply
  the substitution to the right-hand pattern to produce a new term.

\Eclass analyses greatly increase the utility of this generalized form of
  rewriting.
The \textsf{apply} function can look at the analysis data for the matched
  \eclass or any of the \eclasses in the substitution to determine if or how to
  construct the right-hand side term.
These kinds of rewrites can broken down further into two categories:
\begin{itemize}
  \item \textit{Conditional} rewrites like $x / x \to 1$ that are purely
  syntactic but whose validity depends on checking some analysis data;
  \item \textit{Dynamic} rewrites that compute the right-hand side based on
  analysis data.
\end{itemize}

Conditional rewrites are a subset of the more general dynamic rewrites.
Our \egg implementation supports both.
The example in \autoref{sec:impl} and case studies in \autoref{sec:case-studies}
  heavily use generalized rewrites, as it is typically the most convenient way
  to incorporate domain knowledge into the equality saturation
  framework.

\section{Extraction}
\label{sec:tricks-extraction}

Equality saturation typically ends with an extraction phase that selects an
  optimal represented term from an \eclass according to some cost function.
In many domains \cite{herbie, szalinski}, AST size
  (sometimes weighted differently for different operators) suffices as a simple,
  local cost function.
We say a cost function $k$ is local when the cost of a term $f(a_{1}, ...)$ can be
  computed from the function symbol $f$ and the costs of the children.
With such cost functions, extracting an optimal term can be efficiently done
  with a fixed-point traversal over the \egraph that selects the minimum cost
  \enode from each \eclass \cite{herbie}.

Extraction can be formulated as an \eclass analysis when the cost function
  is local.
The analysis data is a tuple $(n, k(n))$ where $n$ is the cheapest \enode
  in that \eclass and $k(n)$ its cost.
The $\textsf{make}(n)$ operation calculates the cost $k(n)$ based on
  the analysis data (which contain the minimum costs) of $n$'s children.
The \textsf{merge} operation simply takes the tuple with lower cost.
The semilattice portion of the analysis invariant then guarantees that the
  analysis data will contain the lowest-cost \enode in each class.
Extract can then proceed recursively;
  if the analysis data for \eclass $c$ gives $f(c_{1}, c_{2}, ...)$ as the optimal \enode,
  the optimal term represented in $c$ is
  $\textsf{extract}(c) = f( \textsf{extract}(c_{1}), \textsf{extract}(c_{2}), ... )$.
% The optimal term represented in an \eclass can then be built recursively,
%   starting with the optimal \enode from the analysis data.
% Extraction can be completed by starting from the desired \eclass and
%   building the term recursively based on the \enode from the analysis data.
This not only further demonstrates the generality of \eclass analyses, but also
  provides the ability to do extraction ``on the fly''; conditional and dynamic
  rewrites can determine their behavior based on the cheapest term in an \eclass.

Extraction (whether done as a separate pass or an \eclass analysis) can also
  benefit from the analysis data.
Typically, a local cost function can only look at the function symbol of the
  \enode $n$ and the costs of $n$'s children.
When an \eclass analysis is attached to the \egraph, however, a cost function
  may observe the data associated with $n$'s \eclass, as well as the data
  associated with $n$'s children.
This allows a cost function to depend on computed facts rather that just purely
  syntactic information.
In other words, the cost of an operator may differ based on its inputs.
\autoref{sec:tensat} provides a motivating case study wherein an \eclass
  analysis computes the size and shape of tensors, and this size information
  informs the cost function.

%%% Local Variables:
%%% TeX-master: "../thesis"
%%% End:
