\chapter{JitSynth: Synthesizing Just-In-Time Compilers for In-Kernel DSLs}
\label{c:jitsynth}
% \newcommand{\sketch}{S}

\usepackage[normalem]{ulem}

% PACKAGES %%%%%%%%%%%%%%%%%%%%%%%%%%%%%%%%%
%% Some recommended packages.
\usepackage{booktabs}   %% For formal tables:
                        %% http://ctan.org/pkg/booktabs
%\usepackage{subcaption} %% For complex figures with subfigures/subcaptions
                        %% http://ctan.org/pkg/subcaption


% \usepackage{courier}            % standard fixed width font
%\usepackage[scaled]{helvet} % see www.ctan.org/get/macros/latex/required/psnfss/psnfss2e.pdf
\usepackage[procnames]{listings}          % format code
\usepackage[shortlabels]{enumitem}      % adjust spacing in enums
\usepackage{algorithm,algorithmicx}
\usepackage{graphicx}
\usepackage{subcaption}
\usepackage{comment}
\usepackage{amsmath}
%\usepackage[hyphens]{url}
%\usepackage[breaklinks]{hyperref}
\usepackage{bbm, dsfont}
\usepackage{mathrsfs}
\usepackage{xcolor}
\usepackage{esvect}
\usepackage{xspace}
\usepackage{array, multirow} % multirow in table
\usepackage{rotating, makecell} % cell rotation
\usepackage[noend]{algpseudocode}
\usepackage{balance}
\usepackage{qtree}
%\usepackage{amsfonts}
\let\Bbbk\relax
\usepackage{amssymb}
\usepackage{multirow}
\usepackage{fancyvrb}
\usepackage{pgfplots}
% \pgfplotsset{compat=1.16}
\usepackage{stmaryrd}
\usepackage{newfloat}
\usepackage{syntax}
\usepackage{tikz}
\usepackage[capitalize]{cleveref}
%\usepackage[firstpage]{draftwatermark}
%%%%%%%%%%%%%%%%%%%%%%%%%%%%%%%%%5%%%%%%%%%%

\def\sectionautorefname{Section}
\def\subsectionautorefname{Section}
\newcommand{\definitionautorefname}{Definition}
% COMMANDS %%%%%%%%%%%%%%%%%%%%%%%%%%%%%%%%%%%
%\newcommand{\toolname}{{\sc Austenite}\xspace}
\newcommand{\toolname}{\textsc{DepSynth}\xspace}
\newcommand{\shardstore}{\textsc{CloudKV}\xspace}
%\newcommand{\depsynth}{{\sc DepSynth}\xspace}
\newcommand{\kv}{storage system\xspace}
\newcommand{\kvs}{storage systems\xspace}
\newcommand{\dstate}{disk state\xspace}
\newcommand{\states}{disk states\xspace}
\newcommand{\dwrite}{disk write\xspace}
\newcommand{\writes}{disk writes\xspace}
% \newcommand{\pctslowdown}{$N$}

\newcommand{\mathword}[1]{\ensuremath{#1}\xspace}
\newcommand{\mathid}[1]{\mathword{\mathit{#1}}}
\newcommand{\putreq}{\texttt{PUT}\xspace}
\newcommand{\get}{\texttt{GET}\xspace}
\newcommand{\delete}{\texttt{DELETE}\xspace}
\newcommand{\flush}{\texttt{FLUSH}\xspace}
\newcommand{\bflush}{\texttt{INDEXFLUSH}\xspace}
\newcommand{\clean}{\texttt{CLEAN}\xspace}
\newcommand{\onbool}{\textit{on}\xspace}
\newcommand{\syncbool}{\textit{sync}\xspace}
\newcommand{\crash}{\mathcal{C}\xspace}
\newcommand{\valid}{\mathcal{V}\xspace}
\newcommand{\true}{\textbf{true}\xspace}
\newcommand{\false}{\textbf{false}\xspace}
\newcommand{\tvar}[1]{\makebox[.9em][r]{$#1$}}
\newcommand{\local}{\mathid{op}}
\newcommand{\concat}{\oplus}
\newcommand{\deprule}[3]{#1\rightsquigarrow_{#3}#2}
\newcommand{\depedge}[2]{#1\rightsquigarrow #2}
\DeclareMathOperator{\extractop}{extract}
\newcommand{\extract}[3]{\extractop(#1, #2, #3)}

\newcommand{\BV}[1]{\ensuremath{\mathid{BV}(#1)}}
\newcommand{\tfun}{\ensuremath{\mathcal{T}}\xspace}
\newcommand{\prog}[1]{\ensuremath{P(#1)}}
\newcommand{\denotes}[1]{\ensuremath{\llbracket #1 \rrbracket}}
\newcommand{\host}{\mathword{\mathcal{H}}}
\newcommand{\size}[1]{\ensuremath{|#1|}}

\newcommand{\cc}[1]{\texttt{#1}}

\newcommand{\todo}[1]{\textbf{\color{red} [[#1]]}}

\newcommand{\tighten}{\looseness=-1}
%%%%%%%%%%%%%%%%%%%%%%%%%%%%%%%%%%%%%%%%%%%%%%

\newcommand{\graphsuff}{\textproc{Sufficient}\xspace}
\newcommand{\acyclic}{\textproc{Acyclic}\xspace}
\newcommand{\graphsearch}{\textproc{GraphSearch}\xspace}
\newcommand{\getwrites}{\textproc{Writes}\xspace}
\newcommand{\getnecpaths}{\textproc{NecessaryPaths}\xspace}
\newcommand{\getdepgraph}{\textproc{GetDependencyGraph}\xspace}
\newcommand{\prune}{\textproc{Prune}\xspace}
\newcommand{\toplevel}{\textproc{DepSynth}\xspace}
\newcommand{\sccsearch}{\textproc{SccGraphSearch}\xspace}
\newcommand{\rulessearch}{\textproc{RulesSearch}\xspace}
\newcommand{\resolvecons}{\textproc{ResolveConflicts}\xspace}
\newcommand{\multisearch}{\textproc{MultiTestSearch}\xspace}
\newcommand{\genrules}{\textproc{GenerateRules}\xspace}

\newcommand{\Map}[1][]{\mathword{\mathcal{M}_{#1}}}

\newcommand{\interpret}[1]{\emph{Interpret}(#1)}
\newcommand{\evaluate}[2]{\emph{Evaluate}_{#2}(#1)}
\newcommand{\consistent}[1]{\emph{Consistent}(#1)}
\newcommand{\run}[3]{\ensuremath{\emph{Run}\left(#1, #2, #3\right)}}
\renewcommand{\valid}[3]{\emph{Valid}_#3(#1, #2)}
\renewcommand{\vec}[1]{\boldsymbol{#1}}
\newcommand{\UNSAT}{\ensuremath{\text{UNSAT}}\xspace}
\newcommand{\UNKNOWN}{\ensuremath{\text{UNKNOWN}}\xspace}
\newcommand{\sys}{\ensuremath{\mathcal{O}}\xspace}

\newcommand{\impl}{\ensuremath{\mathcal{O}}\xspace}
\newcommand{\test}{\ensuremath{T}\xspace}
\newcommand{\tests}{\ensuremath{\mathcal{T}}\xspace}
\newcommand{\consist}{\ensuremath{\emph{Consistent}}\xspace}
\newcommand{\ruleset}{\ensuremath{R}\xspace}
\renewcommand{\wr}{\ensuremath{W}\xspace}
\newcommand{\ord}{\ensuremath{\emph{order}}\xspace}
\newcommand{\gr}{\ensuremath{G}\xspace}
\newcommand{\depsynthalg}{\textsc{DepSynth}\xspace}
\newcommand{\rulesfortest}{\textsc{RulesForTest}\xspace}
\newcommand{\phaseone}{\textsc{Phase1}\xspace}
\newcommand{\phasetwo}{\textsc{Phase2}\xspace}
\newcommand{\crashconsistentalg}{\textsc{CrashConsistent}\xspace}
\newcommand{\rulesforgraph}{\textsc{RulesForGraph}\xspace}

\newcommand{\xsmall}{\fontsize{9pt}{11pt}\selectfont}

% terminology  %%%%%%%%%%%%%%%%%%%%%%%%%%%%%%%


% LANGUAGE DEF %%%%%%%%%%%%%%%%%%%%%%%%%%%%%%%
\lstset{
  columns=flexible,
  keepspaces,
  xleftmargin=1em,
  basicstyle=\ttfamily,
  keywordstyle=\color{green!60!black}\bfseries,
  keywordstyle=[2]\color{green!60!black},
  keywordstyle=[3]\color{blue!60!black}\bfseries,
  stringstyle=\color{red},
  commentstyle=\color{orange!70!black},
  % procnamestyle=\color{blue},
  basicstyle=\small\ttfamily,
  lineskip=-1em,
}

\lstdefinelanguage{rosette}{
  morekeywords=[1]{verify,solve,forall,and,or,assert,s-exp,rosette,set!,begin,define,define-match,define-values,define-syntax,define-syntax-rule,syntax-rules,let,let*,if,when,unless,match-define,lambda,provide,cond,case,else,struct,letrec,for/list,true,false,null,local,require,rename-in,??,define-symbolic,define-symbolic*,not,=>,ite,\#lang,\#:transparent,\#:mutable,equal?,match,list},
  morekeywords=[2]{bv,bvult,bvuge,bvsub,~>,bitvector,integer?},
  morekeywords=[3]{serval:split-pc,serval:bug-on},
  alsoletter={\#,:,?,-,=>,*},
  morecomment=[l]{;},
}

\lstdefinelanguage{c}{
  morekeywords=[1]{struct,switch,case,sizeof,return,break},
  morekeywords=[2]{...},
}

\lstdefinelanguage{sketch}{
  morekeywords=[1]{struct,switch,case,sizeof,return,break},
}

\lstdefinelanguage{py}[]{Python}{
  morekeywords=[1]{with},
  morekeywords=[2]{self, True, False},
  sensitive,
  procnamekeys={def,class},
  procnamestyle=\color{blue},
}

\renewcommand{\algorithmiccomment}[1]{\hfill\textcolor{orange!70!black}{$\triangleright$ #1}}

\renewcommand{\algorithmiccomment}[1]{\hfill{\color{gray}$\triangleright$ \it #1}}
\renewcommand{\alglinenumber}[1]{{\color{gray}\fontsize{6pt}{1em}\selectfont #1}}
% \algrenewcommand\Call[2]{\textproc{#1}(#2)}


%%%%%%%%%%%%%%%%%%%%%%%%%%%%%%%%%%%%%%%%%%%%%%

%\begin{abstract}

Modern operating systems allow user-space applications to submit code for kernel
execution through the use of in-kernel domain specific languages (DSLs).
Applications use these DSLs to customize system policies and add new
functionality. For performance, the kernel executes them via just-in-time (JIT)
compilation. The correctness of these JITs is crucial for the security of the
kernel: bugs in in-kernel JITs have led to numerous critical issues and
patches.\tighten

\smallskip
This paper presents \jitsynth, the first tool for synthesizing verified JITs for
in-kernel DSLs. \jitsynth takes as input interpreters for the source DSL and the
target instruction set architecture. Given these interpreters, and a mapping
from source to target states, \jitsynth synthesizes a verified JIT compiler from
the source to the target. Our key idea is to formulate this synthesis problem as
one of synthesizing a per-instruction compiler for \emph{abstract register
machines}. Our core technical contribution is a new \emph{compiler metasketch}
that enables \jitsynth to efficiently explore the resulting synthesis search
space. To evaluate \jitsynth, we use it to synthesize a JIT from eBPF to RISC-V
and compare to a recently developed Linux JIT\@. The synthesized JIT avoids all
known bugs in the Linux JIT, with an average slowdown of
$\EbpfCompilerSlowdown\times$ in the performance of the generated code. We also
use \jitsynth to synthesize JITs for two additional source-target
pairs.  The results show that \jitsynth offers a promising new way
to develop verified JITs for in-kernel DSLs.\tighten

%\end{abstract}

\section{Overview}
Modern operating systems (OSes) can be customized with user-specified programs
that implement functionality like system call whitelisting, performance
profiling, and power management~\cite{engler:vcode,fleming:ebpf,mccanne:bpf}.
For portability and safety, these programs are written in restricted
domain-specific languages (DSLs), and the kernel executes them via
interpretation and, for better performance, just-in-time (JIT) compilation. The
correctness of in-kernel interpreters and JITs is crucial for the reliability
and security of the kernel, and bugs in their implementations have led to
numerous critical issues and patches~\cite{gpz:1454,paul:cve-2020-8835}.
More broadly, embedded DSLs are
also used to customize---and compromise~\cite{blazakis:jit-spraying,kocher:spectre}---other low-level software, such as font
rendering and anti-virus engines~\cite{chen:vmsec}. Providing formal guarantees
of correctness for in-kernel DSLs is thus a pressing practical and research
problem with applications to a wide range of systems software.\tighten

Prior work has tackled this problem through interactive theorem proving. For
example, the Jitk framework~\cite{wang:jitk} uses the Coq interactive theorem
prover~\cite{coq} to implement and verify the correctness of a JIT compiler for
the classic Berkeley Packet Filter (BPF) language~\cite{mccanne:bpf} in the
Linux kernel. But such an approach presents two key challenges. First, Jitk
imposes a significant burden on DSL developers, requiring them to implement both
the interpreter and the JIT compiler in Coq, and then manually prove the
correctness of the JIT compiler with respect to the interpreter. Second, the
resulting JIT implementation is extracted from Coq into OCaml and cannot be run
in the kernel; rather, it must be run in user space, sacrificing performance and
enlarging the trusted computing base (TCB) by relying on the OCaml runtime as
part of the TCB\@.\tighten

This chapter addresses these challenges with \jitsynth, the first tool for
synthesizing verified JIT compilers for in-kernel DSLs. \jitsynth takes as input
interpreters for the source DSL and the target instruction set architecture
(ISA), and it synthesizes a JIT compiler that is guaranteed to transform each
source program into a semantically equivalent target program. Using \jitsynth,
DSL developers write no proofs or compilers.
Instead, they write the semantics of the source and target
languages in the form of interpreters and a mapping from source to target states,
which \jitsynth trusts to be correct. The
synthesized JIT compiler is implemented in C; thus, it can run directly in the
kernel.\tighten

At first glance, synthesizing a JIT compiler seems intractable. Even the simplest
compiler contains thousands of instructions, whereas existing synthesis
techniques scale to tens of instructions. To tackle this problem in our setting,
we observe that in-kernel DSLs are similar to ISAs: both take the form of
bytecode instructions for an \emph{abstract register machine}, a simple virtual
machine with a program counter, a few registers, and limited memory
store~\cite{wang:jitk}. We also observe that in practice, the target machine has
at least as many resources (registers and memory) as the source machine;
and that JIT compilers for such abstract register machines
perform register allocation statically at compile time.
%
Our main insight is that we can exploit these properties to make synthesis
tractable through \emph{decomposition} and \emph{prioritization}, while
preserving soundness and completeness.\tighten

\jitsynth works by decomposing the JIT synthesis problem into the problem of
synthesizing individual \emph{\minicompilers} for every instruction in the
source language. Each \minicompiler is synthesized by generating a
\emph{compiler metasketch}~\cite{bornholt:synapse}, a set of ordered sketches
that collectively represent \emph{all} instruction sequences in the target
ISA\@. These sketches are then solved by an off-the-shelf synthesis tool based
on reduction to SMT~\cite{torlak:rosette}. The synthesis tool ensures that the
target instruction sequence is semantically equivalent to the source
instruction, according to the input interpreters.  The order in which the
sketches are explored is key to making this search practical, and \jitsynth
contributes two techniques for biasing the search towards tightly constrained,
and therefore tractable, sketches that are likely to contain a correct program. 

First, we observe that source instructions can often be implemented with target
instructions that access the same parts of the state (e.g., only registers).
Based on this observation, we develop \emph{read-write sketches}, which restrict
the synthesis search space to a subset of the target instructions, based on a
sound and precise summary of their semantics. Second, we observe that
hand-written JITs rely on pseudoinstructions to generate common target
sequences, such as loading immediate (constant) values into registers. We use
this observation to develop \emph{pre-load sketches}, which employ synthesized
pseudoinstructions to eliminate the need to repeatedly search for common target
instruction subsequences.   

We have implemented \jitsynth in Rosette~\cite{torlak:rosette} and used it to
synthesize JIT compilers for three widely used in-kernel DSLs. As our main case
study, we used \jitsynth to synthesize a RISC-V~\cite{riscv:isa} compiler for
extended BPF~(eBPF)~\cite{fleming:ebpf}, an extension of classic
BPF~\cite{mccanne:bpf}, used by the Linux kernel. Concurrently with our work,
Linux developers manually built a JIT compiler for the same source and target
pair, and a team of researchers found nine correctness bugs in that compiler
shortly after its release~\cite{nelson:serval}. In contrast, our JIT compiler is
verified by construction; it supports 87 out of 102 eBPF instructions and passes
all the Linux kernel tests within this subset, including the regression tests
for these nine bugs. Our synthesized compiler generates code that is
$\EbpfInterpSpeedup\times$ faster than interpreted code and
$\EbpfCompilerSlowdown\times$ times slower than the code generated by the Linux
JIT\@. We also used  \jitsynth to synthesize a JIT from
libseccomp~\cite{edge:libseccomp}, a policy language for system call
whitelisting, to eBPF, and a JIT from classic BPF to eBPF\@. The synthesized
JITs avoid previously found bugs in the existing generators for these source
target pairs, while incurring, on average, a
$\CbpfSlowdown$--$\LibseccompSlowdown\times$ slowdown in the performance of the
generated code.\tighten


To summarize, this chapter makes the following contributions:
\begin{enumerate}
    \item \jitsynth, the first tool for synthesizing verified JIT compilers for
    in-kernel DSLs, given the semantics of the source and target languages as
    interpreters.\tighten
    \item A novel formulation of the JIT synthesis problem as one of
    synthesizing a per-instruction compiler for \emph{abstract register
    machines}.\tighten
    \item A novel \emph{compiler metasketch} that enables \jitsynth to solve the JIT
    synthesis problem with an off-the-shelf synthesis engine.
    \item An evaluation of \jitsynth's effectiveness, showing that it can
    synthesize verified JIT compilers for three widely used in-kernel DSLs.
\end{enumerate}

The rest of this chapter is organized as follows.
%
\autoref{jitsynth:s:overview} illustrates \jitsynth on a small example.
%
\autoref{s:problem} formalizes the JIT synthesis problem for in-kernel DSLs.
%
\autoref{s:algorithm} presents the \jitsynth algorithm for generating and
solving compiler metasketches.
%
\autoref{jitsynth:s:impl} provides implementation details.
%
\autoref{s:eval} evaluates \jitsynth.
%
\autoref{jitsynth:s:related} discusses related work.
\autoref{jitsynth:s:conclusion} concludes.\tighten


\section{\jitsynth in a nutshell}% TODO bad name
\label{s:overview}

\begin{figure}[h]
  \centering
  \resizebox{\linewidth}{!}{
    \begin{tabular}{llll}

\toprule
\multicolumn{2}{l}{instruction} & description & semantics
\\

\midrule

\multicolumn{2}{l}{eBPF (subset):}

\\
& \cc{addi32}\ $\dst, \mathit{imm32}$
& 32-bit add (high 32 bits cleared)
& $\reg{\dst} \leftarrow 0^{32} \concat (\extract{31}{0}{\reg{\dst}} + \mathit{imm32})$

\\

\midrule

\multicolumn{2}{l}{RISC-V (subset):}
\\
& \cc{lui}\ $\rd, \mathit{imm20}$
& load upper immediate
& $\reg{\rd} \leftarrow \sextll(\mathit{imm20} \concat 0^{12})$

\\

& \cc{addiw}\ $\rd, \rs, \mathit{imm12}$
& 32-bit register-immediate add
& $\reg{\rd} \leftarrow \sextll(\extract{31}{0}{\reg{\rs}} + \sexti(\mathit{imm12}))$

\\

& \cc{add}\ $\rd, \mathit{rs1}, \mathit{rs2}$
& register-register add
& $\reg{\rd} \leftarrow \reg{\mathit{rs1}} + \reg{\mathit{rs2}}$

\\

& \cc{slli}\ $\rd, \rs, \mathit{imm6}$
& register-immediate left shift
& $\reg{\rd} \leftarrow \rs\ \texttt{<<}\ (0^{58} \concat \mathit{imm6})$

\\

& \cc{srli}\ $\rd, \rs, \mathit{imm6}$
& register-immediate logical right shift
& $\reg{\rd} \leftarrow \rs\ \texttt{>>}\ (0^{58} \concat \mathit{imm6})$

\\

& \cc{lb}\ $\rd, \rs, \mathit{imm12}$
& load byte from memory
& $\reg{\rd} \leftarrow \sextll(M[\reg{\rs} + \sextll(\mathit{imm12})])$

\\

& \cc{sb}\ $\mathit{rs1}, \mathit{rs2}, \mathit{imm12}$
& store byte to memory
& $M[\reg{\mathit{rs1}} + \sextll(\mathit{imm12})] \leftarrow \extract{7}{0}{\reg{\mathit{rs2}}} $

\\

\bottomrule

\end{tabular}


  }
  \vspace{-.5em}
  \caption{Subsets of eBPF and RISC-V
used as source and target languages, respectively,
in our running example:
$R[r]$ denotes the value of register~$r$;
$M[a]$ denotes the value at memory address~$a$;
$\concat$ denotes concatenation of bitvectors;
superscripts (e.g., $0^{32}$) denote repetition of bits;
$\sexti(x)$ and $\sextll(x)$ sign-extend $x$ to 32 and 64 bits, respectively;
and $\extractop(i, j, x)$ produces a subrange of bits of $x$ from index $i$ down to $j$.\looseness=-1}
%$r$ denotes a BPF register;
%$\rd$, $\mathit{rs1}$, and $\mathit{rs2}$ denote RISC-V registers;
\label{fig:lang-instrs}
\end{figure}

This section provides an overview of \jitsynth by illustrating how it
synthesizes a toy JIT compiler (\autoref{fig:lang-instrs}). The source language
of the JIT is a tiny subset of eBPF~\cite{fleming:ebpf} consisting of one
instruction, and the target language is a subset of 64-bit
RISC-V~\cite{riscv:isa} consisting of seven instructions. Despite the simplicity
of our languages, the Linux kernel JIT used to produce incorrect code for this
eBPF instruction~\cite{nelson:bpf-riscv-add32-bug}; such miscompilation bugs not
only lead to correctness issues, but also enable adversaries to compromise the
OS kernel by crafting malicious eBPF programs~\cite{wang:jitk}. This section
shows how \jitsynth can be used to synthesize a JIT that is verified with
respect to the semantics of the source and target languages.\tighten

\paragraph{In-kernel languages.} 

\jitsynth expects the source and target languages to be a set of instructions
for manipulating the state of an \emph{abstract register machine}
(\autoref{s:problem}). This state consists of a program counter (\pc), a finite
sequence of general-purpose registers (\regs), and a finite sequence of memory
locations (\mem), all of which store bitvectors (i.e., finite precision
integers). The length of these bitvectors is defined by the language; for
example, both eBPF and RISC-V store 64-bit values in their registers. An
instruction consists of an \emph{opcode} and a finite set of \emph{fields},
which are bitvectors representing either register identifiers or immediate
(constant) values. For instance, the \cc{addi32} instruction in eBPF
has two fields: $\dst$ is a 4-bit value representing the index of the
output register, and $\mathid{imm32}$ is a 32-bit immediate.
(eBPF instructions may have two additional fields $\src$ and $\off$,
which are not shown here as they are not used by \cc{addi32}.)
An abstract
register machine for a language gives meaning to its instructions: the machine
consumes an instruction and a state, and produces a state that is the result of
executing that instruction. \autoref{fig:lang-instrs} shows a high-level
description of the abstract register machines for our languages.\tighten

\paragraph{\jitsynth interface.}

To synthesize a compiler from one language to another, \jitsynth takes as input
their syntax, semantics, and a mapping from source to target states. All three
inputs are given as a program in a \emph{solver-aided host
language}~\cite{torlak:rosette}. \jitsynth uses Rosette as its host, but the
host can be any language with a symbolic
evaluation engine that can reduce the semantics of host programs to SMT
constraints (e.g.,~\cite{solar-lezama:sketch}).
\autoref{fig:toy-inputs} shows the interpreters for the source and
target languages (i.e., emulators for their abstract register machines), as well
as the state-mapping functions \cc{regST}, \cc{pcST}, and \cc{memST} that
\jitsynth uses to determine whether a source state $\sigma_S$ is equivalent to a
target state $\sigma_T$. In particular, \jitsynth deems these states equivalent,
denoted by $\sigma_S \cong \sigma_T$, whenever 
%
$\regs(\sigma_T)[\cc{regST}(r)] = \regs(\sigma_S)[r]$, $\pc(\sigma_T) =
\cc{pcST}(\pc(\sigma_S))$, and $\mem(\sigma_T)[\cc{memST}(a)] =
\mem(\sigma_S)[a]$ 
%
for all registers $r$ and memory addresses $a$.\tighten

\begin{figure}[h]
\centering
\resizebox{\linewidth}{!}{%
\begin{minipage}{1.03\linewidth}
\lstinputlisting[language=rosette,xleftmargin=-.5em,firstline=1]{code/toy-inputs.rkt}%multicols=2,numbers=left,numbersep=.75em,
\end{minipage}}
\vspace{-.5em}
  \caption{Snippets of inputs to \jitsynth: the interpreters for the source (eBPF) and and target (RISC-V) languages and state-mapping functions.}
  \label{fig:toy-inputs}
  \end{figure}

\paragraph{Decomposition into per-instruction compilers.} 

Given these inputs, \jitsynth generates a \emph{per-instruction compiler} from
the source to the target language. To ensure that the resulting compiler is
correct (\autoref{thm:end-to-end-soundness}), and that one will be found if it
exists (\autoref{thm:synthesis-soundness-and-completeness}), \jitsynth puts two
restrictions on its inputs. First, the inputs must be
self-finitizing~\cite{torlak:rosette}, meaning that both the interpreters and
the mapping functions must have a finite symbolic execution tree when applied to
symbolic inputs.
%such as an instruction with symbolic fields or a state with symbolic
%contents. 
Second, the target machine must have at least as many registers and memory
locations as the source machine; these storage cells must be as wide as those of
the source machine; and the state-mapping functions (\cc{pcST}, \cc{regST}, and
\cc{memST}) must be injective. Our toy inputs satisfy these restrictions, as do
the real in-kernel languages evaluated in \autoref{s:eval}.

\paragraph{Synthesis workflow.} 

\jitsynth generates a per-instruction compiler for a given source and target
pair in two stages. The first stage uses an optimized \emph{compiler metasketch}
to synthesize a \minicompiler from every instruction in the source language to a
sequence of instructions in the target language (\autoref{s:algorithm}). The
second stage then simply stitches these mini compilers into a full C compiler
using a trusted outer loop and a switch statement. The first stage is a core
technical contribution of this paper, and we illustrate it next on our toy
example.\tighten


\paragraph{Metasketches.} 

To understand how \jitsynth works, consider the basic problem of determining if
every \cc{addi32} instruction can be emulated by a sequence of $k$ instructions
in toy RISC-V\@. In particular, we are interested in finding a program
$C_\cc{addi32}$ in our host language (which \jitsynth translates to C) that
takes as input a source instruction $s = \cc{addi32}\ \dst, \mathid{imm32}$ and
outputs a semantically equivalent RISC-V program $t = [t_1,\ldots,t_k]$. That
is, for all $\dst, \mathid{imm32}$, and for all equivalent states $\sigma_S \cong
\sigma_T$, we have $\mathid{run}(s, \sigma_S,
\cc{ebpf-interpret})\cong\mathid{run}(t, \sigma_T, \cc{rv-interpret})$, where
$\mathid{run}(e, \sigma, f)$ executes the instruction interpreter $f$ on the
sequence of instructions $e$, starting from the state $\sigma$
(\autoref{def:arm}).\tighten

We can solve this problem by asking the host synthesizer to search for
$C_\cc{addi32}$ in a space of candidate \minicompilers of length $k$. We
describe this space with a syntactic template, or a \emph{sketch}, as shown below:

\begin{lstlisting}[language=rosette,xleftmargin=0em,mathescape=true]
(define (compile-addi32 s)       ; Returns a list of k instruction holes, to be 
  (define dst (ebpf-insn-dst s)) ; filled with toy RISC-V instructions. Each    
  (define imm (ebpf-insn-imm s)) ; hole represents a set of choices, defined 
  (list (??insn dst imm) ...))   ; by the ??insn procedure. 

(define (??insn . sf)            ; Takes as input source instruction fields and
  (define rd  (??reg sf))        ; uses them to construct target field holes. 
  (define rs1 (??reg sf))        ; ??reg and ??imm field holes are bitvector 
  (define rs2 (??reg sf))        ; expressions over sf and arbitrary constants.
  (choose*                       ; Returns an expression that chooses among  
    (rv-insn lui rd rs1 rs2 (??imm 20 sf))   ; lui, addiw,
    ...                                      ; ..., and
    (rv-insn sb  rd rs1 rs2 (??imm 12 sf)))) ; sb instructions.
\end{lstlisting}

Here, \cc{(??insn dst imm)} stands for a missing expression---a hole---that the
synthesizer needs to fill with an instruction from the toy RISC-V language. To
fill an instruction hole, the synthesizer must find an expression that computes
the value of the target instruction's fields. \jitsynth limits this expression
language to bitvector expressions (of any depth) over the fields of the source
instruction and arbitrary bitvector constants.\tighten

Given this sketch, and our correctness specification for $C_\cc{addi32}$, the
synthesizer will search the space defined by the sketch for a program that
satisfies the specification.
%
Below is an example of the resulting toy compiler from eBPF to
RISC-V, synthesized and translated to C by \jitsynth (without the
outer loop):\tighten
\lstinputlisting[language=C,xleftmargin=2em,firstline=1]{code/toy-compiler.c}%multicols=2,numbers=left,numbersep=.75em,

Once we know how to synthesize a compiler of length $k$, we can easily extend
this solution into a naive method for synthesizing a compiler of any length. We
simply enumerate sketches of increasing lengths, $k = 1, 2, 3, \ldots$, invoke
the synthesizer on each generated sketch, and stop as soon as a solution is
found (if ever). The resulting ordered set of sketches forms a
metasketch~\cite{bornholt:synapse}---i.e., a search space and a strategy for
exploring it---that contains all candidate mini compilers (in a subset of the
host language) from the source to the target language. This naive metasketch can
be used to find a mini compiler for our toy example in {493 minutes}. However,
it fails to scale to real in-kernel DSLs (\autoref{s:eval}), motivating the need
for \jitsynth's optimized compiler metasketches.\tighten

\paragraph{Compiler metasketches.} \jitsynth optimizes the naive metasketch by
extending it with two kinds of more tightly constrained sketches, which are
explored first. A constrained sketch of size $k$ usually contains a correct
solution of a given size if one exists, but if not, \jitsynth will eventually
explore the naive sketch of the same length, to maintain completeness. We give
the intuition behind the two optimizations here, and present them in detail in
\autoref{s:algorithm}. 

First, we observe that practical source and target languages include similar
kinds of instructions. For example, both eBPF and RISC-V include instructions
for adding immediate values to registers. This similarity often makes it
possible to emulate a source instruction with a sequence of target instructions
that access the same part of the state (the program counter, registers, or
memory) as the source instruction. For example, \cc{addi32} reads and writes
only registers, not memory, and it can be emulated with RISC-V instructions that
also access only registers. To exploit this observation, we introduce
\emph{read-write sets}, which summarize, soundly and precisely, how an
instruction accesses state. \jitsynth uses these sets to define \emph{read-write
sketches} for a given source instruction, including only target instructions
that access the state in the same way as the source instruction. For instance, a
read-write sketch for \cc{addi32} excludes both \cc{lb} and \cc{sb} instructions
because they read and write memory as well as registers.\tighten

Second, we observe that hand-written JITs use pseudoinstructions to simplify
their implementation of \minicompilers. These are simply subroutines or macros
for generating target sequences that implement common functionality. For
example, the Linux JIT from eBPF to RISC-V includes a pseudoinstruction for
loading 32-bit immediates into registers. \jitsynth mimics the way hand-written
JITs use pseudoinstructions with the help of \emph{pre-load sketches}. These
sketches first use a synthesized pseudoinstruction to create a sequence of
concrete target instructions that load source immediates into scratch registers;
then, they include a compute sequence comprised of read-write instruction holes.
Applying these optimizations to our toy example, \jitsynth finds a \minicompiler
for \cc{addi32} in {5 seconds}---a {roughly $6000\times$} speedup over the naive
metasketch.\tighten


\section{Problem Statement}\label{s:problem}

This section formalizes the compiler synthesis problem for in-kernel DSLs. We
focus on JIT compilers, which, for our purposes, means one-pass
compilers~\cite{engler:vcode}. To start, we define \emph{abstract register
machines} as a way to specify the syntax and semantics of in-kernel languages.
Next, we formulate our compiler synthesis problem as one of synthesizing a set
of sound \emph{\minicompilers} from a single source instruction to a sequence of
target instructions. Finally, we show that these \minicompilers compose into a
sound JIT compiler, which translates every source program into a semantically
equivalent target program.\tighten 


\paragraph{Abstract register machines.}

An abstract register machine (ARM) provides a simple interface for specifying
the syntax and semantics of an in-kernel language. The syntax is given as a set
of {abstract instructions}, and the semantics is given as a {transition
function} over instructions and machine {states}. 

An \emph{abstract instruction} (\autoref{def:instruction}) defines the name
(\mathid{op}) and type signature ($\mathcal{F}$) of an operation in the
underlying language. For example, the abstract instruction $(\mathid{addi32}, r
\mapsto \Reg, imm32 \mapsto \BV{32})$ specifies the name and signature of the
\texttt{addi32} operation from the eBPF language
(\autoref{fig:lang-instrs}). Each abstract instruction represents the (finite)
set of all \emph{concrete instructions} that instantiate the abstract
instruction's parameters with values of the right type. For example,
$\cc{addi32}\, 0, 5$ is a concrete instantiation of the abstract instruction for
\cc{addi32}. In the rest of this paper, we will write ``instruction'' to mean a
concrete instruction.\tighten

\begin{definition}[Abstract and Concrete Instructions]\label{def:instruction}
  An \textup{abstract instruction} $\iota$ is a pair $(\mathid{op}, \mathcal{F})$ where
  \mathid{op} is an opcode and $\mathcal{F}$ is a mapping from \emph{fields}
  to their \emph{types}. Field types include \Reg, denoting register names, and
  \BV{k}, denoting $k$-bit bitvector values. The abstract instruction $\iota$
  represents all \textup{concrete instructions} $p = (\mathid{op}, F)$ with the
  opcode \mathid{op} that bind each field $f\in \mathid{dom}(\mathcal{F})$ to a
  value $F(f)$ of type $\mathcal{F}(f)$. We write $\prog{\iota}$ to denote the
  set of all concrete instructions for $\iota$, and we extend this notation to
  sets of abstract instructions in the usual way, i.e.,
  $\prog{\mathcal{I}}=\bigcup_{\iota\in\mathcal{I}}\prog{\iota}$ for the set
  $\mathcal{I}$.\tighten
\end{definition}

Instructions operate on machine \emph{states} (\autoref{def:state}), and their
semantics are given by the machine's \emph{transition function}
(\autoref{def:arm}). A machine state  consists of a program counter, a map from
register names to register values, and a map from memory addresses to memory
values. Each state component is either a bitvector or a map over bitvectors,
making the set of all states of an ARM finite. The transition function of an ARM
defines an interpreter for the ARM's language by specifying how to compute the
output state for a given instruction and input state. We can apply this
interpreter, together with the ARM's \emph{fuel function}, to define an
\emph{execution} of the machine on a program and an initial state. The fuel
function takes as input a sequence of instructions and returns a natural number
that bounds the number of steps (i.e., state transitions) the machine can make
to execute the given sequence. The inclusion of fuel models the requirement of
in-kernel languages for all program executions to terminate~\cite{wang:jitk}. It
also enables us to use symbolic execution to soundly reduce the semantics of
these languages to SMT constraints, in order to formulate the synthesis queries
in \autoref{s:algorithm:solving}.\tighten 


\begin{definition}[State]\label{def:state}
 A \textup{state} $\sigma$ is a tuple $(\pc, \regs, \mem)$ where $\pc$ is a
 value, \regs is a function from register names to values, and \mem is a
 function from memory addresses to values. Register names, memory addresses, and
 all values are finite-precision integers, or bitvectors. We write
 $\size{\sigma}$ to denote the \emph{size} of the state $\sigma$. The size
 $\size{\sigma}$ is defined to be the tuple $(r, m, k_\pc, k_\regs, k_\mem)$,
 where $r$ is the number of registers in $\sigma$, $m$ is the number of memory
 addresses, and $k_\pc$, $k_\regs$, and $k_\mem$ are the width of the bitvector
 values stored in the \pc, \regs, and \mem, respectively. Two states have the
 same size if $\size{\sigma_i} = \size{\sigma_j}$; one state is smaller than
 another, $\size{\sigma_i} \leq \size{\sigma_j}$, if each element of $\size{\sigma_i}$ 
 is less than or equal to the corresponding element of $\size{\sigma_j}$.\tighten
 \end{definition}


\begin{definition}[Abstract Register Machines and Executions]\label{def:arm}
  An \textup{abstract register machine} $\mathcal{A}$ is a tuple $(\mathcal{I},
  \Sigma, \mathcal{T}, \Phi)$  where $\mathcal{I}$ is a set of abstract
  instructions, $\Sigma$ is a set of states of the same size, $\mathcal{T} :
  \prog{\mathcal{I}} \rightarrow \Sigma \rightarrow \Sigma$ is a
  \textup{transition function} from instructions and states to states, and
  $\Phi: \mathid{List}(\prog{\mathcal{I}}) \rightarrow \mathbb{N}$ is a
  \emph{fuel function} from sequences of instructions to natural numbers. Given
  a state $\sigma_0\in\Sigma$ and a sequence of instructions $\vec{p}$ drawn
  from $\prog{\mathcal{I}}$, we define the \textup{execution} of $\mathcal{A}$
  on $\vec{p}$ and $\sigma_0$ to be the result of applying $\mathcal{T}$ to
  $\vec{p}$ at most $\Phi(\vec{p})$ times. That is, $\arm(\vec{p}, \sigma_0) =
  \mathid{run}(\vec{p}, \sigma_0, \mathcal{T}, \Phi(\vec{p}))$, where  
  \[
    \mathid{run}(\vec{p}, \sigma, \mathcal{T}, k) = 
    \begin{cases}
        \sigma,& \text{if } k = 0 \text{ or } \pc(\sigma) \not\in [0, \size{\vec{p}})\\
        \mathid{run}(\vec{p}, \mathcal{T}(\vec{p}[\pc(\sigma)], \sigma), \mathcal{T}, k-1),   & \text{otherwise.}
    \end{cases}
  \]
\end{definition}

\paragraph{Synthesizing JIT compilers for ARMs.} Given a source and target ARM,
our goal is to synthesize a one-pass JIT compiler that translates source
programs to semantically equivalent target programs. To make synthesis
tractable, we fix the structure of the JIT to consist of an outer loop and a
switch statement that dispatches compilation tasks to a set of
\emph{\minicompilers} (\autoref{def:mini-compiler}). Our synthesis
problem is therefore to find a sound \minicompiler for each abstract
instruction in the source machine (\autoref{def:synthesis-problem}).\tighten

\begin{definition}[\MiniCompiler]\label{def:mini-compiler}
Let $\arm_S = (\mathcal{I}_S, \Sigma_S, \mathcal{T}_S, \Phi_S)$ and
$\arm_T = (\mathcal{I}_T, \Sigma_T, \\ \mathcal{T}_T, \Phi_T)$ be two abstract register
machines, $\cong$ an equivalence relation on their states $\Sigma_S$ and
$\Sigma_T$, and $C: \prog{\iota} \rightarrow
\mathid{List}(\prog{\mathcal{I}_T})$ a function for some
$\iota\in\mathcal{I}_S$. We say that $C$ is a \textup{sound \minicompiler} for
$\iota$ with respect to $\cong$ iff
\[
    \forall \sigma_S\in\Sigma_S,\ \sigma_T\in\Sigma_T,\ p\in\prog{\iota}. \ 
            \sigma_S \cong \sigma_T \Rightarrow
                 \arm_S(p, \sigma_S) \cong \arm_T(C(p), \sigma_T)
\]
\end{definition} 

\begin{definition}[\MiniCompiler Synthesis]\label{def:synthesis-problem}  
Given two abstract register machines $\arm_S = (\mathcal{I}_S, \Sigma_S,
\mathcal{T}_S, \Phi_S)$ and $\arm_T=(\mathcal{I}_T, \Sigma_T,  \mathcal{T}_T, \Phi_T)$, as well
as an equivalence relation $\cong$ on their states, the \textup{\minicompiler
synthesis problem} is to generate a sound \minicompiler $C_\iota$ for each
$\iota\in\mathcal{I}_S$ with respect to $\cong$.\tighten
\end{definition} 

The general version of our synthesis problem, defined above, uses an arbitrary
equivalence relation $\cong$ between the states of the source and target
machines to determine if a source and target program are semantically
equivalent. \jitsynth can, in principle, solve this problem with the naive
metasketch described in \autoref{s:overview}. In practice, however, the naive
metasketch scales poorly, even on small languages such as toy eBPF and RISC-V\@.
So, in this paper, we focus on source and target ARMs that satisfy an additional
assumption on their state equivalence relation: it can be expressed in terms of
injective mappings from source to target states (\autoref{def:state-equiv}).
%When we write $\cong$ from now on, we mean this restricted form of state equivalence. 
This restriction enables \jitsynth to employ optimizations (such as pre-load
sketches described in \autoref{s:algorithm:pld}) that are crucial to scaling
synthesis to real in-kernel languages.\tighten


\begin{definition}[Injective State Equivalence Relation]\label{def:state-equiv}
Let $\arm_S$ and $\arm_T$ be abstract register machines with states $\Sigma_S$
and $\Sigma_T$ such that $\size{\sigma_S}\leq\size{\sigma_T}$ for all
$\sigma_S\in\Sigma_S$ and $\sigma_T\in\Sigma_T$. Let $\mathcal{M}$ be a
\emph{state mapping} $(\mathcal{M}_\pc, \mathcal{M}_\regs, \mathcal{M}_\mem)$
from $\Sigma_S$ and $\Sigma_T$,  where $\mathcal{M}_\pc$ multiplies the program
counter of the states in $\Sigma_S$ by a constant factor, $\mathcal{M}_\regs$ is
an injective map from register names in $\Sigma_S$ to those in $\Sigma_T$, and
$\mathcal{M}_\mem$ is an injective map from memory addresses in $\Sigma_S$ to
those in $\Sigma_T$. We say that two states  $\sigma_S\in\Sigma_S$ and
$\sigma_T\in\Sigma_T$ are equivalent according to $\mathcal{M}$, written
$\sigma_S \cong_\mathcal{M} \sigma_T$, iff $\mathcal{M}_\pc(\pc(\sigma_S)) =
\pc(\sigma_T)$, $\regs(\sigma_S)[r] = \regs(\sigma_T)[\mathcal{M_\regs}(r)]$ for
all register names $r\in\emph{dom}(\regs(\sigma_S))$, and $\mem(\sigma_S)[a] =
\mem(\sigma_T)[\mathcal{M_\mem}(a)]$ for all memory addresses
$a\in\emph{dom}(\mem(\sigma_S))$. The binary relation $\cong_\mathcal{M}$ is
called an \textup{injective state equivalence relation} on $\arm_S$ and
$\arm_T$.
\end{definition}

\paragraph{Soundness of JIT compilers for ARMs.} Finally, we note that a JIT
compiler composed from the synthesized \minicompilers correctly
translates every source program to an equivalent target program.
We formulate and prove this theorem using the Lean theorem prover~\cite{moura:lean}.

\begin{theorem}[Soundness of JIT compilers]\label{thm:end-to-end-soundness}
Let $\arm_S = (\mathcal{I}_S, \Sigma_S, \mathcal{T}_S, \Phi_S)$ and
$\arm_T=(\mathcal{I}_T, \Sigma_T,  \mathcal{T}_T, \Phi_T)$ be abstract register
machines, $\cong_\mathcal{M}$ an injective state equivalence relation on their
states such that $M_\pc(\pc(\sigma_S)) = N_\pc\pc(\sigma_S)$,
and $\{C_1,\ldots,C_{|\mathcal{I}_S|}\}$ a solution to the \minicompiler
synthesis problem for $\arm_S$, $\arm_T$, and $\cong_\mathcal{M}$
where $\forall s\in\prog{\iota}.\ |C_i(s)| = N_\pc$. Let
$\mathcal{C} : \prog{\mathcal{I}_S} \rightarrow
\mathid{List}(\prog{\mathcal{I}_T})$ be a function that maps concrete
instructions $s\in\prog{\iota}$ to the compiler output $C_\iota(s)$ for
$\iota\in\mathcal{I}_S$. If $\vec{s} = s_1, \ldots, s_n$ is a sequence of
concrete instructions drawn from $\mathcal{I}_S$, and $\vec{t} =
\mathcal{C}(s_1)\cdot\ldots\cdot\mathcal{C}(s_n)$ where $\cdot$ stands for
sequence concatenation, then $\forall \sigma_S\in\Sigma_S, \sigma_T\in\Sigma_T.\
\sigma_S \cong_\mathcal{M} \sigma_T \Rightarrow \arm_S(\vec{s}, \sigma_S)
\cong_\mathcal{M} \arm_T(\vec{t}, \sigma_T)$.\tighten
\end{theorem}




\section{Solving the \MiniCompiler Synthesis Problem}
\label{s:algorithm}

This section presents our approach to solving the \minicompiler synthesis
problem defined in \autoref{s:problem}. We employ syntax-guided
synthesis~\cite{solar-lezama:sketch} to search for an implementation of a
\minicompiler in a space of candidate programs. Our core contribution is an
effective way to structure this space using a \emph{compiler metasketch}. This
section presents our algorithm for generating compiler metasketches, describes
its key subroutines and optimizations, and shows how to solve the resulting
sketches with an off-the-shelf synthesis engine.\tighten
%, and highlights key details of our implementation of \jitsynth.\tighten


\subsection{Generating Compiler Metasketches}

\jitsynth synthesizes \minicompilers by generating and solving
\emph{metasketches}~\cite{bornholt:synapse}. A metasketch describes a space of
candidate programs using an ordered set of syntactic templates or
\emph{sketches}~\cite{solar-lezama:sketch}. These sketches take the form of
programs with missing expressions or \emph{holes}, where each hole describes a
finite set of candidate completions. \jitsynth sketches are expressed in a
\emph{host language} \host that serves both as the implementation language for
\minicompilers and the specification language for ARMs. \jitsynth expects the
host to provide a synthesizer for completing sketches and a symbolic evaluator
for reducing ARM semantics to SMT constraints. \jitsynth uses these tools to
generate optimized metasketches for \minicompilers, which we call \emph{compiler
metasketches}.\tighten

\autoref{alg:compiler-metasketch-generator} shows our algorithm for generating
compiler metasketches. The algorithm, \CMS, takes as input an abstract source
instruction $\iota$ for a source machine $\arm_S$, a target machine $\arm_T$,
and a state mapping \Map from $\arm_S$ to $\arm_T$. Given these inputs, it
lazily enumerates an infinite set of \emph{compiler sketches} that collectively
represent the space of all straight-line bitvector programs from $\prog{\iota}$
to $\mathid{List}(\prog{\mathcal{I}_T})$. In particular, each compiler sketch
consists of $k$ target \emph{instruction holes}, constructed from {field holes}
that denote bitvector expressions (over the fields of $\iota$) of depth $d$ or
less. For each length $k$ and depth $d$, the \CMS loop generates three kinds of
compiler sketches: the \emph{pre-load}, the \emph{read-write}, and the
\emph{naive} sketch. The naive sketch (\autoref{s:algorithm:naive}) is the most
general, consisting of all candidate \minicompilers of length $k$ and depth $d$.
But it also scales poorly, so \CMS first yields the pre-load
(\autoref{s:algorithm:pld}) and  read-write (\autoref{s:algorithm:rw}) sketches.
As we will see later, these sketches describe a subset of the programs in the
naive sketch, and they are designed to prioritize exploring small parts of the
search space that are likely to contain a correct \minicompiler for $\iota$, if
one exists.

\begin{figure}[t]
\begin{algorithmic}[1] 
  \Function{CMS}
    {$\iota, \arm_S, \arm_T, \Map$} \Comment{$\iota\in\mathcal{I}_S, \arm_S = (\mathcal{I}_S, \ldots)$}
    \For{$n\in\mathbb{Z}^+$} \Comment{Lazily enumerates all compiler sketches}
      \For{$k \in [1, n], d = n-k$} \Comment{of length $k$ and depth $d$,}
        \State \textbf{yield} $\LCS(k,d,\iota,\arm_S,\arm_T,\Map)$ \Comment{yielding the pre-load sketch first,}
        \State \textbf{yield} $\RW(k,d,\iota,\arm_S,\arm_T,\Map)$ \Comment{read-write sketch next, and}
        \State \textbf{yield} $\Naive(k,d,\iota,\arm_S,\arm_T,\Map)$ \Comment{the most general sketch last.}
      \EndFor
    \EndFor
  \EndFunction
\end{algorithmic}
\caption{Compiler metasketch for the abstract source instruction
$\iota$, source machine $\arm_S$, target machine $\arm_T$, and state mapping
$\mathcal{M}$ from $\arm_S$ to
$\arm_T$.\tighten}\label{alg:compiler-metasketch-generator}
\end{figure}

\subsection{Generating Naive Sketches}\label{s:algorithm:naive}

The most general sketch we consider, $\Naive(k,d,\iota,\arm_S,\arm_T,\Map)$, is
shown in \autoref{fig:naive-sketch}. This sketch consists of $k$ instruction
holes that can be filled with any instruction from $\mathcal{I}_T$. An
instruction hole chooses between expressions of the form $(\mathit{op}_T, H)$,
where $\mathit{op}_T$ is a target opcode, and $H$ specifies the field holes for
that opcode. Each field hole is a bitvector expression (of depth $d$) over the
fields of the input source instruction and arbitrary bitvector constants. This
lets target instructions use the immediates and registers (modulo \Map) of the
source instruction, as well as arbitrary constant values and register names.
Letting field holes include constant register names allows the synthesized
\minicompilers to use target registers unmapped by \Map as temporary, or
scratch, storage. In essence, the naive sketch describes all straight-line
compiler programs that can make free use of standard C arithmetic and bitwise
operators, as well as scratch registers.\tighten 

The space of such programs is intractably large, however, even for small inputs.
For instance, it includes at least $2^{350}$ programs of length $k=5$ and depth
$d\leq 3$ for the toy example from \autoref{jitsynth:s:overview}. \jitsynth therefore
employs two effective heuristics to direct the exploration of this space toward
the most promising candidates first, as defined by the read-write and pre-load
sketches.\tighten 

\begin{figure}
\begin{algorithmic}[1] 
  \Function{Naive}
    {$k, d, \iota , \arm_S, \arm_T, \Map$} \Comment{$\iota\in\mathcal{I}_S, \arm_S = (\mathcal{I}_S, \ldots)$}
    \State $(\mathid{op}, \mathcal{F}) \gets \iota$, $(\mathid{I}_T, \ldots) \gets \arm_T$ \Comment{Source instruction, target instructions.}
    \State $p \gets \mathid{FreshId}()$ \Comment{Identifier for the compiler's input.}
    \State $\mathid{body}\gets []$  \Comment{The body of the compiler is a sequence}
    \For{$0 \leq i < k$} \Comment{of $k$ target instruction holes.}
      \State $I \gets \{\}$  \Comment{The set $I$ of choices for a target instruction hole}
      \For{$(\mathid{op}_T, \mathcal{F}_T)\in\mathcal{I}_T$} \Comment{includes all instructions from $\mathcal{I}_T$.} \label{li:instruction-hole}
        \State $E \gets \{\mathid{Expr}(p.f, \Map) \,|\, f\in\mathid{dom}(\mathcal{F})\}$ \Comment{Any source field can appear in}
        \State $H \gets \{ f\mapsto \mathid{Field}(\mathcal{F}_T(f),d,E) \,|\, f\in\mathid{dom}(\mathcal{F}_T)\} $ \Comment{a target field hole, and}\label{li:field-holes}
        \State $I\gets I \cup \{ \mathit{Expr}((\mathid{op}_T, H), \Map) \}$ \Comment{any constant register or value.}
      \EndFor
      \State $\mathid{body} \gets \mathid{body} \cdot [\mathid{Choose}(I)]$ \Comment{Append a hole over $I$ to the body.}
    \EndFor
    \State \Return $\mathid{Expr}((\lambda p\in\prog{\iota}\, .\, \mathid{body}),\Map)$ \Comment{A \minicompiler sketch for $\iota$.}
  \EndFunction
\end{algorithmic}
\caption{Naive sketch of length $k$ and maximum depth $d$ for $\iota$, $\arm_S$,
$\arm_T$, and \Map. Here, $\mathid{Expr}$ creates an expression in the host language, 
using \Map to map from source to target register names and memory addresses;
$\mathid{Choose}(E)$ is a hole that chooses an expression from the set $E$; and
$\mathid{Field}(\tau,d,E)$ is a hole for a bitvector expression of type
$\tau$ and maximum depth $d$, constructed from arbitrary bitvector constants and
expressions $E$.\tighten}\label{fig:naive-sketch}
\end{figure}

%bvand bvadd bvshl bvor bvxor bvlshr bvashr

\subsection{Generating Read-Write Sketches}\label{s:algorithm:rw} 

The read-write sketch, $\RW(k, d, \iota, \arm_S, \arm_T, \Map)$, is based on the
observation that many practical source and target languages provide similar
functionality, so a source instruction $\iota$ can often be emulated with target
instructions that access the same parts of the state as $\iota$. For example,
the \cc{addi32} instruction from eBPF reads and writes only registers (not,
e.g., memory), and it can be emulated with RISC-V instructions that also touch
only registers (\autoref{jitsynth:s:overview}). Moreover, note that the semantics of
\cc{addi32} ignores the values of its $\mathid{src}$ and $\mathid{off}$
fields, and that the target RISC-V instructions do the same. Based on these
observations, our optimized sketch for \cc{addi32} would therefore consists of
instruction holes that allow only register-register instructions, with field
holes that exclude $\mathid{src}$ and $\mathid{off}$. We first formalize this
intuition with the notion of \emph{read and write sets}, and then describe how \jitsynth
applies such sets to create \RW sketches.\tighten

\paragraph{Read and write sets.} 
Read and write sets provide a compact way to summarize the semantics of an
abstract instruction $\iota$. This summary consists of a set of \emph{state
labels}, where a state label is one of $L_\regs$, $L_\mem$, and $L_\pc$
(\autoref{def:state-labels}). Each label in a summary set represents a state
component (registers, memory, or the program counter) that a concrete instance
of $\iota$ may read or write during some execution. We compute three such sets
of labels for every $\iota$: the read set $\Read{\iota}$, the write set
$\Write{\iota}$, and the write set $\Write{\iota, f}$ for each field $f$ of
$\iota$. \autoref{fig:rw-sets} shows these sets for the toy eBPF and RISC-V
instructions.\tighten



\begin{figure}\centering
  {\small
  \begin{tabular}{llll}
    \toprule
    $\iota$ & $\Read{\iota}$ & $\Write{\iota}$ & $\Write{\iota, \mathid{field}}$ \\
    \midrule
    \cc{addi32} & $\{L_\regs\}$ & $\{L_\regs\}$ & $\mathid{imm}$: $\{L_\regs\}$; $\mathid{off}$: $\emptyset$; $\mathid{src}$: $\emptyset$; $\mathid{dst}$: $\{L_\regs\}$ \\
    \cc{lui} & $\{L_\regs\}$ & $\{L_\regs\}$ & $\mathid{rd}$: $\{L_\regs\}$; $\mathid{imm20}$: $\{L_\regs\}$\\
    \cc{sb} & $\{L_\regs\}$ & $\{L_\mem\}$ & $\mathid{rs1}$: $\{L_\mem\}$; $\mathid{rs2}$: $\{L_\mem\}$; $\mathid{imm12}$: $\{L_\mem\}$ \\
    \bottomrule\end{tabular}
  }

\caption{Read and write sets for the \cc{addi32}, \cc{lui}, and \cc{sb} instructions from \autoref{fig:lang-instrs}.}\label{fig:rw-sets}
\end{figure}

The read set $\Read{\iota}$ specifies which components of the input state may
affect the execution of $\iota$ (\autoref{def:read-set}). For example, if
$\Read{\iota}$ includes $L_\regs$, then some concrete instance of $\iota$
produces different output states when executed on two input states that differ
only in register values. 
% More generally, $L_n \in \Read{\iota}$ means that the behavior of $\iota$ depends on the values read from the component $n$.\tighten
%
The write set $\Write{\iota}$ specifies which components of the output state may
be affected by executing $\iota$ (\autoref{def:write-set}). In particular, if
$\Write{\iota}$ includes $L_\regs$ (or $L_\mem$), then executing some concrete
instance of $\iota$ on an input state produces an output state with different
register (or memory) values. The inclusion of $L_\pc$ is based on a separate
condition, designed to distinguish jump instructions from fall-through
instructions. Both kinds of instructions change the program counter, but
fall-through instructions always change it in the same way. So,
$L_\pc\in\Write{\iota}$ if two instances of $\iota$ can write different values
to the program counter.
%
Finally, the field write set, $\Write{\iota, f}$, specifies the parts of the
output state are affected by the value of the field $f$; $L_n \in
\Write{\iota, f}$ means that two instances of $\iota$ that differ only in $f$
can produce different outputs when applied to the same input state.\tighten

\jitsynth computes all read and write sets from their definitions, by using the
host symbolic evaluator to reduce the reasoning about instruction semantics to
SMT queries. This reduction is possible because we assume that all ARM
interpreters are self-finitizing, as discussed in \autoref{jitsynth:s:overview}.

\begin{definition}[State Labels]\label{def:state-labels} 
  A \emph{state label} is an identifier $L_n$ where $n$ is a state component,
  i.e., $n\in \{\regs,\mem,\pc\}$. We write $N$ for the set of all state
  components, and $\mathcal{L}$ for the set of all state labels. We also use
  state labels to access the corresponding state components:
  $\Get{L_n}{\sigma} = n(\sigma)$ for all $n\in N$.
\end{definition}

\begin{definition}[Read Set]\label{def:read-set}
  Let $\iota\in\mathcal{I}$ be an abstract instruction in $(\mathcal{I},
  \Sigma, \mathcal{T}, \Phi)$. The \textup{read set} of $\iota$,
  $\Read{\iota}$, is the set of all state labels $L_n\in\mathcal{L}$ such that 
  $
    \exists p\in\prog{\iota}.\, 
    \exists L_w \in\Write{\iota}.\,
    \exists \sigma_a, \sigma_b \in\Sigma.\,
    (\Get{L_n}{\sigma_a} \neq \Get{L_n}{\sigma_b} \wedge 
    (\bigwedge_{m\in N \setminus \{ n \}} \Get{L_m}{\sigma_a} = \Get{L_m}{\sigma_b}) \wedge
    \Get{L_w}{\mathcal{T}(p,\sigma_a)} \neq \Get{L_w}{\mathcal{T}(p,\sigma_b)}.
  $\tighten
\end{definition}

\begin{definition}[Write Set]\label{def:write-set}
  Let $\iota\in\mathcal{I}$ be an abstract instruction in $(\mathcal{I}, \Sigma,
  \mathcal{T}, \Phi)$. The \textup{write set} of $\iota$, $\Write{\iota}$,
  includes the state label $L_n \in \{ L_\regs, L_\mem \}$ iff 
  $
    \exists p\in\prog{\iota}.\, 
    \exists \sigma \in\Sigma.\, 
    \Get{L_n}{\sigma} \neq \Get{L_n}{\mathcal{T}(p,\sigma)}
  $, 
  and it includes the state label $L_\pc$ iff 
  $
    \exists p_a, p_b\in\prog{\iota}.\, 
    \exists \sigma \in\Sigma.\,
    \Get{L_\pc}{\mathcal{T}(p_a,\sigma)} \neq \Get{L_\pc}{\mathcal{T}(p_b,\sigma)}.
  $\tighten
\end{definition}

\begin{definition}[Field Write Set]\label{def:field-set}
  Let $f$ be a field of an abstract instruction $\iota =
  (\mathid{op},\mathcal{F})$ in $(\mathcal{I}, \Sigma, \mathcal{T}, \Phi)$. The
  \textup{write set} of $\iota$ and $f$, $\Write{\iota, f}$, includes
  the state label $L_n \in \mathcal{L}$ iff 
  $
    \exists p_a, p_b\in\prog{\iota}.\, 
    \exists \sigma \in\Sigma.\ 
    (p_a.f \neq p_b.f) \wedge
    (\bigwedge_{g\in\mathid{dom}(\mathcal{F})\setminus\{f\}} p_a.g = p_b.g) \wedge
    \Get{L_n}{\mathcal{T}(p_a,\sigma)} \neq \Get{L_n}{\mathcal{T}(p_b,\sigma)}
  $, where $p.f$ denotes $F(f)$ for $p = (op, F)$. 
\end{definition}



% \begin{example}\label{ex:rw-sets}
%   \jitsynth computes the following read and write sets of $\iota = \cc{addi32}$:
%   $\Read{\iota} = \Write{\iota} = \Write{\iota, \mathid{dst}} = \Write{\iota,
%   \mathid{imm}} = \{L_\regs\}$, and $\Write{\iota, \mathid{src}} \\ = \Write{\iota,
%   \mathid{offset}} = \{\}$. The read and write sets for $\cc{lui}$ and its
%   fields are also $\{ L_\regs \}$, while for $\iota = \cc{sb}$, we get
%   $\Read{\iota}  = \{L_\regs \}$ and $\Write{\iota}  = \Write{\iota, f} =
%   \{L_\mem \}$ for each field $f$ of \cc{sb}.\tighten
% \end{example}



\paragraph{Using read and write sets.}  


Given the read and write sets for a source instruction $\iota$ and target
instructions $\mathcal{I}_T$, \jitsynth generates the \RW sketch of length $k$
and depth $d$ by modifying the \Naive algorithm (\autoref{fig:naive-sketch}) as
follows. First, it restricts each target instruction hole (line
\ref{li:instruction-hole}) to choose an instruction $\iota_T\in\mathcal{I}_T$
with the same read and write sets as $\iota$, i.e., $\Read{\iota} =
\Read{\iota_T}$ and  $\Write{\iota} = \Write{\iota_T}$. Second, it restricts the
target field holes (line \ref{li:field-holes}) to use the source fields with the
matching field write set, i.e., the hole for a target field $f_T$ uses the
source field $f$ when $\Write{\iota_T, f_t} = \Write{\iota, f}$. For example,
given the sets from \autoref{fig:rw-sets}, the \RW instruction holes for
\cc{addi32} exclude \cc{sb} but include \cc{lui}, and the field holes for
\cc{lui} use only the $\mathid{dst}$ and $\mathid{imm}$ source fields. More
generally, the \RW sketch for \cc{addi32} consists of register-register
instructions over $\mathid{dst}$ and $\mathid{imm}$, as intended. This sketch
includes $2^{290}$ programs of length $k=5$ and depth $d\leq 3$, resulting in a
$2^{60}$ fold reduction in the size of the search space compared to the \Naive
sketch of the same length and depth. 

% These restrictions capture the hypothesis that $\iota$ can be compiled to
% $\mathcal{I}_T$ instructions that manipulate state in the same way as $\iota$.
% This intuition, in turn, assumes that the states of the source and target
% machines are related by a state mapping (\autoref{def:state-equiv}) rather than
% an arbitrary equivalence relation.


\subsection{Generating Pre-Load Sketches}\label{s:algorithm:pld} 

The pre-load sketch, $\LCS(k, d, \iota, \arm_S, \arm_T, \Map)$, is based on the
observation that hand-written JITs use macros or subroutines to generate
frequently used target instruction sequences. For example, compiling a source
instruction with immediate fields often involves loading the immediates into
scratch registers, and hand-written JITs include a subroutine that generates the
target instructions for performing these loads. The pre-load sketch shown in
\autoref{fig:lcs-sketch} mimics this structure.\tighten 


In particular, \LCS generates a sequence of $m$ concrete instructions that load
the (used) immediate fields of $\iota$, followed by a sequence of $k-m$
instruction holes. The instruction holes can refer to both the source registers
(if any) and the scratch registers (via the arbitrary bitvector constants
included in the \emph{Field} holes). The function
$\mathid{Load}(\mathid{Expr}(p.f), \arm_T, \Map)$ returns a sequence of target
instructions that load the immediate $p.f$ into an unused scratch register. This
function itself is synthesized by \jitsynth using a variant of the \RW
sketch.\tighten 

As an example, the pre-load sketch for \cc{addi32} consists of two $\mathid{Load}$
instructions (\cc{lui} and \cc{addiw} in the generated C code) and $k-2$
instruction holes.  The holes choose among register-register instructions in toy
RISC-V, and they can refer to the $\mathid{dst}$ register of \cc{addi32}, as
well as any scratch register. The resulting sketch includes $2^{100}$ programs
of length $k = 5$ and depth $d\leq 3$, providing a $2^{190}$ fold reduction in
the size of the search space compared to the \RW sketch.\tighten

\begin{figure}[H]
  \begin{algorithmic}[1] 
    \Function{PLD}
      {$k, d, \iota, \arm_S, \arm_T, \Map$}\Comment{$\iota\in\mathcal{I}_S, \arm_S = (\mathcal{I}_S, \ldots)$}
      \State $(\mathid{op}, \mathcal{F}) \gets \iota$, $(\mathid{I}_T, \ldots) \gets \arm_T$ \Comment{Source instruction, target instructions.}
      \State $p \gets \mathid{FreshId}()$ \Comment{Identifier for the compiler's input source instruction.}
      \State $\mathid{body}\gets []$  \Comment{The body of the compiler is a sequence with 2 parts:}
      \State $\mathid{imm} \gets \{ f \,|\, \mathcal{F}(f) = \BV{k} \text{ and } \Write{\iota,f} \neq \emptyset \}$ \Comment{(1) Load each relevant}
      \For{$f \in \mathid{imm}$} \Comment{source immediate into a free scratch register}
        \State $\mathid{body}\gets\mathid{body}\cdot \mathid{Load}(\mathid{Expr}(p.f), \arm_T, \Map)$ \Comment{using the load pseudoinstruction.}
      \EndFor
      % \If{$L_\pc\in\Read{\iota}$} \Comment{If $\iota$ reads the PC, load it into a free scratch register}
      %   \State $\mathid{body}\gets\mathid{body}\cdot \mathid{LoadPC}(\arm_T, \Map)$ \Comment{using the load PC pseudoinstruction.}
      % \EndIf
      \State $m \gets |\mathid{body}|$ \Comment{Let $m$ be the length of the load sequence.}
      \If{$m\geq k$ or $m = 0$} \Return $\bot$ \Comment{Return the empty sketch if $m\not\in (0..k)$.}
      \EndIf
      \For{$m \leq i < k$} \Comment{(2) Create $k-m$ target instruction holes, where the set}
        \State $I \gets \{\}$  \Comment{$I$ of choices for a target instruction hole includes}
        \For{$\iota_T\in\mathcal{I}_T, \iota_T = (\mathid{op}_T, \mathcal{F}_T)$} \Comment{all instructions from $\mathcal{I}_T$ that read-write}  
         \State $\mathid{rw}_T \gets  \Read{\iota_T} \times \Write{\iota_T}$ \Comment{the same state as $\iota$ or just registers.}
          \If{$\mathid{rw}_T = \Read{\iota} \times \Write{\iota}$ or $\mathid{rw}_T \subseteq \{L_\regs\} \times \{L_\regs\}$ }
            \State $\mathid{regs} \gets \{ f \,|\, \mathcal{F}(f) = \Reg \text{ and } \Write{\iota,f} \neq \emptyset \}$ \Comment{Any relevant}
            \State $E \gets \{\mathid{Expr}(p.f, \Map) \,|\, f\in\mathid{regs}\}$ \Comment{source register can appear in}
            \State $H \gets \{ f\mapsto \mathid{Field}(\mathcal{F}_T(f),d,E) \,|\, f\in\mathid{dom}(\mathcal{F}_T)\} $ \Comment{a target field hole,} 
            \State $I\gets I \cup \{ \mathit{Expr}((\mathid{op}_T, H),\Map) \}$ \Comment{and any constant register or value.}
          \EndIf
        \EndFor
        \State $\mathid{body} \gets \mathid{body} \cdot [\mathid{Choose}(I)]$ \Comment{Append a hole over $I$ to the body.}
      \EndFor
      % \If{$L_\pc\in\Write{\iota}$} \Comment{(3) If $\iota$ writes the PC, move a scratch register value}
      %   \State $\mathid{body}\gets\mathid{body}\cdot \mathid{StorePC}(\arm_T, \Map)$ \Comment{into the PC using a pseudoinstruction.}
      % \EndIf
      \State \Return $\mathid{Expr}((\lambda p\in\prog{\iota}\, .\, \mathid{body}),\Map)$ \Comment{A \minicompiler sketch for $\iota$.}
    \EndFunction
  \end{algorithmic}
  \caption{Pre-load sketch of length $k$ and maximum depth $d$ for $\iota$,
  $\arm_S$, $\arm_T$, and \Map. The $\mathid{Load}(E, \arm_T, \Map)$ function
  returns a sequence of target instructions that load the immediate value
  described by the expression $E$ into an unused scratch register; see
  \autoref{fig:naive-sketch} for descriptions of other helper
  functions.\tighten}\label{fig:lcs-sketch}
\end{figure}


\subsection{Solving Compiler Metasketches}\label{s:algorithm:solving}

\jitsynth solves the metasketch $\CMS(\iota, \arm_S, \arm_T, \Map)$ by applying
the host synthesizer to each of the generated sketches in turn until a
\minicompiler is found. If no \minicompiler exists in the search space, this
synthesis process runs forever. To check if a sketch $\mathcal{S}$ contains a
\minicompiler, \jitsynth would ideally ask the host synthesizer to solve the
following query, derived from Definitions \ref{def:mini-compiler}--\ref{def:state-equiv}: 
{\small
\[
    \exists C\in\mathcal{S} . \ 
    \forall \sigma_S\in\Sigma_S,\ \sigma_T\in\Sigma_T,\ p\in\prog{\iota}. \\ 
      \sigma_S \cong_{\Map} \sigma_T \Rightarrow
      \arm_S(p, \sigma_S) \cong_{\Map} \arm_T(C(p), \sigma_T)
\]}%
But recall that the state equivalence check $\cong_{\Map}$ involves universally
quantified formulas over memory addresses and register names. In principle,
these innermost quantifiers are not problematic because they range over finite
domains (bitvectors) so the formula remains decidable. In practice, however,
they lead to intractable SMT queries. We therefore solve a stronger soundness
query (\autoref{def:strong-mini-compiler}) that pulls these quantifiers out to
obtain the standard $\exists\forall$ formula with a quantifier-free body. The  
resulting formula can be solved with CEGIS~\cite{solar-lezama:sketch}, without 
requiring the underlying SMT solver to reason about quantifiers.

\begin{definition}[Strongly Sound \MiniCompiler]\label{def:strong-mini-compiler}
  Let $\arm_S = (\mathcal{I}_S, \Sigma_S, \mathcal{T}_S, \Phi_S)$ and $\arm_T =
  (\mathcal{I}_T, \Sigma_T, \mathcal{T}_T, \Phi_T)$ be two abstract register
  machines, $\cong_{\Map}$ an injective state equivalence relation on their
  states $\Sigma_S$ and $\Sigma_T$, and $C: \prog{\iota} \rightarrow
  \mathid{List}(\prog{\mathcal{I}_T})$ a function for some
  $\iota\in\mathcal{I}_S$. We say that $C$ is a \textup{strongly sound
  \minicompiler} for $\iota_{\Map}$ with respect to $\cong$ iff
  \begin{align*}\small
    & \forall \sigma_S\in\Sigma_S,\ \sigma_T\in\Sigma_T,\ p\in\prog{\iota}, \
    a\in\mathid{dom}(\mem(\sigma_S)),\ r\in\mathid{dom}(\regs(\sigma_S)). \\ 
    & \sigma_S \cong_{\Map,a,r} \sigma_T \Rightarrow
       \arm_S(p, \sigma_S) \cong_{\Map,a,r} \arm_T(C(p), \sigma_T)
\end{align*}
where $\cong_{\Map,a,r}$ stands for the $\cong_{\Map}$ formula with $a$ and $r$
as free variables. 
  \end{definition} 

The \jitsynth synthesis procedure is sound and complete with respect to this
stronger query (\autoref{thm:synthesis-soundness-and-completeness}). The proof
follows from the soundness and completeness of the host synthesizer, and the
construction of the compiler metasketch. We discharge this proof using Lean theorem prover~\cite{moura:lean}.

\begin{theorem}[Strong soundness and completeness of \jitsynth]\label{thm:synthesis-soundness-and-completeness}
  Let $\mathcal{C} = \CMS(\iota, \arm_S, \arm_T, \Map)$ be the compiler
  metasketch for the abstract instruction $\iota$, machines $\arm_S$ and
  $\arm_T$, and the state mapping $\Map$. If \jitsynth terminates and returns a
  program $C$ when applied to $\mathcal{C}$, then $C$ is a strongly sound
  \minicompiler for $\iota$ and $\arm_T$ (soundness). If there is a strongly
  sound \minicompiler in the most general search space $\{\Naive(k, d, \iota,
  \arm_S, \arm_T, \Map) \,|\, k, d\in \mathbb{N}\}$, then \jitsynth will terminate on
  $\mathcal{C}$ and produce a program (completeness). 
\end{theorem}


\section{Implementation}\label{jitsynth:s:impl}
We implemented \jitsynth as described in \autoref{jitsynth:s:overview} using
Rosette~\cite{torlak:rosette} as our host language.
%Without bounding the length of the compiler, the space explored by \jitsynth is infinite, so we use a timeout
%of 48 hours to bound execution.
Since the search spaces for different compiler lengths are disjoint,
the \jitsynth implementation searches these spaces in parallel~\cite{bornholt:synapse}.
%In addition, \jitsynth implements several optimizations
%in order to synthesize real-world JIT compilers, as follows.
%
%
%\paragraph{ARM Fuel Function.}
%For simplicity,
We use $\Phi(\vec{p}) = \texttt{length}(\vec{p})$ as the fuel function for all
languages studied in this work. This provides sufficient fuel for evaluating
programs in these languages that are accepted by the OS kernel. For example, the
Linux kernel requires eBPF programs to be loop-free, and it enforces this
restriction with a conservative static check; programs that fail the check are
not passed to the JIT~\cite{gershuni:crab-ebpf}.

\paragraph{Synthesizing Common Sequences.}
In addition to separately synthesizing pre-loads for parameters,
\jitsynth also synthesizes a few other small sequences used commonly in compiled programs.

One such class of sequences are register extensions.
The target abstract register machine may have larger register values than the source machine.
In these cases, the compiler may often need to either sign-extend the target register values,
filling the upper bits of the register with the sign bit,
or zero-extend, filling the upper bits of the register with 0.
\jitsynth synthesizes a sequence for each of these extension types.
When $L_\regs\in\Read\iota$ of source abstract instruction $\iota$,
\jitsynth will try sketches that extend used registers in both ways.

Additionally, \jitsynth synthesizes sequences
that both load the PC into a temporary register
and write to the PC from a register value.
These operations are needed for compiling jumps,
and are reused in many \minicompilers.

\paragraph{NOP Padding and Removal.}
\jitsynth synthesizes \minicompilers of equal length to ensure that
the target PC for jumps can be computed as a multiple of the source PC.
%
To do so, \jitsynth pads instruction sequences with NOPs to match the length
of the largest \minicompiler.
%
To mitigate the performance impact this incurs, we implemented a trusted compiler
pass that removes the NOP instructions while preserving the correctness of
the compiled code.
\tighten


\section{Evaluation}\label{s:eval}

This section evaluates \jitsynth by answering the following research questions:
\begin{enumerate}[label={},leftmargin=0em]
  \item \textbf{RQ1}: Can \jitsynth synthesize correct and performant compilers for
  real-world source and target languages?
  % \item \textbf{RQ2}: Can \jitsynth synthesize a wide variety of compilers?
  \item \textbf{RQ2}: How effective are the sketch optimizations described in \autoref{s:algorithm}?
\end{enumerate}


\subsection{Synthesizing compilers for real-world source-target pairs}

% \begin{figure}
%   \centering
%   \begin{tabular}{lccc}
\toprule
 & eBPF & RISC-V & Classic BPF \\
\midrule
 Arithmetic instructions & 46 & 33 & 10 \\
 Memory instructions & 12 & 11 & 9 \\
 Jump instructions & 24 & 7 & 11 \\
 Other instructions & 1 & 2 & 2 \\
 Number of registers & 10 & 31 & 2 \\

\bottomrule

\end{tabular}
%   \vspace{1.5em}\newline
%   \begin{tabular}{ll}
\toprule
libseccomp Instruction & Description\\
\midrule
  Arg Check & Check if rule argument passed \\
  Arg Fail & Mark when a rule argment fails \\
  Rule Check Failed & If a rule argument has failed, fall through to next rule \\
  Rule Check Pass & Check to see if the rule has passed \\
  Rule Pass & Perform specified action when rule passes \\
  Rule End & Clean registers after checking rule \\
  Default & Perform the default action \\
\bottomrule
\end{tabular}

%   \caption{Description of eBPF, RISC-V, classic BPF, and libseccomp languages}
%   \label{fig:lang}
% \end{figure}

\begin{figure}[h]
  \resizebox{\textwidth}{!}{
  %% Creator: Matplotlib, PGF backend
%%
%% To include the figure in your LaTeX document, write
%%   \input{<filename>.pgf}
%%
%% Make sure the required packages are loaded in your preamble
%%   \usepackage{pgf}
%%
%% Figures using additional raster images can only be included by \input if
%% they are in the same directory as the main LaTeX file. For loading figures
%% from other directories you can use the `import` package
%%   \usepackage{import}
%% and then include the figures with
%%   \import{<path to file>}{<filename>.pgf}
%%
%% Matplotlib used the following preamble
%%
\begingroup%
\makeatletter%
\begin{pgfpicture}%
\pgfpathrectangle{\pgfpointorigin}{\pgfqpoint{15.000000in}{5.000000in}}%
\pgfusepath{use as bounding box, clip}%
\begin{pgfscope}%
\pgfsetbuttcap%
\pgfsetmiterjoin%
\definecolor{currentfill}{rgb}{1.000000,1.000000,1.000000}%
\pgfsetfillcolor{currentfill}%
\pgfsetlinewidth{0.000000pt}%
\definecolor{currentstroke}{rgb}{1.000000,1.000000,1.000000}%
\pgfsetstrokecolor{currentstroke}%
\pgfsetdash{}{0pt}%
\pgfpathmoveto{\pgfqpoint{0.000000in}{0.000000in}}%
\pgfpathlineto{\pgfqpoint{15.000000in}{0.000000in}}%
\pgfpathlineto{\pgfqpoint{15.000000in}{5.000000in}}%
\pgfpathlineto{\pgfqpoint{0.000000in}{5.000000in}}%
\pgfpathclose%
\pgfusepath{fill}%
\end{pgfscope}%
\begin{pgfscope}%
\pgfsetbuttcap%
\pgfsetmiterjoin%
\definecolor{currentfill}{rgb}{1.000000,1.000000,1.000000}%
\pgfsetfillcolor{currentfill}%
\pgfsetlinewidth{0.000000pt}%
\definecolor{currentstroke}{rgb}{0.000000,0.000000,0.000000}%
\pgfsetstrokecolor{currentstroke}%
\pgfsetstrokeopacity{0.000000}%
\pgfsetdash{}{0pt}%
\pgfpathmoveto{\pgfqpoint{1.228750in}{1.022500in}}%
\pgfpathlineto{\pgfqpoint{14.685000in}{1.022500in}}%
\pgfpathlineto{\pgfqpoint{14.685000in}{4.685000in}}%
\pgfpathlineto{\pgfqpoint{1.228750in}{4.685000in}}%
\pgfpathclose%
\pgfusepath{fill}%
\end{pgfscope}%
\begin{pgfscope}%
\definecolor{textcolor}{rgb}{0.150000,0.150000,0.150000}%
\pgfsetstrokecolor{textcolor}%
\pgfsetfillcolor{textcolor}%
\pgftext[x=2.189911in,y=0.890556in,,top]{\color{textcolor}\sffamily\fontsize{19.250000}{23.100000}\selectfont OpenSSH}%
\end{pgfscope}%
\begin{pgfscope}%
\definecolor{textcolor}{rgb}{0.150000,0.150000,0.150000}%
\pgfsetstrokecolor{textcolor}%
\pgfsetfillcolor{textcolor}%
\pgftext[x=4.112232in,y=0.890556in,,top]{\color{textcolor}\sffamily\fontsize{19.250000}{23.100000}\selectfont NaCl}%
\end{pgfscope}%
\begin{pgfscope}%
\definecolor{textcolor}{rgb}{0.150000,0.150000,0.150000}%
\pgfsetstrokecolor{textcolor}%
\pgfsetfillcolor{textcolor}%
\pgftext[x=6.034554in,y=0.890556in,,top]{\color{textcolor}\sffamily\fontsize{19.250000}{23.100000}\selectfont QEMU}%
\end{pgfscope}%
\begin{pgfscope}%
\definecolor{textcolor}{rgb}{0.150000,0.150000,0.150000}%
\pgfsetstrokecolor{textcolor}%
\pgfsetfillcolor{textcolor}%
\pgftext[x=7.956875in,y=0.890556in,,top]{\color{textcolor}\sffamily\fontsize{19.250000}{23.100000}\selectfont Chrome}%
\end{pgfscope}%
\begin{pgfscope}%
\definecolor{textcolor}{rgb}{0.150000,0.150000,0.150000}%
\pgfsetstrokecolor{textcolor}%
\pgfsetfillcolor{textcolor}%
\pgftext[x=9.879196in,y=0.890556in,,top]{\color{textcolor}\sffamily\fontsize{19.250000}{23.100000}\selectfont Firefox}%
\end{pgfscope}%
\begin{pgfscope}%
\definecolor{textcolor}{rgb}{0.150000,0.150000,0.150000}%
\pgfsetstrokecolor{textcolor}%
\pgfsetfillcolor{textcolor}%
\pgftext[x=11.801518in,y=0.890556in,,top]{\color{textcolor}\sffamily\fontsize{19.250000}{23.100000}\selectfont vsftpd}%
\end{pgfscope}%
\begin{pgfscope}%
\definecolor{textcolor}{rgb}{0.150000,0.150000,0.150000}%
\pgfsetstrokecolor{textcolor}%
\pgfsetfillcolor{textcolor}%
\pgftext[x=13.723839in,y=0.890556in,,top]{\color{textcolor}\sffamily\fontsize{19.250000}{23.100000}\selectfont Tor}%
\end{pgfscope}%
\begin{pgfscope}%
\definecolor{textcolor}{rgb}{0.150000,0.150000,0.150000}%
\pgfsetstrokecolor{textcolor}%
\pgfsetfillcolor{textcolor}%
\pgftext[x=7.956875in,y=0.578932in,,top]{\color{textcolor}\sffamily\fontsize{21.000000}{25.200000}\selectfont Benchmark}%
\end{pgfscope}%
\begin{pgfscope}%
\pgfpathrectangle{\pgfqpoint{1.228750in}{1.022500in}}{\pgfqpoint{13.456250in}{3.662500in}}%
\pgfusepath{clip}%
\pgfsetroundcap%
\pgfsetroundjoin%
\pgfsetlinewidth{1.003750pt}%
\definecolor{currentstroke}{rgb}{0.800000,0.800000,0.800000}%
\pgfsetstrokecolor{currentstroke}%
\pgfsetdash{}{0pt}%
\pgfpathmoveto{\pgfqpoint{1.228750in}{1.022500in}}%
\pgfpathlineto{\pgfqpoint{14.685000in}{1.022500in}}%
\pgfusepath{stroke}%
\end{pgfscope}%
\begin{pgfscope}%
\definecolor{textcolor}{rgb}{0.150000,0.150000,0.150000}%
\pgfsetstrokecolor{textcolor}%
\pgfsetfillcolor{textcolor}%
\pgftext[x=0.961364in,y=0.922481in,left,base]{\color{textcolor}\sffamily\fontsize{19.250000}{23.100000}\selectfont 0}%
\end{pgfscope}%
\begin{pgfscope}%
\pgfpathrectangle{\pgfqpoint{1.228750in}{1.022500in}}{\pgfqpoint{13.456250in}{3.662500in}}%
\pgfusepath{clip}%
\pgfsetroundcap%
\pgfsetroundjoin%
\pgfsetlinewidth{1.003750pt}%
\definecolor{currentstroke}{rgb}{0.800000,0.800000,0.800000}%
\pgfsetstrokecolor{currentstroke}%
\pgfsetdash{}{0pt}%
\pgfpathmoveto{\pgfqpoint{1.228750in}{2.105759in}}%
\pgfpathlineto{\pgfqpoint{14.685000in}{2.105759in}}%
\pgfusepath{stroke}%
\end{pgfscope}%
\begin{pgfscope}%
\definecolor{textcolor}{rgb}{0.150000,0.150000,0.150000}%
\pgfsetstrokecolor{textcolor}%
\pgfsetfillcolor{textcolor}%
\pgftext[x=0.690481in,y=2.005740in,left,base]{\color{textcolor}\sffamily\fontsize{19.250000}{23.100000}\selectfont 200}%
\end{pgfscope}%
\begin{pgfscope}%
\pgfpathrectangle{\pgfqpoint{1.228750in}{1.022500in}}{\pgfqpoint{13.456250in}{3.662500in}}%
\pgfusepath{clip}%
\pgfsetroundcap%
\pgfsetroundjoin%
\pgfsetlinewidth{1.003750pt}%
\definecolor{currentstroke}{rgb}{0.800000,0.800000,0.800000}%
\pgfsetstrokecolor{currentstroke}%
\pgfsetdash{}{0pt}%
\pgfpathmoveto{\pgfqpoint{1.228750in}{3.189019in}}%
\pgfpathlineto{\pgfqpoint{14.685000in}{3.189019in}}%
\pgfusepath{stroke}%
\end{pgfscope}%
\begin{pgfscope}%
\definecolor{textcolor}{rgb}{0.150000,0.150000,0.150000}%
\pgfsetstrokecolor{textcolor}%
\pgfsetfillcolor{textcolor}%
\pgftext[x=0.690481in,y=3.089000in,left,base]{\color{textcolor}\sffamily\fontsize{19.250000}{23.100000}\selectfont 400}%
\end{pgfscope}%
\begin{pgfscope}%
\pgfpathrectangle{\pgfqpoint{1.228750in}{1.022500in}}{\pgfqpoint{13.456250in}{3.662500in}}%
\pgfusepath{clip}%
\pgfsetroundcap%
\pgfsetroundjoin%
\pgfsetlinewidth{1.003750pt}%
\definecolor{currentstroke}{rgb}{0.800000,0.800000,0.800000}%
\pgfsetstrokecolor{currentstroke}%
\pgfsetdash{}{0pt}%
\pgfpathmoveto{\pgfqpoint{1.228750in}{4.272278in}}%
\pgfpathlineto{\pgfqpoint{14.685000in}{4.272278in}}%
\pgfusepath{stroke}%
\end{pgfscope}%
\begin{pgfscope}%
\definecolor{textcolor}{rgb}{0.150000,0.150000,0.150000}%
\pgfsetstrokecolor{textcolor}%
\pgfsetfillcolor{textcolor}%
\pgftext[x=0.690481in,y=4.172259in,left,base]{\color{textcolor}\sffamily\fontsize{19.250000}{23.100000}\selectfont 600}%
\end{pgfscope}%
\begin{pgfscope}%
\definecolor{textcolor}{rgb}{0.150000,0.150000,0.150000}%
\pgfsetstrokecolor{textcolor}%
\pgfsetfillcolor{textcolor}%
\pgftext[x=0.634925in,y=2.853750in,,bottom,rotate=90.000000]{\color{textcolor}\sffamily\fontsize{21.000000}{25.200000}\selectfont Cycles}%
\end{pgfscope}%
\begin{pgfscope}%
\pgfpathrectangle{\pgfqpoint{1.228750in}{1.022500in}}{\pgfqpoint{13.456250in}{3.662500in}}%
\pgfusepath{clip}%
\pgfsetbuttcap%
\pgfsetmiterjoin%
\definecolor{currentfill}{rgb}{0.347059,0.458824,0.641176}%
\pgfsetfillcolor{currentfill}%
\pgfsetlinewidth{1.003750pt}%
\definecolor{currentstroke}{rgb}{1.000000,1.000000,1.000000}%
\pgfsetstrokecolor{currentstroke}%
\pgfsetdash{}{0pt}%
\pgfpathmoveto{\pgfqpoint{1.420982in}{1.022500in}}%
\pgfpathlineto{\pgfqpoint{1.933601in}{1.022500in}}%
\pgfpathlineto{\pgfqpoint{1.933601in}{1.650790in}}%
\pgfpathlineto{\pgfqpoint{1.420982in}{1.650790in}}%
\pgfpathclose%
\pgfusepath{stroke,fill}%
\end{pgfscope}%
\begin{pgfscope}%
\pgfsetbuttcap%
\pgfsetmiterjoin%
\definecolor{currentfill}{rgb}{0.347059,0.458824,0.641176}%
\pgfsetfillcolor{currentfill}%
\pgfsetlinewidth{1.003750pt}%
\definecolor{currentstroke}{rgb}{1.000000,1.000000,1.000000}%
\pgfsetstrokecolor{currentstroke}%
\pgfsetdash{}{0pt}%
\pgfpathrectangle{\pgfqpoint{1.228750in}{1.022500in}}{\pgfqpoint{13.456250in}{3.662500in}}%
\pgfusepath{clip}%
\pgfpathmoveto{\pgfqpoint{1.420982in}{1.022500in}}%
\pgfpathlineto{\pgfqpoint{1.933601in}{1.022500in}}%
\pgfpathlineto{\pgfqpoint{1.933601in}{1.650790in}}%
\pgfpathlineto{\pgfqpoint{1.420982in}{1.650790in}}%
\pgfpathclose%
\pgfusepath{clip}%
\pgfsys@defobject{currentpattern}{\pgfqpoint{0in}{0in}}{\pgfqpoint{1in}{1in}}{%
\begin{pgfscope}%
\pgfpathrectangle{\pgfqpoint{0in}{0in}}{\pgfqpoint{1in}{1in}}%
\pgfusepath{clip}%
\pgfpathmoveto{\pgfqpoint{-0.500000in}{0.500000in}}%
\pgfpathlineto{\pgfqpoint{0.500000in}{1.500000in}}%
\pgfpathmoveto{\pgfqpoint{-0.444444in}{0.444444in}}%
\pgfpathlineto{\pgfqpoint{0.555556in}{1.444444in}}%
\pgfpathmoveto{\pgfqpoint{-0.388889in}{0.388889in}}%
\pgfpathlineto{\pgfqpoint{0.611111in}{1.388889in}}%
\pgfpathmoveto{\pgfqpoint{-0.333333in}{0.333333in}}%
\pgfpathlineto{\pgfqpoint{0.666667in}{1.333333in}}%
\pgfpathmoveto{\pgfqpoint{-0.277778in}{0.277778in}}%
\pgfpathlineto{\pgfqpoint{0.722222in}{1.277778in}}%
\pgfpathmoveto{\pgfqpoint{-0.222222in}{0.222222in}}%
\pgfpathlineto{\pgfqpoint{0.777778in}{1.222222in}}%
\pgfpathmoveto{\pgfqpoint{-0.166667in}{0.166667in}}%
\pgfpathlineto{\pgfqpoint{0.833333in}{1.166667in}}%
\pgfpathmoveto{\pgfqpoint{-0.111111in}{0.111111in}}%
\pgfpathlineto{\pgfqpoint{0.888889in}{1.111111in}}%
\pgfpathmoveto{\pgfqpoint{-0.055556in}{0.055556in}}%
\pgfpathlineto{\pgfqpoint{0.944444in}{1.055556in}}%
\pgfpathmoveto{\pgfqpoint{0.000000in}{0.000000in}}%
\pgfpathlineto{\pgfqpoint{1.000000in}{1.000000in}}%
\pgfpathmoveto{\pgfqpoint{0.055556in}{-0.055556in}}%
\pgfpathlineto{\pgfqpoint{1.055556in}{0.944444in}}%
\pgfpathmoveto{\pgfqpoint{0.111111in}{-0.111111in}}%
\pgfpathlineto{\pgfqpoint{1.111111in}{0.888889in}}%
\pgfpathmoveto{\pgfqpoint{0.166667in}{-0.166667in}}%
\pgfpathlineto{\pgfqpoint{1.166667in}{0.833333in}}%
\pgfpathmoveto{\pgfqpoint{0.222222in}{-0.222222in}}%
\pgfpathlineto{\pgfqpoint{1.222222in}{0.777778in}}%
\pgfpathmoveto{\pgfqpoint{0.277778in}{-0.277778in}}%
\pgfpathlineto{\pgfqpoint{1.277778in}{0.722222in}}%
\pgfpathmoveto{\pgfqpoint{0.333333in}{-0.333333in}}%
\pgfpathlineto{\pgfqpoint{1.333333in}{0.666667in}}%
\pgfpathmoveto{\pgfqpoint{0.388889in}{-0.388889in}}%
\pgfpathlineto{\pgfqpoint{1.388889in}{0.611111in}}%
\pgfpathmoveto{\pgfqpoint{0.444444in}{-0.444444in}}%
\pgfpathlineto{\pgfqpoint{1.444444in}{0.555556in}}%
\pgfpathmoveto{\pgfqpoint{0.500000in}{-0.500000in}}%
\pgfpathlineto{\pgfqpoint{1.500000in}{0.500000in}}%
\pgfusepath{stroke}%
\end{pgfscope}%
}%
\pgfsys@transformshift{1.420982in}{1.022500in}%
\pgfsys@useobject{currentpattern}{}%
\pgfsys@transformshift{1in}{0in}%
\pgfsys@transformshift{-1in}{0in}%
\pgfsys@transformshift{0in}{1in}%
\end{pgfscope}%
\begin{pgfscope}%
\pgfpathrectangle{\pgfqpoint{1.228750in}{1.022500in}}{\pgfqpoint{13.456250in}{3.662500in}}%
\pgfusepath{clip}%
\pgfsetbuttcap%
\pgfsetmiterjoin%
\definecolor{currentfill}{rgb}{0.347059,0.458824,0.641176}%
\pgfsetfillcolor{currentfill}%
\pgfsetlinewidth{1.003750pt}%
\definecolor{currentstroke}{rgb}{1.000000,1.000000,1.000000}%
\pgfsetstrokecolor{currentstroke}%
\pgfsetdash{}{0pt}%
\pgfpathmoveto{\pgfqpoint{3.343304in}{1.022500in}}%
\pgfpathlineto{\pgfqpoint{3.855923in}{1.022500in}}%
\pgfpathlineto{\pgfqpoint{3.855923in}{1.607460in}}%
\pgfpathlineto{\pgfqpoint{3.343304in}{1.607460in}}%
\pgfpathclose%
\pgfusepath{stroke,fill}%
\end{pgfscope}%
\begin{pgfscope}%
\pgfsetbuttcap%
\pgfsetmiterjoin%
\definecolor{currentfill}{rgb}{0.347059,0.458824,0.641176}%
\pgfsetfillcolor{currentfill}%
\pgfsetlinewidth{1.003750pt}%
\definecolor{currentstroke}{rgb}{1.000000,1.000000,1.000000}%
\pgfsetstrokecolor{currentstroke}%
\pgfsetdash{}{0pt}%
\pgfpathrectangle{\pgfqpoint{1.228750in}{1.022500in}}{\pgfqpoint{13.456250in}{3.662500in}}%
\pgfusepath{clip}%
\pgfpathmoveto{\pgfqpoint{3.343304in}{1.022500in}}%
\pgfpathlineto{\pgfqpoint{3.855923in}{1.022500in}}%
\pgfpathlineto{\pgfqpoint{3.855923in}{1.607460in}}%
\pgfpathlineto{\pgfqpoint{3.343304in}{1.607460in}}%
\pgfpathclose%
\pgfusepath{clip}%
\pgfsys@defobject{currentpattern}{\pgfqpoint{0in}{0in}}{\pgfqpoint{1in}{1in}}{%
\begin{pgfscope}%
\pgfpathrectangle{\pgfqpoint{0in}{0in}}{\pgfqpoint{1in}{1in}}%
\pgfusepath{clip}%
\pgfpathmoveto{\pgfqpoint{-0.500000in}{0.500000in}}%
\pgfpathlineto{\pgfqpoint{0.500000in}{1.500000in}}%
\pgfpathmoveto{\pgfqpoint{-0.444444in}{0.444444in}}%
\pgfpathlineto{\pgfqpoint{0.555556in}{1.444444in}}%
\pgfpathmoveto{\pgfqpoint{-0.388889in}{0.388889in}}%
\pgfpathlineto{\pgfqpoint{0.611111in}{1.388889in}}%
\pgfpathmoveto{\pgfqpoint{-0.333333in}{0.333333in}}%
\pgfpathlineto{\pgfqpoint{0.666667in}{1.333333in}}%
\pgfpathmoveto{\pgfqpoint{-0.277778in}{0.277778in}}%
\pgfpathlineto{\pgfqpoint{0.722222in}{1.277778in}}%
\pgfpathmoveto{\pgfqpoint{-0.222222in}{0.222222in}}%
\pgfpathlineto{\pgfqpoint{0.777778in}{1.222222in}}%
\pgfpathmoveto{\pgfqpoint{-0.166667in}{0.166667in}}%
\pgfpathlineto{\pgfqpoint{0.833333in}{1.166667in}}%
\pgfpathmoveto{\pgfqpoint{-0.111111in}{0.111111in}}%
\pgfpathlineto{\pgfqpoint{0.888889in}{1.111111in}}%
\pgfpathmoveto{\pgfqpoint{-0.055556in}{0.055556in}}%
\pgfpathlineto{\pgfqpoint{0.944444in}{1.055556in}}%
\pgfpathmoveto{\pgfqpoint{0.000000in}{0.000000in}}%
\pgfpathlineto{\pgfqpoint{1.000000in}{1.000000in}}%
\pgfpathmoveto{\pgfqpoint{0.055556in}{-0.055556in}}%
\pgfpathlineto{\pgfqpoint{1.055556in}{0.944444in}}%
\pgfpathmoveto{\pgfqpoint{0.111111in}{-0.111111in}}%
\pgfpathlineto{\pgfqpoint{1.111111in}{0.888889in}}%
\pgfpathmoveto{\pgfqpoint{0.166667in}{-0.166667in}}%
\pgfpathlineto{\pgfqpoint{1.166667in}{0.833333in}}%
\pgfpathmoveto{\pgfqpoint{0.222222in}{-0.222222in}}%
\pgfpathlineto{\pgfqpoint{1.222222in}{0.777778in}}%
\pgfpathmoveto{\pgfqpoint{0.277778in}{-0.277778in}}%
\pgfpathlineto{\pgfqpoint{1.277778in}{0.722222in}}%
\pgfpathmoveto{\pgfqpoint{0.333333in}{-0.333333in}}%
\pgfpathlineto{\pgfqpoint{1.333333in}{0.666667in}}%
\pgfpathmoveto{\pgfqpoint{0.388889in}{-0.388889in}}%
\pgfpathlineto{\pgfqpoint{1.388889in}{0.611111in}}%
\pgfpathmoveto{\pgfqpoint{0.444444in}{-0.444444in}}%
\pgfpathlineto{\pgfqpoint{1.444444in}{0.555556in}}%
\pgfpathmoveto{\pgfqpoint{0.500000in}{-0.500000in}}%
\pgfpathlineto{\pgfqpoint{1.500000in}{0.500000in}}%
\pgfusepath{stroke}%
\end{pgfscope}%
}%
\pgfsys@transformshift{3.343304in}{1.022500in}%
\pgfsys@useobject{currentpattern}{}%
\pgfsys@transformshift{1in}{0in}%
\pgfsys@transformshift{-1in}{0in}%
\pgfsys@transformshift{0in}{1in}%
\end{pgfscope}%
\begin{pgfscope}%
\pgfpathrectangle{\pgfqpoint{1.228750in}{1.022500in}}{\pgfqpoint{13.456250in}{3.662500in}}%
\pgfusepath{clip}%
\pgfsetbuttcap%
\pgfsetmiterjoin%
\definecolor{currentfill}{rgb}{0.347059,0.458824,0.641176}%
\pgfsetfillcolor{currentfill}%
\pgfsetlinewidth{1.003750pt}%
\definecolor{currentstroke}{rgb}{1.000000,1.000000,1.000000}%
\pgfsetstrokecolor{currentstroke}%
\pgfsetdash{}{0pt}%
\pgfpathmoveto{\pgfqpoint{5.265625in}{1.022500in}}%
\pgfpathlineto{\pgfqpoint{5.778244in}{1.022500in}}%
\pgfpathlineto{\pgfqpoint{5.778244in}{1.672456in}}%
\pgfpathlineto{\pgfqpoint{5.265625in}{1.672456in}}%
\pgfpathclose%
\pgfusepath{stroke,fill}%
\end{pgfscope}%
\begin{pgfscope}%
\pgfsetbuttcap%
\pgfsetmiterjoin%
\definecolor{currentfill}{rgb}{0.347059,0.458824,0.641176}%
\pgfsetfillcolor{currentfill}%
\pgfsetlinewidth{1.003750pt}%
\definecolor{currentstroke}{rgb}{1.000000,1.000000,1.000000}%
\pgfsetstrokecolor{currentstroke}%
\pgfsetdash{}{0pt}%
\pgfpathrectangle{\pgfqpoint{1.228750in}{1.022500in}}{\pgfqpoint{13.456250in}{3.662500in}}%
\pgfusepath{clip}%
\pgfpathmoveto{\pgfqpoint{5.265625in}{1.022500in}}%
\pgfpathlineto{\pgfqpoint{5.778244in}{1.022500in}}%
\pgfpathlineto{\pgfqpoint{5.778244in}{1.672456in}}%
\pgfpathlineto{\pgfqpoint{5.265625in}{1.672456in}}%
\pgfpathclose%
\pgfusepath{clip}%
\pgfsys@defobject{currentpattern}{\pgfqpoint{0in}{0in}}{\pgfqpoint{1in}{1in}}{%
\begin{pgfscope}%
\pgfpathrectangle{\pgfqpoint{0in}{0in}}{\pgfqpoint{1in}{1in}}%
\pgfusepath{clip}%
\pgfpathmoveto{\pgfqpoint{-0.500000in}{0.500000in}}%
\pgfpathlineto{\pgfqpoint{0.500000in}{1.500000in}}%
\pgfpathmoveto{\pgfqpoint{-0.444444in}{0.444444in}}%
\pgfpathlineto{\pgfqpoint{0.555556in}{1.444444in}}%
\pgfpathmoveto{\pgfqpoint{-0.388889in}{0.388889in}}%
\pgfpathlineto{\pgfqpoint{0.611111in}{1.388889in}}%
\pgfpathmoveto{\pgfqpoint{-0.333333in}{0.333333in}}%
\pgfpathlineto{\pgfqpoint{0.666667in}{1.333333in}}%
\pgfpathmoveto{\pgfqpoint{-0.277778in}{0.277778in}}%
\pgfpathlineto{\pgfqpoint{0.722222in}{1.277778in}}%
\pgfpathmoveto{\pgfqpoint{-0.222222in}{0.222222in}}%
\pgfpathlineto{\pgfqpoint{0.777778in}{1.222222in}}%
\pgfpathmoveto{\pgfqpoint{-0.166667in}{0.166667in}}%
\pgfpathlineto{\pgfqpoint{0.833333in}{1.166667in}}%
\pgfpathmoveto{\pgfqpoint{-0.111111in}{0.111111in}}%
\pgfpathlineto{\pgfqpoint{0.888889in}{1.111111in}}%
\pgfpathmoveto{\pgfqpoint{-0.055556in}{0.055556in}}%
\pgfpathlineto{\pgfqpoint{0.944444in}{1.055556in}}%
\pgfpathmoveto{\pgfqpoint{0.000000in}{0.000000in}}%
\pgfpathlineto{\pgfqpoint{1.000000in}{1.000000in}}%
\pgfpathmoveto{\pgfqpoint{0.055556in}{-0.055556in}}%
\pgfpathlineto{\pgfqpoint{1.055556in}{0.944444in}}%
\pgfpathmoveto{\pgfqpoint{0.111111in}{-0.111111in}}%
\pgfpathlineto{\pgfqpoint{1.111111in}{0.888889in}}%
\pgfpathmoveto{\pgfqpoint{0.166667in}{-0.166667in}}%
\pgfpathlineto{\pgfqpoint{1.166667in}{0.833333in}}%
\pgfpathmoveto{\pgfqpoint{0.222222in}{-0.222222in}}%
\pgfpathlineto{\pgfqpoint{1.222222in}{0.777778in}}%
\pgfpathmoveto{\pgfqpoint{0.277778in}{-0.277778in}}%
\pgfpathlineto{\pgfqpoint{1.277778in}{0.722222in}}%
\pgfpathmoveto{\pgfqpoint{0.333333in}{-0.333333in}}%
\pgfpathlineto{\pgfqpoint{1.333333in}{0.666667in}}%
\pgfpathmoveto{\pgfqpoint{0.388889in}{-0.388889in}}%
\pgfpathlineto{\pgfqpoint{1.388889in}{0.611111in}}%
\pgfpathmoveto{\pgfqpoint{0.444444in}{-0.444444in}}%
\pgfpathlineto{\pgfqpoint{1.444444in}{0.555556in}}%
\pgfpathmoveto{\pgfqpoint{0.500000in}{-0.500000in}}%
\pgfpathlineto{\pgfqpoint{1.500000in}{0.500000in}}%
\pgfusepath{stroke}%
\end{pgfscope}%
}%
\pgfsys@transformshift{5.265625in}{1.022500in}%
\pgfsys@useobject{currentpattern}{}%
\pgfsys@transformshift{1in}{0in}%
\pgfsys@transformshift{-1in}{0in}%
\pgfsys@transformshift{0in}{1in}%
\end{pgfscope}%
\begin{pgfscope}%
\pgfpathrectangle{\pgfqpoint{1.228750in}{1.022500in}}{\pgfqpoint{13.456250in}{3.662500in}}%
\pgfusepath{clip}%
\pgfsetbuttcap%
\pgfsetmiterjoin%
\definecolor{currentfill}{rgb}{0.347059,0.458824,0.641176}%
\pgfsetfillcolor{currentfill}%
\pgfsetlinewidth{1.003750pt}%
\definecolor{currentstroke}{rgb}{1.000000,1.000000,1.000000}%
\pgfsetstrokecolor{currentstroke}%
\pgfsetdash{}{0pt}%
\pgfpathmoveto{\pgfqpoint{7.187946in}{1.022500in}}%
\pgfpathlineto{\pgfqpoint{7.700565in}{1.022500in}}%
\pgfpathlineto{\pgfqpoint{7.700565in}{1.721202in}}%
\pgfpathlineto{\pgfqpoint{7.187946in}{1.721202in}}%
\pgfpathclose%
\pgfusepath{stroke,fill}%
\end{pgfscope}%
\begin{pgfscope}%
\pgfsetbuttcap%
\pgfsetmiterjoin%
\definecolor{currentfill}{rgb}{0.347059,0.458824,0.641176}%
\pgfsetfillcolor{currentfill}%
\pgfsetlinewidth{1.003750pt}%
\definecolor{currentstroke}{rgb}{1.000000,1.000000,1.000000}%
\pgfsetstrokecolor{currentstroke}%
\pgfsetdash{}{0pt}%
\pgfpathrectangle{\pgfqpoint{1.228750in}{1.022500in}}{\pgfqpoint{13.456250in}{3.662500in}}%
\pgfusepath{clip}%
\pgfpathmoveto{\pgfqpoint{7.187946in}{1.022500in}}%
\pgfpathlineto{\pgfqpoint{7.700565in}{1.022500in}}%
\pgfpathlineto{\pgfqpoint{7.700565in}{1.721202in}}%
\pgfpathlineto{\pgfqpoint{7.187946in}{1.721202in}}%
\pgfpathclose%
\pgfusepath{clip}%
\pgfsys@defobject{currentpattern}{\pgfqpoint{0in}{0in}}{\pgfqpoint{1in}{1in}}{%
\begin{pgfscope}%
\pgfpathrectangle{\pgfqpoint{0in}{0in}}{\pgfqpoint{1in}{1in}}%
\pgfusepath{clip}%
\pgfpathmoveto{\pgfqpoint{-0.500000in}{0.500000in}}%
\pgfpathlineto{\pgfqpoint{0.500000in}{1.500000in}}%
\pgfpathmoveto{\pgfqpoint{-0.444444in}{0.444444in}}%
\pgfpathlineto{\pgfqpoint{0.555556in}{1.444444in}}%
\pgfpathmoveto{\pgfqpoint{-0.388889in}{0.388889in}}%
\pgfpathlineto{\pgfqpoint{0.611111in}{1.388889in}}%
\pgfpathmoveto{\pgfqpoint{-0.333333in}{0.333333in}}%
\pgfpathlineto{\pgfqpoint{0.666667in}{1.333333in}}%
\pgfpathmoveto{\pgfqpoint{-0.277778in}{0.277778in}}%
\pgfpathlineto{\pgfqpoint{0.722222in}{1.277778in}}%
\pgfpathmoveto{\pgfqpoint{-0.222222in}{0.222222in}}%
\pgfpathlineto{\pgfqpoint{0.777778in}{1.222222in}}%
\pgfpathmoveto{\pgfqpoint{-0.166667in}{0.166667in}}%
\pgfpathlineto{\pgfqpoint{0.833333in}{1.166667in}}%
\pgfpathmoveto{\pgfqpoint{-0.111111in}{0.111111in}}%
\pgfpathlineto{\pgfqpoint{0.888889in}{1.111111in}}%
\pgfpathmoveto{\pgfqpoint{-0.055556in}{0.055556in}}%
\pgfpathlineto{\pgfqpoint{0.944444in}{1.055556in}}%
\pgfpathmoveto{\pgfqpoint{0.000000in}{0.000000in}}%
\pgfpathlineto{\pgfqpoint{1.000000in}{1.000000in}}%
\pgfpathmoveto{\pgfqpoint{0.055556in}{-0.055556in}}%
\pgfpathlineto{\pgfqpoint{1.055556in}{0.944444in}}%
\pgfpathmoveto{\pgfqpoint{0.111111in}{-0.111111in}}%
\pgfpathlineto{\pgfqpoint{1.111111in}{0.888889in}}%
\pgfpathmoveto{\pgfqpoint{0.166667in}{-0.166667in}}%
\pgfpathlineto{\pgfqpoint{1.166667in}{0.833333in}}%
\pgfpathmoveto{\pgfqpoint{0.222222in}{-0.222222in}}%
\pgfpathlineto{\pgfqpoint{1.222222in}{0.777778in}}%
\pgfpathmoveto{\pgfqpoint{0.277778in}{-0.277778in}}%
\pgfpathlineto{\pgfqpoint{1.277778in}{0.722222in}}%
\pgfpathmoveto{\pgfqpoint{0.333333in}{-0.333333in}}%
\pgfpathlineto{\pgfqpoint{1.333333in}{0.666667in}}%
\pgfpathmoveto{\pgfqpoint{0.388889in}{-0.388889in}}%
\pgfpathlineto{\pgfqpoint{1.388889in}{0.611111in}}%
\pgfpathmoveto{\pgfqpoint{0.444444in}{-0.444444in}}%
\pgfpathlineto{\pgfqpoint{1.444444in}{0.555556in}}%
\pgfpathmoveto{\pgfqpoint{0.500000in}{-0.500000in}}%
\pgfpathlineto{\pgfqpoint{1.500000in}{0.500000in}}%
\pgfusepath{stroke}%
\end{pgfscope}%
}%
\pgfsys@transformshift{7.187946in}{1.022500in}%
\pgfsys@useobject{currentpattern}{}%
\pgfsys@transformshift{1in}{0in}%
\pgfsys@transformshift{-1in}{0in}%
\pgfsys@transformshift{0in}{1in}%
\end{pgfscope}%
\begin{pgfscope}%
\pgfpathrectangle{\pgfqpoint{1.228750in}{1.022500in}}{\pgfqpoint{13.456250in}{3.662500in}}%
\pgfusepath{clip}%
\pgfsetbuttcap%
\pgfsetmiterjoin%
\definecolor{currentfill}{rgb}{0.347059,0.458824,0.641176}%
\pgfsetfillcolor{currentfill}%
\pgfsetlinewidth{1.003750pt}%
\definecolor{currentstroke}{rgb}{1.000000,1.000000,1.000000}%
\pgfsetstrokecolor{currentstroke}%
\pgfsetdash{}{0pt}%
\pgfpathmoveto{\pgfqpoint{9.110268in}{1.022500in}}%
\pgfpathlineto{\pgfqpoint{9.622887in}{1.022500in}}%
\pgfpathlineto{\pgfqpoint{9.622887in}{1.607460in}}%
\pgfpathlineto{\pgfqpoint{9.110268in}{1.607460in}}%
\pgfpathclose%
\pgfusepath{stroke,fill}%
\end{pgfscope}%
\begin{pgfscope}%
\pgfsetbuttcap%
\pgfsetmiterjoin%
\definecolor{currentfill}{rgb}{0.347059,0.458824,0.641176}%
\pgfsetfillcolor{currentfill}%
\pgfsetlinewidth{1.003750pt}%
\definecolor{currentstroke}{rgb}{1.000000,1.000000,1.000000}%
\pgfsetstrokecolor{currentstroke}%
\pgfsetdash{}{0pt}%
\pgfpathrectangle{\pgfqpoint{1.228750in}{1.022500in}}{\pgfqpoint{13.456250in}{3.662500in}}%
\pgfusepath{clip}%
\pgfpathmoveto{\pgfqpoint{9.110268in}{1.022500in}}%
\pgfpathlineto{\pgfqpoint{9.622887in}{1.022500in}}%
\pgfpathlineto{\pgfqpoint{9.622887in}{1.607460in}}%
\pgfpathlineto{\pgfqpoint{9.110268in}{1.607460in}}%
\pgfpathclose%
\pgfusepath{clip}%
\pgfsys@defobject{currentpattern}{\pgfqpoint{0in}{0in}}{\pgfqpoint{1in}{1in}}{%
\begin{pgfscope}%
\pgfpathrectangle{\pgfqpoint{0in}{0in}}{\pgfqpoint{1in}{1in}}%
\pgfusepath{clip}%
\pgfpathmoveto{\pgfqpoint{-0.500000in}{0.500000in}}%
\pgfpathlineto{\pgfqpoint{0.500000in}{1.500000in}}%
\pgfpathmoveto{\pgfqpoint{-0.444444in}{0.444444in}}%
\pgfpathlineto{\pgfqpoint{0.555556in}{1.444444in}}%
\pgfpathmoveto{\pgfqpoint{-0.388889in}{0.388889in}}%
\pgfpathlineto{\pgfqpoint{0.611111in}{1.388889in}}%
\pgfpathmoveto{\pgfqpoint{-0.333333in}{0.333333in}}%
\pgfpathlineto{\pgfqpoint{0.666667in}{1.333333in}}%
\pgfpathmoveto{\pgfqpoint{-0.277778in}{0.277778in}}%
\pgfpathlineto{\pgfqpoint{0.722222in}{1.277778in}}%
\pgfpathmoveto{\pgfqpoint{-0.222222in}{0.222222in}}%
\pgfpathlineto{\pgfqpoint{0.777778in}{1.222222in}}%
\pgfpathmoveto{\pgfqpoint{-0.166667in}{0.166667in}}%
\pgfpathlineto{\pgfqpoint{0.833333in}{1.166667in}}%
\pgfpathmoveto{\pgfqpoint{-0.111111in}{0.111111in}}%
\pgfpathlineto{\pgfqpoint{0.888889in}{1.111111in}}%
\pgfpathmoveto{\pgfqpoint{-0.055556in}{0.055556in}}%
\pgfpathlineto{\pgfqpoint{0.944444in}{1.055556in}}%
\pgfpathmoveto{\pgfqpoint{0.000000in}{0.000000in}}%
\pgfpathlineto{\pgfqpoint{1.000000in}{1.000000in}}%
\pgfpathmoveto{\pgfqpoint{0.055556in}{-0.055556in}}%
\pgfpathlineto{\pgfqpoint{1.055556in}{0.944444in}}%
\pgfpathmoveto{\pgfqpoint{0.111111in}{-0.111111in}}%
\pgfpathlineto{\pgfqpoint{1.111111in}{0.888889in}}%
\pgfpathmoveto{\pgfqpoint{0.166667in}{-0.166667in}}%
\pgfpathlineto{\pgfqpoint{1.166667in}{0.833333in}}%
\pgfpathmoveto{\pgfqpoint{0.222222in}{-0.222222in}}%
\pgfpathlineto{\pgfqpoint{1.222222in}{0.777778in}}%
\pgfpathmoveto{\pgfqpoint{0.277778in}{-0.277778in}}%
\pgfpathlineto{\pgfqpoint{1.277778in}{0.722222in}}%
\pgfpathmoveto{\pgfqpoint{0.333333in}{-0.333333in}}%
\pgfpathlineto{\pgfqpoint{1.333333in}{0.666667in}}%
\pgfpathmoveto{\pgfqpoint{0.388889in}{-0.388889in}}%
\pgfpathlineto{\pgfqpoint{1.388889in}{0.611111in}}%
\pgfpathmoveto{\pgfqpoint{0.444444in}{-0.444444in}}%
\pgfpathlineto{\pgfqpoint{1.444444in}{0.555556in}}%
\pgfpathmoveto{\pgfqpoint{0.500000in}{-0.500000in}}%
\pgfpathlineto{\pgfqpoint{1.500000in}{0.500000in}}%
\pgfusepath{stroke}%
\end{pgfscope}%
}%
\pgfsys@transformshift{9.110268in}{1.022500in}%
\pgfsys@useobject{currentpattern}{}%
\pgfsys@transformshift{1in}{0in}%
\pgfsys@transformshift{-1in}{0in}%
\pgfsys@transformshift{0in}{1in}%
\end{pgfscope}%
\begin{pgfscope}%
\pgfpathrectangle{\pgfqpoint{1.228750in}{1.022500in}}{\pgfqpoint{13.456250in}{3.662500in}}%
\pgfusepath{clip}%
\pgfsetbuttcap%
\pgfsetmiterjoin%
\definecolor{currentfill}{rgb}{0.347059,0.458824,0.641176}%
\pgfsetfillcolor{currentfill}%
\pgfsetlinewidth{1.003750pt}%
\definecolor{currentstroke}{rgb}{1.000000,1.000000,1.000000}%
\pgfsetstrokecolor{currentstroke}%
\pgfsetdash{}{0pt}%
\pgfpathmoveto{\pgfqpoint{11.032589in}{1.022500in}}%
\pgfpathlineto{\pgfqpoint{11.545208in}{1.022500in}}%
\pgfpathlineto{\pgfqpoint{11.545208in}{1.434139in}}%
\pgfpathlineto{\pgfqpoint{11.032589in}{1.434139in}}%
\pgfpathclose%
\pgfusepath{stroke,fill}%
\end{pgfscope}%
\begin{pgfscope}%
\pgfsetbuttcap%
\pgfsetmiterjoin%
\definecolor{currentfill}{rgb}{0.347059,0.458824,0.641176}%
\pgfsetfillcolor{currentfill}%
\pgfsetlinewidth{1.003750pt}%
\definecolor{currentstroke}{rgb}{1.000000,1.000000,1.000000}%
\pgfsetstrokecolor{currentstroke}%
\pgfsetdash{}{0pt}%
\pgfpathrectangle{\pgfqpoint{1.228750in}{1.022500in}}{\pgfqpoint{13.456250in}{3.662500in}}%
\pgfusepath{clip}%
\pgfpathmoveto{\pgfqpoint{11.032589in}{1.022500in}}%
\pgfpathlineto{\pgfqpoint{11.545208in}{1.022500in}}%
\pgfpathlineto{\pgfqpoint{11.545208in}{1.434139in}}%
\pgfpathlineto{\pgfqpoint{11.032589in}{1.434139in}}%
\pgfpathclose%
\pgfusepath{clip}%
\pgfsys@defobject{currentpattern}{\pgfqpoint{0in}{0in}}{\pgfqpoint{1in}{1in}}{%
\begin{pgfscope}%
\pgfpathrectangle{\pgfqpoint{0in}{0in}}{\pgfqpoint{1in}{1in}}%
\pgfusepath{clip}%
\pgfpathmoveto{\pgfqpoint{-0.500000in}{0.500000in}}%
\pgfpathlineto{\pgfqpoint{0.500000in}{1.500000in}}%
\pgfpathmoveto{\pgfqpoint{-0.444444in}{0.444444in}}%
\pgfpathlineto{\pgfqpoint{0.555556in}{1.444444in}}%
\pgfpathmoveto{\pgfqpoint{-0.388889in}{0.388889in}}%
\pgfpathlineto{\pgfqpoint{0.611111in}{1.388889in}}%
\pgfpathmoveto{\pgfqpoint{-0.333333in}{0.333333in}}%
\pgfpathlineto{\pgfqpoint{0.666667in}{1.333333in}}%
\pgfpathmoveto{\pgfqpoint{-0.277778in}{0.277778in}}%
\pgfpathlineto{\pgfqpoint{0.722222in}{1.277778in}}%
\pgfpathmoveto{\pgfqpoint{-0.222222in}{0.222222in}}%
\pgfpathlineto{\pgfqpoint{0.777778in}{1.222222in}}%
\pgfpathmoveto{\pgfqpoint{-0.166667in}{0.166667in}}%
\pgfpathlineto{\pgfqpoint{0.833333in}{1.166667in}}%
\pgfpathmoveto{\pgfqpoint{-0.111111in}{0.111111in}}%
\pgfpathlineto{\pgfqpoint{0.888889in}{1.111111in}}%
\pgfpathmoveto{\pgfqpoint{-0.055556in}{0.055556in}}%
\pgfpathlineto{\pgfqpoint{0.944444in}{1.055556in}}%
\pgfpathmoveto{\pgfqpoint{0.000000in}{0.000000in}}%
\pgfpathlineto{\pgfqpoint{1.000000in}{1.000000in}}%
\pgfpathmoveto{\pgfqpoint{0.055556in}{-0.055556in}}%
\pgfpathlineto{\pgfqpoint{1.055556in}{0.944444in}}%
\pgfpathmoveto{\pgfqpoint{0.111111in}{-0.111111in}}%
\pgfpathlineto{\pgfqpoint{1.111111in}{0.888889in}}%
\pgfpathmoveto{\pgfqpoint{0.166667in}{-0.166667in}}%
\pgfpathlineto{\pgfqpoint{1.166667in}{0.833333in}}%
\pgfpathmoveto{\pgfqpoint{0.222222in}{-0.222222in}}%
\pgfpathlineto{\pgfqpoint{1.222222in}{0.777778in}}%
\pgfpathmoveto{\pgfqpoint{0.277778in}{-0.277778in}}%
\pgfpathlineto{\pgfqpoint{1.277778in}{0.722222in}}%
\pgfpathmoveto{\pgfqpoint{0.333333in}{-0.333333in}}%
\pgfpathlineto{\pgfqpoint{1.333333in}{0.666667in}}%
\pgfpathmoveto{\pgfqpoint{0.388889in}{-0.388889in}}%
\pgfpathlineto{\pgfqpoint{1.388889in}{0.611111in}}%
\pgfpathmoveto{\pgfqpoint{0.444444in}{-0.444444in}}%
\pgfpathlineto{\pgfqpoint{1.444444in}{0.555556in}}%
\pgfpathmoveto{\pgfqpoint{0.500000in}{-0.500000in}}%
\pgfpathlineto{\pgfqpoint{1.500000in}{0.500000in}}%
\pgfusepath{stroke}%
\end{pgfscope}%
}%
\pgfsys@transformshift{11.032589in}{1.022500in}%
\pgfsys@useobject{currentpattern}{}%
\pgfsys@transformshift{1in}{0in}%
\pgfsys@transformshift{-1in}{0in}%
\pgfsys@transformshift{0in}{1in}%
\end{pgfscope}%
\begin{pgfscope}%
\pgfpathrectangle{\pgfqpoint{1.228750in}{1.022500in}}{\pgfqpoint{13.456250in}{3.662500in}}%
\pgfusepath{clip}%
\pgfsetbuttcap%
\pgfsetmiterjoin%
\definecolor{currentfill}{rgb}{0.347059,0.458824,0.641176}%
\pgfsetfillcolor{currentfill}%
\pgfsetlinewidth{1.003750pt}%
\definecolor{currentstroke}{rgb}{1.000000,1.000000,1.000000}%
\pgfsetstrokecolor{currentstroke}%
\pgfsetdash{}{0pt}%
\pgfpathmoveto{\pgfqpoint{12.954911in}{1.022500in}}%
\pgfpathlineto{\pgfqpoint{13.467530in}{1.022500in}}%
\pgfpathlineto{\pgfqpoint{13.467530in}{1.434139in}}%
\pgfpathlineto{\pgfqpoint{12.954911in}{1.434139in}}%
\pgfpathclose%
\pgfusepath{stroke,fill}%
\end{pgfscope}%
\begin{pgfscope}%
\pgfsetbuttcap%
\pgfsetmiterjoin%
\definecolor{currentfill}{rgb}{0.347059,0.458824,0.641176}%
\pgfsetfillcolor{currentfill}%
\pgfsetlinewidth{1.003750pt}%
\definecolor{currentstroke}{rgb}{1.000000,1.000000,1.000000}%
\pgfsetstrokecolor{currentstroke}%
\pgfsetdash{}{0pt}%
\pgfpathrectangle{\pgfqpoint{1.228750in}{1.022500in}}{\pgfqpoint{13.456250in}{3.662500in}}%
\pgfusepath{clip}%
\pgfpathmoveto{\pgfqpoint{12.954911in}{1.022500in}}%
\pgfpathlineto{\pgfqpoint{13.467530in}{1.022500in}}%
\pgfpathlineto{\pgfqpoint{13.467530in}{1.434139in}}%
\pgfpathlineto{\pgfqpoint{12.954911in}{1.434139in}}%
\pgfpathclose%
\pgfusepath{clip}%
\pgfsys@defobject{currentpattern}{\pgfqpoint{0in}{0in}}{\pgfqpoint{1in}{1in}}{%
\begin{pgfscope}%
\pgfpathrectangle{\pgfqpoint{0in}{0in}}{\pgfqpoint{1in}{1in}}%
\pgfusepath{clip}%
\pgfpathmoveto{\pgfqpoint{-0.500000in}{0.500000in}}%
\pgfpathlineto{\pgfqpoint{0.500000in}{1.500000in}}%
\pgfpathmoveto{\pgfqpoint{-0.444444in}{0.444444in}}%
\pgfpathlineto{\pgfqpoint{0.555556in}{1.444444in}}%
\pgfpathmoveto{\pgfqpoint{-0.388889in}{0.388889in}}%
\pgfpathlineto{\pgfqpoint{0.611111in}{1.388889in}}%
\pgfpathmoveto{\pgfqpoint{-0.333333in}{0.333333in}}%
\pgfpathlineto{\pgfqpoint{0.666667in}{1.333333in}}%
\pgfpathmoveto{\pgfqpoint{-0.277778in}{0.277778in}}%
\pgfpathlineto{\pgfqpoint{0.722222in}{1.277778in}}%
\pgfpathmoveto{\pgfqpoint{-0.222222in}{0.222222in}}%
\pgfpathlineto{\pgfqpoint{0.777778in}{1.222222in}}%
\pgfpathmoveto{\pgfqpoint{-0.166667in}{0.166667in}}%
\pgfpathlineto{\pgfqpoint{0.833333in}{1.166667in}}%
\pgfpathmoveto{\pgfqpoint{-0.111111in}{0.111111in}}%
\pgfpathlineto{\pgfqpoint{0.888889in}{1.111111in}}%
\pgfpathmoveto{\pgfqpoint{-0.055556in}{0.055556in}}%
\pgfpathlineto{\pgfqpoint{0.944444in}{1.055556in}}%
\pgfpathmoveto{\pgfqpoint{0.000000in}{0.000000in}}%
\pgfpathlineto{\pgfqpoint{1.000000in}{1.000000in}}%
\pgfpathmoveto{\pgfqpoint{0.055556in}{-0.055556in}}%
\pgfpathlineto{\pgfqpoint{1.055556in}{0.944444in}}%
\pgfpathmoveto{\pgfqpoint{0.111111in}{-0.111111in}}%
\pgfpathlineto{\pgfqpoint{1.111111in}{0.888889in}}%
\pgfpathmoveto{\pgfqpoint{0.166667in}{-0.166667in}}%
\pgfpathlineto{\pgfqpoint{1.166667in}{0.833333in}}%
\pgfpathmoveto{\pgfqpoint{0.222222in}{-0.222222in}}%
\pgfpathlineto{\pgfqpoint{1.222222in}{0.777778in}}%
\pgfpathmoveto{\pgfqpoint{0.277778in}{-0.277778in}}%
\pgfpathlineto{\pgfqpoint{1.277778in}{0.722222in}}%
\pgfpathmoveto{\pgfqpoint{0.333333in}{-0.333333in}}%
\pgfpathlineto{\pgfqpoint{1.333333in}{0.666667in}}%
\pgfpathmoveto{\pgfqpoint{0.388889in}{-0.388889in}}%
\pgfpathlineto{\pgfqpoint{1.388889in}{0.611111in}}%
\pgfpathmoveto{\pgfqpoint{0.444444in}{-0.444444in}}%
\pgfpathlineto{\pgfqpoint{1.444444in}{0.555556in}}%
\pgfpathmoveto{\pgfqpoint{0.500000in}{-0.500000in}}%
\pgfpathlineto{\pgfqpoint{1.500000in}{0.500000in}}%
\pgfusepath{stroke}%
\end{pgfscope}%
}%
\pgfsys@transformshift{12.954911in}{1.022500in}%
\pgfsys@useobject{currentpattern}{}%
\pgfsys@transformshift{1in}{0in}%
\pgfsys@transformshift{-1in}{0in}%
\pgfsys@transformshift{0in}{1in}%
\end{pgfscope}%
\begin{pgfscope}%
\pgfpathrectangle{\pgfqpoint{1.228750in}{1.022500in}}{\pgfqpoint{13.456250in}{3.662500in}}%
\pgfusepath{clip}%
\pgfsetbuttcap%
\pgfsetmiterjoin%
\definecolor{currentfill}{rgb}{0.798529,0.536765,0.389706}%
\pgfsetfillcolor{currentfill}%
\pgfsetlinewidth{1.003750pt}%
\definecolor{currentstroke}{rgb}{1.000000,1.000000,1.000000}%
\pgfsetstrokecolor{currentstroke}%
\pgfsetdash{}{0pt}%
\pgfpathmoveto{\pgfqpoint{1.933601in}{1.022500in}}%
\pgfpathlineto{\pgfqpoint{2.446220in}{1.022500in}}%
\pgfpathlineto{\pgfqpoint{2.446220in}{1.342062in}}%
\pgfpathlineto{\pgfqpoint{1.933601in}{1.342062in}}%
\pgfpathclose%
\pgfusepath{stroke,fill}%
\end{pgfscope}%
\begin{pgfscope}%
\pgfsetbuttcap%
\pgfsetmiterjoin%
\definecolor{currentfill}{rgb}{0.798529,0.536765,0.389706}%
\pgfsetfillcolor{currentfill}%
\pgfsetlinewidth{1.003750pt}%
\definecolor{currentstroke}{rgb}{1.000000,1.000000,1.000000}%
\pgfsetstrokecolor{currentstroke}%
\pgfsetdash{}{0pt}%
\pgfpathrectangle{\pgfqpoint{1.228750in}{1.022500in}}{\pgfqpoint{13.456250in}{3.662500in}}%
\pgfusepath{clip}%
\pgfpathmoveto{\pgfqpoint{1.933601in}{1.022500in}}%
\pgfpathlineto{\pgfqpoint{2.446220in}{1.022500in}}%
\pgfpathlineto{\pgfqpoint{2.446220in}{1.342062in}}%
\pgfpathlineto{\pgfqpoint{1.933601in}{1.342062in}}%
\pgfpathclose%
\pgfusepath{clip}%
\pgfsys@defobject{currentpattern}{\pgfqpoint{0in}{0in}}{\pgfqpoint{1in}{1in}}{%
\begin{pgfscope}%
\pgfpathrectangle{\pgfqpoint{0in}{0in}}{\pgfqpoint{1in}{1in}}%
\pgfusepath{clip}%
\pgfpathmoveto{\pgfqpoint{-0.500000in}{0.500000in}}%
\pgfpathlineto{\pgfqpoint{0.500000in}{-0.500000in}}%
\pgfpathmoveto{\pgfqpoint{-0.444444in}{0.555556in}}%
\pgfpathlineto{\pgfqpoint{0.555556in}{-0.444444in}}%
\pgfpathmoveto{\pgfqpoint{-0.388889in}{0.611111in}}%
\pgfpathlineto{\pgfqpoint{0.611111in}{-0.388889in}}%
\pgfpathmoveto{\pgfqpoint{-0.333333in}{0.666667in}}%
\pgfpathlineto{\pgfqpoint{0.666667in}{-0.333333in}}%
\pgfpathmoveto{\pgfqpoint{-0.277778in}{0.722222in}}%
\pgfpathlineto{\pgfqpoint{0.722222in}{-0.277778in}}%
\pgfpathmoveto{\pgfqpoint{-0.222222in}{0.777778in}}%
\pgfpathlineto{\pgfqpoint{0.777778in}{-0.222222in}}%
\pgfpathmoveto{\pgfqpoint{-0.166667in}{0.833333in}}%
\pgfpathlineto{\pgfqpoint{0.833333in}{-0.166667in}}%
\pgfpathmoveto{\pgfqpoint{-0.111111in}{0.888889in}}%
\pgfpathlineto{\pgfqpoint{0.888889in}{-0.111111in}}%
\pgfpathmoveto{\pgfqpoint{-0.055556in}{0.944444in}}%
\pgfpathlineto{\pgfqpoint{0.944444in}{-0.055556in}}%
\pgfpathmoveto{\pgfqpoint{0.000000in}{1.000000in}}%
\pgfpathlineto{\pgfqpoint{1.000000in}{0.000000in}}%
\pgfpathmoveto{\pgfqpoint{0.055556in}{1.055556in}}%
\pgfpathlineto{\pgfqpoint{1.055556in}{0.055556in}}%
\pgfpathmoveto{\pgfqpoint{0.111111in}{1.111111in}}%
\pgfpathlineto{\pgfqpoint{1.111111in}{0.111111in}}%
\pgfpathmoveto{\pgfqpoint{0.166667in}{1.166667in}}%
\pgfpathlineto{\pgfqpoint{1.166667in}{0.166667in}}%
\pgfpathmoveto{\pgfqpoint{0.222222in}{1.222222in}}%
\pgfpathlineto{\pgfqpoint{1.222222in}{0.222222in}}%
\pgfpathmoveto{\pgfqpoint{0.277778in}{1.277778in}}%
\pgfpathlineto{\pgfqpoint{1.277778in}{0.277778in}}%
\pgfpathmoveto{\pgfqpoint{0.333333in}{1.333333in}}%
\pgfpathlineto{\pgfqpoint{1.333333in}{0.333333in}}%
\pgfpathmoveto{\pgfqpoint{0.388889in}{1.388889in}}%
\pgfpathlineto{\pgfqpoint{1.388889in}{0.388889in}}%
\pgfpathmoveto{\pgfqpoint{0.444444in}{1.444444in}}%
\pgfpathlineto{\pgfqpoint{1.444444in}{0.444444in}}%
\pgfpathmoveto{\pgfqpoint{0.500000in}{1.500000in}}%
\pgfpathlineto{\pgfqpoint{1.500000in}{0.500000in}}%
\pgfusepath{stroke}%
\end{pgfscope}%
}%
\pgfsys@transformshift{1.933601in}{1.022500in}%
\pgfsys@useobject{currentpattern}{}%
\pgfsys@transformshift{1in}{0in}%
\pgfsys@transformshift{-1in}{0in}%
\pgfsys@transformshift{0in}{1in}%
\end{pgfscope}%
\begin{pgfscope}%
\pgfpathrectangle{\pgfqpoint{1.228750in}{1.022500in}}{\pgfqpoint{13.456250in}{3.662500in}}%
\pgfusepath{clip}%
\pgfsetbuttcap%
\pgfsetmiterjoin%
\definecolor{currentfill}{rgb}{0.798529,0.536765,0.389706}%
\pgfsetfillcolor{currentfill}%
\pgfsetlinewidth{1.003750pt}%
\definecolor{currentstroke}{rgb}{1.000000,1.000000,1.000000}%
\pgfsetstrokecolor{currentstroke}%
\pgfsetdash{}{0pt}%
\pgfpathmoveto{\pgfqpoint{3.855923in}{1.022500in}}%
\pgfpathlineto{\pgfqpoint{4.368542in}{1.022500in}}%
\pgfpathlineto{\pgfqpoint{4.368542in}{1.331229in}}%
\pgfpathlineto{\pgfqpoint{3.855923in}{1.331229in}}%
\pgfpathclose%
\pgfusepath{stroke,fill}%
\end{pgfscope}%
\begin{pgfscope}%
\pgfsetbuttcap%
\pgfsetmiterjoin%
\definecolor{currentfill}{rgb}{0.798529,0.536765,0.389706}%
\pgfsetfillcolor{currentfill}%
\pgfsetlinewidth{1.003750pt}%
\definecolor{currentstroke}{rgb}{1.000000,1.000000,1.000000}%
\pgfsetstrokecolor{currentstroke}%
\pgfsetdash{}{0pt}%
\pgfpathrectangle{\pgfqpoint{1.228750in}{1.022500in}}{\pgfqpoint{13.456250in}{3.662500in}}%
\pgfusepath{clip}%
\pgfpathmoveto{\pgfqpoint{3.855923in}{1.022500in}}%
\pgfpathlineto{\pgfqpoint{4.368542in}{1.022500in}}%
\pgfpathlineto{\pgfqpoint{4.368542in}{1.331229in}}%
\pgfpathlineto{\pgfqpoint{3.855923in}{1.331229in}}%
\pgfpathclose%
\pgfusepath{clip}%
\pgfsys@defobject{currentpattern}{\pgfqpoint{0in}{0in}}{\pgfqpoint{1in}{1in}}{%
\begin{pgfscope}%
\pgfpathrectangle{\pgfqpoint{0in}{0in}}{\pgfqpoint{1in}{1in}}%
\pgfusepath{clip}%
\pgfpathmoveto{\pgfqpoint{-0.500000in}{0.500000in}}%
\pgfpathlineto{\pgfqpoint{0.500000in}{-0.500000in}}%
\pgfpathmoveto{\pgfqpoint{-0.444444in}{0.555556in}}%
\pgfpathlineto{\pgfqpoint{0.555556in}{-0.444444in}}%
\pgfpathmoveto{\pgfqpoint{-0.388889in}{0.611111in}}%
\pgfpathlineto{\pgfqpoint{0.611111in}{-0.388889in}}%
\pgfpathmoveto{\pgfqpoint{-0.333333in}{0.666667in}}%
\pgfpathlineto{\pgfqpoint{0.666667in}{-0.333333in}}%
\pgfpathmoveto{\pgfqpoint{-0.277778in}{0.722222in}}%
\pgfpathlineto{\pgfqpoint{0.722222in}{-0.277778in}}%
\pgfpathmoveto{\pgfqpoint{-0.222222in}{0.777778in}}%
\pgfpathlineto{\pgfqpoint{0.777778in}{-0.222222in}}%
\pgfpathmoveto{\pgfqpoint{-0.166667in}{0.833333in}}%
\pgfpathlineto{\pgfqpoint{0.833333in}{-0.166667in}}%
\pgfpathmoveto{\pgfqpoint{-0.111111in}{0.888889in}}%
\pgfpathlineto{\pgfqpoint{0.888889in}{-0.111111in}}%
\pgfpathmoveto{\pgfqpoint{-0.055556in}{0.944444in}}%
\pgfpathlineto{\pgfqpoint{0.944444in}{-0.055556in}}%
\pgfpathmoveto{\pgfqpoint{0.000000in}{1.000000in}}%
\pgfpathlineto{\pgfqpoint{1.000000in}{0.000000in}}%
\pgfpathmoveto{\pgfqpoint{0.055556in}{1.055556in}}%
\pgfpathlineto{\pgfqpoint{1.055556in}{0.055556in}}%
\pgfpathmoveto{\pgfqpoint{0.111111in}{1.111111in}}%
\pgfpathlineto{\pgfqpoint{1.111111in}{0.111111in}}%
\pgfpathmoveto{\pgfqpoint{0.166667in}{1.166667in}}%
\pgfpathlineto{\pgfqpoint{1.166667in}{0.166667in}}%
\pgfpathmoveto{\pgfqpoint{0.222222in}{1.222222in}}%
\pgfpathlineto{\pgfqpoint{1.222222in}{0.222222in}}%
\pgfpathmoveto{\pgfqpoint{0.277778in}{1.277778in}}%
\pgfpathlineto{\pgfqpoint{1.277778in}{0.277778in}}%
\pgfpathmoveto{\pgfqpoint{0.333333in}{1.333333in}}%
\pgfpathlineto{\pgfqpoint{1.333333in}{0.333333in}}%
\pgfpathmoveto{\pgfqpoint{0.388889in}{1.388889in}}%
\pgfpathlineto{\pgfqpoint{1.388889in}{0.388889in}}%
\pgfpathmoveto{\pgfqpoint{0.444444in}{1.444444in}}%
\pgfpathlineto{\pgfqpoint{1.444444in}{0.444444in}}%
\pgfpathmoveto{\pgfqpoint{0.500000in}{1.500000in}}%
\pgfpathlineto{\pgfqpoint{1.500000in}{0.500000in}}%
\pgfusepath{stroke}%
\end{pgfscope}%
}%
\pgfsys@transformshift{3.855923in}{1.022500in}%
\pgfsys@useobject{currentpattern}{}%
\pgfsys@transformshift{1in}{0in}%
\pgfsys@transformshift{-1in}{0in}%
\pgfsys@transformshift{0in}{1in}%
\end{pgfscope}%
\begin{pgfscope}%
\pgfpathrectangle{\pgfqpoint{1.228750in}{1.022500in}}{\pgfqpoint{13.456250in}{3.662500in}}%
\pgfusepath{clip}%
\pgfsetbuttcap%
\pgfsetmiterjoin%
\definecolor{currentfill}{rgb}{0.798529,0.536765,0.389706}%
\pgfsetfillcolor{currentfill}%
\pgfsetlinewidth{1.003750pt}%
\definecolor{currentstroke}{rgb}{1.000000,1.000000,1.000000}%
\pgfsetstrokecolor{currentstroke}%
\pgfsetdash{}{0pt}%
\pgfpathmoveto{\pgfqpoint{5.778244in}{1.022500in}}%
\pgfpathlineto{\pgfqpoint{6.290863in}{1.022500in}}%
\pgfpathlineto{\pgfqpoint{6.290863in}{1.374559in}}%
\pgfpathlineto{\pgfqpoint{5.778244in}{1.374559in}}%
\pgfpathclose%
\pgfusepath{stroke,fill}%
\end{pgfscope}%
\begin{pgfscope}%
\pgfsetbuttcap%
\pgfsetmiterjoin%
\definecolor{currentfill}{rgb}{0.798529,0.536765,0.389706}%
\pgfsetfillcolor{currentfill}%
\pgfsetlinewidth{1.003750pt}%
\definecolor{currentstroke}{rgb}{1.000000,1.000000,1.000000}%
\pgfsetstrokecolor{currentstroke}%
\pgfsetdash{}{0pt}%
\pgfpathrectangle{\pgfqpoint{1.228750in}{1.022500in}}{\pgfqpoint{13.456250in}{3.662500in}}%
\pgfusepath{clip}%
\pgfpathmoveto{\pgfqpoint{5.778244in}{1.022500in}}%
\pgfpathlineto{\pgfqpoint{6.290863in}{1.022500in}}%
\pgfpathlineto{\pgfqpoint{6.290863in}{1.374559in}}%
\pgfpathlineto{\pgfqpoint{5.778244in}{1.374559in}}%
\pgfpathclose%
\pgfusepath{clip}%
\pgfsys@defobject{currentpattern}{\pgfqpoint{0in}{0in}}{\pgfqpoint{1in}{1in}}{%
\begin{pgfscope}%
\pgfpathrectangle{\pgfqpoint{0in}{0in}}{\pgfqpoint{1in}{1in}}%
\pgfusepath{clip}%
\pgfpathmoveto{\pgfqpoint{-0.500000in}{0.500000in}}%
\pgfpathlineto{\pgfqpoint{0.500000in}{-0.500000in}}%
\pgfpathmoveto{\pgfqpoint{-0.444444in}{0.555556in}}%
\pgfpathlineto{\pgfqpoint{0.555556in}{-0.444444in}}%
\pgfpathmoveto{\pgfqpoint{-0.388889in}{0.611111in}}%
\pgfpathlineto{\pgfqpoint{0.611111in}{-0.388889in}}%
\pgfpathmoveto{\pgfqpoint{-0.333333in}{0.666667in}}%
\pgfpathlineto{\pgfqpoint{0.666667in}{-0.333333in}}%
\pgfpathmoveto{\pgfqpoint{-0.277778in}{0.722222in}}%
\pgfpathlineto{\pgfqpoint{0.722222in}{-0.277778in}}%
\pgfpathmoveto{\pgfqpoint{-0.222222in}{0.777778in}}%
\pgfpathlineto{\pgfqpoint{0.777778in}{-0.222222in}}%
\pgfpathmoveto{\pgfqpoint{-0.166667in}{0.833333in}}%
\pgfpathlineto{\pgfqpoint{0.833333in}{-0.166667in}}%
\pgfpathmoveto{\pgfqpoint{-0.111111in}{0.888889in}}%
\pgfpathlineto{\pgfqpoint{0.888889in}{-0.111111in}}%
\pgfpathmoveto{\pgfqpoint{-0.055556in}{0.944444in}}%
\pgfpathlineto{\pgfqpoint{0.944444in}{-0.055556in}}%
\pgfpathmoveto{\pgfqpoint{0.000000in}{1.000000in}}%
\pgfpathlineto{\pgfqpoint{1.000000in}{0.000000in}}%
\pgfpathmoveto{\pgfqpoint{0.055556in}{1.055556in}}%
\pgfpathlineto{\pgfqpoint{1.055556in}{0.055556in}}%
\pgfpathmoveto{\pgfqpoint{0.111111in}{1.111111in}}%
\pgfpathlineto{\pgfqpoint{1.111111in}{0.111111in}}%
\pgfpathmoveto{\pgfqpoint{0.166667in}{1.166667in}}%
\pgfpathlineto{\pgfqpoint{1.166667in}{0.166667in}}%
\pgfpathmoveto{\pgfqpoint{0.222222in}{1.222222in}}%
\pgfpathlineto{\pgfqpoint{1.222222in}{0.222222in}}%
\pgfpathmoveto{\pgfqpoint{0.277778in}{1.277778in}}%
\pgfpathlineto{\pgfqpoint{1.277778in}{0.277778in}}%
\pgfpathmoveto{\pgfqpoint{0.333333in}{1.333333in}}%
\pgfpathlineto{\pgfqpoint{1.333333in}{0.333333in}}%
\pgfpathmoveto{\pgfqpoint{0.388889in}{1.388889in}}%
\pgfpathlineto{\pgfqpoint{1.388889in}{0.388889in}}%
\pgfpathmoveto{\pgfqpoint{0.444444in}{1.444444in}}%
\pgfpathlineto{\pgfqpoint{1.444444in}{0.444444in}}%
\pgfpathmoveto{\pgfqpoint{0.500000in}{1.500000in}}%
\pgfpathlineto{\pgfqpoint{1.500000in}{0.500000in}}%
\pgfusepath{stroke}%
\end{pgfscope}%
}%
\pgfsys@transformshift{5.778244in}{1.022500in}%
\pgfsys@useobject{currentpattern}{}%
\pgfsys@transformshift{1in}{0in}%
\pgfsys@transformshift{-1in}{0in}%
\pgfsys@transformshift{0in}{1in}%
\end{pgfscope}%
\begin{pgfscope}%
\pgfpathrectangle{\pgfqpoint{1.228750in}{1.022500in}}{\pgfqpoint{13.456250in}{3.662500in}}%
\pgfusepath{clip}%
\pgfsetbuttcap%
\pgfsetmiterjoin%
\definecolor{currentfill}{rgb}{0.798529,0.536765,0.389706}%
\pgfsetfillcolor{currentfill}%
\pgfsetlinewidth{1.003750pt}%
\definecolor{currentstroke}{rgb}{1.000000,1.000000,1.000000}%
\pgfsetstrokecolor{currentstroke}%
\pgfsetdash{}{0pt}%
\pgfpathmoveto{\pgfqpoint{7.700565in}{1.022500in}}%
\pgfpathlineto{\pgfqpoint{8.213185in}{1.022500in}}%
\pgfpathlineto{\pgfqpoint{8.213185in}{1.369143in}}%
\pgfpathlineto{\pgfqpoint{7.700565in}{1.369143in}}%
\pgfpathclose%
\pgfusepath{stroke,fill}%
\end{pgfscope}%
\begin{pgfscope}%
\pgfsetbuttcap%
\pgfsetmiterjoin%
\definecolor{currentfill}{rgb}{0.798529,0.536765,0.389706}%
\pgfsetfillcolor{currentfill}%
\pgfsetlinewidth{1.003750pt}%
\definecolor{currentstroke}{rgb}{1.000000,1.000000,1.000000}%
\pgfsetstrokecolor{currentstroke}%
\pgfsetdash{}{0pt}%
\pgfpathrectangle{\pgfqpoint{1.228750in}{1.022500in}}{\pgfqpoint{13.456250in}{3.662500in}}%
\pgfusepath{clip}%
\pgfpathmoveto{\pgfqpoint{7.700565in}{1.022500in}}%
\pgfpathlineto{\pgfqpoint{8.213185in}{1.022500in}}%
\pgfpathlineto{\pgfqpoint{8.213185in}{1.369143in}}%
\pgfpathlineto{\pgfqpoint{7.700565in}{1.369143in}}%
\pgfpathclose%
\pgfusepath{clip}%
\pgfsys@defobject{currentpattern}{\pgfqpoint{0in}{0in}}{\pgfqpoint{1in}{1in}}{%
\begin{pgfscope}%
\pgfpathrectangle{\pgfqpoint{0in}{0in}}{\pgfqpoint{1in}{1in}}%
\pgfusepath{clip}%
\pgfpathmoveto{\pgfqpoint{-0.500000in}{0.500000in}}%
\pgfpathlineto{\pgfqpoint{0.500000in}{-0.500000in}}%
\pgfpathmoveto{\pgfqpoint{-0.444444in}{0.555556in}}%
\pgfpathlineto{\pgfqpoint{0.555556in}{-0.444444in}}%
\pgfpathmoveto{\pgfqpoint{-0.388889in}{0.611111in}}%
\pgfpathlineto{\pgfqpoint{0.611111in}{-0.388889in}}%
\pgfpathmoveto{\pgfqpoint{-0.333333in}{0.666667in}}%
\pgfpathlineto{\pgfqpoint{0.666667in}{-0.333333in}}%
\pgfpathmoveto{\pgfqpoint{-0.277778in}{0.722222in}}%
\pgfpathlineto{\pgfqpoint{0.722222in}{-0.277778in}}%
\pgfpathmoveto{\pgfqpoint{-0.222222in}{0.777778in}}%
\pgfpathlineto{\pgfqpoint{0.777778in}{-0.222222in}}%
\pgfpathmoveto{\pgfqpoint{-0.166667in}{0.833333in}}%
\pgfpathlineto{\pgfqpoint{0.833333in}{-0.166667in}}%
\pgfpathmoveto{\pgfqpoint{-0.111111in}{0.888889in}}%
\pgfpathlineto{\pgfqpoint{0.888889in}{-0.111111in}}%
\pgfpathmoveto{\pgfqpoint{-0.055556in}{0.944444in}}%
\pgfpathlineto{\pgfqpoint{0.944444in}{-0.055556in}}%
\pgfpathmoveto{\pgfqpoint{0.000000in}{1.000000in}}%
\pgfpathlineto{\pgfqpoint{1.000000in}{0.000000in}}%
\pgfpathmoveto{\pgfqpoint{0.055556in}{1.055556in}}%
\pgfpathlineto{\pgfqpoint{1.055556in}{0.055556in}}%
\pgfpathmoveto{\pgfqpoint{0.111111in}{1.111111in}}%
\pgfpathlineto{\pgfqpoint{1.111111in}{0.111111in}}%
\pgfpathmoveto{\pgfqpoint{0.166667in}{1.166667in}}%
\pgfpathlineto{\pgfqpoint{1.166667in}{0.166667in}}%
\pgfpathmoveto{\pgfqpoint{0.222222in}{1.222222in}}%
\pgfpathlineto{\pgfqpoint{1.222222in}{0.222222in}}%
\pgfpathmoveto{\pgfqpoint{0.277778in}{1.277778in}}%
\pgfpathlineto{\pgfqpoint{1.277778in}{0.277778in}}%
\pgfpathmoveto{\pgfqpoint{0.333333in}{1.333333in}}%
\pgfpathlineto{\pgfqpoint{1.333333in}{0.333333in}}%
\pgfpathmoveto{\pgfqpoint{0.388889in}{1.388889in}}%
\pgfpathlineto{\pgfqpoint{1.388889in}{0.388889in}}%
\pgfpathmoveto{\pgfqpoint{0.444444in}{1.444444in}}%
\pgfpathlineto{\pgfqpoint{1.444444in}{0.444444in}}%
\pgfpathmoveto{\pgfqpoint{0.500000in}{1.500000in}}%
\pgfpathlineto{\pgfqpoint{1.500000in}{0.500000in}}%
\pgfusepath{stroke}%
\end{pgfscope}%
}%
\pgfsys@transformshift{7.700565in}{1.022500in}%
\pgfsys@useobject{currentpattern}{}%
\pgfsys@transformshift{1in}{0in}%
\pgfsys@transformshift{-1in}{0in}%
\pgfsys@transformshift{0in}{1in}%
\end{pgfscope}%
\begin{pgfscope}%
\pgfpathrectangle{\pgfqpoint{1.228750in}{1.022500in}}{\pgfqpoint{13.456250in}{3.662500in}}%
\pgfusepath{clip}%
\pgfsetbuttcap%
\pgfsetmiterjoin%
\definecolor{currentfill}{rgb}{0.798529,0.536765,0.389706}%
\pgfsetfillcolor{currentfill}%
\pgfsetlinewidth{1.003750pt}%
\definecolor{currentstroke}{rgb}{1.000000,1.000000,1.000000}%
\pgfsetstrokecolor{currentstroke}%
\pgfsetdash{}{0pt}%
\pgfpathmoveto{\pgfqpoint{9.622887in}{1.022500in}}%
\pgfpathlineto{\pgfqpoint{10.135506in}{1.022500in}}%
\pgfpathlineto{\pgfqpoint{10.135506in}{1.336645in}}%
\pgfpathlineto{\pgfqpoint{9.622887in}{1.336645in}}%
\pgfpathclose%
\pgfusepath{stroke,fill}%
\end{pgfscope}%
\begin{pgfscope}%
\pgfsetbuttcap%
\pgfsetmiterjoin%
\definecolor{currentfill}{rgb}{0.798529,0.536765,0.389706}%
\pgfsetfillcolor{currentfill}%
\pgfsetlinewidth{1.003750pt}%
\definecolor{currentstroke}{rgb}{1.000000,1.000000,1.000000}%
\pgfsetstrokecolor{currentstroke}%
\pgfsetdash{}{0pt}%
\pgfpathrectangle{\pgfqpoint{1.228750in}{1.022500in}}{\pgfqpoint{13.456250in}{3.662500in}}%
\pgfusepath{clip}%
\pgfpathmoveto{\pgfqpoint{9.622887in}{1.022500in}}%
\pgfpathlineto{\pgfqpoint{10.135506in}{1.022500in}}%
\pgfpathlineto{\pgfqpoint{10.135506in}{1.336645in}}%
\pgfpathlineto{\pgfqpoint{9.622887in}{1.336645in}}%
\pgfpathclose%
\pgfusepath{clip}%
\pgfsys@defobject{currentpattern}{\pgfqpoint{0in}{0in}}{\pgfqpoint{1in}{1in}}{%
\begin{pgfscope}%
\pgfpathrectangle{\pgfqpoint{0in}{0in}}{\pgfqpoint{1in}{1in}}%
\pgfusepath{clip}%
\pgfpathmoveto{\pgfqpoint{-0.500000in}{0.500000in}}%
\pgfpathlineto{\pgfqpoint{0.500000in}{-0.500000in}}%
\pgfpathmoveto{\pgfqpoint{-0.444444in}{0.555556in}}%
\pgfpathlineto{\pgfqpoint{0.555556in}{-0.444444in}}%
\pgfpathmoveto{\pgfqpoint{-0.388889in}{0.611111in}}%
\pgfpathlineto{\pgfqpoint{0.611111in}{-0.388889in}}%
\pgfpathmoveto{\pgfqpoint{-0.333333in}{0.666667in}}%
\pgfpathlineto{\pgfqpoint{0.666667in}{-0.333333in}}%
\pgfpathmoveto{\pgfqpoint{-0.277778in}{0.722222in}}%
\pgfpathlineto{\pgfqpoint{0.722222in}{-0.277778in}}%
\pgfpathmoveto{\pgfqpoint{-0.222222in}{0.777778in}}%
\pgfpathlineto{\pgfqpoint{0.777778in}{-0.222222in}}%
\pgfpathmoveto{\pgfqpoint{-0.166667in}{0.833333in}}%
\pgfpathlineto{\pgfqpoint{0.833333in}{-0.166667in}}%
\pgfpathmoveto{\pgfqpoint{-0.111111in}{0.888889in}}%
\pgfpathlineto{\pgfqpoint{0.888889in}{-0.111111in}}%
\pgfpathmoveto{\pgfqpoint{-0.055556in}{0.944444in}}%
\pgfpathlineto{\pgfqpoint{0.944444in}{-0.055556in}}%
\pgfpathmoveto{\pgfqpoint{0.000000in}{1.000000in}}%
\pgfpathlineto{\pgfqpoint{1.000000in}{0.000000in}}%
\pgfpathmoveto{\pgfqpoint{0.055556in}{1.055556in}}%
\pgfpathlineto{\pgfqpoint{1.055556in}{0.055556in}}%
\pgfpathmoveto{\pgfqpoint{0.111111in}{1.111111in}}%
\pgfpathlineto{\pgfqpoint{1.111111in}{0.111111in}}%
\pgfpathmoveto{\pgfqpoint{0.166667in}{1.166667in}}%
\pgfpathlineto{\pgfqpoint{1.166667in}{0.166667in}}%
\pgfpathmoveto{\pgfqpoint{0.222222in}{1.222222in}}%
\pgfpathlineto{\pgfqpoint{1.222222in}{0.222222in}}%
\pgfpathmoveto{\pgfqpoint{0.277778in}{1.277778in}}%
\pgfpathlineto{\pgfqpoint{1.277778in}{0.277778in}}%
\pgfpathmoveto{\pgfqpoint{0.333333in}{1.333333in}}%
\pgfpathlineto{\pgfqpoint{1.333333in}{0.333333in}}%
\pgfpathmoveto{\pgfqpoint{0.388889in}{1.388889in}}%
\pgfpathlineto{\pgfqpoint{1.388889in}{0.388889in}}%
\pgfpathmoveto{\pgfqpoint{0.444444in}{1.444444in}}%
\pgfpathlineto{\pgfqpoint{1.444444in}{0.444444in}}%
\pgfpathmoveto{\pgfqpoint{0.500000in}{1.500000in}}%
\pgfpathlineto{\pgfqpoint{1.500000in}{0.500000in}}%
\pgfusepath{stroke}%
\end{pgfscope}%
}%
\pgfsys@transformshift{9.622887in}{1.022500in}%
\pgfsys@useobject{currentpattern}{}%
\pgfsys@transformshift{1in}{0in}%
\pgfsys@transformshift{-1in}{0in}%
\pgfsys@transformshift{0in}{1in}%
\end{pgfscope}%
\begin{pgfscope}%
\pgfpathrectangle{\pgfqpoint{1.228750in}{1.022500in}}{\pgfqpoint{13.456250in}{3.662500in}}%
\pgfusepath{clip}%
\pgfsetbuttcap%
\pgfsetmiterjoin%
\definecolor{currentfill}{rgb}{0.798529,0.536765,0.389706}%
\pgfsetfillcolor{currentfill}%
\pgfsetlinewidth{1.003750pt}%
\definecolor{currentstroke}{rgb}{1.000000,1.000000,1.000000}%
\pgfsetstrokecolor{currentstroke}%
\pgfsetdash{}{0pt}%
\pgfpathmoveto{\pgfqpoint{11.545208in}{1.022500in}}%
\pgfpathlineto{\pgfqpoint{12.057827in}{1.022500in}}%
\pgfpathlineto{\pgfqpoint{12.057827in}{1.282482in}}%
\pgfpathlineto{\pgfqpoint{11.545208in}{1.282482in}}%
\pgfpathclose%
\pgfusepath{stroke,fill}%
\end{pgfscope}%
\begin{pgfscope}%
\pgfsetbuttcap%
\pgfsetmiterjoin%
\definecolor{currentfill}{rgb}{0.798529,0.536765,0.389706}%
\pgfsetfillcolor{currentfill}%
\pgfsetlinewidth{1.003750pt}%
\definecolor{currentstroke}{rgb}{1.000000,1.000000,1.000000}%
\pgfsetstrokecolor{currentstroke}%
\pgfsetdash{}{0pt}%
\pgfpathrectangle{\pgfqpoint{1.228750in}{1.022500in}}{\pgfqpoint{13.456250in}{3.662500in}}%
\pgfusepath{clip}%
\pgfpathmoveto{\pgfqpoint{11.545208in}{1.022500in}}%
\pgfpathlineto{\pgfqpoint{12.057827in}{1.022500in}}%
\pgfpathlineto{\pgfqpoint{12.057827in}{1.282482in}}%
\pgfpathlineto{\pgfqpoint{11.545208in}{1.282482in}}%
\pgfpathclose%
\pgfusepath{clip}%
\pgfsys@defobject{currentpattern}{\pgfqpoint{0in}{0in}}{\pgfqpoint{1in}{1in}}{%
\begin{pgfscope}%
\pgfpathrectangle{\pgfqpoint{0in}{0in}}{\pgfqpoint{1in}{1in}}%
\pgfusepath{clip}%
\pgfpathmoveto{\pgfqpoint{-0.500000in}{0.500000in}}%
\pgfpathlineto{\pgfqpoint{0.500000in}{-0.500000in}}%
\pgfpathmoveto{\pgfqpoint{-0.444444in}{0.555556in}}%
\pgfpathlineto{\pgfqpoint{0.555556in}{-0.444444in}}%
\pgfpathmoveto{\pgfqpoint{-0.388889in}{0.611111in}}%
\pgfpathlineto{\pgfqpoint{0.611111in}{-0.388889in}}%
\pgfpathmoveto{\pgfqpoint{-0.333333in}{0.666667in}}%
\pgfpathlineto{\pgfqpoint{0.666667in}{-0.333333in}}%
\pgfpathmoveto{\pgfqpoint{-0.277778in}{0.722222in}}%
\pgfpathlineto{\pgfqpoint{0.722222in}{-0.277778in}}%
\pgfpathmoveto{\pgfqpoint{-0.222222in}{0.777778in}}%
\pgfpathlineto{\pgfqpoint{0.777778in}{-0.222222in}}%
\pgfpathmoveto{\pgfqpoint{-0.166667in}{0.833333in}}%
\pgfpathlineto{\pgfqpoint{0.833333in}{-0.166667in}}%
\pgfpathmoveto{\pgfqpoint{-0.111111in}{0.888889in}}%
\pgfpathlineto{\pgfqpoint{0.888889in}{-0.111111in}}%
\pgfpathmoveto{\pgfqpoint{-0.055556in}{0.944444in}}%
\pgfpathlineto{\pgfqpoint{0.944444in}{-0.055556in}}%
\pgfpathmoveto{\pgfqpoint{0.000000in}{1.000000in}}%
\pgfpathlineto{\pgfqpoint{1.000000in}{0.000000in}}%
\pgfpathmoveto{\pgfqpoint{0.055556in}{1.055556in}}%
\pgfpathlineto{\pgfqpoint{1.055556in}{0.055556in}}%
\pgfpathmoveto{\pgfqpoint{0.111111in}{1.111111in}}%
\pgfpathlineto{\pgfqpoint{1.111111in}{0.111111in}}%
\pgfpathmoveto{\pgfqpoint{0.166667in}{1.166667in}}%
\pgfpathlineto{\pgfqpoint{1.166667in}{0.166667in}}%
\pgfpathmoveto{\pgfqpoint{0.222222in}{1.222222in}}%
\pgfpathlineto{\pgfqpoint{1.222222in}{0.222222in}}%
\pgfpathmoveto{\pgfqpoint{0.277778in}{1.277778in}}%
\pgfpathlineto{\pgfqpoint{1.277778in}{0.277778in}}%
\pgfpathmoveto{\pgfqpoint{0.333333in}{1.333333in}}%
\pgfpathlineto{\pgfqpoint{1.333333in}{0.333333in}}%
\pgfpathmoveto{\pgfqpoint{0.388889in}{1.388889in}}%
\pgfpathlineto{\pgfqpoint{1.388889in}{0.388889in}}%
\pgfpathmoveto{\pgfqpoint{0.444444in}{1.444444in}}%
\pgfpathlineto{\pgfqpoint{1.444444in}{0.444444in}}%
\pgfpathmoveto{\pgfqpoint{0.500000in}{1.500000in}}%
\pgfpathlineto{\pgfqpoint{1.500000in}{0.500000in}}%
\pgfusepath{stroke}%
\end{pgfscope}%
}%
\pgfsys@transformshift{11.545208in}{1.022500in}%
\pgfsys@useobject{currentpattern}{}%
\pgfsys@transformshift{1in}{0in}%
\pgfsys@transformshift{-1in}{0in}%
\pgfsys@transformshift{0in}{1in}%
\end{pgfscope}%
\begin{pgfscope}%
\pgfpathrectangle{\pgfqpoint{1.228750in}{1.022500in}}{\pgfqpoint{13.456250in}{3.662500in}}%
\pgfusepath{clip}%
\pgfsetbuttcap%
\pgfsetmiterjoin%
\definecolor{currentfill}{rgb}{0.798529,0.536765,0.389706}%
\pgfsetfillcolor{currentfill}%
\pgfsetlinewidth{1.003750pt}%
\definecolor{currentstroke}{rgb}{1.000000,1.000000,1.000000}%
\pgfsetstrokecolor{currentstroke}%
\pgfsetdash{}{0pt}%
\pgfpathmoveto{\pgfqpoint{13.467530in}{1.022500in}}%
\pgfpathlineto{\pgfqpoint{13.980149in}{1.022500in}}%
\pgfpathlineto{\pgfqpoint{13.980149in}{1.282482in}}%
\pgfpathlineto{\pgfqpoint{13.467530in}{1.282482in}}%
\pgfpathclose%
\pgfusepath{stroke,fill}%
\end{pgfscope}%
\begin{pgfscope}%
\pgfsetbuttcap%
\pgfsetmiterjoin%
\definecolor{currentfill}{rgb}{0.798529,0.536765,0.389706}%
\pgfsetfillcolor{currentfill}%
\pgfsetlinewidth{1.003750pt}%
\definecolor{currentstroke}{rgb}{1.000000,1.000000,1.000000}%
\pgfsetstrokecolor{currentstroke}%
\pgfsetdash{}{0pt}%
\pgfpathrectangle{\pgfqpoint{1.228750in}{1.022500in}}{\pgfqpoint{13.456250in}{3.662500in}}%
\pgfusepath{clip}%
\pgfpathmoveto{\pgfqpoint{13.467530in}{1.022500in}}%
\pgfpathlineto{\pgfqpoint{13.980149in}{1.022500in}}%
\pgfpathlineto{\pgfqpoint{13.980149in}{1.282482in}}%
\pgfpathlineto{\pgfqpoint{13.467530in}{1.282482in}}%
\pgfpathclose%
\pgfusepath{clip}%
\pgfsys@defobject{currentpattern}{\pgfqpoint{0in}{0in}}{\pgfqpoint{1in}{1in}}{%
\begin{pgfscope}%
\pgfpathrectangle{\pgfqpoint{0in}{0in}}{\pgfqpoint{1in}{1in}}%
\pgfusepath{clip}%
\pgfpathmoveto{\pgfqpoint{-0.500000in}{0.500000in}}%
\pgfpathlineto{\pgfqpoint{0.500000in}{-0.500000in}}%
\pgfpathmoveto{\pgfqpoint{-0.444444in}{0.555556in}}%
\pgfpathlineto{\pgfqpoint{0.555556in}{-0.444444in}}%
\pgfpathmoveto{\pgfqpoint{-0.388889in}{0.611111in}}%
\pgfpathlineto{\pgfqpoint{0.611111in}{-0.388889in}}%
\pgfpathmoveto{\pgfqpoint{-0.333333in}{0.666667in}}%
\pgfpathlineto{\pgfqpoint{0.666667in}{-0.333333in}}%
\pgfpathmoveto{\pgfqpoint{-0.277778in}{0.722222in}}%
\pgfpathlineto{\pgfqpoint{0.722222in}{-0.277778in}}%
\pgfpathmoveto{\pgfqpoint{-0.222222in}{0.777778in}}%
\pgfpathlineto{\pgfqpoint{0.777778in}{-0.222222in}}%
\pgfpathmoveto{\pgfqpoint{-0.166667in}{0.833333in}}%
\pgfpathlineto{\pgfqpoint{0.833333in}{-0.166667in}}%
\pgfpathmoveto{\pgfqpoint{-0.111111in}{0.888889in}}%
\pgfpathlineto{\pgfqpoint{0.888889in}{-0.111111in}}%
\pgfpathmoveto{\pgfqpoint{-0.055556in}{0.944444in}}%
\pgfpathlineto{\pgfqpoint{0.944444in}{-0.055556in}}%
\pgfpathmoveto{\pgfqpoint{0.000000in}{1.000000in}}%
\pgfpathlineto{\pgfqpoint{1.000000in}{0.000000in}}%
\pgfpathmoveto{\pgfqpoint{0.055556in}{1.055556in}}%
\pgfpathlineto{\pgfqpoint{1.055556in}{0.055556in}}%
\pgfpathmoveto{\pgfqpoint{0.111111in}{1.111111in}}%
\pgfpathlineto{\pgfqpoint{1.111111in}{0.111111in}}%
\pgfpathmoveto{\pgfqpoint{0.166667in}{1.166667in}}%
\pgfpathlineto{\pgfqpoint{1.166667in}{0.166667in}}%
\pgfpathmoveto{\pgfqpoint{0.222222in}{1.222222in}}%
\pgfpathlineto{\pgfqpoint{1.222222in}{0.222222in}}%
\pgfpathmoveto{\pgfqpoint{0.277778in}{1.277778in}}%
\pgfpathlineto{\pgfqpoint{1.277778in}{0.277778in}}%
\pgfpathmoveto{\pgfqpoint{0.333333in}{1.333333in}}%
\pgfpathlineto{\pgfqpoint{1.333333in}{0.333333in}}%
\pgfpathmoveto{\pgfqpoint{0.388889in}{1.388889in}}%
\pgfpathlineto{\pgfqpoint{1.388889in}{0.388889in}}%
\pgfpathmoveto{\pgfqpoint{0.444444in}{1.444444in}}%
\pgfpathlineto{\pgfqpoint{1.444444in}{0.444444in}}%
\pgfpathmoveto{\pgfqpoint{0.500000in}{1.500000in}}%
\pgfpathlineto{\pgfqpoint{1.500000in}{0.500000in}}%
\pgfusepath{stroke}%
\end{pgfscope}%
}%
\pgfsys@transformshift{13.467530in}{1.022500in}%
\pgfsys@useobject{currentpattern}{}%
\pgfsys@transformshift{1in}{0in}%
\pgfsys@transformshift{-1in}{0in}%
\pgfsys@transformshift{0in}{1in}%
\end{pgfscope}%
\begin{pgfscope}%
\pgfpathrectangle{\pgfqpoint{1.228750in}{1.022500in}}{\pgfqpoint{13.456250in}{3.662500in}}%
\pgfusepath{clip}%
\pgfsetbuttcap%
\pgfsetmiterjoin%
\definecolor{currentfill}{rgb}{0.374020,0.618137,0.429902}%
\pgfsetfillcolor{currentfill}%
\pgfsetlinewidth{1.003750pt}%
\definecolor{currentstroke}{rgb}{1.000000,1.000000,1.000000}%
\pgfsetstrokecolor{currentstroke}%
\pgfsetdash{}{0pt}%
\pgfpathmoveto{\pgfqpoint{2.446220in}{1.022500in}}%
\pgfpathlineto{\pgfqpoint{2.958839in}{1.022500in}}%
\pgfpathlineto{\pgfqpoint{2.958839in}{4.228948in}}%
\pgfpathlineto{\pgfqpoint{2.446220in}{4.228948in}}%
\pgfpathclose%
\pgfusepath{stroke,fill}%
\end{pgfscope}%
\begin{pgfscope}%
\pgfsetbuttcap%
\pgfsetmiterjoin%
\definecolor{currentfill}{rgb}{0.374020,0.618137,0.429902}%
\pgfsetfillcolor{currentfill}%
\pgfsetlinewidth{1.003750pt}%
\definecolor{currentstroke}{rgb}{1.000000,1.000000,1.000000}%
\pgfsetstrokecolor{currentstroke}%
\pgfsetdash{}{0pt}%
\pgfpathrectangle{\pgfqpoint{1.228750in}{1.022500in}}{\pgfqpoint{13.456250in}{3.662500in}}%
\pgfusepath{clip}%
\pgfpathmoveto{\pgfqpoint{2.446220in}{1.022500in}}%
\pgfpathlineto{\pgfqpoint{2.958839in}{1.022500in}}%
\pgfpathlineto{\pgfqpoint{2.958839in}{4.228948in}}%
\pgfpathlineto{\pgfqpoint{2.446220in}{4.228948in}}%
\pgfpathclose%
\pgfusepath{clip}%
\pgfsys@defobject{currentpattern}{\pgfqpoint{0in}{0in}}{\pgfqpoint{1in}{1in}}{%
\begin{pgfscope}%
\pgfpathrectangle{\pgfqpoint{0in}{0in}}{\pgfqpoint{1in}{1in}}%
\pgfusepath{clip}%
\pgfusepath{stroke}%
\end{pgfscope}%
}%
\pgfsys@transformshift{2.446220in}{1.022500in}%
\pgfsys@useobject{currentpattern}{}%
\pgfsys@transformshift{1in}{0in}%
\pgfsys@transformshift{-1in}{0in}%
\pgfsys@transformshift{0in}{1in}%
\pgfsys@useobject{currentpattern}{}%
\pgfsys@transformshift{1in}{0in}%
\pgfsys@transformshift{-1in}{0in}%
\pgfsys@transformshift{0in}{1in}%
\pgfsys@useobject{currentpattern}{}%
\pgfsys@transformshift{1in}{0in}%
\pgfsys@transformshift{-1in}{0in}%
\pgfsys@transformshift{0in}{1in}%
\pgfsys@useobject{currentpattern}{}%
\pgfsys@transformshift{1in}{0in}%
\pgfsys@transformshift{-1in}{0in}%
\pgfsys@transformshift{0in}{1in}%
\end{pgfscope}%
\begin{pgfscope}%
\pgfpathrectangle{\pgfqpoint{1.228750in}{1.022500in}}{\pgfqpoint{13.456250in}{3.662500in}}%
\pgfusepath{clip}%
\pgfsetbuttcap%
\pgfsetmiterjoin%
\definecolor{currentfill}{rgb}{0.374020,0.618137,0.429902}%
\pgfsetfillcolor{currentfill}%
\pgfsetlinewidth{1.003750pt}%
\definecolor{currentstroke}{rgb}{1.000000,1.000000,1.000000}%
\pgfsetstrokecolor{currentstroke}%
\pgfsetdash{}{0pt}%
\pgfpathmoveto{\pgfqpoint{4.368542in}{1.022500in}}%
\pgfpathlineto{\pgfqpoint{4.881161in}{1.022500in}}%
\pgfpathlineto{\pgfqpoint{4.881161in}{4.147703in}}%
\pgfpathlineto{\pgfqpoint{4.368542in}{4.147703in}}%
\pgfpathclose%
\pgfusepath{stroke,fill}%
\end{pgfscope}%
\begin{pgfscope}%
\pgfsetbuttcap%
\pgfsetmiterjoin%
\definecolor{currentfill}{rgb}{0.374020,0.618137,0.429902}%
\pgfsetfillcolor{currentfill}%
\pgfsetlinewidth{1.003750pt}%
\definecolor{currentstroke}{rgb}{1.000000,1.000000,1.000000}%
\pgfsetstrokecolor{currentstroke}%
\pgfsetdash{}{0pt}%
\pgfpathrectangle{\pgfqpoint{1.228750in}{1.022500in}}{\pgfqpoint{13.456250in}{3.662500in}}%
\pgfusepath{clip}%
\pgfpathmoveto{\pgfqpoint{4.368542in}{1.022500in}}%
\pgfpathlineto{\pgfqpoint{4.881161in}{1.022500in}}%
\pgfpathlineto{\pgfqpoint{4.881161in}{4.147703in}}%
\pgfpathlineto{\pgfqpoint{4.368542in}{4.147703in}}%
\pgfpathclose%
\pgfusepath{clip}%
\pgfsys@defobject{currentpattern}{\pgfqpoint{0in}{0in}}{\pgfqpoint{1in}{1in}}{%
\begin{pgfscope}%
\pgfpathrectangle{\pgfqpoint{0in}{0in}}{\pgfqpoint{1in}{1in}}%
\pgfusepath{clip}%
\pgfusepath{stroke}%
\end{pgfscope}%
}%
\pgfsys@transformshift{4.368542in}{1.022500in}%
\pgfsys@useobject{currentpattern}{}%
\pgfsys@transformshift{1in}{0in}%
\pgfsys@transformshift{-1in}{0in}%
\pgfsys@transformshift{0in}{1in}%
\pgfsys@useobject{currentpattern}{}%
\pgfsys@transformshift{1in}{0in}%
\pgfsys@transformshift{-1in}{0in}%
\pgfsys@transformshift{0in}{1in}%
\pgfsys@useobject{currentpattern}{}%
\pgfsys@transformshift{1in}{0in}%
\pgfsys@transformshift{-1in}{0in}%
\pgfsys@transformshift{0in}{1in}%
\pgfsys@useobject{currentpattern}{}%
\pgfsys@transformshift{1in}{0in}%
\pgfsys@transformshift{-1in}{0in}%
\pgfsys@transformshift{0in}{1in}%
\end{pgfscope}%
\begin{pgfscope}%
\pgfpathrectangle{\pgfqpoint{1.228750in}{1.022500in}}{\pgfqpoint{13.456250in}{3.662500in}}%
\pgfusepath{clip}%
\pgfsetbuttcap%
\pgfsetmiterjoin%
\definecolor{currentfill}{rgb}{0.374020,0.618137,0.429902}%
\pgfsetfillcolor{currentfill}%
\pgfsetlinewidth{1.003750pt}%
\definecolor{currentstroke}{rgb}{1.000000,1.000000,1.000000}%
\pgfsetstrokecolor{currentstroke}%
\pgfsetdash{}{0pt}%
\pgfpathmoveto{\pgfqpoint{6.290863in}{1.022500in}}%
\pgfpathlineto{\pgfqpoint{6.803482in}{1.022500in}}%
\pgfpathlineto{\pgfqpoint{6.803482in}{4.510595in}}%
\pgfpathlineto{\pgfqpoint{6.290863in}{4.510595in}}%
\pgfpathclose%
\pgfusepath{stroke,fill}%
\end{pgfscope}%
\begin{pgfscope}%
\pgfsetbuttcap%
\pgfsetmiterjoin%
\definecolor{currentfill}{rgb}{0.374020,0.618137,0.429902}%
\pgfsetfillcolor{currentfill}%
\pgfsetlinewidth{1.003750pt}%
\definecolor{currentstroke}{rgb}{1.000000,1.000000,1.000000}%
\pgfsetstrokecolor{currentstroke}%
\pgfsetdash{}{0pt}%
\pgfpathrectangle{\pgfqpoint{1.228750in}{1.022500in}}{\pgfqpoint{13.456250in}{3.662500in}}%
\pgfusepath{clip}%
\pgfpathmoveto{\pgfqpoint{6.290863in}{1.022500in}}%
\pgfpathlineto{\pgfqpoint{6.803482in}{1.022500in}}%
\pgfpathlineto{\pgfqpoint{6.803482in}{4.510595in}}%
\pgfpathlineto{\pgfqpoint{6.290863in}{4.510595in}}%
\pgfpathclose%
\pgfusepath{clip}%
\pgfsys@defobject{currentpattern}{\pgfqpoint{0in}{0in}}{\pgfqpoint{1in}{1in}}{%
\begin{pgfscope}%
\pgfpathrectangle{\pgfqpoint{0in}{0in}}{\pgfqpoint{1in}{1in}}%
\pgfusepath{clip}%
\pgfusepath{stroke}%
\end{pgfscope}%
}%
\pgfsys@transformshift{6.290863in}{1.022500in}%
\pgfsys@useobject{currentpattern}{}%
\pgfsys@transformshift{1in}{0in}%
\pgfsys@transformshift{-1in}{0in}%
\pgfsys@transformshift{0in}{1in}%
\pgfsys@useobject{currentpattern}{}%
\pgfsys@transformshift{1in}{0in}%
\pgfsys@transformshift{-1in}{0in}%
\pgfsys@transformshift{0in}{1in}%
\pgfsys@useobject{currentpattern}{}%
\pgfsys@transformshift{1in}{0in}%
\pgfsys@transformshift{-1in}{0in}%
\pgfsys@transformshift{0in}{1in}%
\pgfsys@useobject{currentpattern}{}%
\pgfsys@transformshift{1in}{0in}%
\pgfsys@transformshift{-1in}{0in}%
\pgfsys@transformshift{0in}{1in}%
\end{pgfscope}%
\begin{pgfscope}%
\pgfpathrectangle{\pgfqpoint{1.228750in}{1.022500in}}{\pgfqpoint{13.456250in}{3.662500in}}%
\pgfusepath{clip}%
\pgfsetbuttcap%
\pgfsetmiterjoin%
\definecolor{currentfill}{rgb}{0.374020,0.618137,0.429902}%
\pgfsetfillcolor{currentfill}%
\pgfsetlinewidth{1.003750pt}%
\definecolor{currentstroke}{rgb}{1.000000,1.000000,1.000000}%
\pgfsetstrokecolor{currentstroke}%
\pgfsetdash{}{0pt}%
\pgfpathmoveto{\pgfqpoint{8.213185in}{1.022500in}}%
\pgfpathlineto{\pgfqpoint{8.725804in}{1.022500in}}%
\pgfpathlineto{\pgfqpoint{8.725804in}{4.429351in}}%
\pgfpathlineto{\pgfqpoint{8.213185in}{4.429351in}}%
\pgfpathclose%
\pgfusepath{stroke,fill}%
\end{pgfscope}%
\begin{pgfscope}%
\pgfsetbuttcap%
\pgfsetmiterjoin%
\definecolor{currentfill}{rgb}{0.374020,0.618137,0.429902}%
\pgfsetfillcolor{currentfill}%
\pgfsetlinewidth{1.003750pt}%
\definecolor{currentstroke}{rgb}{1.000000,1.000000,1.000000}%
\pgfsetstrokecolor{currentstroke}%
\pgfsetdash{}{0pt}%
\pgfpathrectangle{\pgfqpoint{1.228750in}{1.022500in}}{\pgfqpoint{13.456250in}{3.662500in}}%
\pgfusepath{clip}%
\pgfpathmoveto{\pgfqpoint{8.213185in}{1.022500in}}%
\pgfpathlineto{\pgfqpoint{8.725804in}{1.022500in}}%
\pgfpathlineto{\pgfqpoint{8.725804in}{4.429351in}}%
\pgfpathlineto{\pgfqpoint{8.213185in}{4.429351in}}%
\pgfpathclose%
\pgfusepath{clip}%
\pgfsys@defobject{currentpattern}{\pgfqpoint{0in}{0in}}{\pgfqpoint{1in}{1in}}{%
\begin{pgfscope}%
\pgfpathrectangle{\pgfqpoint{0in}{0in}}{\pgfqpoint{1in}{1in}}%
\pgfusepath{clip}%
\pgfusepath{stroke}%
\end{pgfscope}%
}%
\pgfsys@transformshift{8.213185in}{1.022500in}%
\pgfsys@useobject{currentpattern}{}%
\pgfsys@transformshift{1in}{0in}%
\pgfsys@transformshift{-1in}{0in}%
\pgfsys@transformshift{0in}{1in}%
\pgfsys@useobject{currentpattern}{}%
\pgfsys@transformshift{1in}{0in}%
\pgfsys@transformshift{-1in}{0in}%
\pgfsys@transformshift{0in}{1in}%
\pgfsys@useobject{currentpattern}{}%
\pgfsys@transformshift{1in}{0in}%
\pgfsys@transformshift{-1in}{0in}%
\pgfsys@transformshift{0in}{1in}%
\pgfsys@useobject{currentpattern}{}%
\pgfsys@transformshift{1in}{0in}%
\pgfsys@transformshift{-1in}{0in}%
\pgfsys@transformshift{0in}{1in}%
\end{pgfscope}%
\begin{pgfscope}%
\pgfpathrectangle{\pgfqpoint{1.228750in}{1.022500in}}{\pgfqpoint{13.456250in}{3.662500in}}%
\pgfusepath{clip}%
\pgfsetbuttcap%
\pgfsetmiterjoin%
\definecolor{currentfill}{rgb}{0.374020,0.618137,0.429902}%
\pgfsetfillcolor{currentfill}%
\pgfsetlinewidth{1.003750pt}%
\definecolor{currentstroke}{rgb}{1.000000,1.000000,1.000000}%
\pgfsetstrokecolor{currentstroke}%
\pgfsetdash{}{0pt}%
\pgfpathmoveto{\pgfqpoint{10.135506in}{1.022500in}}%
\pgfpathlineto{\pgfqpoint{10.648125in}{1.022500in}}%
\pgfpathlineto{\pgfqpoint{10.648125in}{4.033961in}}%
\pgfpathlineto{\pgfqpoint{10.135506in}{4.033961in}}%
\pgfpathclose%
\pgfusepath{stroke,fill}%
\end{pgfscope}%
\begin{pgfscope}%
\pgfsetbuttcap%
\pgfsetmiterjoin%
\definecolor{currentfill}{rgb}{0.374020,0.618137,0.429902}%
\pgfsetfillcolor{currentfill}%
\pgfsetlinewidth{1.003750pt}%
\definecolor{currentstroke}{rgb}{1.000000,1.000000,1.000000}%
\pgfsetstrokecolor{currentstroke}%
\pgfsetdash{}{0pt}%
\pgfpathrectangle{\pgfqpoint{1.228750in}{1.022500in}}{\pgfqpoint{13.456250in}{3.662500in}}%
\pgfusepath{clip}%
\pgfpathmoveto{\pgfqpoint{10.135506in}{1.022500in}}%
\pgfpathlineto{\pgfqpoint{10.648125in}{1.022500in}}%
\pgfpathlineto{\pgfqpoint{10.648125in}{4.033961in}}%
\pgfpathlineto{\pgfqpoint{10.135506in}{4.033961in}}%
\pgfpathclose%
\pgfusepath{clip}%
\pgfsys@defobject{currentpattern}{\pgfqpoint{0in}{0in}}{\pgfqpoint{1in}{1in}}{%
\begin{pgfscope}%
\pgfpathrectangle{\pgfqpoint{0in}{0in}}{\pgfqpoint{1in}{1in}}%
\pgfusepath{clip}%
\pgfusepath{stroke}%
\end{pgfscope}%
}%
\pgfsys@transformshift{10.135506in}{1.022500in}%
\pgfsys@useobject{currentpattern}{}%
\pgfsys@transformshift{1in}{0in}%
\pgfsys@transformshift{-1in}{0in}%
\pgfsys@transformshift{0in}{1in}%
\pgfsys@useobject{currentpattern}{}%
\pgfsys@transformshift{1in}{0in}%
\pgfsys@transformshift{-1in}{0in}%
\pgfsys@transformshift{0in}{1in}%
\pgfsys@useobject{currentpattern}{}%
\pgfsys@transformshift{1in}{0in}%
\pgfsys@transformshift{-1in}{0in}%
\pgfsys@transformshift{0in}{1in}%
\pgfsys@useobject{currentpattern}{}%
\pgfsys@transformshift{1in}{0in}%
\pgfsys@transformshift{-1in}{0in}%
\pgfsys@transformshift{0in}{1in}%
\end{pgfscope}%
\begin{pgfscope}%
\pgfpathrectangle{\pgfqpoint{1.228750in}{1.022500in}}{\pgfqpoint{13.456250in}{3.662500in}}%
\pgfusepath{clip}%
\pgfsetbuttcap%
\pgfsetmiterjoin%
\definecolor{currentfill}{rgb}{0.374020,0.618137,0.429902}%
\pgfsetfillcolor{currentfill}%
\pgfsetlinewidth{1.003750pt}%
\definecolor{currentstroke}{rgb}{1.000000,1.000000,1.000000}%
\pgfsetstrokecolor{currentstroke}%
\pgfsetdash{}{0pt}%
\pgfpathmoveto{\pgfqpoint{12.057827in}{1.022500in}}%
\pgfpathlineto{\pgfqpoint{12.570446in}{1.022500in}}%
\pgfpathlineto{\pgfqpoint{12.570446in}{3.243182in}}%
\pgfpathlineto{\pgfqpoint{12.057827in}{3.243182in}}%
\pgfpathclose%
\pgfusepath{stroke,fill}%
\end{pgfscope}%
\begin{pgfscope}%
\pgfsetbuttcap%
\pgfsetmiterjoin%
\definecolor{currentfill}{rgb}{0.374020,0.618137,0.429902}%
\pgfsetfillcolor{currentfill}%
\pgfsetlinewidth{1.003750pt}%
\definecolor{currentstroke}{rgb}{1.000000,1.000000,1.000000}%
\pgfsetstrokecolor{currentstroke}%
\pgfsetdash{}{0pt}%
\pgfpathrectangle{\pgfqpoint{1.228750in}{1.022500in}}{\pgfqpoint{13.456250in}{3.662500in}}%
\pgfusepath{clip}%
\pgfpathmoveto{\pgfqpoint{12.057827in}{1.022500in}}%
\pgfpathlineto{\pgfqpoint{12.570446in}{1.022500in}}%
\pgfpathlineto{\pgfqpoint{12.570446in}{3.243182in}}%
\pgfpathlineto{\pgfqpoint{12.057827in}{3.243182in}}%
\pgfpathclose%
\pgfusepath{clip}%
\pgfsys@defobject{currentpattern}{\pgfqpoint{0in}{0in}}{\pgfqpoint{1in}{1in}}{%
\begin{pgfscope}%
\pgfpathrectangle{\pgfqpoint{0in}{0in}}{\pgfqpoint{1in}{1in}}%
\pgfusepath{clip}%
\pgfusepath{stroke}%
\end{pgfscope}%
}%
\pgfsys@transformshift{12.057827in}{1.022500in}%
\pgfsys@useobject{currentpattern}{}%
\pgfsys@transformshift{1in}{0in}%
\pgfsys@transformshift{-1in}{0in}%
\pgfsys@transformshift{0in}{1in}%
\pgfsys@useobject{currentpattern}{}%
\pgfsys@transformshift{1in}{0in}%
\pgfsys@transformshift{-1in}{0in}%
\pgfsys@transformshift{0in}{1in}%
\pgfsys@useobject{currentpattern}{}%
\pgfsys@transformshift{1in}{0in}%
\pgfsys@transformshift{-1in}{0in}%
\pgfsys@transformshift{0in}{1in}%
\end{pgfscope}%
\begin{pgfscope}%
\pgfpathrectangle{\pgfqpoint{1.228750in}{1.022500in}}{\pgfqpoint{13.456250in}{3.662500in}}%
\pgfusepath{clip}%
\pgfsetbuttcap%
\pgfsetmiterjoin%
\definecolor{currentfill}{rgb}{0.374020,0.618137,0.429902}%
\pgfsetfillcolor{currentfill}%
\pgfsetlinewidth{1.003750pt}%
\definecolor{currentstroke}{rgb}{1.000000,1.000000,1.000000}%
\pgfsetstrokecolor{currentstroke}%
\pgfsetdash{}{0pt}%
\pgfpathmoveto{\pgfqpoint{13.980149in}{1.022500in}}%
\pgfpathlineto{\pgfqpoint{14.492768in}{1.022500in}}%
\pgfpathlineto{\pgfqpoint{14.492768in}{3.254014in}}%
\pgfpathlineto{\pgfqpoint{13.980149in}{3.254014in}}%
\pgfpathclose%
\pgfusepath{stroke,fill}%
\end{pgfscope}%
\begin{pgfscope}%
\pgfsetbuttcap%
\pgfsetmiterjoin%
\definecolor{currentfill}{rgb}{0.374020,0.618137,0.429902}%
\pgfsetfillcolor{currentfill}%
\pgfsetlinewidth{1.003750pt}%
\definecolor{currentstroke}{rgb}{1.000000,1.000000,1.000000}%
\pgfsetstrokecolor{currentstroke}%
\pgfsetdash{}{0pt}%
\pgfpathrectangle{\pgfqpoint{1.228750in}{1.022500in}}{\pgfqpoint{13.456250in}{3.662500in}}%
\pgfusepath{clip}%
\pgfpathmoveto{\pgfqpoint{13.980149in}{1.022500in}}%
\pgfpathlineto{\pgfqpoint{14.492768in}{1.022500in}}%
\pgfpathlineto{\pgfqpoint{14.492768in}{3.254014in}}%
\pgfpathlineto{\pgfqpoint{13.980149in}{3.254014in}}%
\pgfpathclose%
\pgfusepath{clip}%
\pgfsys@defobject{currentpattern}{\pgfqpoint{0in}{0in}}{\pgfqpoint{1in}{1in}}{%
\begin{pgfscope}%
\pgfpathrectangle{\pgfqpoint{0in}{0in}}{\pgfqpoint{1in}{1in}}%
\pgfusepath{clip}%
\pgfusepath{stroke}%
\end{pgfscope}%
}%
\pgfsys@transformshift{13.980149in}{1.022500in}%
\pgfsys@useobject{currentpattern}{}%
\pgfsys@transformshift{1in}{0in}%
\pgfsys@transformshift{-1in}{0in}%
\pgfsys@transformshift{0in}{1in}%
\pgfsys@useobject{currentpattern}{}%
\pgfsys@transformshift{1in}{0in}%
\pgfsys@transformshift{-1in}{0in}%
\pgfsys@transformshift{0in}{1in}%
\pgfsys@useobject{currentpattern}{}%
\pgfsys@transformshift{1in}{0in}%
\pgfsys@transformshift{-1in}{0in}%
\pgfsys@transformshift{0in}{1in}%
\end{pgfscope}%
\begin{pgfscope}%
\pgfpathrectangle{\pgfqpoint{1.228750in}{1.022500in}}{\pgfqpoint{13.456250in}{3.662500in}}%
\pgfusepath{clip}%
\pgfsetroundcap%
\pgfsetroundjoin%
\pgfsetlinewidth{2.710125pt}%
\definecolor{currentstroke}{rgb}{0.260000,0.260000,0.260000}%
\pgfsetstrokecolor{currentstroke}%
\pgfsetdash{}{0pt}%
\pgfusepath{stroke}%
\end{pgfscope}%
\begin{pgfscope}%
\pgfpathrectangle{\pgfqpoint{1.228750in}{1.022500in}}{\pgfqpoint{13.456250in}{3.662500in}}%
\pgfusepath{clip}%
\pgfsetroundcap%
\pgfsetroundjoin%
\pgfsetlinewidth{2.710125pt}%
\definecolor{currentstroke}{rgb}{0.260000,0.260000,0.260000}%
\pgfsetstrokecolor{currentstroke}%
\pgfsetdash{}{0pt}%
\pgfusepath{stroke}%
\end{pgfscope}%
\begin{pgfscope}%
\pgfpathrectangle{\pgfqpoint{1.228750in}{1.022500in}}{\pgfqpoint{13.456250in}{3.662500in}}%
\pgfusepath{clip}%
\pgfsetroundcap%
\pgfsetroundjoin%
\pgfsetlinewidth{2.710125pt}%
\definecolor{currentstroke}{rgb}{0.260000,0.260000,0.260000}%
\pgfsetstrokecolor{currentstroke}%
\pgfsetdash{}{0pt}%
\pgfusepath{stroke}%
\end{pgfscope}%
\begin{pgfscope}%
\pgfpathrectangle{\pgfqpoint{1.228750in}{1.022500in}}{\pgfqpoint{13.456250in}{3.662500in}}%
\pgfusepath{clip}%
\pgfsetroundcap%
\pgfsetroundjoin%
\pgfsetlinewidth{2.710125pt}%
\definecolor{currentstroke}{rgb}{0.260000,0.260000,0.260000}%
\pgfsetstrokecolor{currentstroke}%
\pgfsetdash{}{0pt}%
\pgfusepath{stroke}%
\end{pgfscope}%
\begin{pgfscope}%
\pgfpathrectangle{\pgfqpoint{1.228750in}{1.022500in}}{\pgfqpoint{13.456250in}{3.662500in}}%
\pgfusepath{clip}%
\pgfsetroundcap%
\pgfsetroundjoin%
\pgfsetlinewidth{2.710125pt}%
\definecolor{currentstroke}{rgb}{0.260000,0.260000,0.260000}%
\pgfsetstrokecolor{currentstroke}%
\pgfsetdash{}{0pt}%
\pgfusepath{stroke}%
\end{pgfscope}%
\begin{pgfscope}%
\pgfpathrectangle{\pgfqpoint{1.228750in}{1.022500in}}{\pgfqpoint{13.456250in}{3.662500in}}%
\pgfusepath{clip}%
\pgfsetroundcap%
\pgfsetroundjoin%
\pgfsetlinewidth{2.710125pt}%
\definecolor{currentstroke}{rgb}{0.260000,0.260000,0.260000}%
\pgfsetstrokecolor{currentstroke}%
\pgfsetdash{}{0pt}%
\pgfusepath{stroke}%
\end{pgfscope}%
\begin{pgfscope}%
\pgfpathrectangle{\pgfqpoint{1.228750in}{1.022500in}}{\pgfqpoint{13.456250in}{3.662500in}}%
\pgfusepath{clip}%
\pgfsetroundcap%
\pgfsetroundjoin%
\pgfsetlinewidth{2.710125pt}%
\definecolor{currentstroke}{rgb}{0.260000,0.260000,0.260000}%
\pgfsetstrokecolor{currentstroke}%
\pgfsetdash{}{0pt}%
\pgfusepath{stroke}%
\end{pgfscope}%
\begin{pgfscope}%
\pgfpathrectangle{\pgfqpoint{1.228750in}{1.022500in}}{\pgfqpoint{13.456250in}{3.662500in}}%
\pgfusepath{clip}%
\pgfsetroundcap%
\pgfsetroundjoin%
\pgfsetlinewidth{2.710125pt}%
\definecolor{currentstroke}{rgb}{0.260000,0.260000,0.260000}%
\pgfsetstrokecolor{currentstroke}%
\pgfsetdash{}{0pt}%
\pgfusepath{stroke}%
\end{pgfscope}%
\begin{pgfscope}%
\pgfpathrectangle{\pgfqpoint{1.228750in}{1.022500in}}{\pgfqpoint{13.456250in}{3.662500in}}%
\pgfusepath{clip}%
\pgfsetroundcap%
\pgfsetroundjoin%
\pgfsetlinewidth{2.710125pt}%
\definecolor{currentstroke}{rgb}{0.260000,0.260000,0.260000}%
\pgfsetstrokecolor{currentstroke}%
\pgfsetdash{}{0pt}%
\pgfusepath{stroke}%
\end{pgfscope}%
\begin{pgfscope}%
\pgfpathrectangle{\pgfqpoint{1.228750in}{1.022500in}}{\pgfqpoint{13.456250in}{3.662500in}}%
\pgfusepath{clip}%
\pgfsetroundcap%
\pgfsetroundjoin%
\pgfsetlinewidth{2.710125pt}%
\definecolor{currentstroke}{rgb}{0.260000,0.260000,0.260000}%
\pgfsetstrokecolor{currentstroke}%
\pgfsetdash{}{0pt}%
\pgfusepath{stroke}%
\end{pgfscope}%
\begin{pgfscope}%
\pgfpathrectangle{\pgfqpoint{1.228750in}{1.022500in}}{\pgfqpoint{13.456250in}{3.662500in}}%
\pgfusepath{clip}%
\pgfsetroundcap%
\pgfsetroundjoin%
\pgfsetlinewidth{2.710125pt}%
\definecolor{currentstroke}{rgb}{0.260000,0.260000,0.260000}%
\pgfsetstrokecolor{currentstroke}%
\pgfsetdash{}{0pt}%
\pgfusepath{stroke}%
\end{pgfscope}%
\begin{pgfscope}%
\pgfpathrectangle{\pgfqpoint{1.228750in}{1.022500in}}{\pgfqpoint{13.456250in}{3.662500in}}%
\pgfusepath{clip}%
\pgfsetroundcap%
\pgfsetroundjoin%
\pgfsetlinewidth{2.710125pt}%
\definecolor{currentstroke}{rgb}{0.260000,0.260000,0.260000}%
\pgfsetstrokecolor{currentstroke}%
\pgfsetdash{}{0pt}%
\pgfusepath{stroke}%
\end{pgfscope}%
\begin{pgfscope}%
\pgfpathrectangle{\pgfqpoint{1.228750in}{1.022500in}}{\pgfqpoint{13.456250in}{3.662500in}}%
\pgfusepath{clip}%
\pgfsetroundcap%
\pgfsetroundjoin%
\pgfsetlinewidth{2.710125pt}%
\definecolor{currentstroke}{rgb}{0.260000,0.260000,0.260000}%
\pgfsetstrokecolor{currentstroke}%
\pgfsetdash{}{0pt}%
\pgfusepath{stroke}%
\end{pgfscope}%
\begin{pgfscope}%
\pgfpathrectangle{\pgfqpoint{1.228750in}{1.022500in}}{\pgfqpoint{13.456250in}{3.662500in}}%
\pgfusepath{clip}%
\pgfsetroundcap%
\pgfsetroundjoin%
\pgfsetlinewidth{2.710125pt}%
\definecolor{currentstroke}{rgb}{0.260000,0.260000,0.260000}%
\pgfsetstrokecolor{currentstroke}%
\pgfsetdash{}{0pt}%
\pgfusepath{stroke}%
\end{pgfscope}%
\begin{pgfscope}%
\pgfpathrectangle{\pgfqpoint{1.228750in}{1.022500in}}{\pgfqpoint{13.456250in}{3.662500in}}%
\pgfusepath{clip}%
\pgfsetroundcap%
\pgfsetroundjoin%
\pgfsetlinewidth{2.710125pt}%
\definecolor{currentstroke}{rgb}{0.260000,0.260000,0.260000}%
\pgfsetstrokecolor{currentstroke}%
\pgfsetdash{}{0pt}%
\pgfusepath{stroke}%
\end{pgfscope}%
\begin{pgfscope}%
\pgfpathrectangle{\pgfqpoint{1.228750in}{1.022500in}}{\pgfqpoint{13.456250in}{3.662500in}}%
\pgfusepath{clip}%
\pgfsetroundcap%
\pgfsetroundjoin%
\pgfsetlinewidth{2.710125pt}%
\definecolor{currentstroke}{rgb}{0.260000,0.260000,0.260000}%
\pgfsetstrokecolor{currentstroke}%
\pgfsetdash{}{0pt}%
\pgfusepath{stroke}%
\end{pgfscope}%
\begin{pgfscope}%
\pgfpathrectangle{\pgfqpoint{1.228750in}{1.022500in}}{\pgfqpoint{13.456250in}{3.662500in}}%
\pgfusepath{clip}%
\pgfsetroundcap%
\pgfsetroundjoin%
\pgfsetlinewidth{2.710125pt}%
\definecolor{currentstroke}{rgb}{0.260000,0.260000,0.260000}%
\pgfsetstrokecolor{currentstroke}%
\pgfsetdash{}{0pt}%
\pgfusepath{stroke}%
\end{pgfscope}%
\begin{pgfscope}%
\pgfpathrectangle{\pgfqpoint{1.228750in}{1.022500in}}{\pgfqpoint{13.456250in}{3.662500in}}%
\pgfusepath{clip}%
\pgfsetroundcap%
\pgfsetroundjoin%
\pgfsetlinewidth{2.710125pt}%
\definecolor{currentstroke}{rgb}{0.260000,0.260000,0.260000}%
\pgfsetstrokecolor{currentstroke}%
\pgfsetdash{}{0pt}%
\pgfusepath{stroke}%
\end{pgfscope}%
\begin{pgfscope}%
\pgfpathrectangle{\pgfqpoint{1.228750in}{1.022500in}}{\pgfqpoint{13.456250in}{3.662500in}}%
\pgfusepath{clip}%
\pgfsetroundcap%
\pgfsetroundjoin%
\pgfsetlinewidth{2.710125pt}%
\definecolor{currentstroke}{rgb}{0.260000,0.260000,0.260000}%
\pgfsetstrokecolor{currentstroke}%
\pgfsetdash{}{0pt}%
\pgfusepath{stroke}%
\end{pgfscope}%
\begin{pgfscope}%
\pgfpathrectangle{\pgfqpoint{1.228750in}{1.022500in}}{\pgfqpoint{13.456250in}{3.662500in}}%
\pgfusepath{clip}%
\pgfsetroundcap%
\pgfsetroundjoin%
\pgfsetlinewidth{2.710125pt}%
\definecolor{currentstroke}{rgb}{0.260000,0.260000,0.260000}%
\pgfsetstrokecolor{currentstroke}%
\pgfsetdash{}{0pt}%
\pgfusepath{stroke}%
\end{pgfscope}%
\begin{pgfscope}%
\pgfpathrectangle{\pgfqpoint{1.228750in}{1.022500in}}{\pgfqpoint{13.456250in}{3.662500in}}%
\pgfusepath{clip}%
\pgfsetroundcap%
\pgfsetroundjoin%
\pgfsetlinewidth{2.710125pt}%
\definecolor{currentstroke}{rgb}{0.260000,0.260000,0.260000}%
\pgfsetstrokecolor{currentstroke}%
\pgfsetdash{}{0pt}%
\pgfusepath{stroke}%
\end{pgfscope}%
\begin{pgfscope}%
\pgfsetrectcap%
\pgfsetmiterjoin%
\pgfsetlinewidth{1.254687pt}%
\definecolor{currentstroke}{rgb}{0.800000,0.800000,0.800000}%
\pgfsetstrokecolor{currentstroke}%
\pgfsetdash{}{0pt}%
\pgfpathmoveto{\pgfqpoint{1.228750in}{1.022500in}}%
\pgfpathlineto{\pgfqpoint{14.685000in}{1.022500in}}%
\pgfusepath{stroke}%
\end{pgfscope}%
\begin{pgfscope}%
\pgfsetbuttcap%
\pgfsetmiterjoin%
\definecolor{currentfill}{rgb}{1.000000,1.000000,1.000000}%
\pgfsetfillcolor{currentfill}%
\pgfsetfillopacity{0.800000}%
\pgfsetlinewidth{1.003750pt}%
\definecolor{currentstroke}{rgb}{0.800000,0.800000,0.800000}%
\pgfsetstrokecolor{currentstroke}%
\pgfsetstrokeopacity{0.800000}%
\pgfsetdash{}{0pt}%
\pgfpathmoveto{\pgfqpoint{11.660327in}{3.301867in}}%
\pgfpathlineto{\pgfqpoint{14.497847in}{3.301867in}}%
\pgfpathquadraticcurveto{\pgfqpoint{14.551319in}{3.301867in}}{\pgfqpoint{14.551319in}{3.355339in}}%
\pgfpathlineto{\pgfqpoint{14.551319in}{4.497847in}}%
\pgfpathquadraticcurveto{\pgfqpoint{14.551319in}{4.551319in}}{\pgfqpoint{14.497847in}{4.551319in}}%
\pgfpathlineto{\pgfqpoint{11.660327in}{4.551319in}}%
\pgfpathquadraticcurveto{\pgfqpoint{11.606855in}{4.551319in}}{\pgfqpoint{11.606855in}{4.497847in}}%
\pgfpathlineto{\pgfqpoint{11.606855in}{3.355339in}}%
\pgfpathquadraticcurveto{\pgfqpoint{11.606855in}{3.301867in}}{\pgfqpoint{11.660327in}{3.301867in}}%
\pgfpathclose%
\pgfusepath{stroke,fill}%
\end{pgfscope}%
\begin{pgfscope}%
\pgfsetbuttcap%
\pgfsetmiterjoin%
\definecolor{currentfill}{rgb}{0.347059,0.458824,0.641176}%
\pgfsetfillcolor{currentfill}%
\pgfsetlinewidth{1.003750pt}%
\definecolor{currentstroke}{rgb}{1.000000,1.000000,1.000000}%
\pgfsetstrokecolor{currentstroke}%
\pgfsetdash{}{0pt}%
\pgfpathmoveto{\pgfqpoint{11.713800in}{4.244337in}}%
\pgfpathlineto{\pgfqpoint{12.248522in}{4.244337in}}%
\pgfpathlineto{\pgfqpoint{12.248522in}{4.431489in}}%
\pgfpathlineto{\pgfqpoint{11.713800in}{4.431489in}}%
\pgfpathclose%
\pgfusepath{stroke,fill}%
\end{pgfscope}%
\begin{pgfscope}%
\pgfsetbuttcap%
\pgfsetmiterjoin%
\definecolor{currentfill}{rgb}{0.347059,0.458824,0.641176}%
\pgfsetfillcolor{currentfill}%
\pgfsetlinewidth{1.003750pt}%
\definecolor{currentstroke}{rgb}{1.000000,1.000000,1.000000}%
\pgfsetstrokecolor{currentstroke}%
\pgfsetdash{}{0pt}%
\pgfpathmoveto{\pgfqpoint{11.713800in}{4.244337in}}%
\pgfpathlineto{\pgfqpoint{12.248522in}{4.244337in}}%
\pgfpathlineto{\pgfqpoint{12.248522in}{4.431489in}}%
\pgfpathlineto{\pgfqpoint{11.713800in}{4.431489in}}%
\pgfpathclose%
\pgfusepath{clip}%
\pgfsys@defobject{currentpattern}{\pgfqpoint{0in}{0in}}{\pgfqpoint{1in}{1in}}{%
\begin{pgfscope}%
\pgfpathrectangle{\pgfqpoint{0in}{0in}}{\pgfqpoint{1in}{1in}}%
\pgfusepath{clip}%
\pgfpathmoveto{\pgfqpoint{-0.500000in}{0.500000in}}%
\pgfpathlineto{\pgfqpoint{0.500000in}{1.500000in}}%
\pgfpathmoveto{\pgfqpoint{-0.444444in}{0.444444in}}%
\pgfpathlineto{\pgfqpoint{0.555556in}{1.444444in}}%
\pgfpathmoveto{\pgfqpoint{-0.388889in}{0.388889in}}%
\pgfpathlineto{\pgfqpoint{0.611111in}{1.388889in}}%
\pgfpathmoveto{\pgfqpoint{-0.333333in}{0.333333in}}%
\pgfpathlineto{\pgfqpoint{0.666667in}{1.333333in}}%
\pgfpathmoveto{\pgfqpoint{-0.277778in}{0.277778in}}%
\pgfpathlineto{\pgfqpoint{0.722222in}{1.277778in}}%
\pgfpathmoveto{\pgfqpoint{-0.222222in}{0.222222in}}%
\pgfpathlineto{\pgfqpoint{0.777778in}{1.222222in}}%
\pgfpathmoveto{\pgfqpoint{-0.166667in}{0.166667in}}%
\pgfpathlineto{\pgfqpoint{0.833333in}{1.166667in}}%
\pgfpathmoveto{\pgfqpoint{-0.111111in}{0.111111in}}%
\pgfpathlineto{\pgfqpoint{0.888889in}{1.111111in}}%
\pgfpathmoveto{\pgfqpoint{-0.055556in}{0.055556in}}%
\pgfpathlineto{\pgfqpoint{0.944444in}{1.055556in}}%
\pgfpathmoveto{\pgfqpoint{0.000000in}{0.000000in}}%
\pgfpathlineto{\pgfqpoint{1.000000in}{1.000000in}}%
\pgfpathmoveto{\pgfqpoint{0.055556in}{-0.055556in}}%
\pgfpathlineto{\pgfqpoint{1.055556in}{0.944444in}}%
\pgfpathmoveto{\pgfqpoint{0.111111in}{-0.111111in}}%
\pgfpathlineto{\pgfqpoint{1.111111in}{0.888889in}}%
\pgfpathmoveto{\pgfqpoint{0.166667in}{-0.166667in}}%
\pgfpathlineto{\pgfqpoint{1.166667in}{0.833333in}}%
\pgfpathmoveto{\pgfqpoint{0.222222in}{-0.222222in}}%
\pgfpathlineto{\pgfqpoint{1.222222in}{0.777778in}}%
\pgfpathmoveto{\pgfqpoint{0.277778in}{-0.277778in}}%
\pgfpathlineto{\pgfqpoint{1.277778in}{0.722222in}}%
\pgfpathmoveto{\pgfqpoint{0.333333in}{-0.333333in}}%
\pgfpathlineto{\pgfqpoint{1.333333in}{0.666667in}}%
\pgfpathmoveto{\pgfqpoint{0.388889in}{-0.388889in}}%
\pgfpathlineto{\pgfqpoint{1.388889in}{0.611111in}}%
\pgfpathmoveto{\pgfqpoint{0.444444in}{-0.444444in}}%
\pgfpathlineto{\pgfqpoint{1.444444in}{0.555556in}}%
\pgfpathmoveto{\pgfqpoint{0.500000in}{-0.500000in}}%
\pgfpathlineto{\pgfqpoint{1.500000in}{0.500000in}}%
\pgfusepath{stroke}%
\end{pgfscope}%
}%
\pgfsys@transformshift{11.713800in}{4.244337in}%
\pgfsys@useobject{currentpattern}{}%
\pgfsys@transformshift{1in}{0in}%
\pgfsys@transformshift{-1in}{0in}%
\pgfsys@transformshift{0in}{1in}%
\end{pgfscope}%
\begin{pgfscope}%
\definecolor{textcolor}{rgb}{0.150000,0.150000,0.150000}%
\pgfsetstrokecolor{textcolor}%
\pgfsetfillcolor{textcolor}%
\pgftext[x=12.462411in,y=4.244337in,left,base]{\color{textcolor}\sffamily\fontsize{19.250000}{23.100000}\selectfont JitSynth compiler}%
\end{pgfscope}%
\begin{pgfscope}%
\pgfsetbuttcap%
\pgfsetmiterjoin%
\definecolor{currentfill}{rgb}{0.798529,0.536765,0.389706}%
\pgfsetfillcolor{currentfill}%
\pgfsetlinewidth{1.003750pt}%
\definecolor{currentstroke}{rgb}{1.000000,1.000000,1.000000}%
\pgfsetstrokecolor{currentstroke}%
\pgfsetdash{}{0pt}%
\pgfpathmoveto{\pgfqpoint{11.713800in}{3.854588in}}%
\pgfpathlineto{\pgfqpoint{12.248522in}{3.854588in}}%
\pgfpathlineto{\pgfqpoint{12.248522in}{4.041741in}}%
\pgfpathlineto{\pgfqpoint{11.713800in}{4.041741in}}%
\pgfpathclose%
\pgfusepath{stroke,fill}%
\end{pgfscope}%
\begin{pgfscope}%
\pgfsetbuttcap%
\pgfsetmiterjoin%
\definecolor{currentfill}{rgb}{0.798529,0.536765,0.389706}%
\pgfsetfillcolor{currentfill}%
\pgfsetlinewidth{1.003750pt}%
\definecolor{currentstroke}{rgb}{1.000000,1.000000,1.000000}%
\pgfsetstrokecolor{currentstroke}%
\pgfsetdash{}{0pt}%
\pgfpathmoveto{\pgfqpoint{11.713800in}{3.854588in}}%
\pgfpathlineto{\pgfqpoint{12.248522in}{3.854588in}}%
\pgfpathlineto{\pgfqpoint{12.248522in}{4.041741in}}%
\pgfpathlineto{\pgfqpoint{11.713800in}{4.041741in}}%
\pgfpathclose%
\pgfusepath{clip}%
\pgfsys@defobject{currentpattern}{\pgfqpoint{0in}{0in}}{\pgfqpoint{1in}{1in}}{%
\begin{pgfscope}%
\pgfpathrectangle{\pgfqpoint{0in}{0in}}{\pgfqpoint{1in}{1in}}%
\pgfusepath{clip}%
\pgfpathmoveto{\pgfqpoint{-0.500000in}{0.500000in}}%
\pgfpathlineto{\pgfqpoint{0.500000in}{-0.500000in}}%
\pgfpathmoveto{\pgfqpoint{-0.444444in}{0.555556in}}%
\pgfpathlineto{\pgfqpoint{0.555556in}{-0.444444in}}%
\pgfpathmoveto{\pgfqpoint{-0.388889in}{0.611111in}}%
\pgfpathlineto{\pgfqpoint{0.611111in}{-0.388889in}}%
\pgfpathmoveto{\pgfqpoint{-0.333333in}{0.666667in}}%
\pgfpathlineto{\pgfqpoint{0.666667in}{-0.333333in}}%
\pgfpathmoveto{\pgfqpoint{-0.277778in}{0.722222in}}%
\pgfpathlineto{\pgfqpoint{0.722222in}{-0.277778in}}%
\pgfpathmoveto{\pgfqpoint{-0.222222in}{0.777778in}}%
\pgfpathlineto{\pgfqpoint{0.777778in}{-0.222222in}}%
\pgfpathmoveto{\pgfqpoint{-0.166667in}{0.833333in}}%
\pgfpathlineto{\pgfqpoint{0.833333in}{-0.166667in}}%
\pgfpathmoveto{\pgfqpoint{-0.111111in}{0.888889in}}%
\pgfpathlineto{\pgfqpoint{0.888889in}{-0.111111in}}%
\pgfpathmoveto{\pgfqpoint{-0.055556in}{0.944444in}}%
\pgfpathlineto{\pgfqpoint{0.944444in}{-0.055556in}}%
\pgfpathmoveto{\pgfqpoint{0.000000in}{1.000000in}}%
\pgfpathlineto{\pgfqpoint{1.000000in}{0.000000in}}%
\pgfpathmoveto{\pgfqpoint{0.055556in}{1.055556in}}%
\pgfpathlineto{\pgfqpoint{1.055556in}{0.055556in}}%
\pgfpathmoveto{\pgfqpoint{0.111111in}{1.111111in}}%
\pgfpathlineto{\pgfqpoint{1.111111in}{0.111111in}}%
\pgfpathmoveto{\pgfqpoint{0.166667in}{1.166667in}}%
\pgfpathlineto{\pgfqpoint{1.166667in}{0.166667in}}%
\pgfpathmoveto{\pgfqpoint{0.222222in}{1.222222in}}%
\pgfpathlineto{\pgfqpoint{1.222222in}{0.222222in}}%
\pgfpathmoveto{\pgfqpoint{0.277778in}{1.277778in}}%
\pgfpathlineto{\pgfqpoint{1.277778in}{0.277778in}}%
\pgfpathmoveto{\pgfqpoint{0.333333in}{1.333333in}}%
\pgfpathlineto{\pgfqpoint{1.333333in}{0.333333in}}%
\pgfpathmoveto{\pgfqpoint{0.388889in}{1.388889in}}%
\pgfpathlineto{\pgfqpoint{1.388889in}{0.388889in}}%
\pgfpathmoveto{\pgfqpoint{0.444444in}{1.444444in}}%
\pgfpathlineto{\pgfqpoint{1.444444in}{0.444444in}}%
\pgfpathmoveto{\pgfqpoint{0.500000in}{1.500000in}}%
\pgfpathlineto{\pgfqpoint{1.500000in}{0.500000in}}%
\pgfusepath{stroke}%
\end{pgfscope}%
}%
\pgfsys@transformshift{11.713800in}{3.854588in}%
\pgfsys@useobject{currentpattern}{}%
\pgfsys@transformshift{1in}{0in}%
\pgfsys@transformshift{-1in}{0in}%
\pgfsys@transformshift{0in}{1in}%
\end{pgfscope}%
\begin{pgfscope}%
\definecolor{textcolor}{rgb}{0.150000,0.150000,0.150000}%
\pgfsetstrokecolor{textcolor}%
\pgfsetfillcolor{textcolor}%
\pgftext[x=12.462411in,y=3.854588in,left,base]{\color{textcolor}\sffamily\fontsize{19.250000}{23.100000}\selectfont Linux compiler}%
\end{pgfscope}%
\begin{pgfscope}%
\pgfsetbuttcap%
\pgfsetmiterjoin%
\definecolor{currentfill}{rgb}{0.374020,0.618137,0.429902}%
\pgfsetfillcolor{currentfill}%
\pgfsetlinewidth{1.003750pt}%
\definecolor{currentstroke}{rgb}{1.000000,1.000000,1.000000}%
\pgfsetstrokecolor{currentstroke}%
\pgfsetdash{}{0pt}%
\pgfpathmoveto{\pgfqpoint{11.713800in}{3.464840in}}%
\pgfpathlineto{\pgfqpoint{12.248522in}{3.464840in}}%
\pgfpathlineto{\pgfqpoint{12.248522in}{3.651993in}}%
\pgfpathlineto{\pgfqpoint{11.713800in}{3.651993in}}%
\pgfpathclose%
\pgfusepath{stroke,fill}%
\end{pgfscope}%
\begin{pgfscope}%
\pgfsetbuttcap%
\pgfsetmiterjoin%
\definecolor{currentfill}{rgb}{0.374020,0.618137,0.429902}%
\pgfsetfillcolor{currentfill}%
\pgfsetlinewidth{1.003750pt}%
\definecolor{currentstroke}{rgb}{1.000000,1.000000,1.000000}%
\pgfsetstrokecolor{currentstroke}%
\pgfsetdash{}{0pt}%
\pgfpathmoveto{\pgfqpoint{11.713800in}{3.464840in}}%
\pgfpathlineto{\pgfqpoint{12.248522in}{3.464840in}}%
\pgfpathlineto{\pgfqpoint{12.248522in}{3.651993in}}%
\pgfpathlineto{\pgfqpoint{11.713800in}{3.651993in}}%
\pgfpathclose%
\pgfusepath{clip}%
\pgfsys@defobject{currentpattern}{\pgfqpoint{0in}{0in}}{\pgfqpoint{1in}{1in}}{%
\begin{pgfscope}%
\pgfpathrectangle{\pgfqpoint{0in}{0in}}{\pgfqpoint{1in}{1in}}%
\pgfusepath{clip}%
\pgfusepath{stroke}%
\end{pgfscope}%
}%
\pgfsys@transformshift{11.713800in}{3.464840in}%
\pgfsys@useobject{currentpattern}{}%
\pgfsys@transformshift{1in}{0in}%
\pgfsys@transformshift{-1in}{0in}%
\pgfsys@transformshift{0in}{1in}%
\end{pgfscope}%
\begin{pgfscope}%
\definecolor{textcolor}{rgb}{0.150000,0.150000,0.150000}%
\pgfsetstrokecolor{textcolor}%
\pgfsetfillcolor{textcolor}%
\pgftext[x=12.462411in,y=3.464840in,left,base]{\color{textcolor}\sffamily\fontsize{19.250000}{23.100000}\selectfont Linux interpreter}%
\end{pgfscope}%
\end{pgfpicture}%
\makeatother%
\endgroup%

  }
  \caption{Execution time of eBPF benchmarks on the HiFive Unleashed
  RISC-V development board, using the existing Linux eBPF to RISC-V
  compiler, the \jitsynth compiler, and the Linux eBPF interpreter.
  Measured in processor cycles.}
  \label{fig:b2r-runtime}
\end{figure}

To demonstrate the effectiveness of \jitsynth,
we applied \jitsynth to synthesize compilers for three different
source-target pairs: eBPF to 64-bit RISC-V, classic BPF to eBPF,
and libseccomp to eBPF.
%
This subsection describes our results for each of the synthesized
compilers.
%


\paragraph{eBPF to RISC-V.}

As a case study, we applied \jitsynth to synthesize
a compiler from eBPF to 64-bit RISC-V.
%
It supports 87 of the 102 eBPF instruction opcodes;
unsupported eBPF instructions include function calls,
endianness operations, and atomic instructions.
%
To validate that the synthesized compiler is correct, we ran the existing
eBPF test cases from the Linux kernel; our compiler passes all test cases
it supports. % 228 supported / 234 total ebpf
%
In addition, our compiler avoids bugs previously found in the existing
Linux eBPF-to-RISC-V compiler in Linux~\cite{nelson:bpf-riscv-add32-bug}.
%
To evaluate performance, we compared against the existing Linux compiler.
%
We used the same set of benchmarks used by Jitk~\cite{wang:jitk},
which includes system call filters from widely used applications.
%
Because these benchmarks were originally for classic BPF,
we first compile them to eBPF using the existing
Linux classic-BPF-to-eBPF compiler as a preprocessing step.
%
To run the benchmarks, we execute the generated
code on the HiFive Unleashed RISC-V development board~\cite{sifive:fu540-c000}, measuring
the number of cycles.
%
As input to the filter, we use a system call number that is allowed
by the filter to represent the common case execution.


\autoref{fig:b2r-runtime} shows the results of the performance evaluation.
%
eBPF programs compiled by \jitsynth JIT compilers show an average slowdown of $\EbpfCompilerSlowdown\times$
compared to programs compiled by the existing Linux compiler.
%
This overhead results from additional complexity in the compiled eBPF jump instructions.
%
Linux compilers avoid this complexity by leveraging
bounds on the size of eBPF jump offsets.
%
\jitsynth-compiled programs get an average speedup of $\EbpfInterpSpeedup\times$
compared to interpreting the eBPF programs.
%
This evidence shows that \jitsynth can synthesize a compiler that outperforms
the current Linux eBPF interpreter, and nears the performance of the Linux
compiler, while avoiding bugs.
%

\paragraph{Classic BPF to eBPF.}

\begin{figure}[h]
  \resizebox{\textwidth}{!}{
  %% Creator: Matplotlib, PGF backend
%%
%% To include the figure in your LaTeX document, write
%%   \input{<filename>.pgf}
%%
%% Make sure the required packages are loaded in your preamble
%%   \usepackage{pgf}
%%
%% Figures using additional raster images can only be included by \input if
%% they are in the same directory as the main LaTeX file. For loading figures
%% from other directories you can use the `import` package
%%   \usepackage{import}
%% and then include the figures with
%%   \import{<path to file>}{<filename>.pgf}
%%
%% Matplotlib used the following preamble
%%
\begingroup%
\makeatletter%
\begin{pgfpicture}%
\pgfpathrectangle{\pgfpointorigin}{\pgfqpoint{8.000000in}{5.000000in}}%
\pgfusepath{use as bounding box, clip}%
\begin{pgfscope}%
\pgfsetbuttcap%
\pgfsetmiterjoin%
\definecolor{currentfill}{rgb}{1.000000,1.000000,1.000000}%
\pgfsetfillcolor{currentfill}%
\pgfsetlinewidth{0.000000pt}%
\definecolor{currentstroke}{rgb}{1.000000,1.000000,1.000000}%
\pgfsetstrokecolor{currentstroke}%
\pgfsetdash{}{0pt}%
\pgfpathmoveto{\pgfqpoint{0.000000in}{0.000000in}}%
\pgfpathlineto{\pgfqpoint{8.000000in}{0.000000in}}%
\pgfpathlineto{\pgfqpoint{8.000000in}{5.000000in}}%
\pgfpathlineto{\pgfqpoint{0.000000in}{5.000000in}}%
\pgfpathclose%
\pgfusepath{fill}%
\end{pgfscope}%
\begin{pgfscope}%
\pgfsetbuttcap%
\pgfsetmiterjoin%
\definecolor{currentfill}{rgb}{1.000000,1.000000,1.000000}%
\pgfsetfillcolor{currentfill}%
\pgfsetlinewidth{0.000000pt}%
\definecolor{currentstroke}{rgb}{0.000000,0.000000,0.000000}%
\pgfsetstrokecolor{currentstroke}%
\pgfsetstrokeopacity{0.000000}%
\pgfsetdash{}{0pt}%
\pgfpathmoveto{\pgfqpoint{1.081250in}{1.022500in}}%
\pgfpathlineto{\pgfqpoint{7.685000in}{1.022500in}}%
\pgfpathlineto{\pgfqpoint{7.685000in}{4.381667in}}%
\pgfpathlineto{\pgfqpoint{1.081250in}{4.381667in}}%
\pgfpathclose%
\pgfusepath{fill}%
\end{pgfscope}%
\begin{pgfscope}%
\definecolor{textcolor}{rgb}{0.150000,0.150000,0.150000}%
\pgfsetstrokecolor{textcolor}%
\pgfsetfillcolor{textcolor}%
\pgftext[x=1.552946in,y=0.890556in,,top]{\color{textcolor}\sffamily\fontsize{19.250000}{23.100000}\selectfont OpenSSH}%
\end{pgfscope}%
\begin{pgfscope}%
\definecolor{textcolor}{rgb}{0.150000,0.150000,0.150000}%
\pgfsetstrokecolor{textcolor}%
\pgfsetfillcolor{textcolor}%
\pgftext[x=2.496339in,y=0.890556in,,top]{\color{textcolor}\sffamily\fontsize{19.250000}{23.100000}\selectfont NaCl}%
\end{pgfscope}%
\begin{pgfscope}%
\definecolor{textcolor}{rgb}{0.150000,0.150000,0.150000}%
\pgfsetstrokecolor{textcolor}%
\pgfsetfillcolor{textcolor}%
\pgftext[x=3.439732in,y=0.890556in,,top]{\color{textcolor}\sffamily\fontsize{19.250000}{23.100000}\selectfont QEMU}%
\end{pgfscope}%
\begin{pgfscope}%
\definecolor{textcolor}{rgb}{0.150000,0.150000,0.150000}%
\pgfsetstrokecolor{textcolor}%
\pgfsetfillcolor{textcolor}%
\pgftext[x=4.383125in,y=0.890556in,,top]{\color{textcolor}\sffamily\fontsize{19.250000}{23.100000}\selectfont Chrome}%
\end{pgfscope}%
\begin{pgfscope}%
\definecolor{textcolor}{rgb}{0.150000,0.150000,0.150000}%
\pgfsetstrokecolor{textcolor}%
\pgfsetfillcolor{textcolor}%
\pgftext[x=5.326518in,y=0.890556in,,top]{\color{textcolor}\sffamily\fontsize{19.250000}{23.100000}\selectfont Firefox}%
\end{pgfscope}%
\begin{pgfscope}%
\definecolor{textcolor}{rgb}{0.150000,0.150000,0.150000}%
\pgfsetstrokecolor{textcolor}%
\pgfsetfillcolor{textcolor}%
\pgftext[x=6.269911in,y=0.890556in,,top]{\color{textcolor}\sffamily\fontsize{19.250000}{23.100000}\selectfont vsftpd}%
\end{pgfscope}%
\begin{pgfscope}%
\definecolor{textcolor}{rgb}{0.150000,0.150000,0.150000}%
\pgfsetstrokecolor{textcolor}%
\pgfsetfillcolor{textcolor}%
\pgftext[x=7.213304in,y=0.890556in,,top]{\color{textcolor}\sffamily\fontsize{19.250000}{23.100000}\selectfont Tor}%
\end{pgfscope}%
\begin{pgfscope}%
\definecolor{textcolor}{rgb}{0.150000,0.150000,0.150000}%
\pgfsetstrokecolor{textcolor}%
\pgfsetfillcolor{textcolor}%
\pgftext[x=4.383125in,y=0.578932in,,top]{\color{textcolor}\sffamily\fontsize{21.000000}{25.200000}\selectfont Benchmark}%
\end{pgfscope}%
\begin{pgfscope}%
\pgfpathrectangle{\pgfqpoint{1.081250in}{1.022500in}}{\pgfqpoint{6.603750in}{3.359167in}}%
\pgfusepath{clip}%
\pgfsetroundcap%
\pgfsetroundjoin%
\pgfsetlinewidth{1.003750pt}%
\definecolor{currentstroke}{rgb}{0.800000,0.800000,0.800000}%
\pgfsetstrokecolor{currentstroke}%
\pgfsetdash{}{0pt}%
\pgfpathmoveto{\pgfqpoint{1.081250in}{1.022500in}}%
\pgfpathlineto{\pgfqpoint{7.685000in}{1.022500in}}%
\pgfusepath{stroke}%
\end{pgfscope}%
\begin{pgfscope}%
\definecolor{textcolor}{rgb}{0.150000,0.150000,0.150000}%
\pgfsetstrokecolor{textcolor}%
\pgfsetfillcolor{textcolor}%
\pgftext[x=0.813864in,y=0.922481in,left,base]{\color{textcolor}\sffamily\fontsize{19.250000}{23.100000}\selectfont 0}%
\end{pgfscope}%
\begin{pgfscope}%
\pgfpathrectangle{\pgfqpoint{1.081250in}{1.022500in}}{\pgfqpoint{6.603750in}{3.359167in}}%
\pgfusepath{clip}%
\pgfsetroundcap%
\pgfsetroundjoin%
\pgfsetlinewidth{1.003750pt}%
\definecolor{currentstroke}{rgb}{0.800000,0.800000,0.800000}%
\pgfsetstrokecolor{currentstroke}%
\pgfsetdash{}{0pt}%
\pgfpathmoveto{\pgfqpoint{1.081250in}{1.749592in}}%
\pgfpathlineto{\pgfqpoint{7.685000in}{1.749592in}}%
\pgfusepath{stroke}%
\end{pgfscope}%
\begin{pgfscope}%
\definecolor{textcolor}{rgb}{0.150000,0.150000,0.150000}%
\pgfsetstrokecolor{textcolor}%
\pgfsetfillcolor{textcolor}%
\pgftext[x=0.678422in,y=1.649573in,left,base]{\color{textcolor}\sffamily\fontsize{19.250000}{23.100000}\selectfont 10}%
\end{pgfscope}%
\begin{pgfscope}%
\pgfpathrectangle{\pgfqpoint{1.081250in}{1.022500in}}{\pgfqpoint{6.603750in}{3.359167in}}%
\pgfusepath{clip}%
\pgfsetroundcap%
\pgfsetroundjoin%
\pgfsetlinewidth{1.003750pt}%
\definecolor{currentstroke}{rgb}{0.800000,0.800000,0.800000}%
\pgfsetstrokecolor{currentstroke}%
\pgfsetdash{}{0pt}%
\pgfpathmoveto{\pgfqpoint{1.081250in}{2.476685in}}%
\pgfpathlineto{\pgfqpoint{7.685000in}{2.476685in}}%
\pgfusepath{stroke}%
\end{pgfscope}%
\begin{pgfscope}%
\definecolor{textcolor}{rgb}{0.150000,0.150000,0.150000}%
\pgfsetstrokecolor{textcolor}%
\pgfsetfillcolor{textcolor}%
\pgftext[x=0.678422in,y=2.376665in,left,base]{\color{textcolor}\sffamily\fontsize{19.250000}{23.100000}\selectfont 20}%
\end{pgfscope}%
\begin{pgfscope}%
\pgfpathrectangle{\pgfqpoint{1.081250in}{1.022500in}}{\pgfqpoint{6.603750in}{3.359167in}}%
\pgfusepath{clip}%
\pgfsetroundcap%
\pgfsetroundjoin%
\pgfsetlinewidth{1.003750pt}%
\definecolor{currentstroke}{rgb}{0.800000,0.800000,0.800000}%
\pgfsetstrokecolor{currentstroke}%
\pgfsetdash{}{0pt}%
\pgfpathmoveto{\pgfqpoint{1.081250in}{3.203777in}}%
\pgfpathlineto{\pgfqpoint{7.685000in}{3.203777in}}%
\pgfusepath{stroke}%
\end{pgfscope}%
\begin{pgfscope}%
\definecolor{textcolor}{rgb}{0.150000,0.150000,0.150000}%
\pgfsetstrokecolor{textcolor}%
\pgfsetfillcolor{textcolor}%
\pgftext[x=0.678422in,y=3.103758in,left,base]{\color{textcolor}\sffamily\fontsize{19.250000}{23.100000}\selectfont 30}%
\end{pgfscope}%
\begin{pgfscope}%
\pgfpathrectangle{\pgfqpoint{1.081250in}{1.022500in}}{\pgfqpoint{6.603750in}{3.359167in}}%
\pgfusepath{clip}%
\pgfsetroundcap%
\pgfsetroundjoin%
\pgfsetlinewidth{1.003750pt}%
\definecolor{currentstroke}{rgb}{0.800000,0.800000,0.800000}%
\pgfsetstrokecolor{currentstroke}%
\pgfsetdash{}{0pt}%
\pgfpathmoveto{\pgfqpoint{1.081250in}{3.930869in}}%
\pgfpathlineto{\pgfqpoint{7.685000in}{3.930869in}}%
\pgfusepath{stroke}%
\end{pgfscope}%
\begin{pgfscope}%
\definecolor{textcolor}{rgb}{0.150000,0.150000,0.150000}%
\pgfsetstrokecolor{textcolor}%
\pgfsetfillcolor{textcolor}%
\pgftext[x=0.678422in,y=3.830850in,left,base]{\color{textcolor}\sffamily\fontsize{19.250000}{23.100000}\selectfont 40}%
\end{pgfscope}%
\begin{pgfscope}%
\definecolor{textcolor}{rgb}{0.150000,0.150000,0.150000}%
\pgfsetstrokecolor{textcolor}%
\pgfsetfillcolor{textcolor}%
\pgftext[x=0.622867in,y=2.702083in,,bottom,rotate=90.000000]{\color{textcolor}\sffamily\fontsize{21.000000}{25.200000}\selectfont Instructions executed}%
\end{pgfscope}%
\begin{pgfscope}%
\pgfpathrectangle{\pgfqpoint{1.081250in}{1.022500in}}{\pgfqpoint{6.603750in}{3.359167in}}%
\pgfusepath{clip}%
\pgfsetbuttcap%
\pgfsetmiterjoin%
\definecolor{currentfill}{rgb}{0.347059,0.458824,0.641176}%
\pgfsetfillcolor{currentfill}%
\pgfsetlinewidth{1.003750pt}%
\definecolor{currentstroke}{rgb}{1.000000,1.000000,1.000000}%
\pgfsetstrokecolor{currentstroke}%
\pgfsetdash{}{0pt}%
\pgfpathmoveto{\pgfqpoint{1.175589in}{1.022500in}}%
\pgfpathlineto{\pgfqpoint{1.552946in}{1.022500in}}%
\pgfpathlineto{\pgfqpoint{1.552946in}{2.113139in}}%
\pgfpathlineto{\pgfqpoint{1.175589in}{2.113139in}}%
\pgfpathclose%
\pgfusepath{stroke,fill}%
\end{pgfscope}%
\begin{pgfscope}%
\pgfsetbuttcap%
\pgfsetmiterjoin%
\definecolor{currentfill}{rgb}{0.347059,0.458824,0.641176}%
\pgfsetfillcolor{currentfill}%
\pgfsetlinewidth{1.003750pt}%
\definecolor{currentstroke}{rgb}{1.000000,1.000000,1.000000}%
\pgfsetstrokecolor{currentstroke}%
\pgfsetdash{}{0pt}%
\pgfpathrectangle{\pgfqpoint{1.081250in}{1.022500in}}{\pgfqpoint{6.603750in}{3.359167in}}%
\pgfusepath{clip}%
\pgfpathmoveto{\pgfqpoint{1.175589in}{1.022500in}}%
\pgfpathlineto{\pgfqpoint{1.552946in}{1.022500in}}%
\pgfpathlineto{\pgfqpoint{1.552946in}{2.113139in}}%
\pgfpathlineto{\pgfqpoint{1.175589in}{2.113139in}}%
\pgfpathclose%
\pgfusepath{clip}%
\pgfsys@defobject{currentpattern}{\pgfqpoint{0in}{0in}}{\pgfqpoint{1in}{1in}}{%
\begin{pgfscope}%
\pgfpathrectangle{\pgfqpoint{0in}{0in}}{\pgfqpoint{1in}{1in}}%
\pgfusepath{clip}%
\pgfpathmoveto{\pgfqpoint{-0.500000in}{0.500000in}}%
\pgfpathlineto{\pgfqpoint{0.500000in}{1.500000in}}%
\pgfpathmoveto{\pgfqpoint{-0.444444in}{0.444444in}}%
\pgfpathlineto{\pgfqpoint{0.555556in}{1.444444in}}%
\pgfpathmoveto{\pgfqpoint{-0.388889in}{0.388889in}}%
\pgfpathlineto{\pgfqpoint{0.611111in}{1.388889in}}%
\pgfpathmoveto{\pgfqpoint{-0.333333in}{0.333333in}}%
\pgfpathlineto{\pgfqpoint{0.666667in}{1.333333in}}%
\pgfpathmoveto{\pgfqpoint{-0.277778in}{0.277778in}}%
\pgfpathlineto{\pgfqpoint{0.722222in}{1.277778in}}%
\pgfpathmoveto{\pgfqpoint{-0.222222in}{0.222222in}}%
\pgfpathlineto{\pgfqpoint{0.777778in}{1.222222in}}%
\pgfpathmoveto{\pgfqpoint{-0.166667in}{0.166667in}}%
\pgfpathlineto{\pgfqpoint{0.833333in}{1.166667in}}%
\pgfpathmoveto{\pgfqpoint{-0.111111in}{0.111111in}}%
\pgfpathlineto{\pgfqpoint{0.888889in}{1.111111in}}%
\pgfpathmoveto{\pgfqpoint{-0.055556in}{0.055556in}}%
\pgfpathlineto{\pgfqpoint{0.944444in}{1.055556in}}%
\pgfpathmoveto{\pgfqpoint{0.000000in}{0.000000in}}%
\pgfpathlineto{\pgfqpoint{1.000000in}{1.000000in}}%
\pgfpathmoveto{\pgfqpoint{0.055556in}{-0.055556in}}%
\pgfpathlineto{\pgfqpoint{1.055556in}{0.944444in}}%
\pgfpathmoveto{\pgfqpoint{0.111111in}{-0.111111in}}%
\pgfpathlineto{\pgfqpoint{1.111111in}{0.888889in}}%
\pgfpathmoveto{\pgfqpoint{0.166667in}{-0.166667in}}%
\pgfpathlineto{\pgfqpoint{1.166667in}{0.833333in}}%
\pgfpathmoveto{\pgfqpoint{0.222222in}{-0.222222in}}%
\pgfpathlineto{\pgfqpoint{1.222222in}{0.777778in}}%
\pgfpathmoveto{\pgfqpoint{0.277778in}{-0.277778in}}%
\pgfpathlineto{\pgfqpoint{1.277778in}{0.722222in}}%
\pgfpathmoveto{\pgfqpoint{0.333333in}{-0.333333in}}%
\pgfpathlineto{\pgfqpoint{1.333333in}{0.666667in}}%
\pgfpathmoveto{\pgfqpoint{0.388889in}{-0.388889in}}%
\pgfpathlineto{\pgfqpoint{1.388889in}{0.611111in}}%
\pgfpathmoveto{\pgfqpoint{0.444444in}{-0.444444in}}%
\pgfpathlineto{\pgfqpoint{1.444444in}{0.555556in}}%
\pgfpathmoveto{\pgfqpoint{0.500000in}{-0.500000in}}%
\pgfpathlineto{\pgfqpoint{1.500000in}{0.500000in}}%
\pgfusepath{stroke}%
\end{pgfscope}%
}%
\pgfsys@transformshift{1.175589in}{1.022500in}%
\pgfsys@useobject{currentpattern}{}%
\pgfsys@transformshift{1in}{0in}%
\pgfsys@transformshift{-1in}{0in}%
\pgfsys@transformshift{0in}{1in}%
\pgfsys@useobject{currentpattern}{}%
\pgfsys@transformshift{1in}{0in}%
\pgfsys@transformshift{-1in}{0in}%
\pgfsys@transformshift{0in}{1in}%
\end{pgfscope}%
\begin{pgfscope}%
\pgfpathrectangle{\pgfqpoint{1.081250in}{1.022500in}}{\pgfqpoint{6.603750in}{3.359167in}}%
\pgfusepath{clip}%
\pgfsetbuttcap%
\pgfsetmiterjoin%
\definecolor{currentfill}{rgb}{0.347059,0.458824,0.641176}%
\pgfsetfillcolor{currentfill}%
\pgfsetlinewidth{1.003750pt}%
\definecolor{currentstroke}{rgb}{1.000000,1.000000,1.000000}%
\pgfsetstrokecolor{currentstroke}%
\pgfsetdash{}{0pt}%
\pgfpathmoveto{\pgfqpoint{2.118982in}{1.022500in}}%
\pgfpathlineto{\pgfqpoint{2.496339in}{1.022500in}}%
\pgfpathlineto{\pgfqpoint{2.496339in}{2.040429in}}%
\pgfpathlineto{\pgfqpoint{2.118982in}{2.040429in}}%
\pgfpathclose%
\pgfusepath{stroke,fill}%
\end{pgfscope}%
\begin{pgfscope}%
\pgfsetbuttcap%
\pgfsetmiterjoin%
\definecolor{currentfill}{rgb}{0.347059,0.458824,0.641176}%
\pgfsetfillcolor{currentfill}%
\pgfsetlinewidth{1.003750pt}%
\definecolor{currentstroke}{rgb}{1.000000,1.000000,1.000000}%
\pgfsetstrokecolor{currentstroke}%
\pgfsetdash{}{0pt}%
\pgfpathrectangle{\pgfqpoint{1.081250in}{1.022500in}}{\pgfqpoint{6.603750in}{3.359167in}}%
\pgfusepath{clip}%
\pgfpathmoveto{\pgfqpoint{2.118982in}{1.022500in}}%
\pgfpathlineto{\pgfqpoint{2.496339in}{1.022500in}}%
\pgfpathlineto{\pgfqpoint{2.496339in}{2.040429in}}%
\pgfpathlineto{\pgfqpoint{2.118982in}{2.040429in}}%
\pgfpathclose%
\pgfusepath{clip}%
\pgfsys@defobject{currentpattern}{\pgfqpoint{0in}{0in}}{\pgfqpoint{1in}{1in}}{%
\begin{pgfscope}%
\pgfpathrectangle{\pgfqpoint{0in}{0in}}{\pgfqpoint{1in}{1in}}%
\pgfusepath{clip}%
\pgfpathmoveto{\pgfqpoint{-0.500000in}{0.500000in}}%
\pgfpathlineto{\pgfqpoint{0.500000in}{1.500000in}}%
\pgfpathmoveto{\pgfqpoint{-0.444444in}{0.444444in}}%
\pgfpathlineto{\pgfqpoint{0.555556in}{1.444444in}}%
\pgfpathmoveto{\pgfqpoint{-0.388889in}{0.388889in}}%
\pgfpathlineto{\pgfqpoint{0.611111in}{1.388889in}}%
\pgfpathmoveto{\pgfqpoint{-0.333333in}{0.333333in}}%
\pgfpathlineto{\pgfqpoint{0.666667in}{1.333333in}}%
\pgfpathmoveto{\pgfqpoint{-0.277778in}{0.277778in}}%
\pgfpathlineto{\pgfqpoint{0.722222in}{1.277778in}}%
\pgfpathmoveto{\pgfqpoint{-0.222222in}{0.222222in}}%
\pgfpathlineto{\pgfqpoint{0.777778in}{1.222222in}}%
\pgfpathmoveto{\pgfqpoint{-0.166667in}{0.166667in}}%
\pgfpathlineto{\pgfqpoint{0.833333in}{1.166667in}}%
\pgfpathmoveto{\pgfqpoint{-0.111111in}{0.111111in}}%
\pgfpathlineto{\pgfqpoint{0.888889in}{1.111111in}}%
\pgfpathmoveto{\pgfqpoint{-0.055556in}{0.055556in}}%
\pgfpathlineto{\pgfqpoint{0.944444in}{1.055556in}}%
\pgfpathmoveto{\pgfqpoint{0.000000in}{0.000000in}}%
\pgfpathlineto{\pgfqpoint{1.000000in}{1.000000in}}%
\pgfpathmoveto{\pgfqpoint{0.055556in}{-0.055556in}}%
\pgfpathlineto{\pgfqpoint{1.055556in}{0.944444in}}%
\pgfpathmoveto{\pgfqpoint{0.111111in}{-0.111111in}}%
\pgfpathlineto{\pgfqpoint{1.111111in}{0.888889in}}%
\pgfpathmoveto{\pgfqpoint{0.166667in}{-0.166667in}}%
\pgfpathlineto{\pgfqpoint{1.166667in}{0.833333in}}%
\pgfpathmoveto{\pgfqpoint{0.222222in}{-0.222222in}}%
\pgfpathlineto{\pgfqpoint{1.222222in}{0.777778in}}%
\pgfpathmoveto{\pgfqpoint{0.277778in}{-0.277778in}}%
\pgfpathlineto{\pgfqpoint{1.277778in}{0.722222in}}%
\pgfpathmoveto{\pgfqpoint{0.333333in}{-0.333333in}}%
\pgfpathlineto{\pgfqpoint{1.333333in}{0.666667in}}%
\pgfpathmoveto{\pgfqpoint{0.388889in}{-0.388889in}}%
\pgfpathlineto{\pgfqpoint{1.388889in}{0.611111in}}%
\pgfpathmoveto{\pgfqpoint{0.444444in}{-0.444444in}}%
\pgfpathlineto{\pgfqpoint{1.444444in}{0.555556in}}%
\pgfpathmoveto{\pgfqpoint{0.500000in}{-0.500000in}}%
\pgfpathlineto{\pgfqpoint{1.500000in}{0.500000in}}%
\pgfusepath{stroke}%
\end{pgfscope}%
}%
\pgfsys@transformshift{2.118982in}{1.022500in}%
\pgfsys@useobject{currentpattern}{}%
\pgfsys@transformshift{1in}{0in}%
\pgfsys@transformshift{-1in}{0in}%
\pgfsys@transformshift{0in}{1in}%
\pgfsys@useobject{currentpattern}{}%
\pgfsys@transformshift{1in}{0in}%
\pgfsys@transformshift{-1in}{0in}%
\pgfsys@transformshift{0in}{1in}%
\end{pgfscope}%
\begin{pgfscope}%
\pgfpathrectangle{\pgfqpoint{1.081250in}{1.022500in}}{\pgfqpoint{6.603750in}{3.359167in}}%
\pgfusepath{clip}%
\pgfsetbuttcap%
\pgfsetmiterjoin%
\definecolor{currentfill}{rgb}{0.347059,0.458824,0.641176}%
\pgfsetfillcolor{currentfill}%
\pgfsetlinewidth{1.003750pt}%
\definecolor{currentstroke}{rgb}{1.000000,1.000000,1.000000}%
\pgfsetstrokecolor{currentstroke}%
\pgfsetdash{}{0pt}%
\pgfpathmoveto{\pgfqpoint{3.062375in}{1.022500in}}%
\pgfpathlineto{\pgfqpoint{3.439732in}{1.022500in}}%
\pgfpathlineto{\pgfqpoint{3.439732in}{2.258557in}}%
\pgfpathlineto{\pgfqpoint{3.062375in}{2.258557in}}%
\pgfpathclose%
\pgfusepath{stroke,fill}%
\end{pgfscope}%
\begin{pgfscope}%
\pgfsetbuttcap%
\pgfsetmiterjoin%
\definecolor{currentfill}{rgb}{0.347059,0.458824,0.641176}%
\pgfsetfillcolor{currentfill}%
\pgfsetlinewidth{1.003750pt}%
\definecolor{currentstroke}{rgb}{1.000000,1.000000,1.000000}%
\pgfsetstrokecolor{currentstroke}%
\pgfsetdash{}{0pt}%
\pgfpathrectangle{\pgfqpoint{1.081250in}{1.022500in}}{\pgfqpoint{6.603750in}{3.359167in}}%
\pgfusepath{clip}%
\pgfpathmoveto{\pgfqpoint{3.062375in}{1.022500in}}%
\pgfpathlineto{\pgfqpoint{3.439732in}{1.022500in}}%
\pgfpathlineto{\pgfqpoint{3.439732in}{2.258557in}}%
\pgfpathlineto{\pgfqpoint{3.062375in}{2.258557in}}%
\pgfpathclose%
\pgfusepath{clip}%
\pgfsys@defobject{currentpattern}{\pgfqpoint{0in}{0in}}{\pgfqpoint{1in}{1in}}{%
\begin{pgfscope}%
\pgfpathrectangle{\pgfqpoint{0in}{0in}}{\pgfqpoint{1in}{1in}}%
\pgfusepath{clip}%
\pgfpathmoveto{\pgfqpoint{-0.500000in}{0.500000in}}%
\pgfpathlineto{\pgfqpoint{0.500000in}{1.500000in}}%
\pgfpathmoveto{\pgfqpoint{-0.444444in}{0.444444in}}%
\pgfpathlineto{\pgfqpoint{0.555556in}{1.444444in}}%
\pgfpathmoveto{\pgfqpoint{-0.388889in}{0.388889in}}%
\pgfpathlineto{\pgfqpoint{0.611111in}{1.388889in}}%
\pgfpathmoveto{\pgfqpoint{-0.333333in}{0.333333in}}%
\pgfpathlineto{\pgfqpoint{0.666667in}{1.333333in}}%
\pgfpathmoveto{\pgfqpoint{-0.277778in}{0.277778in}}%
\pgfpathlineto{\pgfqpoint{0.722222in}{1.277778in}}%
\pgfpathmoveto{\pgfqpoint{-0.222222in}{0.222222in}}%
\pgfpathlineto{\pgfqpoint{0.777778in}{1.222222in}}%
\pgfpathmoveto{\pgfqpoint{-0.166667in}{0.166667in}}%
\pgfpathlineto{\pgfqpoint{0.833333in}{1.166667in}}%
\pgfpathmoveto{\pgfqpoint{-0.111111in}{0.111111in}}%
\pgfpathlineto{\pgfqpoint{0.888889in}{1.111111in}}%
\pgfpathmoveto{\pgfqpoint{-0.055556in}{0.055556in}}%
\pgfpathlineto{\pgfqpoint{0.944444in}{1.055556in}}%
\pgfpathmoveto{\pgfqpoint{0.000000in}{0.000000in}}%
\pgfpathlineto{\pgfqpoint{1.000000in}{1.000000in}}%
\pgfpathmoveto{\pgfqpoint{0.055556in}{-0.055556in}}%
\pgfpathlineto{\pgfqpoint{1.055556in}{0.944444in}}%
\pgfpathmoveto{\pgfqpoint{0.111111in}{-0.111111in}}%
\pgfpathlineto{\pgfqpoint{1.111111in}{0.888889in}}%
\pgfpathmoveto{\pgfqpoint{0.166667in}{-0.166667in}}%
\pgfpathlineto{\pgfqpoint{1.166667in}{0.833333in}}%
\pgfpathmoveto{\pgfqpoint{0.222222in}{-0.222222in}}%
\pgfpathlineto{\pgfqpoint{1.222222in}{0.777778in}}%
\pgfpathmoveto{\pgfqpoint{0.277778in}{-0.277778in}}%
\pgfpathlineto{\pgfqpoint{1.277778in}{0.722222in}}%
\pgfpathmoveto{\pgfqpoint{0.333333in}{-0.333333in}}%
\pgfpathlineto{\pgfqpoint{1.333333in}{0.666667in}}%
\pgfpathmoveto{\pgfqpoint{0.388889in}{-0.388889in}}%
\pgfpathlineto{\pgfqpoint{1.388889in}{0.611111in}}%
\pgfpathmoveto{\pgfqpoint{0.444444in}{-0.444444in}}%
\pgfpathlineto{\pgfqpoint{1.444444in}{0.555556in}}%
\pgfpathmoveto{\pgfqpoint{0.500000in}{-0.500000in}}%
\pgfpathlineto{\pgfqpoint{1.500000in}{0.500000in}}%
\pgfusepath{stroke}%
\end{pgfscope}%
}%
\pgfsys@transformshift{3.062375in}{1.022500in}%
\pgfsys@useobject{currentpattern}{}%
\pgfsys@transformshift{1in}{0in}%
\pgfsys@transformshift{-1in}{0in}%
\pgfsys@transformshift{0in}{1in}%
\pgfsys@useobject{currentpattern}{}%
\pgfsys@transformshift{1in}{0in}%
\pgfsys@transformshift{-1in}{0in}%
\pgfsys@transformshift{0in}{1in}%
\end{pgfscope}%
\begin{pgfscope}%
\pgfpathrectangle{\pgfqpoint{1.081250in}{1.022500in}}{\pgfqpoint{6.603750in}{3.359167in}}%
\pgfusepath{clip}%
\pgfsetbuttcap%
\pgfsetmiterjoin%
\definecolor{currentfill}{rgb}{0.347059,0.458824,0.641176}%
\pgfsetfillcolor{currentfill}%
\pgfsetlinewidth{1.003750pt}%
\definecolor{currentstroke}{rgb}{1.000000,1.000000,1.000000}%
\pgfsetstrokecolor{currentstroke}%
\pgfsetdash{}{0pt}%
\pgfpathmoveto{\pgfqpoint{4.005768in}{1.022500in}}%
\pgfpathlineto{\pgfqpoint{4.383125in}{1.022500in}}%
\pgfpathlineto{\pgfqpoint{4.383125in}{2.258557in}}%
\pgfpathlineto{\pgfqpoint{4.005768in}{2.258557in}}%
\pgfpathclose%
\pgfusepath{stroke,fill}%
\end{pgfscope}%
\begin{pgfscope}%
\pgfsetbuttcap%
\pgfsetmiterjoin%
\definecolor{currentfill}{rgb}{0.347059,0.458824,0.641176}%
\pgfsetfillcolor{currentfill}%
\pgfsetlinewidth{1.003750pt}%
\definecolor{currentstroke}{rgb}{1.000000,1.000000,1.000000}%
\pgfsetstrokecolor{currentstroke}%
\pgfsetdash{}{0pt}%
\pgfpathrectangle{\pgfqpoint{1.081250in}{1.022500in}}{\pgfqpoint{6.603750in}{3.359167in}}%
\pgfusepath{clip}%
\pgfpathmoveto{\pgfqpoint{4.005768in}{1.022500in}}%
\pgfpathlineto{\pgfqpoint{4.383125in}{1.022500in}}%
\pgfpathlineto{\pgfqpoint{4.383125in}{2.258557in}}%
\pgfpathlineto{\pgfqpoint{4.005768in}{2.258557in}}%
\pgfpathclose%
\pgfusepath{clip}%
\pgfsys@defobject{currentpattern}{\pgfqpoint{0in}{0in}}{\pgfqpoint{1in}{1in}}{%
\begin{pgfscope}%
\pgfpathrectangle{\pgfqpoint{0in}{0in}}{\pgfqpoint{1in}{1in}}%
\pgfusepath{clip}%
\pgfpathmoveto{\pgfqpoint{-0.500000in}{0.500000in}}%
\pgfpathlineto{\pgfqpoint{0.500000in}{1.500000in}}%
\pgfpathmoveto{\pgfqpoint{-0.444444in}{0.444444in}}%
\pgfpathlineto{\pgfqpoint{0.555556in}{1.444444in}}%
\pgfpathmoveto{\pgfqpoint{-0.388889in}{0.388889in}}%
\pgfpathlineto{\pgfqpoint{0.611111in}{1.388889in}}%
\pgfpathmoveto{\pgfqpoint{-0.333333in}{0.333333in}}%
\pgfpathlineto{\pgfqpoint{0.666667in}{1.333333in}}%
\pgfpathmoveto{\pgfqpoint{-0.277778in}{0.277778in}}%
\pgfpathlineto{\pgfqpoint{0.722222in}{1.277778in}}%
\pgfpathmoveto{\pgfqpoint{-0.222222in}{0.222222in}}%
\pgfpathlineto{\pgfqpoint{0.777778in}{1.222222in}}%
\pgfpathmoveto{\pgfqpoint{-0.166667in}{0.166667in}}%
\pgfpathlineto{\pgfqpoint{0.833333in}{1.166667in}}%
\pgfpathmoveto{\pgfqpoint{-0.111111in}{0.111111in}}%
\pgfpathlineto{\pgfqpoint{0.888889in}{1.111111in}}%
\pgfpathmoveto{\pgfqpoint{-0.055556in}{0.055556in}}%
\pgfpathlineto{\pgfqpoint{0.944444in}{1.055556in}}%
\pgfpathmoveto{\pgfqpoint{0.000000in}{0.000000in}}%
\pgfpathlineto{\pgfqpoint{1.000000in}{1.000000in}}%
\pgfpathmoveto{\pgfqpoint{0.055556in}{-0.055556in}}%
\pgfpathlineto{\pgfqpoint{1.055556in}{0.944444in}}%
\pgfpathmoveto{\pgfqpoint{0.111111in}{-0.111111in}}%
\pgfpathlineto{\pgfqpoint{1.111111in}{0.888889in}}%
\pgfpathmoveto{\pgfqpoint{0.166667in}{-0.166667in}}%
\pgfpathlineto{\pgfqpoint{1.166667in}{0.833333in}}%
\pgfpathmoveto{\pgfqpoint{0.222222in}{-0.222222in}}%
\pgfpathlineto{\pgfqpoint{1.222222in}{0.777778in}}%
\pgfpathmoveto{\pgfqpoint{0.277778in}{-0.277778in}}%
\pgfpathlineto{\pgfqpoint{1.277778in}{0.722222in}}%
\pgfpathmoveto{\pgfqpoint{0.333333in}{-0.333333in}}%
\pgfpathlineto{\pgfqpoint{1.333333in}{0.666667in}}%
\pgfpathmoveto{\pgfqpoint{0.388889in}{-0.388889in}}%
\pgfpathlineto{\pgfqpoint{1.388889in}{0.611111in}}%
\pgfpathmoveto{\pgfqpoint{0.444444in}{-0.444444in}}%
\pgfpathlineto{\pgfqpoint{1.444444in}{0.555556in}}%
\pgfpathmoveto{\pgfqpoint{0.500000in}{-0.500000in}}%
\pgfpathlineto{\pgfqpoint{1.500000in}{0.500000in}}%
\pgfusepath{stroke}%
\end{pgfscope}%
}%
\pgfsys@transformshift{4.005768in}{1.022500in}%
\pgfsys@useobject{currentpattern}{}%
\pgfsys@transformshift{1in}{0in}%
\pgfsys@transformshift{-1in}{0in}%
\pgfsys@transformshift{0in}{1in}%
\pgfsys@useobject{currentpattern}{}%
\pgfsys@transformshift{1in}{0in}%
\pgfsys@transformshift{-1in}{0in}%
\pgfsys@transformshift{0in}{1in}%
\end{pgfscope}%
\begin{pgfscope}%
\pgfpathrectangle{\pgfqpoint{1.081250in}{1.022500in}}{\pgfqpoint{6.603750in}{3.359167in}}%
\pgfusepath{clip}%
\pgfsetbuttcap%
\pgfsetmiterjoin%
\definecolor{currentfill}{rgb}{0.347059,0.458824,0.641176}%
\pgfsetfillcolor{currentfill}%
\pgfsetlinewidth{1.003750pt}%
\definecolor{currentstroke}{rgb}{1.000000,1.000000,1.000000}%
\pgfsetstrokecolor{currentstroke}%
\pgfsetdash{}{0pt}%
\pgfpathmoveto{\pgfqpoint{4.949161in}{1.022500in}}%
\pgfpathlineto{\pgfqpoint{5.326518in}{1.022500in}}%
\pgfpathlineto{\pgfqpoint{5.326518in}{2.040429in}}%
\pgfpathlineto{\pgfqpoint{4.949161in}{2.040429in}}%
\pgfpathclose%
\pgfusepath{stroke,fill}%
\end{pgfscope}%
\begin{pgfscope}%
\pgfsetbuttcap%
\pgfsetmiterjoin%
\definecolor{currentfill}{rgb}{0.347059,0.458824,0.641176}%
\pgfsetfillcolor{currentfill}%
\pgfsetlinewidth{1.003750pt}%
\definecolor{currentstroke}{rgb}{1.000000,1.000000,1.000000}%
\pgfsetstrokecolor{currentstroke}%
\pgfsetdash{}{0pt}%
\pgfpathrectangle{\pgfqpoint{1.081250in}{1.022500in}}{\pgfqpoint{6.603750in}{3.359167in}}%
\pgfusepath{clip}%
\pgfpathmoveto{\pgfqpoint{4.949161in}{1.022500in}}%
\pgfpathlineto{\pgfqpoint{5.326518in}{1.022500in}}%
\pgfpathlineto{\pgfqpoint{5.326518in}{2.040429in}}%
\pgfpathlineto{\pgfqpoint{4.949161in}{2.040429in}}%
\pgfpathclose%
\pgfusepath{clip}%
\pgfsys@defobject{currentpattern}{\pgfqpoint{0in}{0in}}{\pgfqpoint{1in}{1in}}{%
\begin{pgfscope}%
\pgfpathrectangle{\pgfqpoint{0in}{0in}}{\pgfqpoint{1in}{1in}}%
\pgfusepath{clip}%
\pgfpathmoveto{\pgfqpoint{-0.500000in}{0.500000in}}%
\pgfpathlineto{\pgfqpoint{0.500000in}{1.500000in}}%
\pgfpathmoveto{\pgfqpoint{-0.444444in}{0.444444in}}%
\pgfpathlineto{\pgfqpoint{0.555556in}{1.444444in}}%
\pgfpathmoveto{\pgfqpoint{-0.388889in}{0.388889in}}%
\pgfpathlineto{\pgfqpoint{0.611111in}{1.388889in}}%
\pgfpathmoveto{\pgfqpoint{-0.333333in}{0.333333in}}%
\pgfpathlineto{\pgfqpoint{0.666667in}{1.333333in}}%
\pgfpathmoveto{\pgfqpoint{-0.277778in}{0.277778in}}%
\pgfpathlineto{\pgfqpoint{0.722222in}{1.277778in}}%
\pgfpathmoveto{\pgfqpoint{-0.222222in}{0.222222in}}%
\pgfpathlineto{\pgfqpoint{0.777778in}{1.222222in}}%
\pgfpathmoveto{\pgfqpoint{-0.166667in}{0.166667in}}%
\pgfpathlineto{\pgfqpoint{0.833333in}{1.166667in}}%
\pgfpathmoveto{\pgfqpoint{-0.111111in}{0.111111in}}%
\pgfpathlineto{\pgfqpoint{0.888889in}{1.111111in}}%
\pgfpathmoveto{\pgfqpoint{-0.055556in}{0.055556in}}%
\pgfpathlineto{\pgfqpoint{0.944444in}{1.055556in}}%
\pgfpathmoveto{\pgfqpoint{0.000000in}{0.000000in}}%
\pgfpathlineto{\pgfqpoint{1.000000in}{1.000000in}}%
\pgfpathmoveto{\pgfqpoint{0.055556in}{-0.055556in}}%
\pgfpathlineto{\pgfqpoint{1.055556in}{0.944444in}}%
\pgfpathmoveto{\pgfqpoint{0.111111in}{-0.111111in}}%
\pgfpathlineto{\pgfqpoint{1.111111in}{0.888889in}}%
\pgfpathmoveto{\pgfqpoint{0.166667in}{-0.166667in}}%
\pgfpathlineto{\pgfqpoint{1.166667in}{0.833333in}}%
\pgfpathmoveto{\pgfqpoint{0.222222in}{-0.222222in}}%
\pgfpathlineto{\pgfqpoint{1.222222in}{0.777778in}}%
\pgfpathmoveto{\pgfqpoint{0.277778in}{-0.277778in}}%
\pgfpathlineto{\pgfqpoint{1.277778in}{0.722222in}}%
\pgfpathmoveto{\pgfqpoint{0.333333in}{-0.333333in}}%
\pgfpathlineto{\pgfqpoint{1.333333in}{0.666667in}}%
\pgfpathmoveto{\pgfqpoint{0.388889in}{-0.388889in}}%
\pgfpathlineto{\pgfqpoint{1.388889in}{0.611111in}}%
\pgfpathmoveto{\pgfqpoint{0.444444in}{-0.444444in}}%
\pgfpathlineto{\pgfqpoint{1.444444in}{0.555556in}}%
\pgfpathmoveto{\pgfqpoint{0.500000in}{-0.500000in}}%
\pgfpathlineto{\pgfqpoint{1.500000in}{0.500000in}}%
\pgfusepath{stroke}%
\end{pgfscope}%
}%
\pgfsys@transformshift{4.949161in}{1.022500in}%
\pgfsys@useobject{currentpattern}{}%
\pgfsys@transformshift{1in}{0in}%
\pgfsys@transformshift{-1in}{0in}%
\pgfsys@transformshift{0in}{1in}%
\pgfsys@useobject{currentpattern}{}%
\pgfsys@transformshift{1in}{0in}%
\pgfsys@transformshift{-1in}{0in}%
\pgfsys@transformshift{0in}{1in}%
\end{pgfscope}%
\begin{pgfscope}%
\pgfpathrectangle{\pgfqpoint{1.081250in}{1.022500in}}{\pgfqpoint{6.603750in}{3.359167in}}%
\pgfusepath{clip}%
\pgfsetbuttcap%
\pgfsetmiterjoin%
\definecolor{currentfill}{rgb}{0.347059,0.458824,0.641176}%
\pgfsetfillcolor{currentfill}%
\pgfsetlinewidth{1.003750pt}%
\definecolor{currentstroke}{rgb}{1.000000,1.000000,1.000000}%
\pgfsetstrokecolor{currentstroke}%
\pgfsetdash{}{0pt}%
\pgfpathmoveto{\pgfqpoint{5.892554in}{1.022500in}}%
\pgfpathlineto{\pgfqpoint{6.269911in}{1.022500in}}%
\pgfpathlineto{\pgfqpoint{6.269911in}{1.749592in}}%
\pgfpathlineto{\pgfqpoint{5.892554in}{1.749592in}}%
\pgfpathclose%
\pgfusepath{stroke,fill}%
\end{pgfscope}%
\begin{pgfscope}%
\pgfsetbuttcap%
\pgfsetmiterjoin%
\definecolor{currentfill}{rgb}{0.347059,0.458824,0.641176}%
\pgfsetfillcolor{currentfill}%
\pgfsetlinewidth{1.003750pt}%
\definecolor{currentstroke}{rgb}{1.000000,1.000000,1.000000}%
\pgfsetstrokecolor{currentstroke}%
\pgfsetdash{}{0pt}%
\pgfpathrectangle{\pgfqpoint{1.081250in}{1.022500in}}{\pgfqpoint{6.603750in}{3.359167in}}%
\pgfusepath{clip}%
\pgfpathmoveto{\pgfqpoint{5.892554in}{1.022500in}}%
\pgfpathlineto{\pgfqpoint{6.269911in}{1.022500in}}%
\pgfpathlineto{\pgfqpoint{6.269911in}{1.749592in}}%
\pgfpathlineto{\pgfqpoint{5.892554in}{1.749592in}}%
\pgfpathclose%
\pgfusepath{clip}%
\pgfsys@defobject{currentpattern}{\pgfqpoint{0in}{0in}}{\pgfqpoint{1in}{1in}}{%
\begin{pgfscope}%
\pgfpathrectangle{\pgfqpoint{0in}{0in}}{\pgfqpoint{1in}{1in}}%
\pgfusepath{clip}%
\pgfpathmoveto{\pgfqpoint{-0.500000in}{0.500000in}}%
\pgfpathlineto{\pgfqpoint{0.500000in}{1.500000in}}%
\pgfpathmoveto{\pgfqpoint{-0.444444in}{0.444444in}}%
\pgfpathlineto{\pgfqpoint{0.555556in}{1.444444in}}%
\pgfpathmoveto{\pgfqpoint{-0.388889in}{0.388889in}}%
\pgfpathlineto{\pgfqpoint{0.611111in}{1.388889in}}%
\pgfpathmoveto{\pgfqpoint{-0.333333in}{0.333333in}}%
\pgfpathlineto{\pgfqpoint{0.666667in}{1.333333in}}%
\pgfpathmoveto{\pgfqpoint{-0.277778in}{0.277778in}}%
\pgfpathlineto{\pgfqpoint{0.722222in}{1.277778in}}%
\pgfpathmoveto{\pgfqpoint{-0.222222in}{0.222222in}}%
\pgfpathlineto{\pgfqpoint{0.777778in}{1.222222in}}%
\pgfpathmoveto{\pgfqpoint{-0.166667in}{0.166667in}}%
\pgfpathlineto{\pgfqpoint{0.833333in}{1.166667in}}%
\pgfpathmoveto{\pgfqpoint{-0.111111in}{0.111111in}}%
\pgfpathlineto{\pgfqpoint{0.888889in}{1.111111in}}%
\pgfpathmoveto{\pgfqpoint{-0.055556in}{0.055556in}}%
\pgfpathlineto{\pgfqpoint{0.944444in}{1.055556in}}%
\pgfpathmoveto{\pgfqpoint{0.000000in}{0.000000in}}%
\pgfpathlineto{\pgfqpoint{1.000000in}{1.000000in}}%
\pgfpathmoveto{\pgfqpoint{0.055556in}{-0.055556in}}%
\pgfpathlineto{\pgfqpoint{1.055556in}{0.944444in}}%
\pgfpathmoveto{\pgfqpoint{0.111111in}{-0.111111in}}%
\pgfpathlineto{\pgfqpoint{1.111111in}{0.888889in}}%
\pgfpathmoveto{\pgfqpoint{0.166667in}{-0.166667in}}%
\pgfpathlineto{\pgfqpoint{1.166667in}{0.833333in}}%
\pgfpathmoveto{\pgfqpoint{0.222222in}{-0.222222in}}%
\pgfpathlineto{\pgfqpoint{1.222222in}{0.777778in}}%
\pgfpathmoveto{\pgfqpoint{0.277778in}{-0.277778in}}%
\pgfpathlineto{\pgfqpoint{1.277778in}{0.722222in}}%
\pgfpathmoveto{\pgfqpoint{0.333333in}{-0.333333in}}%
\pgfpathlineto{\pgfqpoint{1.333333in}{0.666667in}}%
\pgfpathmoveto{\pgfqpoint{0.388889in}{-0.388889in}}%
\pgfpathlineto{\pgfqpoint{1.388889in}{0.611111in}}%
\pgfpathmoveto{\pgfqpoint{0.444444in}{-0.444444in}}%
\pgfpathlineto{\pgfqpoint{1.444444in}{0.555556in}}%
\pgfpathmoveto{\pgfqpoint{0.500000in}{-0.500000in}}%
\pgfpathlineto{\pgfqpoint{1.500000in}{0.500000in}}%
\pgfusepath{stroke}%
\end{pgfscope}%
}%
\pgfsys@transformshift{5.892554in}{1.022500in}%
\pgfsys@useobject{currentpattern}{}%
\pgfsys@transformshift{1in}{0in}%
\pgfsys@transformshift{-1in}{0in}%
\pgfsys@transformshift{0in}{1in}%
\end{pgfscope}%
\begin{pgfscope}%
\pgfpathrectangle{\pgfqpoint{1.081250in}{1.022500in}}{\pgfqpoint{6.603750in}{3.359167in}}%
\pgfusepath{clip}%
\pgfsetbuttcap%
\pgfsetmiterjoin%
\definecolor{currentfill}{rgb}{0.347059,0.458824,0.641176}%
\pgfsetfillcolor{currentfill}%
\pgfsetlinewidth{1.003750pt}%
\definecolor{currentstroke}{rgb}{1.000000,1.000000,1.000000}%
\pgfsetstrokecolor{currentstroke}%
\pgfsetdash{}{0pt}%
\pgfpathmoveto{\pgfqpoint{6.835946in}{1.022500in}}%
\pgfpathlineto{\pgfqpoint{7.213304in}{1.022500in}}%
\pgfpathlineto{\pgfqpoint{7.213304in}{1.749592in}}%
\pgfpathlineto{\pgfqpoint{6.835946in}{1.749592in}}%
\pgfpathclose%
\pgfusepath{stroke,fill}%
\end{pgfscope}%
\begin{pgfscope}%
\pgfsetbuttcap%
\pgfsetmiterjoin%
\definecolor{currentfill}{rgb}{0.347059,0.458824,0.641176}%
\pgfsetfillcolor{currentfill}%
\pgfsetlinewidth{1.003750pt}%
\definecolor{currentstroke}{rgb}{1.000000,1.000000,1.000000}%
\pgfsetstrokecolor{currentstroke}%
\pgfsetdash{}{0pt}%
\pgfpathrectangle{\pgfqpoint{1.081250in}{1.022500in}}{\pgfqpoint{6.603750in}{3.359167in}}%
\pgfusepath{clip}%
\pgfpathmoveto{\pgfqpoint{6.835946in}{1.022500in}}%
\pgfpathlineto{\pgfqpoint{7.213304in}{1.022500in}}%
\pgfpathlineto{\pgfqpoint{7.213304in}{1.749592in}}%
\pgfpathlineto{\pgfqpoint{6.835946in}{1.749592in}}%
\pgfpathclose%
\pgfusepath{clip}%
\pgfsys@defobject{currentpattern}{\pgfqpoint{0in}{0in}}{\pgfqpoint{1in}{1in}}{%
\begin{pgfscope}%
\pgfpathrectangle{\pgfqpoint{0in}{0in}}{\pgfqpoint{1in}{1in}}%
\pgfusepath{clip}%
\pgfpathmoveto{\pgfqpoint{-0.500000in}{0.500000in}}%
\pgfpathlineto{\pgfqpoint{0.500000in}{1.500000in}}%
\pgfpathmoveto{\pgfqpoint{-0.444444in}{0.444444in}}%
\pgfpathlineto{\pgfqpoint{0.555556in}{1.444444in}}%
\pgfpathmoveto{\pgfqpoint{-0.388889in}{0.388889in}}%
\pgfpathlineto{\pgfqpoint{0.611111in}{1.388889in}}%
\pgfpathmoveto{\pgfqpoint{-0.333333in}{0.333333in}}%
\pgfpathlineto{\pgfqpoint{0.666667in}{1.333333in}}%
\pgfpathmoveto{\pgfqpoint{-0.277778in}{0.277778in}}%
\pgfpathlineto{\pgfqpoint{0.722222in}{1.277778in}}%
\pgfpathmoveto{\pgfqpoint{-0.222222in}{0.222222in}}%
\pgfpathlineto{\pgfqpoint{0.777778in}{1.222222in}}%
\pgfpathmoveto{\pgfqpoint{-0.166667in}{0.166667in}}%
\pgfpathlineto{\pgfqpoint{0.833333in}{1.166667in}}%
\pgfpathmoveto{\pgfqpoint{-0.111111in}{0.111111in}}%
\pgfpathlineto{\pgfqpoint{0.888889in}{1.111111in}}%
\pgfpathmoveto{\pgfqpoint{-0.055556in}{0.055556in}}%
\pgfpathlineto{\pgfqpoint{0.944444in}{1.055556in}}%
\pgfpathmoveto{\pgfqpoint{0.000000in}{0.000000in}}%
\pgfpathlineto{\pgfqpoint{1.000000in}{1.000000in}}%
\pgfpathmoveto{\pgfqpoint{0.055556in}{-0.055556in}}%
\pgfpathlineto{\pgfqpoint{1.055556in}{0.944444in}}%
\pgfpathmoveto{\pgfqpoint{0.111111in}{-0.111111in}}%
\pgfpathlineto{\pgfqpoint{1.111111in}{0.888889in}}%
\pgfpathmoveto{\pgfqpoint{0.166667in}{-0.166667in}}%
\pgfpathlineto{\pgfqpoint{1.166667in}{0.833333in}}%
\pgfpathmoveto{\pgfqpoint{0.222222in}{-0.222222in}}%
\pgfpathlineto{\pgfqpoint{1.222222in}{0.777778in}}%
\pgfpathmoveto{\pgfqpoint{0.277778in}{-0.277778in}}%
\pgfpathlineto{\pgfqpoint{1.277778in}{0.722222in}}%
\pgfpathmoveto{\pgfqpoint{0.333333in}{-0.333333in}}%
\pgfpathlineto{\pgfqpoint{1.333333in}{0.666667in}}%
\pgfpathmoveto{\pgfqpoint{0.388889in}{-0.388889in}}%
\pgfpathlineto{\pgfqpoint{1.388889in}{0.611111in}}%
\pgfpathmoveto{\pgfqpoint{0.444444in}{-0.444444in}}%
\pgfpathlineto{\pgfqpoint{1.444444in}{0.555556in}}%
\pgfpathmoveto{\pgfqpoint{0.500000in}{-0.500000in}}%
\pgfpathlineto{\pgfqpoint{1.500000in}{0.500000in}}%
\pgfusepath{stroke}%
\end{pgfscope}%
}%
\pgfsys@transformshift{6.835946in}{1.022500in}%
\pgfsys@useobject{currentpattern}{}%
\pgfsys@transformshift{1in}{0in}%
\pgfsys@transformshift{-1in}{0in}%
\pgfsys@transformshift{0in}{1in}%
\end{pgfscope}%
\begin{pgfscope}%
\pgfpathrectangle{\pgfqpoint{1.081250in}{1.022500in}}{\pgfqpoint{6.603750in}{3.359167in}}%
\pgfusepath{clip}%
\pgfsetbuttcap%
\pgfsetmiterjoin%
\definecolor{currentfill}{rgb}{0.798529,0.536765,0.389706}%
\pgfsetfillcolor{currentfill}%
\pgfsetlinewidth{1.003750pt}%
\definecolor{currentstroke}{rgb}{1.000000,1.000000,1.000000}%
\pgfsetstrokecolor{currentstroke}%
\pgfsetdash{}{0pt}%
\pgfpathmoveto{\pgfqpoint{1.552946in}{1.022500in}}%
\pgfpathlineto{\pgfqpoint{1.930304in}{1.022500in}}%
\pgfpathlineto{\pgfqpoint{1.930304in}{3.567323in}}%
\pgfpathlineto{\pgfqpoint{1.552946in}{3.567323in}}%
\pgfpathclose%
\pgfusepath{stroke,fill}%
\end{pgfscope}%
\begin{pgfscope}%
\pgfsetbuttcap%
\pgfsetmiterjoin%
\definecolor{currentfill}{rgb}{0.798529,0.536765,0.389706}%
\pgfsetfillcolor{currentfill}%
\pgfsetlinewidth{1.003750pt}%
\definecolor{currentstroke}{rgb}{1.000000,1.000000,1.000000}%
\pgfsetstrokecolor{currentstroke}%
\pgfsetdash{}{0pt}%
\pgfpathrectangle{\pgfqpoint{1.081250in}{1.022500in}}{\pgfqpoint{6.603750in}{3.359167in}}%
\pgfusepath{clip}%
\pgfpathmoveto{\pgfqpoint{1.552946in}{1.022500in}}%
\pgfpathlineto{\pgfqpoint{1.930304in}{1.022500in}}%
\pgfpathlineto{\pgfqpoint{1.930304in}{3.567323in}}%
\pgfpathlineto{\pgfqpoint{1.552946in}{3.567323in}}%
\pgfpathclose%
\pgfusepath{clip}%
\pgfsys@defobject{currentpattern}{\pgfqpoint{0in}{0in}}{\pgfqpoint{1in}{1in}}{%
\begin{pgfscope}%
\pgfpathrectangle{\pgfqpoint{0in}{0in}}{\pgfqpoint{1in}{1in}}%
\pgfusepath{clip}%
\pgfusepath{stroke}%
\end{pgfscope}%
}%
\pgfsys@transformshift{1.552946in}{1.022500in}%
\pgfsys@useobject{currentpattern}{}%
\pgfsys@transformshift{1in}{0in}%
\pgfsys@transformshift{-1in}{0in}%
\pgfsys@transformshift{0in}{1in}%
\pgfsys@useobject{currentpattern}{}%
\pgfsys@transformshift{1in}{0in}%
\pgfsys@transformshift{-1in}{0in}%
\pgfsys@transformshift{0in}{1in}%
\pgfsys@useobject{currentpattern}{}%
\pgfsys@transformshift{1in}{0in}%
\pgfsys@transformshift{-1in}{0in}%
\pgfsys@transformshift{0in}{1in}%
\end{pgfscope}%
\begin{pgfscope}%
\pgfpathrectangle{\pgfqpoint{1.081250in}{1.022500in}}{\pgfqpoint{6.603750in}{3.359167in}}%
\pgfusepath{clip}%
\pgfsetbuttcap%
\pgfsetmiterjoin%
\definecolor{currentfill}{rgb}{0.798529,0.536765,0.389706}%
\pgfsetfillcolor{currentfill}%
\pgfsetlinewidth{1.003750pt}%
\definecolor{currentstroke}{rgb}{1.000000,1.000000,1.000000}%
\pgfsetstrokecolor{currentstroke}%
\pgfsetdash{}{0pt}%
\pgfpathmoveto{\pgfqpoint{2.496339in}{1.022500in}}%
\pgfpathlineto{\pgfqpoint{2.873696in}{1.022500in}}%
\pgfpathlineto{\pgfqpoint{2.873696in}{3.349196in}}%
\pgfpathlineto{\pgfqpoint{2.496339in}{3.349196in}}%
\pgfpathclose%
\pgfusepath{stroke,fill}%
\end{pgfscope}%
\begin{pgfscope}%
\pgfsetbuttcap%
\pgfsetmiterjoin%
\definecolor{currentfill}{rgb}{0.798529,0.536765,0.389706}%
\pgfsetfillcolor{currentfill}%
\pgfsetlinewidth{1.003750pt}%
\definecolor{currentstroke}{rgb}{1.000000,1.000000,1.000000}%
\pgfsetstrokecolor{currentstroke}%
\pgfsetdash{}{0pt}%
\pgfpathrectangle{\pgfqpoint{1.081250in}{1.022500in}}{\pgfqpoint{6.603750in}{3.359167in}}%
\pgfusepath{clip}%
\pgfpathmoveto{\pgfqpoint{2.496339in}{1.022500in}}%
\pgfpathlineto{\pgfqpoint{2.873696in}{1.022500in}}%
\pgfpathlineto{\pgfqpoint{2.873696in}{3.349196in}}%
\pgfpathlineto{\pgfqpoint{2.496339in}{3.349196in}}%
\pgfpathclose%
\pgfusepath{clip}%
\pgfsys@defobject{currentpattern}{\pgfqpoint{0in}{0in}}{\pgfqpoint{1in}{1in}}{%
\begin{pgfscope}%
\pgfpathrectangle{\pgfqpoint{0in}{0in}}{\pgfqpoint{1in}{1in}}%
\pgfusepath{clip}%
\pgfusepath{stroke}%
\end{pgfscope}%
}%
\pgfsys@transformshift{2.496339in}{1.022500in}%
\pgfsys@useobject{currentpattern}{}%
\pgfsys@transformshift{1in}{0in}%
\pgfsys@transformshift{-1in}{0in}%
\pgfsys@transformshift{0in}{1in}%
\pgfsys@useobject{currentpattern}{}%
\pgfsys@transformshift{1in}{0in}%
\pgfsys@transformshift{-1in}{0in}%
\pgfsys@transformshift{0in}{1in}%
\pgfsys@useobject{currentpattern}{}%
\pgfsys@transformshift{1in}{0in}%
\pgfsys@transformshift{-1in}{0in}%
\pgfsys@transformshift{0in}{1in}%
\end{pgfscope}%
\begin{pgfscope}%
\pgfpathrectangle{\pgfqpoint{1.081250in}{1.022500in}}{\pgfqpoint{6.603750in}{3.359167in}}%
\pgfusepath{clip}%
\pgfsetbuttcap%
\pgfsetmiterjoin%
\definecolor{currentfill}{rgb}{0.798529,0.536765,0.389706}%
\pgfsetfillcolor{currentfill}%
\pgfsetlinewidth{1.003750pt}%
\definecolor{currentstroke}{rgb}{1.000000,1.000000,1.000000}%
\pgfsetstrokecolor{currentstroke}%
\pgfsetdash{}{0pt}%
\pgfpathmoveto{\pgfqpoint{3.439732in}{1.022500in}}%
\pgfpathlineto{\pgfqpoint{3.817089in}{1.022500in}}%
\pgfpathlineto{\pgfqpoint{3.817089in}{4.076288in}}%
\pgfpathlineto{\pgfqpoint{3.439732in}{4.076288in}}%
\pgfpathclose%
\pgfusepath{stroke,fill}%
\end{pgfscope}%
\begin{pgfscope}%
\pgfsetbuttcap%
\pgfsetmiterjoin%
\definecolor{currentfill}{rgb}{0.798529,0.536765,0.389706}%
\pgfsetfillcolor{currentfill}%
\pgfsetlinewidth{1.003750pt}%
\definecolor{currentstroke}{rgb}{1.000000,1.000000,1.000000}%
\pgfsetstrokecolor{currentstroke}%
\pgfsetdash{}{0pt}%
\pgfpathrectangle{\pgfqpoint{1.081250in}{1.022500in}}{\pgfqpoint{6.603750in}{3.359167in}}%
\pgfusepath{clip}%
\pgfpathmoveto{\pgfqpoint{3.439732in}{1.022500in}}%
\pgfpathlineto{\pgfqpoint{3.817089in}{1.022500in}}%
\pgfpathlineto{\pgfqpoint{3.817089in}{4.076288in}}%
\pgfpathlineto{\pgfqpoint{3.439732in}{4.076288in}}%
\pgfpathclose%
\pgfusepath{clip}%
\pgfsys@defobject{currentpattern}{\pgfqpoint{0in}{0in}}{\pgfqpoint{1in}{1in}}{%
\begin{pgfscope}%
\pgfpathrectangle{\pgfqpoint{0in}{0in}}{\pgfqpoint{1in}{1in}}%
\pgfusepath{clip}%
\pgfusepath{stroke}%
\end{pgfscope}%
}%
\pgfsys@transformshift{3.439732in}{1.022500in}%
\pgfsys@useobject{currentpattern}{}%
\pgfsys@transformshift{1in}{0in}%
\pgfsys@transformshift{-1in}{0in}%
\pgfsys@transformshift{0in}{1in}%
\pgfsys@useobject{currentpattern}{}%
\pgfsys@transformshift{1in}{0in}%
\pgfsys@transformshift{-1in}{0in}%
\pgfsys@transformshift{0in}{1in}%
\pgfsys@useobject{currentpattern}{}%
\pgfsys@transformshift{1in}{0in}%
\pgfsys@transformshift{-1in}{0in}%
\pgfsys@transformshift{0in}{1in}%
\pgfsys@useobject{currentpattern}{}%
\pgfsys@transformshift{1in}{0in}%
\pgfsys@transformshift{-1in}{0in}%
\pgfsys@transformshift{0in}{1in}%
\end{pgfscope}%
\begin{pgfscope}%
\pgfpathrectangle{\pgfqpoint{1.081250in}{1.022500in}}{\pgfqpoint{6.603750in}{3.359167in}}%
\pgfusepath{clip}%
\pgfsetbuttcap%
\pgfsetmiterjoin%
\definecolor{currentfill}{rgb}{0.798529,0.536765,0.389706}%
\pgfsetfillcolor{currentfill}%
\pgfsetlinewidth{1.003750pt}%
\definecolor{currentstroke}{rgb}{1.000000,1.000000,1.000000}%
\pgfsetstrokecolor{currentstroke}%
\pgfsetdash{}{0pt}%
\pgfpathmoveto{\pgfqpoint{4.383125in}{1.022500in}}%
\pgfpathlineto{\pgfqpoint{4.760482in}{1.022500in}}%
\pgfpathlineto{\pgfqpoint{4.760482in}{4.221706in}}%
\pgfpathlineto{\pgfqpoint{4.383125in}{4.221706in}}%
\pgfpathclose%
\pgfusepath{stroke,fill}%
\end{pgfscope}%
\begin{pgfscope}%
\pgfsetbuttcap%
\pgfsetmiterjoin%
\definecolor{currentfill}{rgb}{0.798529,0.536765,0.389706}%
\pgfsetfillcolor{currentfill}%
\pgfsetlinewidth{1.003750pt}%
\definecolor{currentstroke}{rgb}{1.000000,1.000000,1.000000}%
\pgfsetstrokecolor{currentstroke}%
\pgfsetdash{}{0pt}%
\pgfpathrectangle{\pgfqpoint{1.081250in}{1.022500in}}{\pgfqpoint{6.603750in}{3.359167in}}%
\pgfusepath{clip}%
\pgfpathmoveto{\pgfqpoint{4.383125in}{1.022500in}}%
\pgfpathlineto{\pgfqpoint{4.760482in}{1.022500in}}%
\pgfpathlineto{\pgfqpoint{4.760482in}{4.221706in}}%
\pgfpathlineto{\pgfqpoint{4.383125in}{4.221706in}}%
\pgfpathclose%
\pgfusepath{clip}%
\pgfsys@defobject{currentpattern}{\pgfqpoint{0in}{0in}}{\pgfqpoint{1in}{1in}}{%
\begin{pgfscope}%
\pgfpathrectangle{\pgfqpoint{0in}{0in}}{\pgfqpoint{1in}{1in}}%
\pgfusepath{clip}%
\pgfusepath{stroke}%
\end{pgfscope}%
}%
\pgfsys@transformshift{4.383125in}{1.022500in}%
\pgfsys@useobject{currentpattern}{}%
\pgfsys@transformshift{1in}{0in}%
\pgfsys@transformshift{-1in}{0in}%
\pgfsys@transformshift{0in}{1in}%
\pgfsys@useobject{currentpattern}{}%
\pgfsys@transformshift{1in}{0in}%
\pgfsys@transformshift{-1in}{0in}%
\pgfsys@transformshift{0in}{1in}%
\pgfsys@useobject{currentpattern}{}%
\pgfsys@transformshift{1in}{0in}%
\pgfsys@transformshift{-1in}{0in}%
\pgfsys@transformshift{0in}{1in}%
\pgfsys@useobject{currentpattern}{}%
\pgfsys@transformshift{1in}{0in}%
\pgfsys@transformshift{-1in}{0in}%
\pgfsys@transformshift{0in}{1in}%
\end{pgfscope}%
\begin{pgfscope}%
\pgfpathrectangle{\pgfqpoint{1.081250in}{1.022500in}}{\pgfqpoint{6.603750in}{3.359167in}}%
\pgfusepath{clip}%
\pgfsetbuttcap%
\pgfsetmiterjoin%
\definecolor{currentfill}{rgb}{0.798529,0.536765,0.389706}%
\pgfsetfillcolor{currentfill}%
\pgfsetlinewidth{1.003750pt}%
\definecolor{currentstroke}{rgb}{1.000000,1.000000,1.000000}%
\pgfsetstrokecolor{currentstroke}%
\pgfsetdash{}{0pt}%
\pgfpathmoveto{\pgfqpoint{5.326518in}{1.022500in}}%
\pgfpathlineto{\pgfqpoint{5.703875in}{1.022500in}}%
\pgfpathlineto{\pgfqpoint{5.703875in}{3.349196in}}%
\pgfpathlineto{\pgfqpoint{5.326518in}{3.349196in}}%
\pgfpathclose%
\pgfusepath{stroke,fill}%
\end{pgfscope}%
\begin{pgfscope}%
\pgfsetbuttcap%
\pgfsetmiterjoin%
\definecolor{currentfill}{rgb}{0.798529,0.536765,0.389706}%
\pgfsetfillcolor{currentfill}%
\pgfsetlinewidth{1.003750pt}%
\definecolor{currentstroke}{rgb}{1.000000,1.000000,1.000000}%
\pgfsetstrokecolor{currentstroke}%
\pgfsetdash{}{0pt}%
\pgfpathrectangle{\pgfqpoint{1.081250in}{1.022500in}}{\pgfqpoint{6.603750in}{3.359167in}}%
\pgfusepath{clip}%
\pgfpathmoveto{\pgfqpoint{5.326518in}{1.022500in}}%
\pgfpathlineto{\pgfqpoint{5.703875in}{1.022500in}}%
\pgfpathlineto{\pgfqpoint{5.703875in}{3.349196in}}%
\pgfpathlineto{\pgfqpoint{5.326518in}{3.349196in}}%
\pgfpathclose%
\pgfusepath{clip}%
\pgfsys@defobject{currentpattern}{\pgfqpoint{0in}{0in}}{\pgfqpoint{1in}{1in}}{%
\begin{pgfscope}%
\pgfpathrectangle{\pgfqpoint{0in}{0in}}{\pgfqpoint{1in}{1in}}%
\pgfusepath{clip}%
\pgfusepath{stroke}%
\end{pgfscope}%
}%
\pgfsys@transformshift{5.326518in}{1.022500in}%
\pgfsys@useobject{currentpattern}{}%
\pgfsys@transformshift{1in}{0in}%
\pgfsys@transformshift{-1in}{0in}%
\pgfsys@transformshift{0in}{1in}%
\pgfsys@useobject{currentpattern}{}%
\pgfsys@transformshift{1in}{0in}%
\pgfsys@transformshift{-1in}{0in}%
\pgfsys@transformshift{0in}{1in}%
\pgfsys@useobject{currentpattern}{}%
\pgfsys@transformshift{1in}{0in}%
\pgfsys@transformshift{-1in}{0in}%
\pgfsys@transformshift{0in}{1in}%
\end{pgfscope}%
\begin{pgfscope}%
\pgfpathrectangle{\pgfqpoint{1.081250in}{1.022500in}}{\pgfqpoint{6.603750in}{3.359167in}}%
\pgfusepath{clip}%
\pgfsetbuttcap%
\pgfsetmiterjoin%
\definecolor{currentfill}{rgb}{0.798529,0.536765,0.389706}%
\pgfsetfillcolor{currentfill}%
\pgfsetlinewidth{1.003750pt}%
\definecolor{currentstroke}{rgb}{1.000000,1.000000,1.000000}%
\pgfsetstrokecolor{currentstroke}%
\pgfsetdash{}{0pt}%
\pgfpathmoveto{\pgfqpoint{6.269911in}{1.022500in}}%
\pgfpathlineto{\pgfqpoint{6.647268in}{1.022500in}}%
\pgfpathlineto{\pgfqpoint{6.647268in}{2.476685in}}%
\pgfpathlineto{\pgfqpoint{6.269911in}{2.476685in}}%
\pgfpathclose%
\pgfusepath{stroke,fill}%
\end{pgfscope}%
\begin{pgfscope}%
\pgfsetbuttcap%
\pgfsetmiterjoin%
\definecolor{currentfill}{rgb}{0.798529,0.536765,0.389706}%
\pgfsetfillcolor{currentfill}%
\pgfsetlinewidth{1.003750pt}%
\definecolor{currentstroke}{rgb}{1.000000,1.000000,1.000000}%
\pgfsetstrokecolor{currentstroke}%
\pgfsetdash{}{0pt}%
\pgfpathrectangle{\pgfqpoint{1.081250in}{1.022500in}}{\pgfqpoint{6.603750in}{3.359167in}}%
\pgfusepath{clip}%
\pgfpathmoveto{\pgfqpoint{6.269911in}{1.022500in}}%
\pgfpathlineto{\pgfqpoint{6.647268in}{1.022500in}}%
\pgfpathlineto{\pgfqpoint{6.647268in}{2.476685in}}%
\pgfpathlineto{\pgfqpoint{6.269911in}{2.476685in}}%
\pgfpathclose%
\pgfusepath{clip}%
\pgfsys@defobject{currentpattern}{\pgfqpoint{0in}{0in}}{\pgfqpoint{1in}{1in}}{%
\begin{pgfscope}%
\pgfpathrectangle{\pgfqpoint{0in}{0in}}{\pgfqpoint{1in}{1in}}%
\pgfusepath{clip}%
\pgfusepath{stroke}%
\end{pgfscope}%
}%
\pgfsys@transformshift{6.269911in}{1.022500in}%
\pgfsys@useobject{currentpattern}{}%
\pgfsys@transformshift{1in}{0in}%
\pgfsys@transformshift{-1in}{0in}%
\pgfsys@transformshift{0in}{1in}%
\pgfsys@useobject{currentpattern}{}%
\pgfsys@transformshift{1in}{0in}%
\pgfsys@transformshift{-1in}{0in}%
\pgfsys@transformshift{0in}{1in}%
\end{pgfscope}%
\begin{pgfscope}%
\pgfpathrectangle{\pgfqpoint{1.081250in}{1.022500in}}{\pgfqpoint{6.603750in}{3.359167in}}%
\pgfusepath{clip}%
\pgfsetbuttcap%
\pgfsetmiterjoin%
\definecolor{currentfill}{rgb}{0.798529,0.536765,0.389706}%
\pgfsetfillcolor{currentfill}%
\pgfsetlinewidth{1.003750pt}%
\definecolor{currentstroke}{rgb}{1.000000,1.000000,1.000000}%
\pgfsetstrokecolor{currentstroke}%
\pgfsetdash{}{0pt}%
\pgfpathmoveto{\pgfqpoint{7.213304in}{1.022500in}}%
\pgfpathlineto{\pgfqpoint{7.590661in}{1.022500in}}%
\pgfpathlineto{\pgfqpoint{7.590661in}{2.476685in}}%
\pgfpathlineto{\pgfqpoint{7.213304in}{2.476685in}}%
\pgfpathclose%
\pgfusepath{stroke,fill}%
\end{pgfscope}%
\begin{pgfscope}%
\pgfsetbuttcap%
\pgfsetmiterjoin%
\definecolor{currentfill}{rgb}{0.798529,0.536765,0.389706}%
\pgfsetfillcolor{currentfill}%
\pgfsetlinewidth{1.003750pt}%
\definecolor{currentstroke}{rgb}{1.000000,1.000000,1.000000}%
\pgfsetstrokecolor{currentstroke}%
\pgfsetdash{}{0pt}%
\pgfpathrectangle{\pgfqpoint{1.081250in}{1.022500in}}{\pgfqpoint{6.603750in}{3.359167in}}%
\pgfusepath{clip}%
\pgfpathmoveto{\pgfqpoint{7.213304in}{1.022500in}}%
\pgfpathlineto{\pgfqpoint{7.590661in}{1.022500in}}%
\pgfpathlineto{\pgfqpoint{7.590661in}{2.476685in}}%
\pgfpathlineto{\pgfqpoint{7.213304in}{2.476685in}}%
\pgfpathclose%
\pgfusepath{clip}%
\pgfsys@defobject{currentpattern}{\pgfqpoint{0in}{0in}}{\pgfqpoint{1in}{1in}}{%
\begin{pgfscope}%
\pgfpathrectangle{\pgfqpoint{0in}{0in}}{\pgfqpoint{1in}{1in}}%
\pgfusepath{clip}%
\pgfusepath{stroke}%
\end{pgfscope}%
}%
\pgfsys@transformshift{7.213304in}{1.022500in}%
\pgfsys@useobject{currentpattern}{}%
\pgfsys@transformshift{1in}{0in}%
\pgfsys@transformshift{-1in}{0in}%
\pgfsys@transformshift{0in}{1in}%
\pgfsys@useobject{currentpattern}{}%
\pgfsys@transformshift{1in}{0in}%
\pgfsys@transformshift{-1in}{0in}%
\pgfsys@transformshift{0in}{1in}%
\end{pgfscope}%
\begin{pgfscope}%
\pgfpathrectangle{\pgfqpoint{1.081250in}{1.022500in}}{\pgfqpoint{6.603750in}{3.359167in}}%
\pgfusepath{clip}%
\pgfsetroundcap%
\pgfsetroundjoin%
\pgfsetlinewidth{2.710125pt}%
\definecolor{currentstroke}{rgb}{0.260000,0.260000,0.260000}%
\pgfsetstrokecolor{currentstroke}%
\pgfsetdash{}{0pt}%
\pgfusepath{stroke}%
\end{pgfscope}%
\begin{pgfscope}%
\pgfpathrectangle{\pgfqpoint{1.081250in}{1.022500in}}{\pgfqpoint{6.603750in}{3.359167in}}%
\pgfusepath{clip}%
\pgfsetroundcap%
\pgfsetroundjoin%
\pgfsetlinewidth{2.710125pt}%
\definecolor{currentstroke}{rgb}{0.260000,0.260000,0.260000}%
\pgfsetstrokecolor{currentstroke}%
\pgfsetdash{}{0pt}%
\pgfusepath{stroke}%
\end{pgfscope}%
\begin{pgfscope}%
\pgfpathrectangle{\pgfqpoint{1.081250in}{1.022500in}}{\pgfqpoint{6.603750in}{3.359167in}}%
\pgfusepath{clip}%
\pgfsetroundcap%
\pgfsetroundjoin%
\pgfsetlinewidth{2.710125pt}%
\definecolor{currentstroke}{rgb}{0.260000,0.260000,0.260000}%
\pgfsetstrokecolor{currentstroke}%
\pgfsetdash{}{0pt}%
\pgfusepath{stroke}%
\end{pgfscope}%
\begin{pgfscope}%
\pgfpathrectangle{\pgfqpoint{1.081250in}{1.022500in}}{\pgfqpoint{6.603750in}{3.359167in}}%
\pgfusepath{clip}%
\pgfsetroundcap%
\pgfsetroundjoin%
\pgfsetlinewidth{2.710125pt}%
\definecolor{currentstroke}{rgb}{0.260000,0.260000,0.260000}%
\pgfsetstrokecolor{currentstroke}%
\pgfsetdash{}{0pt}%
\pgfusepath{stroke}%
\end{pgfscope}%
\begin{pgfscope}%
\pgfpathrectangle{\pgfqpoint{1.081250in}{1.022500in}}{\pgfqpoint{6.603750in}{3.359167in}}%
\pgfusepath{clip}%
\pgfsetroundcap%
\pgfsetroundjoin%
\pgfsetlinewidth{2.710125pt}%
\definecolor{currentstroke}{rgb}{0.260000,0.260000,0.260000}%
\pgfsetstrokecolor{currentstroke}%
\pgfsetdash{}{0pt}%
\pgfusepath{stroke}%
\end{pgfscope}%
\begin{pgfscope}%
\pgfpathrectangle{\pgfqpoint{1.081250in}{1.022500in}}{\pgfqpoint{6.603750in}{3.359167in}}%
\pgfusepath{clip}%
\pgfsetroundcap%
\pgfsetroundjoin%
\pgfsetlinewidth{2.710125pt}%
\definecolor{currentstroke}{rgb}{0.260000,0.260000,0.260000}%
\pgfsetstrokecolor{currentstroke}%
\pgfsetdash{}{0pt}%
\pgfusepath{stroke}%
\end{pgfscope}%
\begin{pgfscope}%
\pgfpathrectangle{\pgfqpoint{1.081250in}{1.022500in}}{\pgfqpoint{6.603750in}{3.359167in}}%
\pgfusepath{clip}%
\pgfsetroundcap%
\pgfsetroundjoin%
\pgfsetlinewidth{2.710125pt}%
\definecolor{currentstroke}{rgb}{0.260000,0.260000,0.260000}%
\pgfsetstrokecolor{currentstroke}%
\pgfsetdash{}{0pt}%
\pgfusepath{stroke}%
\end{pgfscope}%
\begin{pgfscope}%
\pgfpathrectangle{\pgfqpoint{1.081250in}{1.022500in}}{\pgfqpoint{6.603750in}{3.359167in}}%
\pgfusepath{clip}%
\pgfsetroundcap%
\pgfsetroundjoin%
\pgfsetlinewidth{2.710125pt}%
\definecolor{currentstroke}{rgb}{0.260000,0.260000,0.260000}%
\pgfsetstrokecolor{currentstroke}%
\pgfsetdash{}{0pt}%
\pgfusepath{stroke}%
\end{pgfscope}%
\begin{pgfscope}%
\pgfpathrectangle{\pgfqpoint{1.081250in}{1.022500in}}{\pgfqpoint{6.603750in}{3.359167in}}%
\pgfusepath{clip}%
\pgfsetroundcap%
\pgfsetroundjoin%
\pgfsetlinewidth{2.710125pt}%
\definecolor{currentstroke}{rgb}{0.260000,0.260000,0.260000}%
\pgfsetstrokecolor{currentstroke}%
\pgfsetdash{}{0pt}%
\pgfusepath{stroke}%
\end{pgfscope}%
\begin{pgfscope}%
\pgfpathrectangle{\pgfqpoint{1.081250in}{1.022500in}}{\pgfqpoint{6.603750in}{3.359167in}}%
\pgfusepath{clip}%
\pgfsetroundcap%
\pgfsetroundjoin%
\pgfsetlinewidth{2.710125pt}%
\definecolor{currentstroke}{rgb}{0.260000,0.260000,0.260000}%
\pgfsetstrokecolor{currentstroke}%
\pgfsetdash{}{0pt}%
\pgfusepath{stroke}%
\end{pgfscope}%
\begin{pgfscope}%
\pgfpathrectangle{\pgfqpoint{1.081250in}{1.022500in}}{\pgfqpoint{6.603750in}{3.359167in}}%
\pgfusepath{clip}%
\pgfsetroundcap%
\pgfsetroundjoin%
\pgfsetlinewidth{2.710125pt}%
\definecolor{currentstroke}{rgb}{0.260000,0.260000,0.260000}%
\pgfsetstrokecolor{currentstroke}%
\pgfsetdash{}{0pt}%
\pgfusepath{stroke}%
\end{pgfscope}%
\begin{pgfscope}%
\pgfpathrectangle{\pgfqpoint{1.081250in}{1.022500in}}{\pgfqpoint{6.603750in}{3.359167in}}%
\pgfusepath{clip}%
\pgfsetroundcap%
\pgfsetroundjoin%
\pgfsetlinewidth{2.710125pt}%
\definecolor{currentstroke}{rgb}{0.260000,0.260000,0.260000}%
\pgfsetstrokecolor{currentstroke}%
\pgfsetdash{}{0pt}%
\pgfusepath{stroke}%
\end{pgfscope}%
\begin{pgfscope}%
\pgfpathrectangle{\pgfqpoint{1.081250in}{1.022500in}}{\pgfqpoint{6.603750in}{3.359167in}}%
\pgfusepath{clip}%
\pgfsetroundcap%
\pgfsetroundjoin%
\pgfsetlinewidth{2.710125pt}%
\definecolor{currentstroke}{rgb}{0.260000,0.260000,0.260000}%
\pgfsetstrokecolor{currentstroke}%
\pgfsetdash{}{0pt}%
\pgfusepath{stroke}%
\end{pgfscope}%
\begin{pgfscope}%
\pgfpathrectangle{\pgfqpoint{1.081250in}{1.022500in}}{\pgfqpoint{6.603750in}{3.359167in}}%
\pgfusepath{clip}%
\pgfsetroundcap%
\pgfsetroundjoin%
\pgfsetlinewidth{2.710125pt}%
\definecolor{currentstroke}{rgb}{0.260000,0.260000,0.260000}%
\pgfsetstrokecolor{currentstroke}%
\pgfsetdash{}{0pt}%
\pgfusepath{stroke}%
\end{pgfscope}%
\begin{pgfscope}%
\pgfsetrectcap%
\pgfsetmiterjoin%
\pgfsetlinewidth{1.254687pt}%
\definecolor{currentstroke}{rgb}{0.800000,0.800000,0.800000}%
\pgfsetstrokecolor{currentstroke}%
\pgfsetdash{}{0pt}%
\pgfpathmoveto{\pgfqpoint{1.081250in}{1.022500in}}%
\pgfpathlineto{\pgfqpoint{7.685000in}{1.022500in}}%
\pgfusepath{stroke}%
\end{pgfscope}%
\begin{pgfscope}%
\definecolor{textcolor}{rgb}{0.150000,0.150000,0.150000}%
\pgfsetstrokecolor{textcolor}%
\pgfsetfillcolor{textcolor}%
\pgftext[x=4.383125in,y=4.465000in,,base]{\color{textcolor}\sffamily\fontsize{21.000000}{25.200000}\selectfont Classic BPF to eBPF benchmarks}%
\end{pgfscope}%
\begin{pgfscope}%
\pgfsetbuttcap%
\pgfsetmiterjoin%
\definecolor{currentfill}{rgb}{1.000000,1.000000,1.000000}%
\pgfsetfillcolor{currentfill}%
\pgfsetfillopacity{0.800000}%
\pgfsetlinewidth{1.003750pt}%
\definecolor{currentstroke}{rgb}{0.800000,0.800000,0.800000}%
\pgfsetstrokecolor{currentstroke}%
\pgfsetstrokeopacity{0.800000}%
\pgfsetdash{}{0pt}%
\pgfpathmoveto{\pgfqpoint{5.701377in}{3.388281in}}%
\pgfpathlineto{\pgfqpoint{7.497847in}{3.388281in}}%
\pgfpathquadraticcurveto{\pgfqpoint{7.551319in}{3.388281in}}{\pgfqpoint{7.551319in}{3.441754in}}%
\pgfpathlineto{\pgfqpoint{7.551319in}{4.194514in}}%
\pgfpathquadraticcurveto{\pgfqpoint{7.551319in}{4.247986in}}{\pgfqpoint{7.497847in}{4.247986in}}%
\pgfpathlineto{\pgfqpoint{5.701377in}{4.247986in}}%
\pgfpathquadraticcurveto{\pgfqpoint{5.647905in}{4.247986in}}{\pgfqpoint{5.647905in}{4.194514in}}%
\pgfpathlineto{\pgfqpoint{5.647905in}{3.441754in}}%
\pgfpathquadraticcurveto{\pgfqpoint{5.647905in}{3.388281in}}{\pgfqpoint{5.701377in}{3.388281in}}%
\pgfpathclose%
\pgfusepath{stroke,fill}%
\end{pgfscope}%
\begin{pgfscope}%
\pgfsetbuttcap%
\pgfsetmiterjoin%
\definecolor{currentfill}{rgb}{0.347059,0.458824,0.641176}%
\pgfsetfillcolor{currentfill}%
\pgfsetlinewidth{1.003750pt}%
\definecolor{currentstroke}{rgb}{1.000000,1.000000,1.000000}%
\pgfsetstrokecolor{currentstroke}%
\pgfsetdash{}{0pt}%
\pgfpathmoveto{\pgfqpoint{5.754849in}{3.941003in}}%
\pgfpathlineto{\pgfqpoint{6.289571in}{3.941003in}}%
\pgfpathlineto{\pgfqpoint{6.289571in}{4.128156in}}%
\pgfpathlineto{\pgfqpoint{5.754849in}{4.128156in}}%
\pgfpathclose%
\pgfusepath{stroke,fill}%
\end{pgfscope}%
\begin{pgfscope}%
\pgfsetbuttcap%
\pgfsetmiterjoin%
\definecolor{currentfill}{rgb}{0.347059,0.458824,0.641176}%
\pgfsetfillcolor{currentfill}%
\pgfsetlinewidth{1.003750pt}%
\definecolor{currentstroke}{rgb}{1.000000,1.000000,1.000000}%
\pgfsetstrokecolor{currentstroke}%
\pgfsetdash{}{0pt}%
\pgfpathmoveto{\pgfqpoint{5.754849in}{3.941003in}}%
\pgfpathlineto{\pgfqpoint{6.289571in}{3.941003in}}%
\pgfpathlineto{\pgfqpoint{6.289571in}{4.128156in}}%
\pgfpathlineto{\pgfqpoint{5.754849in}{4.128156in}}%
\pgfpathclose%
\pgfusepath{clip}%
\pgfsys@defobject{currentpattern}{\pgfqpoint{0in}{0in}}{\pgfqpoint{1in}{1in}}{%
\begin{pgfscope}%
\pgfpathrectangle{\pgfqpoint{0in}{0in}}{\pgfqpoint{1in}{1in}}%
\pgfusepath{clip}%
\pgfpathmoveto{\pgfqpoint{-0.500000in}{0.500000in}}%
\pgfpathlineto{\pgfqpoint{0.500000in}{1.500000in}}%
\pgfpathmoveto{\pgfqpoint{-0.444444in}{0.444444in}}%
\pgfpathlineto{\pgfqpoint{0.555556in}{1.444444in}}%
\pgfpathmoveto{\pgfqpoint{-0.388889in}{0.388889in}}%
\pgfpathlineto{\pgfqpoint{0.611111in}{1.388889in}}%
\pgfpathmoveto{\pgfqpoint{-0.333333in}{0.333333in}}%
\pgfpathlineto{\pgfqpoint{0.666667in}{1.333333in}}%
\pgfpathmoveto{\pgfqpoint{-0.277778in}{0.277778in}}%
\pgfpathlineto{\pgfqpoint{0.722222in}{1.277778in}}%
\pgfpathmoveto{\pgfqpoint{-0.222222in}{0.222222in}}%
\pgfpathlineto{\pgfqpoint{0.777778in}{1.222222in}}%
\pgfpathmoveto{\pgfqpoint{-0.166667in}{0.166667in}}%
\pgfpathlineto{\pgfqpoint{0.833333in}{1.166667in}}%
\pgfpathmoveto{\pgfqpoint{-0.111111in}{0.111111in}}%
\pgfpathlineto{\pgfqpoint{0.888889in}{1.111111in}}%
\pgfpathmoveto{\pgfqpoint{-0.055556in}{0.055556in}}%
\pgfpathlineto{\pgfqpoint{0.944444in}{1.055556in}}%
\pgfpathmoveto{\pgfqpoint{0.000000in}{0.000000in}}%
\pgfpathlineto{\pgfqpoint{1.000000in}{1.000000in}}%
\pgfpathmoveto{\pgfqpoint{0.055556in}{-0.055556in}}%
\pgfpathlineto{\pgfqpoint{1.055556in}{0.944444in}}%
\pgfpathmoveto{\pgfqpoint{0.111111in}{-0.111111in}}%
\pgfpathlineto{\pgfqpoint{1.111111in}{0.888889in}}%
\pgfpathmoveto{\pgfqpoint{0.166667in}{-0.166667in}}%
\pgfpathlineto{\pgfqpoint{1.166667in}{0.833333in}}%
\pgfpathmoveto{\pgfqpoint{0.222222in}{-0.222222in}}%
\pgfpathlineto{\pgfqpoint{1.222222in}{0.777778in}}%
\pgfpathmoveto{\pgfqpoint{0.277778in}{-0.277778in}}%
\pgfpathlineto{\pgfqpoint{1.277778in}{0.722222in}}%
\pgfpathmoveto{\pgfqpoint{0.333333in}{-0.333333in}}%
\pgfpathlineto{\pgfqpoint{1.333333in}{0.666667in}}%
\pgfpathmoveto{\pgfqpoint{0.388889in}{-0.388889in}}%
\pgfpathlineto{\pgfqpoint{1.388889in}{0.611111in}}%
\pgfpathmoveto{\pgfqpoint{0.444444in}{-0.444444in}}%
\pgfpathlineto{\pgfqpoint{1.444444in}{0.555556in}}%
\pgfpathmoveto{\pgfqpoint{0.500000in}{-0.500000in}}%
\pgfpathlineto{\pgfqpoint{1.500000in}{0.500000in}}%
\pgfusepath{stroke}%
\end{pgfscope}%
}%
\pgfsys@transformshift{5.754849in}{3.941003in}%
\pgfsys@useobject{currentpattern}{}%
\pgfsys@transformshift{1in}{0in}%
\pgfsys@transformshift{-1in}{0in}%
\pgfsys@transformshift{0in}{1in}%
\end{pgfscope}%
\begin{pgfscope}%
\definecolor{textcolor}{rgb}{0.150000,0.150000,0.150000}%
\pgfsetstrokecolor{textcolor}%
\pgfsetfillcolor{textcolor}%
\pgftext[x=6.503460in,y=3.941003in,left,base]{\color{textcolor}\sffamily\fontsize{19.250000}{23.100000}\selectfont Linux}%
\end{pgfscope}%
\begin{pgfscope}%
\pgfsetbuttcap%
\pgfsetmiterjoin%
\definecolor{currentfill}{rgb}{0.798529,0.536765,0.389706}%
\pgfsetfillcolor{currentfill}%
\pgfsetlinewidth{1.003750pt}%
\definecolor{currentstroke}{rgb}{1.000000,1.000000,1.000000}%
\pgfsetstrokecolor{currentstroke}%
\pgfsetdash{}{0pt}%
\pgfpathmoveto{\pgfqpoint{5.754849in}{3.551255in}}%
\pgfpathlineto{\pgfqpoint{6.289571in}{3.551255in}}%
\pgfpathlineto{\pgfqpoint{6.289571in}{3.738408in}}%
\pgfpathlineto{\pgfqpoint{5.754849in}{3.738408in}}%
\pgfpathclose%
\pgfusepath{stroke,fill}%
\end{pgfscope}%
\begin{pgfscope}%
\pgfsetbuttcap%
\pgfsetmiterjoin%
\definecolor{currentfill}{rgb}{0.798529,0.536765,0.389706}%
\pgfsetfillcolor{currentfill}%
\pgfsetlinewidth{1.003750pt}%
\definecolor{currentstroke}{rgb}{1.000000,1.000000,1.000000}%
\pgfsetstrokecolor{currentstroke}%
\pgfsetdash{}{0pt}%
\pgfpathmoveto{\pgfqpoint{5.754849in}{3.551255in}}%
\pgfpathlineto{\pgfqpoint{6.289571in}{3.551255in}}%
\pgfpathlineto{\pgfqpoint{6.289571in}{3.738408in}}%
\pgfpathlineto{\pgfqpoint{5.754849in}{3.738408in}}%
\pgfpathclose%
\pgfusepath{clip}%
\pgfsys@defobject{currentpattern}{\pgfqpoint{0in}{0in}}{\pgfqpoint{1in}{1in}}{%
\begin{pgfscope}%
\pgfpathrectangle{\pgfqpoint{0in}{0in}}{\pgfqpoint{1in}{1in}}%
\pgfusepath{clip}%
\pgfusepath{stroke}%
\end{pgfscope}%
}%
\pgfsys@transformshift{5.754849in}{3.551255in}%
\pgfsys@useobject{currentpattern}{}%
\pgfsys@transformshift{1in}{0in}%
\pgfsys@transformshift{-1in}{0in}%
\pgfsys@transformshift{0in}{1in}%
\end{pgfscope}%
\begin{pgfscope}%
\definecolor{textcolor}{rgb}{0.150000,0.150000,0.150000}%
\pgfsetstrokecolor{textcolor}%
\pgfsetfillcolor{textcolor}%
\pgftext[x=6.503460in,y=3.551255in,left,base]{\color{textcolor}\sffamily\fontsize{19.250000}{23.100000}\selectfont JitSynth}%
\end{pgfscope}%
\end{pgfpicture}%
\makeatother%
\endgroup%

  %% Creator: Matplotlib, PGF backend
%%
%% To include the figure in your LaTeX document, write
%%   \input{<filename>.pgf}
%%
%% Make sure the required packages are loaded in your preamble
%%   \usepackage{pgf}
%%
%% Figures using additional raster images can only be included by \input if
%% they are in the same directory as the main LaTeX file. For loading figures
%% from other directories you can use the `import` package
%%   \usepackage{import}
%% and then include the figures with
%%   \import{<path to file>}{<filename>.pgf}
%%
%% Matplotlib used the following preamble
%%
\begingroup%
\makeatletter%
\begin{pgfpicture}%
\pgfpathrectangle{\pgfpointorigin}{\pgfqpoint{8.000000in}{5.000000in}}%
\pgfusepath{use as bounding box, clip}%
\begin{pgfscope}%
\pgfsetbuttcap%
\pgfsetmiterjoin%
\definecolor{currentfill}{rgb}{1.000000,1.000000,1.000000}%
\pgfsetfillcolor{currentfill}%
\pgfsetlinewidth{0.000000pt}%
\definecolor{currentstroke}{rgb}{1.000000,1.000000,1.000000}%
\pgfsetstrokecolor{currentstroke}%
\pgfsetdash{}{0pt}%
\pgfpathmoveto{\pgfqpoint{0.000000in}{0.000000in}}%
\pgfpathlineto{\pgfqpoint{8.000000in}{0.000000in}}%
\pgfpathlineto{\pgfqpoint{8.000000in}{5.000000in}}%
\pgfpathlineto{\pgfqpoint{0.000000in}{5.000000in}}%
\pgfpathclose%
\pgfusepath{fill}%
\end{pgfscope}%
\begin{pgfscope}%
\pgfsetbuttcap%
\pgfsetmiterjoin%
\definecolor{currentfill}{rgb}{1.000000,1.000000,1.000000}%
\pgfsetfillcolor{currentfill}%
\pgfsetlinewidth{0.000000pt}%
\definecolor{currentstroke}{rgb}{0.000000,0.000000,0.000000}%
\pgfsetstrokecolor{currentstroke}%
\pgfsetstrokeopacity{0.000000}%
\pgfsetdash{}{0pt}%
\pgfpathmoveto{\pgfqpoint{1.230000in}{1.022500in}}%
\pgfpathlineto{\pgfqpoint{7.685000in}{1.022500in}}%
\pgfpathlineto{\pgfqpoint{7.685000in}{4.381667in}}%
\pgfpathlineto{\pgfqpoint{1.230000in}{4.381667in}}%
\pgfpathclose%
\pgfusepath{fill}%
\end{pgfscope}%
\begin{pgfscope}%
\definecolor{textcolor}{rgb}{0.150000,0.150000,0.150000}%
\pgfsetstrokecolor{textcolor}%
\pgfsetfillcolor{textcolor}%
\pgftext[x=1.875500in,y=0.890556in,,top]{\color{textcolor}\sffamily\fontsize{19.250000}{23.100000}\selectfont ctags}%
\end{pgfscope}%
\begin{pgfscope}%
\definecolor{textcolor}{rgb}{0.150000,0.150000,0.150000}%
\pgfsetstrokecolor{textcolor}%
\pgfsetfillcolor{textcolor}%
\pgftext[x=3.166500in,y=0.890556in,,top]{\color{textcolor}\sffamily\fontsize{19.250000}{23.100000}\selectfont lepton}%
\end{pgfscope}%
\begin{pgfscope}%
\definecolor{textcolor}{rgb}{0.150000,0.150000,0.150000}%
\pgfsetstrokecolor{textcolor}%
\pgfsetfillcolor{textcolor}%
\pgftext[x=4.457500in,y=0.890556in,,top]{\color{textcolor}\sffamily\fontsize{19.250000}{23.100000}\selectfont libreoffice}%
\end{pgfscope}%
\begin{pgfscope}%
\definecolor{textcolor}{rgb}{0.150000,0.150000,0.150000}%
\pgfsetstrokecolor{textcolor}%
\pgfsetfillcolor{textcolor}%
\pgftext[x=5.748500in,y=0.890556in,,top]{\color{textcolor}\sffamily\fontsize{19.250000}{23.100000}\selectfont openssh}%
\end{pgfscope}%
\begin{pgfscope}%
\definecolor{textcolor}{rgb}{0.150000,0.150000,0.150000}%
\pgfsetstrokecolor{textcolor}%
\pgfsetfillcolor{textcolor}%
\pgftext[x=7.039500in,y=0.890556in,,top]{\color{textcolor}\sffamily\fontsize{19.250000}{23.100000}\selectfont vsftpd}%
\end{pgfscope}%
\begin{pgfscope}%
\definecolor{textcolor}{rgb}{0.150000,0.150000,0.150000}%
\pgfsetstrokecolor{textcolor}%
\pgfsetfillcolor{textcolor}%
\pgftext[x=4.457500in,y=0.578932in,,top]{\color{textcolor}\sffamily\fontsize{21.000000}{25.200000}\selectfont Benchmark}%
\end{pgfscope}%
\begin{pgfscope}%
\pgfpathrectangle{\pgfqpoint{1.230000in}{1.022500in}}{\pgfqpoint{6.455000in}{3.359167in}}%
\pgfusepath{clip}%
\pgfsetroundcap%
\pgfsetroundjoin%
\pgfsetlinewidth{1.003750pt}%
\definecolor{currentstroke}{rgb}{0.800000,0.800000,0.800000}%
\pgfsetstrokecolor{currentstroke}%
\pgfsetdash{}{0pt}%
\pgfpathmoveto{\pgfqpoint{1.230000in}{1.022500in}}%
\pgfpathlineto{\pgfqpoint{7.685000in}{1.022500in}}%
\pgfusepath{stroke}%
\end{pgfscope}%
\begin{pgfscope}%
\definecolor{textcolor}{rgb}{0.150000,0.150000,0.150000}%
\pgfsetstrokecolor{textcolor}%
\pgfsetfillcolor{textcolor}%
\pgftext[x=0.962614in,y=0.922481in,left,base]{\color{textcolor}\sffamily\fontsize{19.250000}{23.100000}\selectfont 0}%
\end{pgfscope}%
\begin{pgfscope}%
\pgfpathrectangle{\pgfqpoint{1.230000in}{1.022500in}}{\pgfqpoint{6.455000in}{3.359167in}}%
\pgfusepath{clip}%
\pgfsetroundcap%
\pgfsetroundjoin%
\pgfsetlinewidth{1.003750pt}%
\definecolor{currentstroke}{rgb}{0.800000,0.800000,0.800000}%
\pgfsetstrokecolor{currentstroke}%
\pgfsetdash{}{0pt}%
\pgfpathmoveto{\pgfqpoint{1.230000in}{2.061203in}}%
\pgfpathlineto{\pgfqpoint{7.685000in}{2.061203in}}%
\pgfusepath{stroke}%
\end{pgfscope}%
\begin{pgfscope}%
\definecolor{textcolor}{rgb}{0.150000,0.150000,0.150000}%
\pgfsetstrokecolor{textcolor}%
\pgfsetfillcolor{textcolor}%
\pgftext[x=0.691731in,y=1.961184in,left,base]{\color{textcolor}\sffamily\fontsize{19.250000}{23.100000}\selectfont 100}%
\end{pgfscope}%
\begin{pgfscope}%
\pgfpathrectangle{\pgfqpoint{1.230000in}{1.022500in}}{\pgfqpoint{6.455000in}{3.359167in}}%
\pgfusepath{clip}%
\pgfsetroundcap%
\pgfsetroundjoin%
\pgfsetlinewidth{1.003750pt}%
\definecolor{currentstroke}{rgb}{0.800000,0.800000,0.800000}%
\pgfsetstrokecolor{currentstroke}%
\pgfsetdash{}{0pt}%
\pgfpathmoveto{\pgfqpoint{1.230000in}{3.099907in}}%
\pgfpathlineto{\pgfqpoint{7.685000in}{3.099907in}}%
\pgfusepath{stroke}%
\end{pgfscope}%
\begin{pgfscope}%
\definecolor{textcolor}{rgb}{0.150000,0.150000,0.150000}%
\pgfsetstrokecolor{textcolor}%
\pgfsetfillcolor{textcolor}%
\pgftext[x=0.691731in,y=2.999887in,left,base]{\color{textcolor}\sffamily\fontsize{19.250000}{23.100000}\selectfont 200}%
\end{pgfscope}%
\begin{pgfscope}%
\pgfpathrectangle{\pgfqpoint{1.230000in}{1.022500in}}{\pgfqpoint{6.455000in}{3.359167in}}%
\pgfusepath{clip}%
\pgfsetroundcap%
\pgfsetroundjoin%
\pgfsetlinewidth{1.003750pt}%
\definecolor{currentstroke}{rgb}{0.800000,0.800000,0.800000}%
\pgfsetstrokecolor{currentstroke}%
\pgfsetdash{}{0pt}%
\pgfpathmoveto{\pgfqpoint{1.230000in}{4.138610in}}%
\pgfpathlineto{\pgfqpoint{7.685000in}{4.138610in}}%
\pgfusepath{stroke}%
\end{pgfscope}%
\begin{pgfscope}%
\definecolor{textcolor}{rgb}{0.150000,0.150000,0.150000}%
\pgfsetstrokecolor{textcolor}%
\pgfsetfillcolor{textcolor}%
\pgftext[x=0.691731in,y=4.038591in,left,base]{\color{textcolor}\sffamily\fontsize{19.250000}{23.100000}\selectfont 300}%
\end{pgfscope}%
\begin{pgfscope}%
\definecolor{textcolor}{rgb}{0.150000,0.150000,0.150000}%
\pgfsetstrokecolor{textcolor}%
\pgfsetfillcolor{textcolor}%
\pgftext[x=0.636175in,y=2.702083in,,bottom,rotate=90.000000]{\color{textcolor}\sffamily\fontsize{21.000000}{25.200000}\selectfont Instructions executed}%
\end{pgfscope}%
\begin{pgfscope}%
\pgfpathrectangle{\pgfqpoint{1.230000in}{1.022500in}}{\pgfqpoint{6.455000in}{3.359167in}}%
\pgfusepath{clip}%
\pgfsetbuttcap%
\pgfsetmiterjoin%
\definecolor{currentfill}{rgb}{0.347059,0.458824,0.641176}%
\pgfsetfillcolor{currentfill}%
\pgfsetlinewidth{1.003750pt}%
\definecolor{currentstroke}{rgb}{1.000000,1.000000,1.000000}%
\pgfsetstrokecolor{currentstroke}%
\pgfsetdash{}{0pt}%
\pgfpathmoveto{\pgfqpoint{1.359100in}{1.022500in}}%
\pgfpathlineto{\pgfqpoint{1.875500in}{1.022500in}}%
\pgfpathlineto{\pgfqpoint{1.875500in}{1.136757in}}%
\pgfpathlineto{\pgfqpoint{1.359100in}{1.136757in}}%
\pgfpathclose%
\pgfusepath{stroke,fill}%
\end{pgfscope}%
\begin{pgfscope}%
\pgfsetbuttcap%
\pgfsetmiterjoin%
\definecolor{currentfill}{rgb}{0.347059,0.458824,0.641176}%
\pgfsetfillcolor{currentfill}%
\pgfsetlinewidth{1.003750pt}%
\definecolor{currentstroke}{rgb}{1.000000,1.000000,1.000000}%
\pgfsetstrokecolor{currentstroke}%
\pgfsetdash{}{0pt}%
\pgfpathrectangle{\pgfqpoint{1.230000in}{1.022500in}}{\pgfqpoint{6.455000in}{3.359167in}}%
\pgfusepath{clip}%
\pgfpathmoveto{\pgfqpoint{1.359100in}{1.022500in}}%
\pgfpathlineto{\pgfqpoint{1.875500in}{1.022500in}}%
\pgfpathlineto{\pgfqpoint{1.875500in}{1.136757in}}%
\pgfpathlineto{\pgfqpoint{1.359100in}{1.136757in}}%
\pgfpathclose%
\pgfusepath{clip}%
\pgfsys@defobject{currentpattern}{\pgfqpoint{0in}{0in}}{\pgfqpoint{1in}{1in}}{%
\begin{pgfscope}%
\pgfpathrectangle{\pgfqpoint{0in}{0in}}{\pgfqpoint{1in}{1in}}%
\pgfusepath{clip}%
\pgfpathmoveto{\pgfqpoint{-0.500000in}{0.500000in}}%
\pgfpathlineto{\pgfqpoint{0.500000in}{1.500000in}}%
\pgfpathmoveto{\pgfqpoint{-0.444444in}{0.444444in}}%
\pgfpathlineto{\pgfqpoint{0.555556in}{1.444444in}}%
\pgfpathmoveto{\pgfqpoint{-0.388889in}{0.388889in}}%
\pgfpathlineto{\pgfqpoint{0.611111in}{1.388889in}}%
\pgfpathmoveto{\pgfqpoint{-0.333333in}{0.333333in}}%
\pgfpathlineto{\pgfqpoint{0.666667in}{1.333333in}}%
\pgfpathmoveto{\pgfqpoint{-0.277778in}{0.277778in}}%
\pgfpathlineto{\pgfqpoint{0.722222in}{1.277778in}}%
\pgfpathmoveto{\pgfqpoint{-0.222222in}{0.222222in}}%
\pgfpathlineto{\pgfqpoint{0.777778in}{1.222222in}}%
\pgfpathmoveto{\pgfqpoint{-0.166667in}{0.166667in}}%
\pgfpathlineto{\pgfqpoint{0.833333in}{1.166667in}}%
\pgfpathmoveto{\pgfqpoint{-0.111111in}{0.111111in}}%
\pgfpathlineto{\pgfqpoint{0.888889in}{1.111111in}}%
\pgfpathmoveto{\pgfqpoint{-0.055556in}{0.055556in}}%
\pgfpathlineto{\pgfqpoint{0.944444in}{1.055556in}}%
\pgfpathmoveto{\pgfqpoint{0.000000in}{0.000000in}}%
\pgfpathlineto{\pgfqpoint{1.000000in}{1.000000in}}%
\pgfpathmoveto{\pgfqpoint{0.055556in}{-0.055556in}}%
\pgfpathlineto{\pgfqpoint{1.055556in}{0.944444in}}%
\pgfpathmoveto{\pgfqpoint{0.111111in}{-0.111111in}}%
\pgfpathlineto{\pgfqpoint{1.111111in}{0.888889in}}%
\pgfpathmoveto{\pgfqpoint{0.166667in}{-0.166667in}}%
\pgfpathlineto{\pgfqpoint{1.166667in}{0.833333in}}%
\pgfpathmoveto{\pgfqpoint{0.222222in}{-0.222222in}}%
\pgfpathlineto{\pgfqpoint{1.222222in}{0.777778in}}%
\pgfpathmoveto{\pgfqpoint{0.277778in}{-0.277778in}}%
\pgfpathlineto{\pgfqpoint{1.277778in}{0.722222in}}%
\pgfpathmoveto{\pgfqpoint{0.333333in}{-0.333333in}}%
\pgfpathlineto{\pgfqpoint{1.333333in}{0.666667in}}%
\pgfpathmoveto{\pgfqpoint{0.388889in}{-0.388889in}}%
\pgfpathlineto{\pgfqpoint{1.388889in}{0.611111in}}%
\pgfpathmoveto{\pgfqpoint{0.444444in}{-0.444444in}}%
\pgfpathlineto{\pgfqpoint{1.444444in}{0.555556in}}%
\pgfpathmoveto{\pgfqpoint{0.500000in}{-0.500000in}}%
\pgfpathlineto{\pgfqpoint{1.500000in}{0.500000in}}%
\pgfusepath{stroke}%
\end{pgfscope}%
}%
\pgfsys@transformshift{1.359100in}{1.022500in}%
\pgfsys@useobject{currentpattern}{}%
\pgfsys@transformshift{1in}{0in}%
\pgfsys@transformshift{-1in}{0in}%
\pgfsys@transformshift{0in}{1in}%
\end{pgfscope}%
\begin{pgfscope}%
\pgfpathrectangle{\pgfqpoint{1.230000in}{1.022500in}}{\pgfqpoint{6.455000in}{3.359167in}}%
\pgfusepath{clip}%
\pgfsetbuttcap%
\pgfsetmiterjoin%
\definecolor{currentfill}{rgb}{0.347059,0.458824,0.641176}%
\pgfsetfillcolor{currentfill}%
\pgfsetlinewidth{1.003750pt}%
\definecolor{currentstroke}{rgb}{1.000000,1.000000,1.000000}%
\pgfsetstrokecolor{currentstroke}%
\pgfsetdash{}{0pt}%
\pgfpathmoveto{\pgfqpoint{2.650100in}{1.022500in}}%
\pgfpathlineto{\pgfqpoint{3.166500in}{1.022500in}}%
\pgfpathlineto{\pgfqpoint{3.166500in}{1.209467in}}%
\pgfpathlineto{\pgfqpoint{2.650100in}{1.209467in}}%
\pgfpathclose%
\pgfusepath{stroke,fill}%
\end{pgfscope}%
\begin{pgfscope}%
\pgfsetbuttcap%
\pgfsetmiterjoin%
\definecolor{currentfill}{rgb}{0.347059,0.458824,0.641176}%
\pgfsetfillcolor{currentfill}%
\pgfsetlinewidth{1.003750pt}%
\definecolor{currentstroke}{rgb}{1.000000,1.000000,1.000000}%
\pgfsetstrokecolor{currentstroke}%
\pgfsetdash{}{0pt}%
\pgfpathrectangle{\pgfqpoint{1.230000in}{1.022500in}}{\pgfqpoint{6.455000in}{3.359167in}}%
\pgfusepath{clip}%
\pgfpathmoveto{\pgfqpoint{2.650100in}{1.022500in}}%
\pgfpathlineto{\pgfqpoint{3.166500in}{1.022500in}}%
\pgfpathlineto{\pgfqpoint{3.166500in}{1.209467in}}%
\pgfpathlineto{\pgfqpoint{2.650100in}{1.209467in}}%
\pgfpathclose%
\pgfusepath{clip}%
\pgfsys@defobject{currentpattern}{\pgfqpoint{0in}{0in}}{\pgfqpoint{1in}{1in}}{%
\begin{pgfscope}%
\pgfpathrectangle{\pgfqpoint{0in}{0in}}{\pgfqpoint{1in}{1in}}%
\pgfusepath{clip}%
\pgfpathmoveto{\pgfqpoint{-0.500000in}{0.500000in}}%
\pgfpathlineto{\pgfqpoint{0.500000in}{1.500000in}}%
\pgfpathmoveto{\pgfqpoint{-0.444444in}{0.444444in}}%
\pgfpathlineto{\pgfqpoint{0.555556in}{1.444444in}}%
\pgfpathmoveto{\pgfqpoint{-0.388889in}{0.388889in}}%
\pgfpathlineto{\pgfqpoint{0.611111in}{1.388889in}}%
\pgfpathmoveto{\pgfqpoint{-0.333333in}{0.333333in}}%
\pgfpathlineto{\pgfqpoint{0.666667in}{1.333333in}}%
\pgfpathmoveto{\pgfqpoint{-0.277778in}{0.277778in}}%
\pgfpathlineto{\pgfqpoint{0.722222in}{1.277778in}}%
\pgfpathmoveto{\pgfqpoint{-0.222222in}{0.222222in}}%
\pgfpathlineto{\pgfqpoint{0.777778in}{1.222222in}}%
\pgfpathmoveto{\pgfqpoint{-0.166667in}{0.166667in}}%
\pgfpathlineto{\pgfqpoint{0.833333in}{1.166667in}}%
\pgfpathmoveto{\pgfqpoint{-0.111111in}{0.111111in}}%
\pgfpathlineto{\pgfqpoint{0.888889in}{1.111111in}}%
\pgfpathmoveto{\pgfqpoint{-0.055556in}{0.055556in}}%
\pgfpathlineto{\pgfqpoint{0.944444in}{1.055556in}}%
\pgfpathmoveto{\pgfqpoint{0.000000in}{0.000000in}}%
\pgfpathlineto{\pgfqpoint{1.000000in}{1.000000in}}%
\pgfpathmoveto{\pgfqpoint{0.055556in}{-0.055556in}}%
\pgfpathlineto{\pgfqpoint{1.055556in}{0.944444in}}%
\pgfpathmoveto{\pgfqpoint{0.111111in}{-0.111111in}}%
\pgfpathlineto{\pgfqpoint{1.111111in}{0.888889in}}%
\pgfpathmoveto{\pgfqpoint{0.166667in}{-0.166667in}}%
\pgfpathlineto{\pgfqpoint{1.166667in}{0.833333in}}%
\pgfpathmoveto{\pgfqpoint{0.222222in}{-0.222222in}}%
\pgfpathlineto{\pgfqpoint{1.222222in}{0.777778in}}%
\pgfpathmoveto{\pgfqpoint{0.277778in}{-0.277778in}}%
\pgfpathlineto{\pgfqpoint{1.277778in}{0.722222in}}%
\pgfpathmoveto{\pgfqpoint{0.333333in}{-0.333333in}}%
\pgfpathlineto{\pgfqpoint{1.333333in}{0.666667in}}%
\pgfpathmoveto{\pgfqpoint{0.388889in}{-0.388889in}}%
\pgfpathlineto{\pgfqpoint{1.388889in}{0.611111in}}%
\pgfpathmoveto{\pgfqpoint{0.444444in}{-0.444444in}}%
\pgfpathlineto{\pgfqpoint{1.444444in}{0.555556in}}%
\pgfpathmoveto{\pgfqpoint{0.500000in}{-0.500000in}}%
\pgfpathlineto{\pgfqpoint{1.500000in}{0.500000in}}%
\pgfusepath{stroke}%
\end{pgfscope}%
}%
\pgfsys@transformshift{2.650100in}{1.022500in}%
\pgfsys@useobject{currentpattern}{}%
\pgfsys@transformshift{1in}{0in}%
\pgfsys@transformshift{-1in}{0in}%
\pgfsys@transformshift{0in}{1in}%
\end{pgfscope}%
\begin{pgfscope}%
\pgfpathrectangle{\pgfqpoint{1.230000in}{1.022500in}}{\pgfqpoint{6.455000in}{3.359167in}}%
\pgfusepath{clip}%
\pgfsetbuttcap%
\pgfsetmiterjoin%
\definecolor{currentfill}{rgb}{0.347059,0.458824,0.641176}%
\pgfsetfillcolor{currentfill}%
\pgfsetlinewidth{1.003750pt}%
\definecolor{currentstroke}{rgb}{1.000000,1.000000,1.000000}%
\pgfsetstrokecolor{currentstroke}%
\pgfsetdash{}{0pt}%
\pgfpathmoveto{\pgfqpoint{3.941100in}{1.022500in}}%
\pgfpathlineto{\pgfqpoint{4.457500in}{1.022500in}}%
\pgfpathlineto{\pgfqpoint{4.457500in}{2.030042in}}%
\pgfpathlineto{\pgfqpoint{3.941100in}{2.030042in}}%
\pgfpathclose%
\pgfusepath{stroke,fill}%
\end{pgfscope}%
\begin{pgfscope}%
\pgfsetbuttcap%
\pgfsetmiterjoin%
\definecolor{currentfill}{rgb}{0.347059,0.458824,0.641176}%
\pgfsetfillcolor{currentfill}%
\pgfsetlinewidth{1.003750pt}%
\definecolor{currentstroke}{rgb}{1.000000,1.000000,1.000000}%
\pgfsetstrokecolor{currentstroke}%
\pgfsetdash{}{0pt}%
\pgfpathrectangle{\pgfqpoint{1.230000in}{1.022500in}}{\pgfqpoint{6.455000in}{3.359167in}}%
\pgfusepath{clip}%
\pgfpathmoveto{\pgfqpoint{3.941100in}{1.022500in}}%
\pgfpathlineto{\pgfqpoint{4.457500in}{1.022500in}}%
\pgfpathlineto{\pgfqpoint{4.457500in}{2.030042in}}%
\pgfpathlineto{\pgfqpoint{3.941100in}{2.030042in}}%
\pgfpathclose%
\pgfusepath{clip}%
\pgfsys@defobject{currentpattern}{\pgfqpoint{0in}{0in}}{\pgfqpoint{1in}{1in}}{%
\begin{pgfscope}%
\pgfpathrectangle{\pgfqpoint{0in}{0in}}{\pgfqpoint{1in}{1in}}%
\pgfusepath{clip}%
\pgfpathmoveto{\pgfqpoint{-0.500000in}{0.500000in}}%
\pgfpathlineto{\pgfqpoint{0.500000in}{1.500000in}}%
\pgfpathmoveto{\pgfqpoint{-0.444444in}{0.444444in}}%
\pgfpathlineto{\pgfqpoint{0.555556in}{1.444444in}}%
\pgfpathmoveto{\pgfqpoint{-0.388889in}{0.388889in}}%
\pgfpathlineto{\pgfqpoint{0.611111in}{1.388889in}}%
\pgfpathmoveto{\pgfqpoint{-0.333333in}{0.333333in}}%
\pgfpathlineto{\pgfqpoint{0.666667in}{1.333333in}}%
\pgfpathmoveto{\pgfqpoint{-0.277778in}{0.277778in}}%
\pgfpathlineto{\pgfqpoint{0.722222in}{1.277778in}}%
\pgfpathmoveto{\pgfqpoint{-0.222222in}{0.222222in}}%
\pgfpathlineto{\pgfqpoint{0.777778in}{1.222222in}}%
\pgfpathmoveto{\pgfqpoint{-0.166667in}{0.166667in}}%
\pgfpathlineto{\pgfqpoint{0.833333in}{1.166667in}}%
\pgfpathmoveto{\pgfqpoint{-0.111111in}{0.111111in}}%
\pgfpathlineto{\pgfqpoint{0.888889in}{1.111111in}}%
\pgfpathmoveto{\pgfqpoint{-0.055556in}{0.055556in}}%
\pgfpathlineto{\pgfqpoint{0.944444in}{1.055556in}}%
\pgfpathmoveto{\pgfqpoint{0.000000in}{0.000000in}}%
\pgfpathlineto{\pgfqpoint{1.000000in}{1.000000in}}%
\pgfpathmoveto{\pgfqpoint{0.055556in}{-0.055556in}}%
\pgfpathlineto{\pgfqpoint{1.055556in}{0.944444in}}%
\pgfpathmoveto{\pgfqpoint{0.111111in}{-0.111111in}}%
\pgfpathlineto{\pgfqpoint{1.111111in}{0.888889in}}%
\pgfpathmoveto{\pgfqpoint{0.166667in}{-0.166667in}}%
\pgfpathlineto{\pgfqpoint{1.166667in}{0.833333in}}%
\pgfpathmoveto{\pgfqpoint{0.222222in}{-0.222222in}}%
\pgfpathlineto{\pgfqpoint{1.222222in}{0.777778in}}%
\pgfpathmoveto{\pgfqpoint{0.277778in}{-0.277778in}}%
\pgfpathlineto{\pgfqpoint{1.277778in}{0.722222in}}%
\pgfpathmoveto{\pgfqpoint{0.333333in}{-0.333333in}}%
\pgfpathlineto{\pgfqpoint{1.333333in}{0.666667in}}%
\pgfpathmoveto{\pgfqpoint{0.388889in}{-0.388889in}}%
\pgfpathlineto{\pgfqpoint{1.388889in}{0.611111in}}%
\pgfpathmoveto{\pgfqpoint{0.444444in}{-0.444444in}}%
\pgfpathlineto{\pgfqpoint{1.444444in}{0.555556in}}%
\pgfpathmoveto{\pgfqpoint{0.500000in}{-0.500000in}}%
\pgfpathlineto{\pgfqpoint{1.500000in}{0.500000in}}%
\pgfusepath{stroke}%
\end{pgfscope}%
}%
\pgfsys@transformshift{3.941100in}{1.022500in}%
\pgfsys@useobject{currentpattern}{}%
\pgfsys@transformshift{1in}{0in}%
\pgfsys@transformshift{-1in}{0in}%
\pgfsys@transformshift{0in}{1in}%
\pgfsys@useobject{currentpattern}{}%
\pgfsys@transformshift{1in}{0in}%
\pgfsys@transformshift{-1in}{0in}%
\pgfsys@transformshift{0in}{1in}%
\end{pgfscope}%
\begin{pgfscope}%
\pgfpathrectangle{\pgfqpoint{1.230000in}{1.022500in}}{\pgfqpoint{6.455000in}{3.359167in}}%
\pgfusepath{clip}%
\pgfsetbuttcap%
\pgfsetmiterjoin%
\definecolor{currentfill}{rgb}{0.347059,0.458824,0.641176}%
\pgfsetfillcolor{currentfill}%
\pgfsetlinewidth{1.003750pt}%
\definecolor{currentstroke}{rgb}{1.000000,1.000000,1.000000}%
\pgfsetstrokecolor{currentstroke}%
\pgfsetdash{}{0pt}%
\pgfpathmoveto{\pgfqpoint{5.232100in}{1.022500in}}%
\pgfpathlineto{\pgfqpoint{5.748500in}{1.022500in}}%
\pgfpathlineto{\pgfqpoint{5.748500in}{1.635335in}}%
\pgfpathlineto{\pgfqpoint{5.232100in}{1.635335in}}%
\pgfpathclose%
\pgfusepath{stroke,fill}%
\end{pgfscope}%
\begin{pgfscope}%
\pgfsetbuttcap%
\pgfsetmiterjoin%
\definecolor{currentfill}{rgb}{0.347059,0.458824,0.641176}%
\pgfsetfillcolor{currentfill}%
\pgfsetlinewidth{1.003750pt}%
\definecolor{currentstroke}{rgb}{1.000000,1.000000,1.000000}%
\pgfsetstrokecolor{currentstroke}%
\pgfsetdash{}{0pt}%
\pgfpathrectangle{\pgfqpoint{1.230000in}{1.022500in}}{\pgfqpoint{6.455000in}{3.359167in}}%
\pgfusepath{clip}%
\pgfpathmoveto{\pgfqpoint{5.232100in}{1.022500in}}%
\pgfpathlineto{\pgfqpoint{5.748500in}{1.022500in}}%
\pgfpathlineto{\pgfqpoint{5.748500in}{1.635335in}}%
\pgfpathlineto{\pgfqpoint{5.232100in}{1.635335in}}%
\pgfpathclose%
\pgfusepath{clip}%
\pgfsys@defobject{currentpattern}{\pgfqpoint{0in}{0in}}{\pgfqpoint{1in}{1in}}{%
\begin{pgfscope}%
\pgfpathrectangle{\pgfqpoint{0in}{0in}}{\pgfqpoint{1in}{1in}}%
\pgfusepath{clip}%
\pgfpathmoveto{\pgfqpoint{-0.500000in}{0.500000in}}%
\pgfpathlineto{\pgfqpoint{0.500000in}{1.500000in}}%
\pgfpathmoveto{\pgfqpoint{-0.444444in}{0.444444in}}%
\pgfpathlineto{\pgfqpoint{0.555556in}{1.444444in}}%
\pgfpathmoveto{\pgfqpoint{-0.388889in}{0.388889in}}%
\pgfpathlineto{\pgfqpoint{0.611111in}{1.388889in}}%
\pgfpathmoveto{\pgfqpoint{-0.333333in}{0.333333in}}%
\pgfpathlineto{\pgfqpoint{0.666667in}{1.333333in}}%
\pgfpathmoveto{\pgfqpoint{-0.277778in}{0.277778in}}%
\pgfpathlineto{\pgfqpoint{0.722222in}{1.277778in}}%
\pgfpathmoveto{\pgfqpoint{-0.222222in}{0.222222in}}%
\pgfpathlineto{\pgfqpoint{0.777778in}{1.222222in}}%
\pgfpathmoveto{\pgfqpoint{-0.166667in}{0.166667in}}%
\pgfpathlineto{\pgfqpoint{0.833333in}{1.166667in}}%
\pgfpathmoveto{\pgfqpoint{-0.111111in}{0.111111in}}%
\pgfpathlineto{\pgfqpoint{0.888889in}{1.111111in}}%
\pgfpathmoveto{\pgfqpoint{-0.055556in}{0.055556in}}%
\pgfpathlineto{\pgfqpoint{0.944444in}{1.055556in}}%
\pgfpathmoveto{\pgfqpoint{0.000000in}{0.000000in}}%
\pgfpathlineto{\pgfqpoint{1.000000in}{1.000000in}}%
\pgfpathmoveto{\pgfqpoint{0.055556in}{-0.055556in}}%
\pgfpathlineto{\pgfqpoint{1.055556in}{0.944444in}}%
\pgfpathmoveto{\pgfqpoint{0.111111in}{-0.111111in}}%
\pgfpathlineto{\pgfqpoint{1.111111in}{0.888889in}}%
\pgfpathmoveto{\pgfqpoint{0.166667in}{-0.166667in}}%
\pgfpathlineto{\pgfqpoint{1.166667in}{0.833333in}}%
\pgfpathmoveto{\pgfqpoint{0.222222in}{-0.222222in}}%
\pgfpathlineto{\pgfqpoint{1.222222in}{0.777778in}}%
\pgfpathmoveto{\pgfqpoint{0.277778in}{-0.277778in}}%
\pgfpathlineto{\pgfqpoint{1.277778in}{0.722222in}}%
\pgfpathmoveto{\pgfqpoint{0.333333in}{-0.333333in}}%
\pgfpathlineto{\pgfqpoint{1.333333in}{0.666667in}}%
\pgfpathmoveto{\pgfqpoint{0.388889in}{-0.388889in}}%
\pgfpathlineto{\pgfqpoint{1.388889in}{0.611111in}}%
\pgfpathmoveto{\pgfqpoint{0.444444in}{-0.444444in}}%
\pgfpathlineto{\pgfqpoint{1.444444in}{0.555556in}}%
\pgfpathmoveto{\pgfqpoint{0.500000in}{-0.500000in}}%
\pgfpathlineto{\pgfqpoint{1.500000in}{0.500000in}}%
\pgfusepath{stroke}%
\end{pgfscope}%
}%
\pgfsys@transformshift{5.232100in}{1.022500in}%
\pgfsys@useobject{currentpattern}{}%
\pgfsys@transformshift{1in}{0in}%
\pgfsys@transformshift{-1in}{0in}%
\pgfsys@transformshift{0in}{1in}%
\end{pgfscope}%
\begin{pgfscope}%
\pgfpathrectangle{\pgfqpoint{1.230000in}{1.022500in}}{\pgfqpoint{6.455000in}{3.359167in}}%
\pgfusepath{clip}%
\pgfsetbuttcap%
\pgfsetmiterjoin%
\definecolor{currentfill}{rgb}{0.347059,0.458824,0.641176}%
\pgfsetfillcolor{currentfill}%
\pgfsetlinewidth{1.003750pt}%
\definecolor{currentstroke}{rgb}{1.000000,1.000000,1.000000}%
\pgfsetstrokecolor{currentstroke}%
\pgfsetdash{}{0pt}%
\pgfpathmoveto{\pgfqpoint{6.523100in}{1.022500in}}%
\pgfpathlineto{\pgfqpoint{7.039500in}{1.022500in}}%
\pgfpathlineto{\pgfqpoint{7.039500in}{1.095209in}}%
\pgfpathlineto{\pgfqpoint{6.523100in}{1.095209in}}%
\pgfpathclose%
\pgfusepath{stroke,fill}%
\end{pgfscope}%
\begin{pgfscope}%
\pgfsetbuttcap%
\pgfsetmiterjoin%
\definecolor{currentfill}{rgb}{0.347059,0.458824,0.641176}%
\pgfsetfillcolor{currentfill}%
\pgfsetlinewidth{1.003750pt}%
\definecolor{currentstroke}{rgb}{1.000000,1.000000,1.000000}%
\pgfsetstrokecolor{currentstroke}%
\pgfsetdash{}{0pt}%
\pgfpathrectangle{\pgfqpoint{1.230000in}{1.022500in}}{\pgfqpoint{6.455000in}{3.359167in}}%
\pgfusepath{clip}%
\pgfpathmoveto{\pgfqpoint{6.523100in}{1.022500in}}%
\pgfpathlineto{\pgfqpoint{7.039500in}{1.022500in}}%
\pgfpathlineto{\pgfqpoint{7.039500in}{1.095209in}}%
\pgfpathlineto{\pgfqpoint{6.523100in}{1.095209in}}%
\pgfpathclose%
\pgfusepath{clip}%
\pgfsys@defobject{currentpattern}{\pgfqpoint{0in}{0in}}{\pgfqpoint{1in}{1in}}{%
\begin{pgfscope}%
\pgfpathrectangle{\pgfqpoint{0in}{0in}}{\pgfqpoint{1in}{1in}}%
\pgfusepath{clip}%
\pgfpathmoveto{\pgfqpoint{-0.500000in}{0.500000in}}%
\pgfpathlineto{\pgfqpoint{0.500000in}{1.500000in}}%
\pgfpathmoveto{\pgfqpoint{-0.444444in}{0.444444in}}%
\pgfpathlineto{\pgfqpoint{0.555556in}{1.444444in}}%
\pgfpathmoveto{\pgfqpoint{-0.388889in}{0.388889in}}%
\pgfpathlineto{\pgfqpoint{0.611111in}{1.388889in}}%
\pgfpathmoveto{\pgfqpoint{-0.333333in}{0.333333in}}%
\pgfpathlineto{\pgfqpoint{0.666667in}{1.333333in}}%
\pgfpathmoveto{\pgfqpoint{-0.277778in}{0.277778in}}%
\pgfpathlineto{\pgfqpoint{0.722222in}{1.277778in}}%
\pgfpathmoveto{\pgfqpoint{-0.222222in}{0.222222in}}%
\pgfpathlineto{\pgfqpoint{0.777778in}{1.222222in}}%
\pgfpathmoveto{\pgfqpoint{-0.166667in}{0.166667in}}%
\pgfpathlineto{\pgfqpoint{0.833333in}{1.166667in}}%
\pgfpathmoveto{\pgfqpoint{-0.111111in}{0.111111in}}%
\pgfpathlineto{\pgfqpoint{0.888889in}{1.111111in}}%
\pgfpathmoveto{\pgfqpoint{-0.055556in}{0.055556in}}%
\pgfpathlineto{\pgfqpoint{0.944444in}{1.055556in}}%
\pgfpathmoveto{\pgfqpoint{0.000000in}{0.000000in}}%
\pgfpathlineto{\pgfqpoint{1.000000in}{1.000000in}}%
\pgfpathmoveto{\pgfqpoint{0.055556in}{-0.055556in}}%
\pgfpathlineto{\pgfqpoint{1.055556in}{0.944444in}}%
\pgfpathmoveto{\pgfqpoint{0.111111in}{-0.111111in}}%
\pgfpathlineto{\pgfqpoint{1.111111in}{0.888889in}}%
\pgfpathmoveto{\pgfqpoint{0.166667in}{-0.166667in}}%
\pgfpathlineto{\pgfqpoint{1.166667in}{0.833333in}}%
\pgfpathmoveto{\pgfqpoint{0.222222in}{-0.222222in}}%
\pgfpathlineto{\pgfqpoint{1.222222in}{0.777778in}}%
\pgfpathmoveto{\pgfqpoint{0.277778in}{-0.277778in}}%
\pgfpathlineto{\pgfqpoint{1.277778in}{0.722222in}}%
\pgfpathmoveto{\pgfqpoint{0.333333in}{-0.333333in}}%
\pgfpathlineto{\pgfqpoint{1.333333in}{0.666667in}}%
\pgfpathmoveto{\pgfqpoint{0.388889in}{-0.388889in}}%
\pgfpathlineto{\pgfqpoint{1.388889in}{0.611111in}}%
\pgfpathmoveto{\pgfqpoint{0.444444in}{-0.444444in}}%
\pgfpathlineto{\pgfqpoint{1.444444in}{0.555556in}}%
\pgfpathmoveto{\pgfqpoint{0.500000in}{-0.500000in}}%
\pgfpathlineto{\pgfqpoint{1.500000in}{0.500000in}}%
\pgfusepath{stroke}%
\end{pgfscope}%
}%
\pgfsys@transformshift{6.523100in}{1.022500in}%
\pgfsys@useobject{currentpattern}{}%
\pgfsys@transformshift{1in}{0in}%
\pgfsys@transformshift{-1in}{0in}%
\pgfsys@transformshift{0in}{1in}%
\end{pgfscope}%
\begin{pgfscope}%
\pgfpathrectangle{\pgfqpoint{1.230000in}{1.022500in}}{\pgfqpoint{6.455000in}{3.359167in}}%
\pgfusepath{clip}%
\pgfsetbuttcap%
\pgfsetmiterjoin%
\definecolor{currentfill}{rgb}{0.798529,0.536765,0.389706}%
\pgfsetfillcolor{currentfill}%
\pgfsetlinewidth{1.003750pt}%
\definecolor{currentstroke}{rgb}{1.000000,1.000000,1.000000}%
\pgfsetstrokecolor{currentstroke}%
\pgfsetdash{}{0pt}%
\pgfpathmoveto{\pgfqpoint{1.875500in}{1.022500in}}%
\pgfpathlineto{\pgfqpoint{2.391900in}{1.022500in}}%
\pgfpathlineto{\pgfqpoint{2.391900in}{1.302950in}}%
\pgfpathlineto{\pgfqpoint{1.875500in}{1.302950in}}%
\pgfpathclose%
\pgfusepath{stroke,fill}%
\end{pgfscope}%
\begin{pgfscope}%
\pgfsetbuttcap%
\pgfsetmiterjoin%
\definecolor{currentfill}{rgb}{0.798529,0.536765,0.389706}%
\pgfsetfillcolor{currentfill}%
\pgfsetlinewidth{1.003750pt}%
\definecolor{currentstroke}{rgb}{1.000000,1.000000,1.000000}%
\pgfsetstrokecolor{currentstroke}%
\pgfsetdash{}{0pt}%
\pgfpathrectangle{\pgfqpoint{1.230000in}{1.022500in}}{\pgfqpoint{6.455000in}{3.359167in}}%
\pgfusepath{clip}%
\pgfpathmoveto{\pgfqpoint{1.875500in}{1.022500in}}%
\pgfpathlineto{\pgfqpoint{2.391900in}{1.022500in}}%
\pgfpathlineto{\pgfqpoint{2.391900in}{1.302950in}}%
\pgfpathlineto{\pgfqpoint{1.875500in}{1.302950in}}%
\pgfpathclose%
\pgfusepath{clip}%
\pgfsys@defobject{currentpattern}{\pgfqpoint{0in}{0in}}{\pgfqpoint{1in}{1in}}{%
\begin{pgfscope}%
\pgfpathrectangle{\pgfqpoint{0in}{0in}}{\pgfqpoint{1in}{1in}}%
\pgfusepath{clip}%
\pgfusepath{stroke}%
\end{pgfscope}%
}%
\pgfsys@transformshift{1.875500in}{1.022500in}%
\pgfsys@useobject{currentpattern}{}%
\pgfsys@transformshift{1in}{0in}%
\pgfsys@transformshift{-1in}{0in}%
\pgfsys@transformshift{0in}{1in}%
\end{pgfscope}%
\begin{pgfscope}%
\pgfpathrectangle{\pgfqpoint{1.230000in}{1.022500in}}{\pgfqpoint{6.455000in}{3.359167in}}%
\pgfusepath{clip}%
\pgfsetbuttcap%
\pgfsetmiterjoin%
\definecolor{currentfill}{rgb}{0.798529,0.536765,0.389706}%
\pgfsetfillcolor{currentfill}%
\pgfsetlinewidth{1.003750pt}%
\definecolor{currentstroke}{rgb}{1.000000,1.000000,1.000000}%
\pgfsetstrokecolor{currentstroke}%
\pgfsetdash{}{0pt}%
\pgfpathmoveto{\pgfqpoint{3.166500in}{1.022500in}}%
\pgfpathlineto{\pgfqpoint{3.682900in}{1.022500in}}%
\pgfpathlineto{\pgfqpoint{3.682900in}{1.666496in}}%
\pgfpathlineto{\pgfqpoint{3.166500in}{1.666496in}}%
\pgfpathclose%
\pgfusepath{stroke,fill}%
\end{pgfscope}%
\begin{pgfscope}%
\pgfsetbuttcap%
\pgfsetmiterjoin%
\definecolor{currentfill}{rgb}{0.798529,0.536765,0.389706}%
\pgfsetfillcolor{currentfill}%
\pgfsetlinewidth{1.003750pt}%
\definecolor{currentstroke}{rgb}{1.000000,1.000000,1.000000}%
\pgfsetstrokecolor{currentstroke}%
\pgfsetdash{}{0pt}%
\pgfpathrectangle{\pgfqpoint{1.230000in}{1.022500in}}{\pgfqpoint{6.455000in}{3.359167in}}%
\pgfusepath{clip}%
\pgfpathmoveto{\pgfqpoint{3.166500in}{1.022500in}}%
\pgfpathlineto{\pgfqpoint{3.682900in}{1.022500in}}%
\pgfpathlineto{\pgfqpoint{3.682900in}{1.666496in}}%
\pgfpathlineto{\pgfqpoint{3.166500in}{1.666496in}}%
\pgfpathclose%
\pgfusepath{clip}%
\pgfsys@defobject{currentpattern}{\pgfqpoint{0in}{0in}}{\pgfqpoint{1in}{1in}}{%
\begin{pgfscope}%
\pgfpathrectangle{\pgfqpoint{0in}{0in}}{\pgfqpoint{1in}{1in}}%
\pgfusepath{clip}%
\pgfusepath{stroke}%
\end{pgfscope}%
}%
\pgfsys@transformshift{3.166500in}{1.022500in}%
\pgfsys@useobject{currentpattern}{}%
\pgfsys@transformshift{1in}{0in}%
\pgfsys@transformshift{-1in}{0in}%
\pgfsys@transformshift{0in}{1in}%
\end{pgfscope}%
\begin{pgfscope}%
\pgfpathrectangle{\pgfqpoint{1.230000in}{1.022500in}}{\pgfqpoint{6.455000in}{3.359167in}}%
\pgfusepath{clip}%
\pgfsetbuttcap%
\pgfsetmiterjoin%
\definecolor{currentfill}{rgb}{0.798529,0.536765,0.389706}%
\pgfsetfillcolor{currentfill}%
\pgfsetlinewidth{1.003750pt}%
\definecolor{currentstroke}{rgb}{1.000000,1.000000,1.000000}%
\pgfsetstrokecolor{currentstroke}%
\pgfsetdash{}{0pt}%
\pgfpathmoveto{\pgfqpoint{4.457500in}{1.022500in}}%
\pgfpathlineto{\pgfqpoint{4.973900in}{1.022500in}}%
\pgfpathlineto{\pgfqpoint{4.973900in}{4.221706in}}%
\pgfpathlineto{\pgfqpoint{4.457500in}{4.221706in}}%
\pgfpathclose%
\pgfusepath{stroke,fill}%
\end{pgfscope}%
\begin{pgfscope}%
\pgfsetbuttcap%
\pgfsetmiterjoin%
\definecolor{currentfill}{rgb}{0.798529,0.536765,0.389706}%
\pgfsetfillcolor{currentfill}%
\pgfsetlinewidth{1.003750pt}%
\definecolor{currentstroke}{rgb}{1.000000,1.000000,1.000000}%
\pgfsetstrokecolor{currentstroke}%
\pgfsetdash{}{0pt}%
\pgfpathrectangle{\pgfqpoint{1.230000in}{1.022500in}}{\pgfqpoint{6.455000in}{3.359167in}}%
\pgfusepath{clip}%
\pgfpathmoveto{\pgfqpoint{4.457500in}{1.022500in}}%
\pgfpathlineto{\pgfqpoint{4.973900in}{1.022500in}}%
\pgfpathlineto{\pgfqpoint{4.973900in}{4.221706in}}%
\pgfpathlineto{\pgfqpoint{4.457500in}{4.221706in}}%
\pgfpathclose%
\pgfusepath{clip}%
\pgfsys@defobject{currentpattern}{\pgfqpoint{0in}{0in}}{\pgfqpoint{1in}{1in}}{%
\begin{pgfscope}%
\pgfpathrectangle{\pgfqpoint{0in}{0in}}{\pgfqpoint{1in}{1in}}%
\pgfusepath{clip}%
\pgfusepath{stroke}%
\end{pgfscope}%
}%
\pgfsys@transformshift{4.457500in}{1.022500in}%
\pgfsys@useobject{currentpattern}{}%
\pgfsys@transformshift{1in}{0in}%
\pgfsys@transformshift{-1in}{0in}%
\pgfsys@transformshift{0in}{1in}%
\pgfsys@useobject{currentpattern}{}%
\pgfsys@transformshift{1in}{0in}%
\pgfsys@transformshift{-1in}{0in}%
\pgfsys@transformshift{0in}{1in}%
\pgfsys@useobject{currentpattern}{}%
\pgfsys@transformshift{1in}{0in}%
\pgfsys@transformshift{-1in}{0in}%
\pgfsys@transformshift{0in}{1in}%
\pgfsys@useobject{currentpattern}{}%
\pgfsys@transformshift{1in}{0in}%
\pgfsys@transformshift{-1in}{0in}%
\pgfsys@transformshift{0in}{1in}%
\end{pgfscope}%
\begin{pgfscope}%
\pgfpathrectangle{\pgfqpoint{1.230000in}{1.022500in}}{\pgfqpoint{6.455000in}{3.359167in}}%
\pgfusepath{clip}%
\pgfsetbuttcap%
\pgfsetmiterjoin%
\definecolor{currentfill}{rgb}{0.798529,0.536765,0.389706}%
\pgfsetfillcolor{currentfill}%
\pgfsetlinewidth{1.003750pt}%
\definecolor{currentstroke}{rgb}{1.000000,1.000000,1.000000}%
\pgfsetstrokecolor{currentstroke}%
\pgfsetdash{}{0pt}%
\pgfpathmoveto{\pgfqpoint{5.748500in}{1.022500in}}%
\pgfpathlineto{\pgfqpoint{6.264900in}{1.022500in}}%
\pgfpathlineto{\pgfqpoint{6.264900in}{2.840231in}}%
\pgfpathlineto{\pgfqpoint{5.748500in}{2.840231in}}%
\pgfpathclose%
\pgfusepath{stroke,fill}%
\end{pgfscope}%
\begin{pgfscope}%
\pgfsetbuttcap%
\pgfsetmiterjoin%
\definecolor{currentfill}{rgb}{0.798529,0.536765,0.389706}%
\pgfsetfillcolor{currentfill}%
\pgfsetlinewidth{1.003750pt}%
\definecolor{currentstroke}{rgb}{1.000000,1.000000,1.000000}%
\pgfsetstrokecolor{currentstroke}%
\pgfsetdash{}{0pt}%
\pgfpathrectangle{\pgfqpoint{1.230000in}{1.022500in}}{\pgfqpoint{6.455000in}{3.359167in}}%
\pgfusepath{clip}%
\pgfpathmoveto{\pgfqpoint{5.748500in}{1.022500in}}%
\pgfpathlineto{\pgfqpoint{6.264900in}{1.022500in}}%
\pgfpathlineto{\pgfqpoint{6.264900in}{2.840231in}}%
\pgfpathlineto{\pgfqpoint{5.748500in}{2.840231in}}%
\pgfpathclose%
\pgfusepath{clip}%
\pgfsys@defobject{currentpattern}{\pgfqpoint{0in}{0in}}{\pgfqpoint{1in}{1in}}{%
\begin{pgfscope}%
\pgfpathrectangle{\pgfqpoint{0in}{0in}}{\pgfqpoint{1in}{1in}}%
\pgfusepath{clip}%
\pgfusepath{stroke}%
\end{pgfscope}%
}%
\pgfsys@transformshift{5.748500in}{1.022500in}%
\pgfsys@useobject{currentpattern}{}%
\pgfsys@transformshift{1in}{0in}%
\pgfsys@transformshift{-1in}{0in}%
\pgfsys@transformshift{0in}{1in}%
\pgfsys@useobject{currentpattern}{}%
\pgfsys@transformshift{1in}{0in}%
\pgfsys@transformshift{-1in}{0in}%
\pgfsys@transformshift{0in}{1in}%
\end{pgfscope}%
\begin{pgfscope}%
\pgfpathrectangle{\pgfqpoint{1.230000in}{1.022500in}}{\pgfqpoint{6.455000in}{3.359167in}}%
\pgfusepath{clip}%
\pgfsetbuttcap%
\pgfsetmiterjoin%
\definecolor{currentfill}{rgb}{0.798529,0.536765,0.389706}%
\pgfsetfillcolor{currentfill}%
\pgfsetlinewidth{1.003750pt}%
\definecolor{currentstroke}{rgb}{1.000000,1.000000,1.000000}%
\pgfsetstrokecolor{currentstroke}%
\pgfsetdash{}{0pt}%
\pgfpathmoveto{\pgfqpoint{7.039500in}{1.022500in}}%
\pgfpathlineto{\pgfqpoint{7.555900in}{1.022500in}}%
\pgfpathlineto{\pgfqpoint{7.555900in}{1.095209in}}%
\pgfpathlineto{\pgfqpoint{7.039500in}{1.095209in}}%
\pgfpathclose%
\pgfusepath{stroke,fill}%
\end{pgfscope}%
\begin{pgfscope}%
\pgfsetbuttcap%
\pgfsetmiterjoin%
\definecolor{currentfill}{rgb}{0.798529,0.536765,0.389706}%
\pgfsetfillcolor{currentfill}%
\pgfsetlinewidth{1.003750pt}%
\definecolor{currentstroke}{rgb}{1.000000,1.000000,1.000000}%
\pgfsetstrokecolor{currentstroke}%
\pgfsetdash{}{0pt}%
\pgfpathrectangle{\pgfqpoint{1.230000in}{1.022500in}}{\pgfqpoint{6.455000in}{3.359167in}}%
\pgfusepath{clip}%
\pgfpathmoveto{\pgfqpoint{7.039500in}{1.022500in}}%
\pgfpathlineto{\pgfqpoint{7.555900in}{1.022500in}}%
\pgfpathlineto{\pgfqpoint{7.555900in}{1.095209in}}%
\pgfpathlineto{\pgfqpoint{7.039500in}{1.095209in}}%
\pgfpathclose%
\pgfusepath{clip}%
\pgfsys@defobject{currentpattern}{\pgfqpoint{0in}{0in}}{\pgfqpoint{1in}{1in}}{%
\begin{pgfscope}%
\pgfpathrectangle{\pgfqpoint{0in}{0in}}{\pgfqpoint{1in}{1in}}%
\pgfusepath{clip}%
\pgfusepath{stroke}%
\end{pgfscope}%
}%
\pgfsys@transformshift{7.039500in}{1.022500in}%
\pgfsys@useobject{currentpattern}{}%
\pgfsys@transformshift{1in}{0in}%
\pgfsys@transformshift{-1in}{0in}%
\pgfsys@transformshift{0in}{1in}%
\end{pgfscope}%
\begin{pgfscope}%
\pgfpathrectangle{\pgfqpoint{1.230000in}{1.022500in}}{\pgfqpoint{6.455000in}{3.359167in}}%
\pgfusepath{clip}%
\pgfsetroundcap%
\pgfsetroundjoin%
\pgfsetlinewidth{2.710125pt}%
\definecolor{currentstroke}{rgb}{0.260000,0.260000,0.260000}%
\pgfsetstrokecolor{currentstroke}%
\pgfsetdash{}{0pt}%
\pgfusepath{stroke}%
\end{pgfscope}%
\begin{pgfscope}%
\pgfpathrectangle{\pgfqpoint{1.230000in}{1.022500in}}{\pgfqpoint{6.455000in}{3.359167in}}%
\pgfusepath{clip}%
\pgfsetroundcap%
\pgfsetroundjoin%
\pgfsetlinewidth{2.710125pt}%
\definecolor{currentstroke}{rgb}{0.260000,0.260000,0.260000}%
\pgfsetstrokecolor{currentstroke}%
\pgfsetdash{}{0pt}%
\pgfusepath{stroke}%
\end{pgfscope}%
\begin{pgfscope}%
\pgfpathrectangle{\pgfqpoint{1.230000in}{1.022500in}}{\pgfqpoint{6.455000in}{3.359167in}}%
\pgfusepath{clip}%
\pgfsetroundcap%
\pgfsetroundjoin%
\pgfsetlinewidth{2.710125pt}%
\definecolor{currentstroke}{rgb}{0.260000,0.260000,0.260000}%
\pgfsetstrokecolor{currentstroke}%
\pgfsetdash{}{0pt}%
\pgfusepath{stroke}%
\end{pgfscope}%
\begin{pgfscope}%
\pgfpathrectangle{\pgfqpoint{1.230000in}{1.022500in}}{\pgfqpoint{6.455000in}{3.359167in}}%
\pgfusepath{clip}%
\pgfsetroundcap%
\pgfsetroundjoin%
\pgfsetlinewidth{2.710125pt}%
\definecolor{currentstroke}{rgb}{0.260000,0.260000,0.260000}%
\pgfsetstrokecolor{currentstroke}%
\pgfsetdash{}{0pt}%
\pgfusepath{stroke}%
\end{pgfscope}%
\begin{pgfscope}%
\pgfpathrectangle{\pgfqpoint{1.230000in}{1.022500in}}{\pgfqpoint{6.455000in}{3.359167in}}%
\pgfusepath{clip}%
\pgfsetroundcap%
\pgfsetroundjoin%
\pgfsetlinewidth{2.710125pt}%
\definecolor{currentstroke}{rgb}{0.260000,0.260000,0.260000}%
\pgfsetstrokecolor{currentstroke}%
\pgfsetdash{}{0pt}%
\pgfusepath{stroke}%
\end{pgfscope}%
\begin{pgfscope}%
\pgfpathrectangle{\pgfqpoint{1.230000in}{1.022500in}}{\pgfqpoint{6.455000in}{3.359167in}}%
\pgfusepath{clip}%
\pgfsetroundcap%
\pgfsetroundjoin%
\pgfsetlinewidth{2.710125pt}%
\definecolor{currentstroke}{rgb}{0.260000,0.260000,0.260000}%
\pgfsetstrokecolor{currentstroke}%
\pgfsetdash{}{0pt}%
\pgfusepath{stroke}%
\end{pgfscope}%
\begin{pgfscope}%
\pgfpathrectangle{\pgfqpoint{1.230000in}{1.022500in}}{\pgfqpoint{6.455000in}{3.359167in}}%
\pgfusepath{clip}%
\pgfsetroundcap%
\pgfsetroundjoin%
\pgfsetlinewidth{2.710125pt}%
\definecolor{currentstroke}{rgb}{0.260000,0.260000,0.260000}%
\pgfsetstrokecolor{currentstroke}%
\pgfsetdash{}{0pt}%
\pgfusepath{stroke}%
\end{pgfscope}%
\begin{pgfscope}%
\pgfpathrectangle{\pgfqpoint{1.230000in}{1.022500in}}{\pgfqpoint{6.455000in}{3.359167in}}%
\pgfusepath{clip}%
\pgfsetroundcap%
\pgfsetroundjoin%
\pgfsetlinewidth{2.710125pt}%
\definecolor{currentstroke}{rgb}{0.260000,0.260000,0.260000}%
\pgfsetstrokecolor{currentstroke}%
\pgfsetdash{}{0pt}%
\pgfusepath{stroke}%
\end{pgfscope}%
\begin{pgfscope}%
\pgfpathrectangle{\pgfqpoint{1.230000in}{1.022500in}}{\pgfqpoint{6.455000in}{3.359167in}}%
\pgfusepath{clip}%
\pgfsetroundcap%
\pgfsetroundjoin%
\pgfsetlinewidth{2.710125pt}%
\definecolor{currentstroke}{rgb}{0.260000,0.260000,0.260000}%
\pgfsetstrokecolor{currentstroke}%
\pgfsetdash{}{0pt}%
\pgfusepath{stroke}%
\end{pgfscope}%
\begin{pgfscope}%
\pgfpathrectangle{\pgfqpoint{1.230000in}{1.022500in}}{\pgfqpoint{6.455000in}{3.359167in}}%
\pgfusepath{clip}%
\pgfsetroundcap%
\pgfsetroundjoin%
\pgfsetlinewidth{2.710125pt}%
\definecolor{currentstroke}{rgb}{0.260000,0.260000,0.260000}%
\pgfsetstrokecolor{currentstroke}%
\pgfsetdash{}{0pt}%
\pgfusepath{stroke}%
\end{pgfscope}%
\begin{pgfscope}%
\pgfsetrectcap%
\pgfsetmiterjoin%
\pgfsetlinewidth{1.254687pt}%
\definecolor{currentstroke}{rgb}{0.800000,0.800000,0.800000}%
\pgfsetstrokecolor{currentstroke}%
\pgfsetdash{}{0pt}%
\pgfpathmoveto{\pgfqpoint{1.230000in}{1.022500in}}%
\pgfpathlineto{\pgfqpoint{7.685000in}{1.022500in}}%
\pgfusepath{stroke}%
\end{pgfscope}%
\begin{pgfscope}%
\definecolor{textcolor}{rgb}{0.150000,0.150000,0.150000}%
\pgfsetstrokecolor{textcolor}%
\pgfsetfillcolor{textcolor}%
\pgftext[x=4.457500in,y=4.465000in,,base]{\color{textcolor}\sffamily\fontsize{21.000000}{25.200000}\selectfont libseccomp to eBPF benchmarks}%
\end{pgfscope}%
\begin{pgfscope}%
\pgfsetbuttcap%
\pgfsetmiterjoin%
\definecolor{currentfill}{rgb}{1.000000,1.000000,1.000000}%
\pgfsetfillcolor{currentfill}%
\pgfsetfillopacity{0.800000}%
\pgfsetlinewidth{1.003750pt}%
\definecolor{currentstroke}{rgb}{0.800000,0.800000,0.800000}%
\pgfsetstrokecolor{currentstroke}%
\pgfsetstrokeopacity{0.800000}%
\pgfsetdash{}{0pt}%
\pgfpathmoveto{\pgfqpoint{5.420190in}{3.388281in}}%
\pgfpathlineto{\pgfqpoint{7.497847in}{3.388281in}}%
\pgfpathquadraticcurveto{\pgfqpoint{7.551319in}{3.388281in}}{\pgfqpoint{7.551319in}{3.441754in}}%
\pgfpathlineto{\pgfqpoint{7.551319in}{4.194514in}}%
\pgfpathquadraticcurveto{\pgfqpoint{7.551319in}{4.247986in}}{\pgfqpoint{7.497847in}{4.247986in}}%
\pgfpathlineto{\pgfqpoint{5.420190in}{4.247986in}}%
\pgfpathquadraticcurveto{\pgfqpoint{5.366718in}{4.247986in}}{\pgfqpoint{5.366718in}{4.194514in}}%
\pgfpathlineto{\pgfqpoint{5.366718in}{3.441754in}}%
\pgfpathquadraticcurveto{\pgfqpoint{5.366718in}{3.388281in}}{\pgfqpoint{5.420190in}{3.388281in}}%
\pgfpathclose%
\pgfusepath{stroke,fill}%
\end{pgfscope}%
\begin{pgfscope}%
\pgfsetbuttcap%
\pgfsetmiterjoin%
\definecolor{currentfill}{rgb}{0.347059,0.458824,0.641176}%
\pgfsetfillcolor{currentfill}%
\pgfsetlinewidth{1.003750pt}%
\definecolor{currentstroke}{rgb}{1.000000,1.000000,1.000000}%
\pgfsetstrokecolor{currentstroke}%
\pgfsetdash{}{0pt}%
\pgfpathmoveto{\pgfqpoint{5.473662in}{3.941003in}}%
\pgfpathlineto{\pgfqpoint{6.008384in}{3.941003in}}%
\pgfpathlineto{\pgfqpoint{6.008384in}{4.128156in}}%
\pgfpathlineto{\pgfqpoint{5.473662in}{4.128156in}}%
\pgfpathclose%
\pgfusepath{stroke,fill}%
\end{pgfscope}%
\begin{pgfscope}%
\pgfsetbuttcap%
\pgfsetmiterjoin%
\definecolor{currentfill}{rgb}{0.347059,0.458824,0.641176}%
\pgfsetfillcolor{currentfill}%
\pgfsetlinewidth{1.003750pt}%
\definecolor{currentstroke}{rgb}{1.000000,1.000000,1.000000}%
\pgfsetstrokecolor{currentstroke}%
\pgfsetdash{}{0pt}%
\pgfpathmoveto{\pgfqpoint{5.473662in}{3.941003in}}%
\pgfpathlineto{\pgfqpoint{6.008384in}{3.941003in}}%
\pgfpathlineto{\pgfqpoint{6.008384in}{4.128156in}}%
\pgfpathlineto{\pgfqpoint{5.473662in}{4.128156in}}%
\pgfpathclose%
\pgfusepath{clip}%
\pgfsys@defobject{currentpattern}{\pgfqpoint{0in}{0in}}{\pgfqpoint{1in}{1in}}{%
\begin{pgfscope}%
\pgfpathrectangle{\pgfqpoint{0in}{0in}}{\pgfqpoint{1in}{1in}}%
\pgfusepath{clip}%
\pgfpathmoveto{\pgfqpoint{-0.500000in}{0.500000in}}%
\pgfpathlineto{\pgfqpoint{0.500000in}{1.500000in}}%
\pgfpathmoveto{\pgfqpoint{-0.444444in}{0.444444in}}%
\pgfpathlineto{\pgfqpoint{0.555556in}{1.444444in}}%
\pgfpathmoveto{\pgfqpoint{-0.388889in}{0.388889in}}%
\pgfpathlineto{\pgfqpoint{0.611111in}{1.388889in}}%
\pgfpathmoveto{\pgfqpoint{-0.333333in}{0.333333in}}%
\pgfpathlineto{\pgfqpoint{0.666667in}{1.333333in}}%
\pgfpathmoveto{\pgfqpoint{-0.277778in}{0.277778in}}%
\pgfpathlineto{\pgfqpoint{0.722222in}{1.277778in}}%
\pgfpathmoveto{\pgfqpoint{-0.222222in}{0.222222in}}%
\pgfpathlineto{\pgfqpoint{0.777778in}{1.222222in}}%
\pgfpathmoveto{\pgfqpoint{-0.166667in}{0.166667in}}%
\pgfpathlineto{\pgfqpoint{0.833333in}{1.166667in}}%
\pgfpathmoveto{\pgfqpoint{-0.111111in}{0.111111in}}%
\pgfpathlineto{\pgfqpoint{0.888889in}{1.111111in}}%
\pgfpathmoveto{\pgfqpoint{-0.055556in}{0.055556in}}%
\pgfpathlineto{\pgfqpoint{0.944444in}{1.055556in}}%
\pgfpathmoveto{\pgfqpoint{0.000000in}{0.000000in}}%
\pgfpathlineto{\pgfqpoint{1.000000in}{1.000000in}}%
\pgfpathmoveto{\pgfqpoint{0.055556in}{-0.055556in}}%
\pgfpathlineto{\pgfqpoint{1.055556in}{0.944444in}}%
\pgfpathmoveto{\pgfqpoint{0.111111in}{-0.111111in}}%
\pgfpathlineto{\pgfqpoint{1.111111in}{0.888889in}}%
\pgfpathmoveto{\pgfqpoint{0.166667in}{-0.166667in}}%
\pgfpathlineto{\pgfqpoint{1.166667in}{0.833333in}}%
\pgfpathmoveto{\pgfqpoint{0.222222in}{-0.222222in}}%
\pgfpathlineto{\pgfqpoint{1.222222in}{0.777778in}}%
\pgfpathmoveto{\pgfqpoint{0.277778in}{-0.277778in}}%
\pgfpathlineto{\pgfqpoint{1.277778in}{0.722222in}}%
\pgfpathmoveto{\pgfqpoint{0.333333in}{-0.333333in}}%
\pgfpathlineto{\pgfqpoint{1.333333in}{0.666667in}}%
\pgfpathmoveto{\pgfqpoint{0.388889in}{-0.388889in}}%
\pgfpathlineto{\pgfqpoint{1.388889in}{0.611111in}}%
\pgfpathmoveto{\pgfqpoint{0.444444in}{-0.444444in}}%
\pgfpathlineto{\pgfqpoint{1.444444in}{0.555556in}}%
\pgfpathmoveto{\pgfqpoint{0.500000in}{-0.500000in}}%
\pgfpathlineto{\pgfqpoint{1.500000in}{0.500000in}}%
\pgfusepath{stroke}%
\end{pgfscope}%
}%
\pgfsys@transformshift{5.473662in}{3.941003in}%
\pgfsys@useobject{currentpattern}{}%
\pgfsys@transformshift{1in}{0in}%
\pgfsys@transformshift{-1in}{0in}%
\pgfsys@transformshift{0in}{1in}%
\end{pgfscope}%
\begin{pgfscope}%
\definecolor{textcolor}{rgb}{0.150000,0.150000,0.150000}%
\pgfsetstrokecolor{textcolor}%
\pgfsetfillcolor{textcolor}%
\pgftext[x=6.222273in,y=3.941003in,left,base]{\color{textcolor}\sffamily\fontsize{19.250000}{23.100000}\selectfont libseccomp}%
\end{pgfscope}%
\begin{pgfscope}%
\pgfsetbuttcap%
\pgfsetmiterjoin%
\definecolor{currentfill}{rgb}{0.798529,0.536765,0.389706}%
\pgfsetfillcolor{currentfill}%
\pgfsetlinewidth{1.003750pt}%
\definecolor{currentstroke}{rgb}{1.000000,1.000000,1.000000}%
\pgfsetstrokecolor{currentstroke}%
\pgfsetdash{}{0pt}%
\pgfpathmoveto{\pgfqpoint{5.473662in}{3.551255in}}%
\pgfpathlineto{\pgfqpoint{6.008384in}{3.551255in}}%
\pgfpathlineto{\pgfqpoint{6.008384in}{3.738408in}}%
\pgfpathlineto{\pgfqpoint{5.473662in}{3.738408in}}%
\pgfpathclose%
\pgfusepath{stroke,fill}%
\end{pgfscope}%
\begin{pgfscope}%
\pgfsetbuttcap%
\pgfsetmiterjoin%
\definecolor{currentfill}{rgb}{0.798529,0.536765,0.389706}%
\pgfsetfillcolor{currentfill}%
\pgfsetlinewidth{1.003750pt}%
\definecolor{currentstroke}{rgb}{1.000000,1.000000,1.000000}%
\pgfsetstrokecolor{currentstroke}%
\pgfsetdash{}{0pt}%
\pgfpathmoveto{\pgfqpoint{5.473662in}{3.551255in}}%
\pgfpathlineto{\pgfqpoint{6.008384in}{3.551255in}}%
\pgfpathlineto{\pgfqpoint{6.008384in}{3.738408in}}%
\pgfpathlineto{\pgfqpoint{5.473662in}{3.738408in}}%
\pgfpathclose%
\pgfusepath{clip}%
\pgfsys@defobject{currentpattern}{\pgfqpoint{0in}{0in}}{\pgfqpoint{1in}{1in}}{%
\begin{pgfscope}%
\pgfpathrectangle{\pgfqpoint{0in}{0in}}{\pgfqpoint{1in}{1in}}%
\pgfusepath{clip}%
\pgfusepath{stroke}%
\end{pgfscope}%
}%
\pgfsys@transformshift{5.473662in}{3.551255in}%
\pgfsys@useobject{currentpattern}{}%
\pgfsys@transformshift{1in}{0in}%
\pgfsys@transformshift{-1in}{0in}%
\pgfsys@transformshift{0in}{1in}%
\end{pgfscope}%
\begin{pgfscope}%
\definecolor{textcolor}{rgb}{0.150000,0.150000,0.150000}%
\pgfsetstrokecolor{textcolor}%
\pgfsetfillcolor{textcolor}%
\pgftext[x=6.222273in,y=3.551255in,left,base]{\color{textcolor}\sffamily\fontsize{19.250000}{23.100000}\selectfont JitSynth}%
\end{pgfscope}%
\end{pgfpicture}%
\makeatother%
\endgroup%

  }
  \caption{Performance of code generated by \jitsynth compilers
  compared to existing compilers for the classic BPF to eBPF benchmarks
  (left) and the libseccomp to eBPF benchmarks (right). Measured in number
  of instructions executed.}
  \label{fig:o2b-l2b-runtime}
\end{figure}

Classic BPF is the original, simpler version of BPF used for packet filtering which
was later extended to eBPF in Linux.
%
Since many applications still use classic BPF, Linux must first compile classic BPF
to eBPF as an intermediary step before compiling to machine instructions.
%
As a second case study, we used \jitsynth to synthesize a compiler from classic BPF to eBPF.
%
Our synthesized compiler supports all classic BPF opcodes.
%
To evaluate performance, we compare against the existing
Linux classic-BPF-to-eBPF compiler.
%
Similar to the RISC-V benchmarks, we run each eBPF program
with input that is allowed by the filter.
%
Because eBPF does not run directly on hardware, we measure
the number of instructions executed instead of processor cycles.


\autoref{fig:o2b-l2b-runtime} shows the performance results.
%
Classic BPF programs generated by \jitsynth compilers execute
an average of $\CbpfSlowdown\times$ more instructions
than those compiled by Linux.


\paragraph{libseccomp to eBPF.}

libseccomp is a library used to simplify construction of BPF system call filters.
%
The existing libseccomp implementation compiles to classic BPF; we instead choose
to compile to eBPF because classic BPF has only two registers, which does not satisfy
the assumptions of \jitsynth.
%
Since libseccomp is a library and does not have distinct instructions, libseccomp itself does not meet the definition of an abstract register machine; we instead introduce
an intermediate libseccomp language which does satisfy this definition.
%
Our full libseccomp to eBPF compiler is composed of both a trusted program to translate from libseccomp to our intermediate language and a synthesized compiler from our intermediate language to eBPF.

To evaluate performance, we select a set of benchmark filters from real-world
applications that use libseccomp, and measure the number of eBPF instructions
executed for an input the filter allows.
%
Because no existing compiler exists from libseccomp to eBPF directly,
we compare against the composition of the existing libseccomp-to-classic-BPF and classic-BPF-to-eBPF compilers.


\autoref{fig:o2b-l2b-runtime} shows the performance results.
%
libseccomp programs generated by \jitsynth execute
$\LibseccompSlowdown\times$ more instructions on average
compared to the existing libseccomp-to-eBPF compiler stack.
%
However, the synthesized compiler avoids bugs previously found
in the libseccomp-to-classic-BPF compiler~\cite{lsc:bug}.
%
% This shows that \jitsynth can synthesize a compiler for a new
% source-target pair that approaches the performance of the existing
% compiler stack. % This sentence is super weird
%
% At the same time, the synthesized compiler avoids bugs previously
% found in the libseccomp-to-classic-BPF compiler~\cite{lsc:bug}.
%



% \if 0
% \autoref{fig:lang} describes the instructions in both instruction sets.
% %
% Our compiler supports a subset of
% %

% To validate that the synthesized compiler is correct, we ran the existing eBPF
% test suite in the Linux kernel.
% %
% In addition to passing these tests, our compiler also avoids bugs previously
% found in the existing eBPF to RISC-V compiler in Linux~\cite{nelson:bpf-riscv-add32-bug}.
% %
% This evidence shows that \jitsynth can synthesize reliable compilers for real-world DSLs.


% We evaluated the performance of code generated by the synthesized compiler against
% both the existing Linux eBPF to RISC-V compiler and the Linux eBPF interpreter.
% %
% Our evaluation uses the same set of benchmarks used by Jitk~\cite{wang:jitk}, which includes system call filters from widely used applications.
% %OpenSSH, vsftpd, Native Client (NaCl), QEMU, Firefox, Chrome, and Tor.
% %
% Because these benchmarks were originally for classic BPF,
% we first compile them to eBPF using the existing
% Linux classic BPF to eBPF compiler as a preprocessing step.
% %
% We ran all benchmarks on the HiFive Unleashed RISC-V development board, and measured
% the total number of processor cycles executed in the common case, for each filter
% and for each execution method.



% \autoref{fig:b2r-runtime} shows the results of the performance evaluation.
% %
% eBPF programs compiled by \jitsynth JIT compilers show an average slowdown of $\EbpfCompilerSlowdown\times$
% compared to programs compiled by the existing Linux compiler.
% %
% This overhead is from \todo{XXX}.
% %
% \jitsynth-compiled programs get an average speedup of $\EbpfInterpSpeedup\times$
% compared to interpreting the eBPF programs, preserving the benefit
% of JIT compilation.
% %
% %In addition, \jitsynth compilers avoid \todo{previously found bugs in the Linux compiler}.
% %
% This is evidence that \jitsynth can synthesize a compiler that outperforms
% %
% This shows that \jitsynth can synthesize a compiler for a real-world source-target pair,
% which out-performs the existing interpreter and nears the performance of the existing compiler.
% %
% ISA-level optimizations could reduce this performance gap between the \jitsynth compiler
% and the Linux compiler, but we consider this to be outside the scope of our work.
% %
% \fi

% We also evaluated the performance of code generated by the \jitsynth eBPF-to-RISC-V compiler, comparing it to both the existing Linux compiler as well as the Linux eBPF interpreter.
% To give a more realistic comparison, we also applied a simple NOP-removal pass on top of the \jitsynth compiler.
% This pass, done by trusted code, removes NOP instructions from code generated by the \jitsynth compiler
% while preserving the semantics of the code.
% \todo{it's odd to bring up NOP removal for the first time here.
% maybe add this part to the actual JIT (rather than a patch)?}
%To further improve the speed of code generated by the \jitsynth compiler, further optimization passes could also be made.

\if 0
\subsection{Synthesizing compilers for multiple source-target pairs}


%NB: should this subsection and the previous section be merged? they
%    seem to be making similar claims

%NB: this section is very repetitive. We probably can factor
%    the libseccomp->ebpf and cbpf->ebpf portions to minimize
%    the space it takes up

To demonstrate the generality of our approach, we also apply \jitsynth to
synthesize compilers for two additional source and target pairs:
classic BPF to eBPF, and libseccomp to eBPF.
%
Classic BPF is the original, simpler version of BPF used for packet filtering which
was later extended to eBPF in Linux.
%
Since many applications still use classic BPF, Linux must first compile classic BPF
to eBPF as an intermediary before compiling to machine instructions.
%
libseccomp is a library used to simplify construction of BPF system call filters.
%
The existing libseccomp implementation compiles to classic BPF; we instead choose
to compile to eBPF as classic BPF has only two registers, which does not satisfy
the assumptions of \jitsynth.
%
Since libseccomp is a library and does not have distinct instructions, libseccomp itself does not meet the definition of an abstract register machine; we instead introduce
an intermediate libseccomp language which does satisfy this definition.
%
Our full libseccomp to eBPF compiler is composed of both a trusted program to translate from libseccomp to our intermediate language and a synthesized compiler from our intermediate language to eBPF.


To evaluate the compilers generated by \jitsynth, we compile a set of benchmarks
to eBPF and perform a worst-case execution analysis on the generated code, and
compare against the existing implementation.
%
Because no existing compiler exists from libseccomp to eBPF directly,
we compare against the composition of the libseccomp to classic BPF implementation and the
existing Linux classic BPF to eBPF compiler.
%
The classic BPF benchmarks are the same used in Jitk; the libseccomp benchmarks
are selected from a handful of real-world applications.
%
\autoref{fig:o2b-l2b-runtime} shows the results of the analysis.
%
Classic BPF program generated by \jitsynth compilers execute
an average of $\CbpfSlowdown\times$ more instructions in the worst case
than those compiled Linux.
%
Similarly, libseccomp programs generated by \jitsynth show a
$\LibseccompSlowdown\times$ slowdown compared to the existing libseccomp
to eBPF compiler stack.
%
These results show that \jitsynth can generate compilers which generate
code that approaches the performance of existing compilers, while avoiding
bugs like those previously found in the libseccomp compiler~\cite{lsc:bug}.
%


\if 0
We evaluate our synthesized classic BPF to eBPF compiler using the same benchmarks
as in Jitk.
%
To compare our compiler against the existing Linux compiler, we compile each
benchmark to eBPF, and perform a worst-case execution analysis of the generated
code, measuring the number of eBPF instructions executed.
%
\autoref{fig:o2b-l2b-runtime} shows the results of this analysis.
%
BPF programs compiled by \jitsynth compilers execute an average of
$\CbpfSlowdown\times$ more instructions in the worst case than those
compiled by Linux.
%
This shows that the \jitsynth-synthesized compiler generates code
near the performance of that generated by Linux.
%
% The results for both source-target pairs are shown in \autoref{fig:o2b-l2b-runtime}.

% Classic BPF is the original version of BPF used for packet filtering, later expanded into eBPF.
% Since many applications still use classic BPF, compiling to eBPF is an essential step for running these filters.
% In particular, each of the applications discussed in the previous section emits classic BPF filters, which are compiled into eBPF in the kernel.
% In this section, we compare the compiler that \jitsynth synthesizes to the existing classic BPF to eBPF compiler in the Linux kernel using these same filters.
% For each filter, we evaluate the worst-case execution time of the compiled eBPF code in number of instructions.
% On average, filters compiled by the \jitsynth compiler suffer an average slowdown of \todo{$2.94\times$} compared to the Linux kernel compiler for classic BPF.

In addition to classic BPF, we also synthesize a compiler for libseccomp,
a library used to build classic BPF system call filters.
%
Though libseccomp itself compiles to classic BPF, we choose to generate a compiler from libseccomp to eBPF.
%
We do so because no compiler currently exists from libseccomp to eBPF, and because
classic BPF has only two registers, which does not meet our assumptions for \jitsynth.
%
Since libseccomp is a library and does not have distinct instructions, libseccomp itself does not match the assumptions of an abstract register machine.
%
However, we introduce an intermediate libseccomp language which does meet these assumptions.
%
Our full libseccomp to eBPF compiler is composed of both a trusted program to translate from libseccomp to our intermediate language and a synthesized compiler from our intermediate language to eBPF.
%
This compiler avoids bugs found in the current libseccomp implementation~\cite{lsc:bug}.

We evaluate our compiler using benchmarks taken from a handful of real-world applications
that use libseccomp, including Ctags, Lepton, LibreOffice, OpenSSH, and vsftpd.
%
To compare against the existing libseccomp compiler, we compile the libseccomp
filters to classic BPF using the existing compiler, and then into eBPF using
the Linux classic BPF to eBPF compiler.
%
We compare this result the code generated by our synthesized libseccomp to eBPF compiler,
using the same worst-case execution analysis.
%
The \jitsynth compiler has a $\LibseccompSlowdown\times$ compared to the
existing libseccomp to eBPF compiler stack.
%
% NB: What's the point to make here?
%
\fi
\fi

\subsection{Effectiveness of sketch optimizations}

\begin{figure}[h]
  % \resizebox{\textwidth}{!}{
  % %% Creator: Matplotlib, PGF backend
%%
%% To include the figure in your LaTeX document, write
%%   \input{<filename>.pgf}
%%
%% Make sure the required packages are loaded in your preamble
%%   \usepackage{pgf}
%%
%% Figures using additional raster images can only be included by \input if
%% they are in the same directory as the main LaTeX file. For loading figures
%% from other directories you can use the `import` package
%%   \usepackage{import}
%% and then include the figures with
%%   \import{<path to file>}{<filename>.pgf}
%%
%% Matplotlib used the following preamble
%%
\begingroup%
\makeatletter%
\begin{pgfpicture}%
\pgfpathrectangle{\pgfpointorigin}{\pgfqpoint{6.400000in}{4.800000in}}%
\pgfusepath{use as bounding box, clip}%
\begin{pgfscope}%
\pgfsetbuttcap%
\pgfsetmiterjoin%
\definecolor{currentfill}{rgb}{1.000000,1.000000,1.000000}%
\pgfsetfillcolor{currentfill}%
\pgfsetlinewidth{0.000000pt}%
\definecolor{currentstroke}{rgb}{1.000000,1.000000,1.000000}%
\pgfsetstrokecolor{currentstroke}%
\pgfsetdash{}{0pt}%
\pgfpathmoveto{\pgfqpoint{0.000000in}{0.000000in}}%
\pgfpathlineto{\pgfqpoint{6.400000in}{0.000000in}}%
\pgfpathlineto{\pgfqpoint{6.400000in}{4.800000in}}%
\pgfpathlineto{\pgfqpoint{0.000000in}{4.800000in}}%
\pgfpathclose%
\pgfusepath{fill}%
\end{pgfscope}%
\begin{pgfscope}%
\pgfsetbuttcap%
\pgfsetmiterjoin%
\definecolor{currentfill}{rgb}{1.000000,1.000000,1.000000}%
\pgfsetfillcolor{currentfill}%
\pgfsetlinewidth{0.000000pt}%
\definecolor{currentstroke}{rgb}{0.000000,0.000000,0.000000}%
\pgfsetstrokecolor{currentstroke}%
\pgfsetstrokeopacity{0.000000}%
\pgfsetdash{}{0pt}%
\pgfpathmoveto{\pgfqpoint{1.230000in}{1.022500in}}%
\pgfpathlineto{\pgfqpoint{6.077630in}{1.022500in}}%
\pgfpathlineto{\pgfqpoint{6.077630in}{4.078625in}}%
\pgfpathlineto{\pgfqpoint{1.230000in}{4.078625in}}%
\pgfpathclose%
\pgfusepath{fill}%
\end{pgfscope}%
\begin{pgfscope}%
\pgfpathrectangle{\pgfqpoint{1.230000in}{1.022500in}}{\pgfqpoint{4.847630in}{3.056125in}}%
\pgfusepath{clip}%
\pgfsetroundcap%
\pgfsetroundjoin%
\pgfsetlinewidth{1.003750pt}%
\definecolor{currentstroke}{rgb}{0.800000,0.800000,0.800000}%
\pgfsetstrokecolor{currentstroke}%
\pgfsetstrokeopacity{0.400000}%
\pgfsetdash{}{0pt}%
\pgfpathmoveto{\pgfqpoint{1.450347in}{1.022500in}}%
\pgfpathlineto{\pgfqpoint{1.450347in}{4.078625in}}%
\pgfusepath{stroke}%
\end{pgfscope}%
\begin{pgfscope}%
\definecolor{textcolor}{rgb}{0.150000,0.150000,0.150000}%
\pgfsetstrokecolor{textcolor}%
\pgfsetfillcolor{textcolor}%
\pgftext[x=1.450347in,y=0.890556in,,top]{\color{textcolor}\sffamily\fontsize{19.250000}{23.100000}\selectfont 0}%
\end{pgfscope}%
\begin{pgfscope}%
\pgfpathrectangle{\pgfqpoint{1.230000in}{1.022500in}}{\pgfqpoint{4.847630in}{3.056125in}}%
\pgfusepath{clip}%
\pgfsetroundcap%
\pgfsetroundjoin%
\pgfsetlinewidth{1.003750pt}%
\definecolor{currentstroke}{rgb}{0.800000,0.800000,0.800000}%
\pgfsetstrokecolor{currentstroke}%
\pgfsetstrokeopacity{0.400000}%
\pgfsetdash{}{0pt}%
\pgfpathmoveto{\pgfqpoint{2.849592in}{1.022500in}}%
\pgfpathlineto{\pgfqpoint{2.849592in}{4.078625in}}%
\pgfusepath{stroke}%
\end{pgfscope}%
\begin{pgfscope}%
\definecolor{textcolor}{rgb}{0.150000,0.150000,0.150000}%
\pgfsetstrokecolor{textcolor}%
\pgfsetfillcolor{textcolor}%
\pgftext[x=2.849592in,y=0.890556in,,top]{\color{textcolor}\sffamily\fontsize{19.250000}{23.100000}\selectfont 50000}%
\end{pgfscope}%
\begin{pgfscope}%
\pgfpathrectangle{\pgfqpoint{1.230000in}{1.022500in}}{\pgfqpoint{4.847630in}{3.056125in}}%
\pgfusepath{clip}%
\pgfsetroundcap%
\pgfsetroundjoin%
\pgfsetlinewidth{1.003750pt}%
\definecolor{currentstroke}{rgb}{0.800000,0.800000,0.800000}%
\pgfsetstrokecolor{currentstroke}%
\pgfsetstrokeopacity{0.400000}%
\pgfsetdash{}{0pt}%
\pgfpathmoveto{\pgfqpoint{4.248837in}{1.022500in}}%
\pgfpathlineto{\pgfqpoint{4.248837in}{4.078625in}}%
\pgfusepath{stroke}%
\end{pgfscope}%
\begin{pgfscope}%
\definecolor{textcolor}{rgb}{0.150000,0.150000,0.150000}%
\pgfsetstrokecolor{textcolor}%
\pgfsetfillcolor{textcolor}%
\pgftext[x=4.248837in,y=0.890556in,,top]{\color{textcolor}\sffamily\fontsize{19.250000}{23.100000}\selectfont 100000}%
\end{pgfscope}%
\begin{pgfscope}%
\pgfpathrectangle{\pgfqpoint{1.230000in}{1.022500in}}{\pgfqpoint{4.847630in}{3.056125in}}%
\pgfusepath{clip}%
\pgfsetroundcap%
\pgfsetroundjoin%
\pgfsetlinewidth{1.003750pt}%
\definecolor{currentstroke}{rgb}{0.800000,0.800000,0.800000}%
\pgfsetstrokecolor{currentstroke}%
\pgfsetstrokeopacity{0.400000}%
\pgfsetdash{}{0pt}%
\pgfpathmoveto{\pgfqpoint{5.648082in}{1.022500in}}%
\pgfpathlineto{\pgfqpoint{5.648082in}{4.078625in}}%
\pgfusepath{stroke}%
\end{pgfscope}%
\begin{pgfscope}%
\definecolor{textcolor}{rgb}{0.150000,0.150000,0.150000}%
\pgfsetstrokecolor{textcolor}%
\pgfsetfillcolor{textcolor}%
\pgftext[x=5.648082in,y=0.890556in,,top]{\color{textcolor}\sffamily\fontsize{19.250000}{23.100000}\selectfont 150000}%
\end{pgfscope}%
\begin{pgfscope}%
\definecolor{textcolor}{rgb}{0.150000,0.150000,0.150000}%
\pgfsetstrokecolor{textcolor}%
\pgfsetfillcolor{textcolor}%
\pgftext[x=3.653815in,y=0.578932in,,top]{\color{textcolor}\sffamily\fontsize{21.000000}{25.200000}\selectfont Time (s)}%
\end{pgfscope}%
\begin{pgfscope}%
\pgfpathrectangle{\pgfqpoint{1.230000in}{1.022500in}}{\pgfqpoint{4.847630in}{3.056125in}}%
\pgfusepath{clip}%
\pgfsetroundcap%
\pgfsetroundjoin%
\pgfsetlinewidth{1.003750pt}%
\definecolor{currentstroke}{rgb}{0.800000,0.800000,0.800000}%
\pgfsetstrokecolor{currentstroke}%
\pgfsetstrokeopacity{0.400000}%
\pgfsetdash{}{0pt}%
\pgfpathmoveto{\pgfqpoint{1.230000in}{1.193435in}}%
\pgfpathlineto{\pgfqpoint{6.077630in}{1.193435in}}%
\pgfusepath{stroke}%
\end{pgfscope}%
\begin{pgfscope}%
\definecolor{textcolor}{rgb}{0.150000,0.150000,0.150000}%
\pgfsetstrokecolor{textcolor}%
\pgfsetfillcolor{textcolor}%
\pgftext[x=0.962614in,y=1.093416in,left,base]{\color{textcolor}\sffamily\fontsize{19.250000}{23.100000}\selectfont 0}%
\end{pgfscope}%
\begin{pgfscope}%
\pgfpathrectangle{\pgfqpoint{1.230000in}{1.022500in}}{\pgfqpoint{4.847630in}{3.056125in}}%
\pgfusepath{clip}%
\pgfsetroundcap%
\pgfsetroundjoin%
\pgfsetlinewidth{1.003750pt}%
\definecolor{currentstroke}{rgb}{0.800000,0.800000,0.800000}%
\pgfsetstrokecolor{currentstroke}%
\pgfsetstrokeopacity{0.400000}%
\pgfsetdash{}{0pt}%
\pgfpathmoveto{\pgfqpoint{1.230000in}{1.880004in}}%
\pgfpathlineto{\pgfqpoint{6.077630in}{1.880004in}}%
\pgfusepath{stroke}%
\end{pgfscope}%
\begin{pgfscope}%
\definecolor{textcolor}{rgb}{0.150000,0.150000,0.150000}%
\pgfsetstrokecolor{textcolor}%
\pgfsetfillcolor{textcolor}%
\pgftext[x=0.827172in,y=1.779985in,left,base]{\color{textcolor}\sffamily\fontsize{19.250000}{23.100000}\selectfont 25}%
\end{pgfscope}%
\begin{pgfscope}%
\pgfpathrectangle{\pgfqpoint{1.230000in}{1.022500in}}{\pgfqpoint{4.847630in}{3.056125in}}%
\pgfusepath{clip}%
\pgfsetroundcap%
\pgfsetroundjoin%
\pgfsetlinewidth{1.003750pt}%
\definecolor{currentstroke}{rgb}{0.800000,0.800000,0.800000}%
\pgfsetstrokecolor{currentstroke}%
\pgfsetstrokeopacity{0.400000}%
\pgfsetdash{}{0pt}%
\pgfpathmoveto{\pgfqpoint{1.230000in}{2.566573in}}%
\pgfpathlineto{\pgfqpoint{6.077630in}{2.566573in}}%
\pgfusepath{stroke}%
\end{pgfscope}%
\begin{pgfscope}%
\definecolor{textcolor}{rgb}{0.150000,0.150000,0.150000}%
\pgfsetstrokecolor{textcolor}%
\pgfsetfillcolor{textcolor}%
\pgftext[x=0.827172in,y=2.466554in,left,base]{\color{textcolor}\sffamily\fontsize{19.250000}{23.100000}\selectfont 50}%
\end{pgfscope}%
\begin{pgfscope}%
\pgfpathrectangle{\pgfqpoint{1.230000in}{1.022500in}}{\pgfqpoint{4.847630in}{3.056125in}}%
\pgfusepath{clip}%
\pgfsetroundcap%
\pgfsetroundjoin%
\pgfsetlinewidth{1.003750pt}%
\definecolor{currentstroke}{rgb}{0.800000,0.800000,0.800000}%
\pgfsetstrokecolor{currentstroke}%
\pgfsetstrokeopacity{0.400000}%
\pgfsetdash{}{0pt}%
\pgfpathmoveto{\pgfqpoint{1.230000in}{3.253141in}}%
\pgfpathlineto{\pgfqpoint{6.077630in}{3.253141in}}%
\pgfusepath{stroke}%
\end{pgfscope}%
\begin{pgfscope}%
\definecolor{textcolor}{rgb}{0.150000,0.150000,0.150000}%
\pgfsetstrokecolor{textcolor}%
\pgfsetfillcolor{textcolor}%
\pgftext[x=0.827172in,y=3.153122in,left,base]{\color{textcolor}\sffamily\fontsize{19.250000}{23.100000}\selectfont 75}%
\end{pgfscope}%
\begin{pgfscope}%
\pgfpathrectangle{\pgfqpoint{1.230000in}{1.022500in}}{\pgfqpoint{4.847630in}{3.056125in}}%
\pgfusepath{clip}%
\pgfsetroundcap%
\pgfsetroundjoin%
\pgfsetlinewidth{1.003750pt}%
\definecolor{currentstroke}{rgb}{0.800000,0.800000,0.800000}%
\pgfsetstrokecolor{currentstroke}%
\pgfsetstrokeopacity{0.400000}%
\pgfsetdash{}{0pt}%
\pgfpathmoveto{\pgfqpoint{1.230000in}{3.939710in}}%
\pgfpathlineto{\pgfqpoint{6.077630in}{3.939710in}}%
\pgfusepath{stroke}%
\end{pgfscope}%
\begin{pgfscope}%
\definecolor{textcolor}{rgb}{0.150000,0.150000,0.150000}%
\pgfsetstrokecolor{textcolor}%
\pgfsetfillcolor{textcolor}%
\pgftext[x=0.691731in,y=3.839691in,left,base]{\color{textcolor}\sffamily\fontsize{19.250000}{23.100000}\selectfont 100}%
\end{pgfscope}%
\begin{pgfscope}%
\definecolor{textcolor}{rgb}{0.150000,0.150000,0.150000}%
\pgfsetstrokecolor{textcolor}%
\pgfsetfillcolor{textcolor}%
\pgftext[x=0.636175in,y=2.550563in,,bottom,rotate=90.000000]{\color{textcolor}\sffamily\fontsize{21.000000}{25.200000}\selectfont Percent of instructions synthesized}%
\end{pgfscope}%
\begin{pgfscope}%
\pgfpathrectangle{\pgfqpoint{1.230000in}{1.022500in}}{\pgfqpoint{4.847630in}{3.056125in}}%
\pgfusepath{clip}%
\pgfsetroundcap%
\pgfsetroundjoin%
\pgfsetlinewidth{1.505625pt}%
\definecolor{currentstroke}{rgb}{0.298039,0.447059,0.690196}%
\pgfsetstrokecolor{currentstroke}%
\pgfsetdash{}{0pt}%
\pgfpathmoveto{\pgfqpoint{1.450347in}{1.193435in}}%
\pgfpathlineto{\pgfqpoint{1.456252in}{1.193435in}}%
\pgfpathlineto{\pgfqpoint{1.456252in}{1.225002in}}%
\pgfpathlineto{\pgfqpoint{1.457511in}{1.225002in}}%
\pgfpathlineto{\pgfqpoint{1.457511in}{1.256568in}}%
\pgfpathlineto{\pgfqpoint{1.463220in}{1.256568in}}%
\pgfpathlineto{\pgfqpoint{1.463220in}{1.288134in}}%
\pgfpathlineto{\pgfqpoint{1.464479in}{1.288134in}}%
\pgfpathlineto{\pgfqpoint{1.464479in}{1.319701in}}%
\pgfpathlineto{\pgfqpoint{1.470860in}{1.319701in}}%
\pgfpathlineto{\pgfqpoint{1.470860in}{1.351267in}}%
\pgfpathlineto{\pgfqpoint{1.472147in}{1.351267in}}%
\pgfpathlineto{\pgfqpoint{1.472147in}{1.382834in}}%
\pgfpathlineto{\pgfqpoint{1.478388in}{1.382834in}}%
\pgfpathlineto{\pgfqpoint{1.478388in}{1.414400in}}%
\pgfpathlineto{\pgfqpoint{1.479647in}{1.414400in}}%
\pgfpathlineto{\pgfqpoint{1.479647in}{1.445966in}}%
\pgfpathlineto{\pgfqpoint{1.485608in}{1.445966in}}%
\pgfpathlineto{\pgfqpoint{1.485608in}{1.477533in}}%
\pgfpathlineto{\pgfqpoint{1.486867in}{1.477533in}}%
\pgfpathlineto{\pgfqpoint{1.486867in}{1.509099in}}%
\pgfpathlineto{\pgfqpoint{1.492716in}{1.509099in}}%
\pgfpathlineto{\pgfqpoint{1.492716in}{1.540665in}}%
\pgfpathlineto{\pgfqpoint{1.493975in}{1.540665in}}%
\pgfpathlineto{\pgfqpoint{1.493975in}{1.572232in}}%
\pgfpathlineto{\pgfqpoint{1.500020in}{1.572232in}}%
\pgfpathlineto{\pgfqpoint{1.500020in}{1.603798in}}%
\pgfpathlineto{\pgfqpoint{1.501363in}{1.603798in}}%
\pgfpathlineto{\pgfqpoint{1.501363in}{1.635365in}}%
\pgfpathlineto{\pgfqpoint{1.507380in}{1.635365in}}%
\pgfpathlineto{\pgfqpoint{1.507380in}{1.666931in}}%
\pgfpathlineto{\pgfqpoint{1.508723in}{1.666931in}}%
\pgfpathlineto{\pgfqpoint{1.508723in}{1.698497in}}%
\pgfpathlineto{\pgfqpoint{1.509983in}{1.698497in}}%
\pgfpathlineto{\pgfqpoint{1.509983in}{1.730064in}}%
\pgfpathlineto{\pgfqpoint{1.515999in}{1.730064in}}%
\pgfpathlineto{\pgfqpoint{1.515999in}{1.761630in}}%
\pgfpathlineto{\pgfqpoint{1.517259in}{1.761630in}}%
\pgfpathlineto{\pgfqpoint{1.517259in}{1.793196in}}%
\pgfpathlineto{\pgfqpoint{1.523247in}{1.793196in}}%
\pgfpathlineto{\pgfqpoint{1.523247in}{1.824763in}}%
\pgfpathlineto{\pgfqpoint{1.524507in}{1.824763in}}%
\pgfpathlineto{\pgfqpoint{1.524507in}{1.856329in}}%
\pgfpathlineto{\pgfqpoint{1.530104in}{1.856329in}}%
\pgfpathlineto{\pgfqpoint{1.530104in}{1.887896in}}%
\pgfpathlineto{\pgfqpoint{1.531363in}{1.887896in}}%
\pgfpathlineto{\pgfqpoint{1.531363in}{1.919462in}}%
\pgfpathlineto{\pgfqpoint{1.537240in}{1.919462in}}%
\pgfpathlineto{\pgfqpoint{1.537240in}{1.951028in}}%
\pgfpathlineto{\pgfqpoint{1.538611in}{1.951028in}}%
\pgfpathlineto{\pgfqpoint{1.538611in}{1.982595in}}%
\pgfpathlineto{\pgfqpoint{1.544544in}{1.982595in}}%
\pgfpathlineto{\pgfqpoint{1.544544in}{2.014161in}}%
\pgfpathlineto{\pgfqpoint{1.560327in}{2.014161in}}%
\pgfpathlineto{\pgfqpoint{1.560327in}{2.045727in}}%
\pgfpathlineto{\pgfqpoint{1.575271in}{2.045727in}}%
\pgfpathlineto{\pgfqpoint{1.575271in}{2.077294in}}%
\pgfpathlineto{\pgfqpoint{1.590523in}{2.077294in}}%
\pgfpathlineto{\pgfqpoint{1.590523in}{2.108860in}}%
\pgfpathlineto{\pgfqpoint{1.605803in}{2.108860in}}%
\pgfpathlineto{\pgfqpoint{1.605803in}{2.140427in}}%
\pgfpathlineto{\pgfqpoint{1.631913in}{2.140427in}}%
\pgfpathlineto{\pgfqpoint{1.631913in}{2.171993in}}%
\pgfpathlineto{\pgfqpoint{1.654972in}{2.171993in}}%
\pgfpathlineto{\pgfqpoint{1.654972in}{2.203559in}}%
\pgfpathlineto{\pgfqpoint{1.680187in}{2.203559in}}%
\pgfpathlineto{\pgfqpoint{1.680187in}{2.235126in}}%
\pgfpathlineto{\pgfqpoint{1.716287in}{2.235126in}}%
\pgfpathlineto{\pgfqpoint{1.716287in}{2.266692in}}%
\pgfpathlineto{\pgfqpoint{1.732015in}{2.266692in}}%
\pgfpathlineto{\pgfqpoint{1.732015in}{2.298259in}}%
\pgfpathlineto{\pgfqpoint{1.748134in}{2.298259in}}%
\pgfpathlineto{\pgfqpoint{1.748134in}{2.329825in}}%
\pgfpathlineto{\pgfqpoint{1.762882in}{2.329825in}}%
\pgfpathlineto{\pgfqpoint{1.762882in}{2.361391in}}%
\pgfpathlineto{\pgfqpoint{1.778050in}{2.361391in}}%
\pgfpathlineto{\pgfqpoint{1.778050in}{2.392958in}}%
\pgfpathlineto{\pgfqpoint{1.815886in}{2.392958in}}%
\pgfpathlineto{\pgfqpoint{1.815886in}{2.424524in}}%
\pgfpathlineto{\pgfqpoint{1.820643in}{2.424524in}}%
\pgfpathlineto{\pgfqpoint{1.820643in}{2.456090in}}%
\pgfpathlineto{\pgfqpoint{1.837714in}{2.456090in}}%
\pgfpathlineto{\pgfqpoint{1.837714in}{2.487657in}}%
\pgfpathlineto{\pgfqpoint{1.843087in}{2.487657in}}%
\pgfpathlineto{\pgfqpoint{1.843087in}{2.519223in}}%
\pgfpathlineto{\pgfqpoint{1.857443in}{2.519223in}}%
\pgfpathlineto{\pgfqpoint{1.857443in}{2.550790in}}%
\pgfpathlineto{\pgfqpoint{1.866426in}{2.550790in}}%
\pgfpathlineto{\pgfqpoint{1.866426in}{2.582356in}}%
\pgfpathlineto{\pgfqpoint{1.882406in}{2.582356in}}%
\pgfpathlineto{\pgfqpoint{1.882406in}{2.613922in}}%
\pgfpathlineto{\pgfqpoint{1.893348in}{2.613922in}}%
\pgfpathlineto{\pgfqpoint{1.893348in}{2.645489in}}%
\pgfpathlineto{\pgfqpoint{1.908432in}{2.645489in}}%
\pgfpathlineto{\pgfqpoint{1.908432in}{2.677055in}}%
\pgfpathlineto{\pgfqpoint{1.917247in}{2.677055in}}%
\pgfpathlineto{\pgfqpoint{1.917247in}{2.708621in}}%
\pgfpathlineto{\pgfqpoint{1.934374in}{2.708621in}}%
\pgfpathlineto{\pgfqpoint{1.934374in}{2.740188in}}%
\pgfpathlineto{\pgfqpoint{1.943665in}{2.740188in}}%
\pgfpathlineto{\pgfqpoint{1.943665in}{2.771754in}}%
\pgfpathlineto{\pgfqpoint{1.956062in}{2.771754in}}%
\pgfpathlineto{\pgfqpoint{1.956062in}{2.803321in}}%
\pgfpathlineto{\pgfqpoint{1.967704in}{2.803321in}}%
\pgfpathlineto{\pgfqpoint{1.967704in}{2.834887in}}%
\pgfpathlineto{\pgfqpoint{1.991491in}{2.834887in}}%
\pgfpathlineto{\pgfqpoint{1.991491in}{2.866453in}}%
\pgfpathlineto{\pgfqpoint{1.997480in}{2.866453in}}%
\pgfpathlineto{\pgfqpoint{1.997480in}{2.898020in}}%
\pgfpathlineto{\pgfqpoint{2.008925in}{2.898020in}}%
\pgfpathlineto{\pgfqpoint{2.008925in}{2.929586in}}%
\pgfpathlineto{\pgfqpoint{2.030446in}{2.929586in}}%
\pgfpathlineto{\pgfqpoint{2.030446in}{2.961152in}}%
\pgfpathlineto{\pgfqpoint{2.035819in}{2.961152in}}%
\pgfpathlineto{\pgfqpoint{2.035819in}{2.992719in}}%
\pgfpathlineto{\pgfqpoint{2.084821in}{2.992719in}}%
\pgfpathlineto{\pgfqpoint{2.084821in}{3.024285in}}%
\pgfpathlineto{\pgfqpoint{2.096630in}{3.024285in}}%
\pgfpathlineto{\pgfqpoint{2.096630in}{3.055852in}}%
\pgfpathlineto{\pgfqpoint{2.105921in}{3.055852in}}%
\pgfpathlineto{\pgfqpoint{2.105921in}{3.087418in}}%
\pgfpathlineto{\pgfqpoint{2.108216in}{3.087418in}}%
\pgfpathlineto{\pgfqpoint{2.108216in}{3.118984in}}%
\pgfpathlineto{\pgfqpoint{2.127246in}{3.118984in}}%
\pgfpathlineto{\pgfqpoint{2.127246in}{3.150551in}}%
\pgfpathlineto{\pgfqpoint{2.132731in}{3.150551in}}%
\pgfpathlineto{\pgfqpoint{2.132731in}{3.182117in}}%
\pgfpathlineto{\pgfqpoint{2.133990in}{3.182117in}}%
\pgfpathlineto{\pgfqpoint{2.133990in}{3.213684in}}%
\pgfpathlineto{\pgfqpoint{2.144568in}{3.213684in}}%
\pgfpathlineto{\pgfqpoint{2.144568in}{3.245250in}}%
\pgfpathlineto{\pgfqpoint{2.157945in}{3.245250in}}%
\pgfpathlineto{\pgfqpoint{2.157945in}{3.276816in}}%
\pgfpathlineto{\pgfqpoint{2.168915in}{3.276816in}}%
\pgfpathlineto{\pgfqpoint{2.168915in}{3.308383in}}%
\pgfpathlineto{\pgfqpoint{2.181173in}{3.308383in}}%
\pgfpathlineto{\pgfqpoint{2.181173in}{3.339949in}}%
\pgfpathlineto{\pgfqpoint{2.192031in}{3.339949in}}%
\pgfpathlineto{\pgfqpoint{2.192031in}{3.371515in}}%
\pgfpathlineto{\pgfqpoint{2.205995in}{3.371515in}}%
\pgfpathlineto{\pgfqpoint{2.205995in}{3.403082in}}%
\pgfpathlineto{\pgfqpoint{2.217973in}{3.403082in}}%
\pgfpathlineto{\pgfqpoint{2.217973in}{3.434648in}}%
\pgfpathlineto{\pgfqpoint{2.231014in}{3.434648in}}%
\pgfpathlineto{\pgfqpoint{2.231014in}{3.466215in}}%
\pgfpathlineto{\pgfqpoint{2.244362in}{3.466215in}}%
\pgfpathlineto{\pgfqpoint{2.244362in}{3.497781in}}%
\pgfpathlineto{\pgfqpoint{2.256956in}{3.497781in}}%
\pgfpathlineto{\pgfqpoint{2.256956in}{3.529347in}}%
\pgfpathlineto{\pgfqpoint{2.268849in}{3.529347in}}%
\pgfpathlineto{\pgfqpoint{2.268849in}{3.560914in}}%
\pgfpathlineto{\pgfqpoint{4.695336in}{3.560914in}}%
\pgfpathlineto{\pgfqpoint{4.695336in}{3.592480in}}%
\pgfpathlineto{\pgfqpoint{5.678949in}{3.592480in}}%
\pgfpathlineto{\pgfqpoint{5.678949in}{3.624046in}}%
\pgfpathlineto{\pgfqpoint{5.692690in}{3.624046in}}%
\pgfpathlineto{\pgfqpoint{5.692690in}{3.655613in}}%
\pgfpathlineto{\pgfqpoint{5.706627in}{3.655613in}}%
\pgfpathlineto{\pgfqpoint{5.706627in}{3.687179in}}%
\pgfpathlineto{\pgfqpoint{5.720143in}{3.687179in}}%
\pgfpathlineto{\pgfqpoint{5.720143in}{3.718746in}}%
\pgfpathlineto{\pgfqpoint{5.734723in}{3.718746in}}%
\pgfpathlineto{\pgfqpoint{5.734723in}{3.750312in}}%
\pgfpathlineto{\pgfqpoint{5.748520in}{3.750312in}}%
\pgfpathlineto{\pgfqpoint{5.748520in}{3.781878in}}%
\pgfpathlineto{\pgfqpoint{5.762288in}{3.781878in}}%
\pgfpathlineto{\pgfqpoint{5.762288in}{3.813445in}}%
\pgfpathlineto{\pgfqpoint{5.775022in}{3.813445in}}%
\pgfpathlineto{\pgfqpoint{5.775022in}{3.845011in}}%
\pgfpathlineto{\pgfqpoint{5.789210in}{3.845011in}}%
\pgfpathlineto{\pgfqpoint{5.789210in}{3.876577in}}%
\pgfpathlineto{\pgfqpoint{5.804546in}{3.876577in}}%
\pgfpathlineto{\pgfqpoint{5.804546in}{3.908144in}}%
\pgfpathlineto{\pgfqpoint{5.819434in}{3.908144in}}%
\pgfpathlineto{\pgfqpoint{5.819434in}{3.939710in}}%
\pgfpathlineto{\pgfqpoint{5.819434in}{3.939710in}}%
\pgfusepath{stroke}%
\end{pgfscope}%
\begin{pgfscope}%
\pgfpathrectangle{\pgfqpoint{1.230000in}{1.022500in}}{\pgfqpoint{4.847630in}{3.056125in}}%
\pgfusepath{clip}%
\pgfsetbuttcap%
\pgfsetroundjoin%
\definecolor{currentfill}{rgb}{0.298039,0.447059,0.690196}%
\pgfsetfillcolor{currentfill}%
\pgfsetlinewidth{1.003750pt}%
\definecolor{currentstroke}{rgb}{0.298039,0.447059,0.690196}%
\pgfsetstrokecolor{currentstroke}%
\pgfsetdash{}{0pt}%
\pgfsys@defobject{currentmarker}{\pgfqpoint{-0.020833in}{-0.020833in}}{\pgfqpoint{0.020833in}{0.020833in}}{%
\pgfpathmoveto{\pgfqpoint{0.000000in}{-0.020833in}}%
\pgfpathcurveto{\pgfqpoint{0.005525in}{-0.020833in}}{\pgfqpoint{0.010825in}{-0.018638in}}{\pgfqpoint{0.014731in}{-0.014731in}}%
\pgfpathcurveto{\pgfqpoint{0.018638in}{-0.010825in}}{\pgfqpoint{0.020833in}{-0.005525in}}{\pgfqpoint{0.020833in}{0.000000in}}%
\pgfpathcurveto{\pgfqpoint{0.020833in}{0.005525in}}{\pgfqpoint{0.018638in}{0.010825in}}{\pgfqpoint{0.014731in}{0.014731in}}%
\pgfpathcurveto{\pgfqpoint{0.010825in}{0.018638in}}{\pgfqpoint{0.005525in}{0.020833in}}{\pgfqpoint{0.000000in}{0.020833in}}%
\pgfpathcurveto{\pgfqpoint{-0.005525in}{0.020833in}}{\pgfqpoint{-0.010825in}{0.018638in}}{\pgfqpoint{-0.014731in}{0.014731in}}%
\pgfpathcurveto{\pgfqpoint{-0.018638in}{0.010825in}}{\pgfqpoint{-0.020833in}{0.005525in}}{\pgfqpoint{-0.020833in}{0.000000in}}%
\pgfpathcurveto{\pgfqpoint{-0.020833in}{-0.005525in}}{\pgfqpoint{-0.018638in}{-0.010825in}}{\pgfqpoint{-0.014731in}{-0.014731in}}%
\pgfpathcurveto{\pgfqpoint{-0.010825in}{-0.018638in}}{\pgfqpoint{-0.005525in}{-0.020833in}}{\pgfqpoint{0.000000in}{-0.020833in}}%
\pgfpathclose%
\pgfusepath{stroke,fill}%
}%
\begin{pgfscope}%
\pgfsys@transformshift{1.450347in}{1.193435in}%
\pgfsys@useobject{currentmarker}{}%
\end{pgfscope}%
\begin{pgfscope}%
\pgfsys@transformshift{5.819434in}{3.939710in}%
\pgfsys@useobject{currentmarker}{}%
\end{pgfscope}%
\end{pgfscope}%
\begin{pgfscope}%
\pgfpathrectangle{\pgfqpoint{1.230000in}{1.022500in}}{\pgfqpoint{4.847630in}{3.056125in}}%
\pgfusepath{clip}%
\pgfsetbuttcap%
\pgfsetroundjoin%
\pgfsetlinewidth{1.505625pt}%
\definecolor{currentstroke}{rgb}{0.866667,0.517647,0.321569}%
\pgfsetstrokecolor{currentstroke}%
\pgfsetdash{{5.550000pt}{2.400000pt}}{0.000000pt}%
\pgfpathmoveto{\pgfqpoint{1.450347in}{1.193435in}}%
\pgfpathlineto{\pgfqpoint{5.819434in}{1.193435in}}%
\pgfpathlineto{\pgfqpoint{5.819434in}{1.193435in}}%
\pgfusepath{stroke}%
\end{pgfscope}%
\begin{pgfscope}%
\pgfpathrectangle{\pgfqpoint{1.230000in}{1.022500in}}{\pgfqpoint{4.847630in}{3.056125in}}%
\pgfusepath{clip}%
\pgfsetbuttcap%
\pgfsetroundjoin%
\definecolor{currentfill}{rgb}{0.866667,0.517647,0.321569}%
\pgfsetfillcolor{currentfill}%
\pgfsetlinewidth{1.003750pt}%
\definecolor{currentstroke}{rgb}{0.866667,0.517647,0.321569}%
\pgfsetstrokecolor{currentstroke}%
\pgfsetdash{}{0pt}%
\pgfsys@defobject{currentmarker}{\pgfqpoint{-0.020833in}{-0.020833in}}{\pgfqpoint{0.020833in}{0.020833in}}{%
\pgfpathmoveto{\pgfqpoint{0.000000in}{-0.020833in}}%
\pgfpathcurveto{\pgfqpoint{0.005525in}{-0.020833in}}{\pgfqpoint{0.010825in}{-0.018638in}}{\pgfqpoint{0.014731in}{-0.014731in}}%
\pgfpathcurveto{\pgfqpoint{0.018638in}{-0.010825in}}{\pgfqpoint{0.020833in}{-0.005525in}}{\pgfqpoint{0.020833in}{0.000000in}}%
\pgfpathcurveto{\pgfqpoint{0.020833in}{0.005525in}}{\pgfqpoint{0.018638in}{0.010825in}}{\pgfqpoint{0.014731in}{0.014731in}}%
\pgfpathcurveto{\pgfqpoint{0.010825in}{0.018638in}}{\pgfqpoint{0.005525in}{0.020833in}}{\pgfqpoint{0.000000in}{0.020833in}}%
\pgfpathcurveto{\pgfqpoint{-0.005525in}{0.020833in}}{\pgfqpoint{-0.010825in}{0.018638in}}{\pgfqpoint{-0.014731in}{0.014731in}}%
\pgfpathcurveto{\pgfqpoint{-0.018638in}{0.010825in}}{\pgfqpoint{-0.020833in}{0.005525in}}{\pgfqpoint{-0.020833in}{0.000000in}}%
\pgfpathcurveto{\pgfqpoint{-0.020833in}{-0.005525in}}{\pgfqpoint{-0.018638in}{-0.010825in}}{\pgfqpoint{-0.014731in}{-0.014731in}}%
\pgfpathcurveto{\pgfqpoint{-0.010825in}{-0.018638in}}{\pgfqpoint{-0.005525in}{-0.020833in}}{\pgfqpoint{0.000000in}{-0.020833in}}%
\pgfpathclose%
\pgfusepath{stroke,fill}%
}%
\begin{pgfscope}%
\pgfsys@transformshift{1.450347in}{1.193435in}%
\pgfsys@useobject{currentmarker}{}%
\end{pgfscope}%
\begin{pgfscope}%
\pgfsys@transformshift{5.819434in}{1.193435in}%
\pgfsys@useobject{currentmarker}{}%
\end{pgfscope}%
\end{pgfscope}%
\begin{pgfscope}%
\pgfpathrectangle{\pgfqpoint{1.230000in}{1.022500in}}{\pgfqpoint{4.847630in}{3.056125in}}%
\pgfusepath{clip}%
\pgfsetbuttcap%
\pgfsetroundjoin%
\pgfsetlinewidth{1.505625pt}%
\definecolor{currentstroke}{rgb}{0.333333,0.658824,0.407843}%
\pgfsetstrokecolor{currentstroke}%
\pgfsetdash{{9.600000pt}{2.400000pt}{1.500000pt}{2.400000pt}}{0.000000pt}%
\pgfpathmoveto{\pgfqpoint{1.450347in}{1.193435in}}%
\pgfpathlineto{\pgfqpoint{5.819434in}{1.193435in}}%
\pgfpathlineto{\pgfqpoint{5.819434in}{1.193435in}}%
\pgfusepath{stroke}%
\end{pgfscope}%
\begin{pgfscope}%
\pgfpathrectangle{\pgfqpoint{1.230000in}{1.022500in}}{\pgfqpoint{4.847630in}{3.056125in}}%
\pgfusepath{clip}%
\pgfsetbuttcap%
\pgfsetroundjoin%
\definecolor{currentfill}{rgb}{0.333333,0.658824,0.407843}%
\pgfsetfillcolor{currentfill}%
\pgfsetlinewidth{1.003750pt}%
\definecolor{currentstroke}{rgb}{0.333333,0.658824,0.407843}%
\pgfsetstrokecolor{currentstroke}%
\pgfsetdash{}{0pt}%
\pgfsys@defobject{currentmarker}{\pgfqpoint{-0.020833in}{-0.020833in}}{\pgfqpoint{0.020833in}{0.020833in}}{%
\pgfpathmoveto{\pgfqpoint{0.000000in}{-0.020833in}}%
\pgfpathcurveto{\pgfqpoint{0.005525in}{-0.020833in}}{\pgfqpoint{0.010825in}{-0.018638in}}{\pgfqpoint{0.014731in}{-0.014731in}}%
\pgfpathcurveto{\pgfqpoint{0.018638in}{-0.010825in}}{\pgfqpoint{0.020833in}{-0.005525in}}{\pgfqpoint{0.020833in}{0.000000in}}%
\pgfpathcurveto{\pgfqpoint{0.020833in}{0.005525in}}{\pgfqpoint{0.018638in}{0.010825in}}{\pgfqpoint{0.014731in}{0.014731in}}%
\pgfpathcurveto{\pgfqpoint{0.010825in}{0.018638in}}{\pgfqpoint{0.005525in}{0.020833in}}{\pgfqpoint{0.000000in}{0.020833in}}%
\pgfpathcurveto{\pgfqpoint{-0.005525in}{0.020833in}}{\pgfqpoint{-0.010825in}{0.018638in}}{\pgfqpoint{-0.014731in}{0.014731in}}%
\pgfpathcurveto{\pgfqpoint{-0.018638in}{0.010825in}}{\pgfqpoint{-0.020833in}{0.005525in}}{\pgfqpoint{-0.020833in}{0.000000in}}%
\pgfpathcurveto{\pgfqpoint{-0.020833in}{-0.005525in}}{\pgfqpoint{-0.018638in}{-0.010825in}}{\pgfqpoint{-0.014731in}{-0.014731in}}%
\pgfpathcurveto{\pgfqpoint{-0.010825in}{-0.018638in}}{\pgfqpoint{-0.005525in}{-0.020833in}}{\pgfqpoint{0.000000in}{-0.020833in}}%
\pgfpathclose%
\pgfusepath{stroke,fill}%
}%
\begin{pgfscope}%
\pgfsys@transformshift{1.450347in}{1.193435in}%
\pgfsys@useobject{currentmarker}{}%
\end{pgfscope}%
\begin{pgfscope}%
\pgfsys@transformshift{5.819434in}{1.193435in}%
\pgfsys@useobject{currentmarker}{}%
\end{pgfscope}%
\end{pgfscope}%
\begin{pgfscope}%
\pgfsetrectcap%
\pgfsetmiterjoin%
\pgfsetlinewidth{1.254687pt}%
\definecolor{currentstroke}{rgb}{0.800000,0.800000,0.800000}%
\pgfsetstrokecolor{currentstroke}%
\pgfsetdash{}{0pt}%
\pgfpathmoveto{\pgfqpoint{1.230000in}{1.022500in}}%
\pgfpathlineto{\pgfqpoint{1.230000in}{4.078625in}}%
\pgfusepath{stroke}%
\end{pgfscope}%
\begin{pgfscope}%
\pgfsetrectcap%
\pgfsetmiterjoin%
\pgfsetlinewidth{1.254687pt}%
\definecolor{currentstroke}{rgb}{0.800000,0.800000,0.800000}%
\pgfsetstrokecolor{currentstroke}%
\pgfsetdash{}{0pt}%
\pgfpathmoveto{\pgfqpoint{6.077630in}{1.022500in}}%
\pgfpathlineto{\pgfqpoint{6.077630in}{4.078625in}}%
\pgfusepath{stroke}%
\end{pgfscope}%
\begin{pgfscope}%
\pgfsetrectcap%
\pgfsetmiterjoin%
\pgfsetlinewidth{1.254687pt}%
\definecolor{currentstroke}{rgb}{0.800000,0.800000,0.800000}%
\pgfsetstrokecolor{currentstroke}%
\pgfsetdash{}{0pt}%
\pgfpathmoveto{\pgfqpoint{1.230000in}{1.022500in}}%
\pgfpathlineto{\pgfqpoint{6.077630in}{1.022500in}}%
\pgfusepath{stroke}%
\end{pgfscope}%
\begin{pgfscope}%
\pgfsetrectcap%
\pgfsetmiterjoin%
\pgfsetlinewidth{1.254687pt}%
\definecolor{currentstroke}{rgb}{0.800000,0.800000,0.800000}%
\pgfsetstrokecolor{currentstroke}%
\pgfsetdash{}{0pt}%
\pgfpathmoveto{\pgfqpoint{1.230000in}{4.078625in}}%
\pgfpathlineto{\pgfqpoint{6.077630in}{4.078625in}}%
\pgfusepath{stroke}%
\end{pgfscope}%
\begin{pgfscope}%
\definecolor{textcolor}{rgb}{0.150000,0.150000,0.150000}%
\pgfsetstrokecolor{textcolor}%
\pgfsetfillcolor{textcolor}%
\pgftext[x=3.653815in,y=4.161958in,,base]{\color{textcolor}\sffamily\fontsize{21.000000}{25.200000}\selectfont Synthesis time for eBPF to RISC-V}%
\end{pgfscope}%
\begin{pgfscope}%
\pgfsetbuttcap%
\pgfsetmiterjoin%
\definecolor{currentfill}{rgb}{1.000000,1.000000,1.000000}%
\pgfsetfillcolor{currentfill}%
\pgfsetfillopacity{0.800000}%
\pgfsetlinewidth{1.003750pt}%
\definecolor{currentstroke}{rgb}{0.800000,0.800000,0.800000}%
\pgfsetstrokecolor{currentstroke}%
\pgfsetstrokeopacity{0.800000}%
\pgfsetdash{}{0pt}%
\pgfpathmoveto{\pgfqpoint{3.821155in}{1.925836in}}%
\pgfpathlineto{\pgfqpoint{5.890477in}{1.925836in}}%
\pgfpathquadraticcurveto{\pgfqpoint{5.943950in}{1.925836in}}{\pgfqpoint{5.943950in}{1.979308in}}%
\pgfpathlineto{\pgfqpoint{5.943950in}{3.121817in}}%
\pgfpathquadraticcurveto{\pgfqpoint{5.943950in}{3.175289in}}{\pgfqpoint{5.890477in}{3.175289in}}%
\pgfpathlineto{\pgfqpoint{3.821155in}{3.175289in}}%
\pgfpathquadraticcurveto{\pgfqpoint{3.767683in}{3.175289in}}{\pgfqpoint{3.767683in}{3.121817in}}%
\pgfpathlineto{\pgfqpoint{3.767683in}{1.979308in}}%
\pgfpathquadraticcurveto{\pgfqpoint{3.767683in}{1.925836in}}{\pgfqpoint{3.821155in}{1.925836in}}%
\pgfpathclose%
\pgfusepath{stroke,fill}%
\end{pgfscope}%
\begin{pgfscope}%
\pgfsetroundcap%
\pgfsetroundjoin%
\pgfsetlinewidth{1.505625pt}%
\definecolor{currentstroke}{rgb}{0.298039,0.447059,0.690196}%
\pgfsetstrokecolor{currentstroke}%
\pgfsetdash{}{0pt}%
\pgfpathmoveto{\pgfqpoint{3.874627in}{2.961882in}}%
\pgfpathlineto{\pgfqpoint{4.409350in}{2.961882in}}%
\pgfusepath{stroke}%
\end{pgfscope}%
\begin{pgfscope}%
\pgfsetbuttcap%
\pgfsetroundjoin%
\definecolor{currentfill}{rgb}{0.298039,0.447059,0.690196}%
\pgfsetfillcolor{currentfill}%
\pgfsetlinewidth{1.003750pt}%
\definecolor{currentstroke}{rgb}{0.298039,0.447059,0.690196}%
\pgfsetstrokecolor{currentstroke}%
\pgfsetdash{}{0pt}%
\pgfsys@defobject{currentmarker}{\pgfqpoint{-0.020833in}{-0.020833in}}{\pgfqpoint{0.020833in}{0.020833in}}{%
\pgfpathmoveto{\pgfqpoint{0.000000in}{-0.020833in}}%
\pgfpathcurveto{\pgfqpoint{0.005525in}{-0.020833in}}{\pgfqpoint{0.010825in}{-0.018638in}}{\pgfqpoint{0.014731in}{-0.014731in}}%
\pgfpathcurveto{\pgfqpoint{0.018638in}{-0.010825in}}{\pgfqpoint{0.020833in}{-0.005525in}}{\pgfqpoint{0.020833in}{0.000000in}}%
\pgfpathcurveto{\pgfqpoint{0.020833in}{0.005525in}}{\pgfqpoint{0.018638in}{0.010825in}}{\pgfqpoint{0.014731in}{0.014731in}}%
\pgfpathcurveto{\pgfqpoint{0.010825in}{0.018638in}}{\pgfqpoint{0.005525in}{0.020833in}}{\pgfqpoint{0.000000in}{0.020833in}}%
\pgfpathcurveto{\pgfqpoint{-0.005525in}{0.020833in}}{\pgfqpoint{-0.010825in}{0.018638in}}{\pgfqpoint{-0.014731in}{0.014731in}}%
\pgfpathcurveto{\pgfqpoint{-0.018638in}{0.010825in}}{\pgfqpoint{-0.020833in}{0.005525in}}{\pgfqpoint{-0.020833in}{0.000000in}}%
\pgfpathcurveto{\pgfqpoint{-0.020833in}{-0.005525in}}{\pgfqpoint{-0.018638in}{-0.010825in}}{\pgfqpoint{-0.014731in}{-0.014731in}}%
\pgfpathcurveto{\pgfqpoint{-0.010825in}{-0.018638in}}{\pgfqpoint{-0.005525in}{-0.020833in}}{\pgfqpoint{0.000000in}{-0.020833in}}%
\pgfpathclose%
\pgfusepath{stroke,fill}%
}%
\begin{pgfscope}%
\pgfsys@transformshift{4.141989in}{2.961882in}%
\pgfsys@useobject{currentmarker}{}%
\end{pgfscope}%
\end{pgfscope}%
\begin{pgfscope}%
\definecolor{textcolor}{rgb}{0.150000,0.150000,0.150000}%
\pgfsetstrokecolor{textcolor}%
\pgfsetfillcolor{textcolor}%
\pgftext[x=4.623239in,y=2.868306in,left,base]{\color{textcolor}\sffamily\fontsize{19.250000}{23.100000}\selectfont Pre-load}%
\end{pgfscope}%
\begin{pgfscope}%
\pgfsetbuttcap%
\pgfsetroundjoin%
\pgfsetlinewidth{1.505625pt}%
\definecolor{currentstroke}{rgb}{0.866667,0.517647,0.321569}%
\pgfsetstrokecolor{currentstroke}%
\pgfsetdash{{5.550000pt}{2.400000pt}}{0.000000pt}%
\pgfpathmoveto{\pgfqpoint{3.874627in}{2.572134in}}%
\pgfpathlineto{\pgfqpoint{4.409350in}{2.572134in}}%
\pgfusepath{stroke}%
\end{pgfscope}%
\begin{pgfscope}%
\pgfsetbuttcap%
\pgfsetroundjoin%
\definecolor{currentfill}{rgb}{0.866667,0.517647,0.321569}%
\pgfsetfillcolor{currentfill}%
\pgfsetlinewidth{1.003750pt}%
\definecolor{currentstroke}{rgb}{0.866667,0.517647,0.321569}%
\pgfsetstrokecolor{currentstroke}%
\pgfsetdash{}{0pt}%
\pgfsys@defobject{currentmarker}{\pgfqpoint{-0.020833in}{-0.020833in}}{\pgfqpoint{0.020833in}{0.020833in}}{%
\pgfpathmoveto{\pgfqpoint{0.000000in}{-0.020833in}}%
\pgfpathcurveto{\pgfqpoint{0.005525in}{-0.020833in}}{\pgfqpoint{0.010825in}{-0.018638in}}{\pgfqpoint{0.014731in}{-0.014731in}}%
\pgfpathcurveto{\pgfqpoint{0.018638in}{-0.010825in}}{\pgfqpoint{0.020833in}{-0.005525in}}{\pgfqpoint{0.020833in}{0.000000in}}%
\pgfpathcurveto{\pgfqpoint{0.020833in}{0.005525in}}{\pgfqpoint{0.018638in}{0.010825in}}{\pgfqpoint{0.014731in}{0.014731in}}%
\pgfpathcurveto{\pgfqpoint{0.010825in}{0.018638in}}{\pgfqpoint{0.005525in}{0.020833in}}{\pgfqpoint{0.000000in}{0.020833in}}%
\pgfpathcurveto{\pgfqpoint{-0.005525in}{0.020833in}}{\pgfqpoint{-0.010825in}{0.018638in}}{\pgfqpoint{-0.014731in}{0.014731in}}%
\pgfpathcurveto{\pgfqpoint{-0.018638in}{0.010825in}}{\pgfqpoint{-0.020833in}{0.005525in}}{\pgfqpoint{-0.020833in}{0.000000in}}%
\pgfpathcurveto{\pgfqpoint{-0.020833in}{-0.005525in}}{\pgfqpoint{-0.018638in}{-0.010825in}}{\pgfqpoint{-0.014731in}{-0.014731in}}%
\pgfpathcurveto{\pgfqpoint{-0.010825in}{-0.018638in}}{\pgfqpoint{-0.005525in}{-0.020833in}}{\pgfqpoint{0.000000in}{-0.020833in}}%
\pgfpathclose%
\pgfusepath{stroke,fill}%
}%
\begin{pgfscope}%
\pgfsys@transformshift{4.141989in}{2.572134in}%
\pgfsys@useobject{currentmarker}{}%
\end{pgfscope}%
\end{pgfscope}%
\begin{pgfscope}%
\definecolor{textcolor}{rgb}{0.150000,0.150000,0.150000}%
\pgfsetstrokecolor{textcolor}%
\pgfsetfillcolor{textcolor}%
\pgftext[x=4.623239in,y=2.478558in,left,base]{\color{textcolor}\sffamily\fontsize{19.250000}{23.100000}\selectfont Read-write}%
\end{pgfscope}%
\begin{pgfscope}%
\pgfsetbuttcap%
\pgfsetroundjoin%
\pgfsetlinewidth{1.505625pt}%
\definecolor{currentstroke}{rgb}{0.333333,0.658824,0.407843}%
\pgfsetstrokecolor{currentstroke}%
\pgfsetdash{{9.600000pt}{2.400000pt}{1.500000pt}{2.400000pt}}{0.000000pt}%
\pgfpathmoveto{\pgfqpoint{3.874627in}{2.182386in}}%
\pgfpathlineto{\pgfqpoint{4.409350in}{2.182386in}}%
\pgfusepath{stroke}%
\end{pgfscope}%
\begin{pgfscope}%
\pgfsetbuttcap%
\pgfsetroundjoin%
\definecolor{currentfill}{rgb}{0.333333,0.658824,0.407843}%
\pgfsetfillcolor{currentfill}%
\pgfsetlinewidth{1.003750pt}%
\definecolor{currentstroke}{rgb}{0.333333,0.658824,0.407843}%
\pgfsetstrokecolor{currentstroke}%
\pgfsetdash{}{0pt}%
\pgfsys@defobject{currentmarker}{\pgfqpoint{-0.020833in}{-0.020833in}}{\pgfqpoint{0.020833in}{0.020833in}}{%
\pgfpathmoveto{\pgfqpoint{0.000000in}{-0.020833in}}%
\pgfpathcurveto{\pgfqpoint{0.005525in}{-0.020833in}}{\pgfqpoint{0.010825in}{-0.018638in}}{\pgfqpoint{0.014731in}{-0.014731in}}%
\pgfpathcurveto{\pgfqpoint{0.018638in}{-0.010825in}}{\pgfqpoint{0.020833in}{-0.005525in}}{\pgfqpoint{0.020833in}{0.000000in}}%
\pgfpathcurveto{\pgfqpoint{0.020833in}{0.005525in}}{\pgfqpoint{0.018638in}{0.010825in}}{\pgfqpoint{0.014731in}{0.014731in}}%
\pgfpathcurveto{\pgfqpoint{0.010825in}{0.018638in}}{\pgfqpoint{0.005525in}{0.020833in}}{\pgfqpoint{0.000000in}{0.020833in}}%
\pgfpathcurveto{\pgfqpoint{-0.005525in}{0.020833in}}{\pgfqpoint{-0.010825in}{0.018638in}}{\pgfqpoint{-0.014731in}{0.014731in}}%
\pgfpathcurveto{\pgfqpoint{-0.018638in}{0.010825in}}{\pgfqpoint{-0.020833in}{0.005525in}}{\pgfqpoint{-0.020833in}{0.000000in}}%
\pgfpathcurveto{\pgfqpoint{-0.020833in}{-0.005525in}}{\pgfqpoint{-0.018638in}{-0.010825in}}{\pgfqpoint{-0.014731in}{-0.014731in}}%
\pgfpathcurveto{\pgfqpoint{-0.010825in}{-0.018638in}}{\pgfqpoint{-0.005525in}{-0.020833in}}{\pgfqpoint{0.000000in}{-0.020833in}}%
\pgfpathclose%
\pgfusepath{stroke,fill}%
}%
\begin{pgfscope}%
\pgfsys@transformshift{4.141989in}{2.182386in}%
\pgfsys@useobject{currentmarker}{}%
\end{pgfscope}%
\end{pgfscope}%
\begin{pgfscope}%
\definecolor{textcolor}{rgb}{0.150000,0.150000,0.150000}%
\pgfsetstrokecolor{textcolor}%
\pgfsetfillcolor{textcolor}%
\pgftext[x=4.623239in,y=2.088810in,left,base]{\color{textcolor}\sffamily\fontsize{19.250000}{23.100000}\selectfont Naïve}%
\end{pgfscope}%
\begin{pgfscope}%
\pgfpathrectangle{\pgfqpoint{1.230000in}{1.022500in}}{\pgfqpoint{4.847630in}{3.056125in}}%
\pgfusepath{clip}%
\pgfsetbuttcap%
\pgfsetroundjoin%
\definecolor{currentfill}{rgb}{1.000000,0.000000,0.000000}%
\pgfsetfillcolor{currentfill}%
\pgfsetlinewidth{1.505625pt}%
\definecolor{currentstroke}{rgb}{1.000000,0.000000,0.000000}%
\pgfsetstrokecolor{currentstroke}%
\pgfsetdash{}{0pt}%
\pgfpathmoveto{\pgfqpoint{5.765642in}{1.139644in}}%
\pgfpathlineto{\pgfqpoint{5.873225in}{1.247227in}}%
\pgfpathmoveto{\pgfqpoint{5.765642in}{1.247227in}}%
\pgfpathlineto{\pgfqpoint{5.873225in}{1.139644in}}%
\pgfusepath{stroke,fill}%
\end{pgfscope}%
\begin{pgfscope}%
\pgfpathrectangle{\pgfqpoint{1.230000in}{1.022500in}}{\pgfqpoint{4.847630in}{3.056125in}}%
\pgfusepath{clip}%
\pgfsetbuttcap%
\pgfsetroundjoin%
\definecolor{currentfill}{rgb}{1.000000,0.000000,0.000000}%
\pgfsetfillcolor{currentfill}%
\pgfsetlinewidth{1.505625pt}%
\definecolor{currentstroke}{rgb}{1.000000,0.000000,0.000000}%
\pgfsetstrokecolor{currentstroke}%
\pgfsetdash{}{0pt}%
\pgfpathmoveto{\pgfqpoint{5.765642in}{1.139644in}}%
\pgfpathlineto{\pgfqpoint{5.873225in}{1.247227in}}%
\pgfpathmoveto{\pgfqpoint{5.765642in}{1.247227in}}%
\pgfpathlineto{\pgfqpoint{5.873225in}{1.139644in}}%
\pgfusepath{stroke,fill}%
\end{pgfscope}%
\end{pgfpicture}%
\makeatother%
\endgroup%

  % %% Creator: Matplotlib, PGF backend
%%
%% To include the figure in your LaTeX document, write
%%   \input{<filename>.pgf}
%%
%% Make sure the required packages are loaded in your preamble
%%   \usepackage{pgf}
%%
%% Figures using additional raster images can only be included by \input if
%% they are in the same directory as the main LaTeX file. For loading figures
%% from other directories you can use the `import` package
%%   \usepackage{import}
%% and then include the figures with
%%   \import{<path to file>}{<filename>.pgf}
%%
%% Matplotlib used the following preamble
%%
\begingroup%
\makeatletter%
\begin{pgfpicture}%
\pgfpathrectangle{\pgfpointorigin}{\pgfqpoint{6.400000in}{4.800000in}}%
\pgfusepath{use as bounding box, clip}%
\begin{pgfscope}%
\pgfsetbuttcap%
\pgfsetmiterjoin%
\definecolor{currentfill}{rgb}{1.000000,1.000000,1.000000}%
\pgfsetfillcolor{currentfill}%
\pgfsetlinewidth{0.000000pt}%
\definecolor{currentstroke}{rgb}{1.000000,1.000000,1.000000}%
\pgfsetstrokecolor{currentstroke}%
\pgfsetdash{}{0pt}%
\pgfpathmoveto{\pgfqpoint{0.000000in}{0.000000in}}%
\pgfpathlineto{\pgfqpoint{6.400000in}{0.000000in}}%
\pgfpathlineto{\pgfqpoint{6.400000in}{4.800000in}}%
\pgfpathlineto{\pgfqpoint{0.000000in}{4.800000in}}%
\pgfpathclose%
\pgfusepath{fill}%
\end{pgfscope}%
\begin{pgfscope}%
\pgfsetbuttcap%
\pgfsetmiterjoin%
\definecolor{currentfill}{rgb}{1.000000,1.000000,1.000000}%
\pgfsetfillcolor{currentfill}%
\pgfsetlinewidth{0.000000pt}%
\definecolor{currentstroke}{rgb}{0.000000,0.000000,0.000000}%
\pgfsetstrokecolor{currentstroke}%
\pgfsetstrokeopacity{0.000000}%
\pgfsetdash{}{0pt}%
\pgfpathmoveto{\pgfqpoint{1.230000in}{1.022500in}}%
\pgfpathlineto{\pgfqpoint{6.037500in}{1.022500in}}%
\pgfpathlineto{\pgfqpoint{6.037500in}{4.078625in}}%
\pgfpathlineto{\pgfqpoint{1.230000in}{4.078625in}}%
\pgfpathclose%
\pgfusepath{fill}%
\end{pgfscope}%
\begin{pgfscope}%
\pgfpathrectangle{\pgfqpoint{1.230000in}{1.022500in}}{\pgfqpoint{4.807500in}{3.056125in}}%
\pgfusepath{clip}%
\pgfsetroundcap%
\pgfsetroundjoin%
\pgfsetlinewidth{1.003750pt}%
\definecolor{currentstroke}{rgb}{0.800000,0.800000,0.800000}%
\pgfsetstrokecolor{currentstroke}%
\pgfsetstrokeopacity{0.400000}%
\pgfsetdash{}{0pt}%
\pgfpathmoveto{\pgfqpoint{1.448523in}{1.022500in}}%
\pgfpathlineto{\pgfqpoint{1.448523in}{4.078625in}}%
\pgfusepath{stroke}%
\end{pgfscope}%
\begin{pgfscope}%
\definecolor{textcolor}{rgb}{0.150000,0.150000,0.150000}%
\pgfsetstrokecolor{textcolor}%
\pgfsetfillcolor{textcolor}%
\pgftext[x=1.448523in,y=0.890556in,,top]{\color{textcolor}\sffamily\fontsize{19.250000}{23.100000}\selectfont 0}%
\end{pgfscope}%
\begin{pgfscope}%
\pgfpathrectangle{\pgfqpoint{1.230000in}{1.022500in}}{\pgfqpoint{4.807500in}{3.056125in}}%
\pgfusepath{clip}%
\pgfsetroundcap%
\pgfsetroundjoin%
\pgfsetlinewidth{1.003750pt}%
\definecolor{currentstroke}{rgb}{0.800000,0.800000,0.800000}%
\pgfsetstrokecolor{currentstroke}%
\pgfsetstrokeopacity{0.400000}%
\pgfsetdash{}{0pt}%
\pgfpathmoveto{\pgfqpoint{2.433053in}{1.022500in}}%
\pgfpathlineto{\pgfqpoint{2.433053in}{4.078625in}}%
\pgfusepath{stroke}%
\end{pgfscope}%
\begin{pgfscope}%
\definecolor{textcolor}{rgb}{0.150000,0.150000,0.150000}%
\pgfsetstrokecolor{textcolor}%
\pgfsetfillcolor{textcolor}%
\pgftext[x=2.433053in,y=0.890556in,,top]{\color{textcolor}\sffamily\fontsize{19.250000}{23.100000}\selectfont 1000}%
\end{pgfscope}%
\begin{pgfscope}%
\pgfpathrectangle{\pgfqpoint{1.230000in}{1.022500in}}{\pgfqpoint{4.807500in}{3.056125in}}%
\pgfusepath{clip}%
\pgfsetroundcap%
\pgfsetroundjoin%
\pgfsetlinewidth{1.003750pt}%
\definecolor{currentstroke}{rgb}{0.800000,0.800000,0.800000}%
\pgfsetstrokecolor{currentstroke}%
\pgfsetstrokeopacity{0.400000}%
\pgfsetdash{}{0pt}%
\pgfpathmoveto{\pgfqpoint{3.417584in}{1.022500in}}%
\pgfpathlineto{\pgfqpoint{3.417584in}{4.078625in}}%
\pgfusepath{stroke}%
\end{pgfscope}%
\begin{pgfscope}%
\definecolor{textcolor}{rgb}{0.150000,0.150000,0.150000}%
\pgfsetstrokecolor{textcolor}%
\pgfsetfillcolor{textcolor}%
\pgftext[x=3.417584in,y=0.890556in,,top]{\color{textcolor}\sffamily\fontsize{19.250000}{23.100000}\selectfont 2000}%
\end{pgfscope}%
\begin{pgfscope}%
\pgfpathrectangle{\pgfqpoint{1.230000in}{1.022500in}}{\pgfqpoint{4.807500in}{3.056125in}}%
\pgfusepath{clip}%
\pgfsetroundcap%
\pgfsetroundjoin%
\pgfsetlinewidth{1.003750pt}%
\definecolor{currentstroke}{rgb}{0.800000,0.800000,0.800000}%
\pgfsetstrokecolor{currentstroke}%
\pgfsetstrokeopacity{0.400000}%
\pgfsetdash{}{0pt}%
\pgfpathmoveto{\pgfqpoint{4.402114in}{1.022500in}}%
\pgfpathlineto{\pgfqpoint{4.402114in}{4.078625in}}%
\pgfusepath{stroke}%
\end{pgfscope}%
\begin{pgfscope}%
\definecolor{textcolor}{rgb}{0.150000,0.150000,0.150000}%
\pgfsetstrokecolor{textcolor}%
\pgfsetfillcolor{textcolor}%
\pgftext[x=4.402114in,y=0.890556in,,top]{\color{textcolor}\sffamily\fontsize{19.250000}{23.100000}\selectfont 3000}%
\end{pgfscope}%
\begin{pgfscope}%
\pgfpathrectangle{\pgfqpoint{1.230000in}{1.022500in}}{\pgfqpoint{4.807500in}{3.056125in}}%
\pgfusepath{clip}%
\pgfsetroundcap%
\pgfsetroundjoin%
\pgfsetlinewidth{1.003750pt}%
\definecolor{currentstroke}{rgb}{0.800000,0.800000,0.800000}%
\pgfsetstrokecolor{currentstroke}%
\pgfsetstrokeopacity{0.400000}%
\pgfsetdash{}{0pt}%
\pgfpathmoveto{\pgfqpoint{5.386644in}{1.022500in}}%
\pgfpathlineto{\pgfqpoint{5.386644in}{4.078625in}}%
\pgfusepath{stroke}%
\end{pgfscope}%
\begin{pgfscope}%
\definecolor{textcolor}{rgb}{0.150000,0.150000,0.150000}%
\pgfsetstrokecolor{textcolor}%
\pgfsetfillcolor{textcolor}%
\pgftext[x=5.386644in,y=0.890556in,,top]{\color{textcolor}\sffamily\fontsize{19.250000}{23.100000}\selectfont 4000}%
\end{pgfscope}%
\begin{pgfscope}%
\definecolor{textcolor}{rgb}{0.150000,0.150000,0.150000}%
\pgfsetstrokecolor{textcolor}%
\pgfsetfillcolor{textcolor}%
\pgftext[x=3.633750in,y=0.578932in,,top]{\color{textcolor}\sffamily\fontsize{21.000000}{25.200000}\selectfont Time (s)}%
\end{pgfscope}%
\begin{pgfscope}%
\pgfpathrectangle{\pgfqpoint{1.230000in}{1.022500in}}{\pgfqpoint{4.807500in}{3.056125in}}%
\pgfusepath{clip}%
\pgfsetroundcap%
\pgfsetroundjoin%
\pgfsetlinewidth{1.003750pt}%
\definecolor{currentstroke}{rgb}{0.800000,0.800000,0.800000}%
\pgfsetstrokecolor{currentstroke}%
\pgfsetstrokeopacity{0.400000}%
\pgfsetdash{}{0pt}%
\pgfpathmoveto{\pgfqpoint{1.230000in}{1.161415in}}%
\pgfpathlineto{\pgfqpoint{6.037500in}{1.161415in}}%
\pgfusepath{stroke}%
\end{pgfscope}%
\begin{pgfscope}%
\definecolor{textcolor}{rgb}{0.150000,0.150000,0.150000}%
\pgfsetstrokecolor{textcolor}%
\pgfsetfillcolor{textcolor}%
\pgftext[x=0.962614in,y=1.061396in,left,base]{\color{textcolor}\sffamily\fontsize{19.250000}{23.100000}\selectfont 0}%
\end{pgfscope}%
\begin{pgfscope}%
\pgfpathrectangle{\pgfqpoint{1.230000in}{1.022500in}}{\pgfqpoint{4.807500in}{3.056125in}}%
\pgfusepath{clip}%
\pgfsetroundcap%
\pgfsetroundjoin%
\pgfsetlinewidth{1.003750pt}%
\definecolor{currentstroke}{rgb}{0.800000,0.800000,0.800000}%
\pgfsetstrokecolor{currentstroke}%
\pgfsetstrokeopacity{0.400000}%
\pgfsetdash{}{0pt}%
\pgfpathmoveto{\pgfqpoint{1.230000in}{1.855989in}}%
\pgfpathlineto{\pgfqpoint{6.037500in}{1.855989in}}%
\pgfusepath{stroke}%
\end{pgfscope}%
\begin{pgfscope}%
\definecolor{textcolor}{rgb}{0.150000,0.150000,0.150000}%
\pgfsetstrokecolor{textcolor}%
\pgfsetfillcolor{textcolor}%
\pgftext[x=0.827172in,y=1.755969in,left,base]{\color{textcolor}\sffamily\fontsize{19.250000}{23.100000}\selectfont 25}%
\end{pgfscope}%
\begin{pgfscope}%
\pgfpathrectangle{\pgfqpoint{1.230000in}{1.022500in}}{\pgfqpoint{4.807500in}{3.056125in}}%
\pgfusepath{clip}%
\pgfsetroundcap%
\pgfsetroundjoin%
\pgfsetlinewidth{1.003750pt}%
\definecolor{currentstroke}{rgb}{0.800000,0.800000,0.800000}%
\pgfsetstrokecolor{currentstroke}%
\pgfsetstrokeopacity{0.400000}%
\pgfsetdash{}{0pt}%
\pgfpathmoveto{\pgfqpoint{1.230000in}{2.550562in}}%
\pgfpathlineto{\pgfqpoint{6.037500in}{2.550562in}}%
\pgfusepath{stroke}%
\end{pgfscope}%
\begin{pgfscope}%
\definecolor{textcolor}{rgb}{0.150000,0.150000,0.150000}%
\pgfsetstrokecolor{textcolor}%
\pgfsetfillcolor{textcolor}%
\pgftext[x=0.827172in,y=2.450543in,left,base]{\color{textcolor}\sffamily\fontsize{19.250000}{23.100000}\selectfont 50}%
\end{pgfscope}%
\begin{pgfscope}%
\pgfpathrectangle{\pgfqpoint{1.230000in}{1.022500in}}{\pgfqpoint{4.807500in}{3.056125in}}%
\pgfusepath{clip}%
\pgfsetroundcap%
\pgfsetroundjoin%
\pgfsetlinewidth{1.003750pt}%
\definecolor{currentstroke}{rgb}{0.800000,0.800000,0.800000}%
\pgfsetstrokecolor{currentstroke}%
\pgfsetstrokeopacity{0.400000}%
\pgfsetdash{}{0pt}%
\pgfpathmoveto{\pgfqpoint{1.230000in}{3.245136in}}%
\pgfpathlineto{\pgfqpoint{6.037500in}{3.245136in}}%
\pgfusepath{stroke}%
\end{pgfscope}%
\begin{pgfscope}%
\definecolor{textcolor}{rgb}{0.150000,0.150000,0.150000}%
\pgfsetstrokecolor{textcolor}%
\pgfsetfillcolor{textcolor}%
\pgftext[x=0.827172in,y=3.145117in,left,base]{\color{textcolor}\sffamily\fontsize{19.250000}{23.100000}\selectfont 75}%
\end{pgfscope}%
\begin{pgfscope}%
\pgfpathrectangle{\pgfqpoint{1.230000in}{1.022500in}}{\pgfqpoint{4.807500in}{3.056125in}}%
\pgfusepath{clip}%
\pgfsetroundcap%
\pgfsetroundjoin%
\pgfsetlinewidth{1.003750pt}%
\definecolor{currentstroke}{rgb}{0.800000,0.800000,0.800000}%
\pgfsetstrokecolor{currentstroke}%
\pgfsetstrokeopacity{0.400000}%
\pgfsetdash{}{0pt}%
\pgfpathmoveto{\pgfqpoint{1.230000in}{3.939710in}}%
\pgfpathlineto{\pgfqpoint{6.037500in}{3.939710in}}%
\pgfusepath{stroke}%
\end{pgfscope}%
\begin{pgfscope}%
\definecolor{textcolor}{rgb}{0.150000,0.150000,0.150000}%
\pgfsetstrokecolor{textcolor}%
\pgfsetfillcolor{textcolor}%
\pgftext[x=0.691731in,y=3.839691in,left,base]{\color{textcolor}\sffamily\fontsize{19.250000}{23.100000}\selectfont 100}%
\end{pgfscope}%
\begin{pgfscope}%
\definecolor{textcolor}{rgb}{0.150000,0.150000,0.150000}%
\pgfsetstrokecolor{textcolor}%
\pgfsetfillcolor{textcolor}%
\pgftext[x=0.636175in,y=2.550563in,,bottom,rotate=90.000000]{\color{textcolor}\sffamily\fontsize{21.000000}{25.200000}\selectfont Percent of instructions synthesized}%
\end{pgfscope}%
\begin{pgfscope}%
\pgfpathrectangle{\pgfqpoint{1.230000in}{1.022500in}}{\pgfqpoint{4.807500in}{3.056125in}}%
\pgfusepath{clip}%
\pgfsetroundcap%
\pgfsetroundjoin%
\pgfsetlinewidth{1.505625pt}%
\definecolor{currentstroke}{rgb}{0.298039,0.447059,0.690196}%
\pgfsetstrokecolor{currentstroke}%
\pgfsetdash{}{0pt}%
\pgfpathmoveto{\pgfqpoint{1.448523in}{1.161415in}}%
\pgfpathlineto{\pgfqpoint{1.492827in}{1.161415in}}%
\pgfpathlineto{\pgfqpoint{1.492827in}{1.227565in}}%
\pgfpathlineto{\pgfqpoint{1.537130in}{1.227565in}}%
\pgfpathlineto{\pgfqpoint{1.537130in}{1.293715in}}%
\pgfpathlineto{\pgfqpoint{1.581434in}{1.293715in}}%
\pgfpathlineto{\pgfqpoint{1.581434in}{1.359864in}}%
\pgfpathlineto{\pgfqpoint{1.625738in}{1.359864in}}%
\pgfpathlineto{\pgfqpoint{1.625738in}{1.426014in}}%
\pgfpathlineto{\pgfqpoint{1.671027in}{1.426014in}}%
\pgfpathlineto{\pgfqpoint{1.671027in}{1.492164in}}%
\pgfpathlineto{\pgfqpoint{1.715330in}{1.492164in}}%
\pgfpathlineto{\pgfqpoint{1.715330in}{1.558314in}}%
\pgfpathlineto{\pgfqpoint{1.759634in}{1.558314in}}%
\pgfpathlineto{\pgfqpoint{1.759634in}{1.624464in}}%
\pgfpathlineto{\pgfqpoint{1.803938in}{1.624464in}}%
\pgfpathlineto{\pgfqpoint{1.803938in}{1.690614in}}%
\pgfpathlineto{\pgfqpoint{1.848242in}{1.690614in}}%
\pgfpathlineto{\pgfqpoint{1.848242in}{1.756764in}}%
\pgfpathlineto{\pgfqpoint{1.893530in}{1.756764in}}%
\pgfpathlineto{\pgfqpoint{1.893530in}{1.822914in}}%
\pgfpathlineto{\pgfqpoint{1.937834in}{1.822914in}}%
\pgfpathlineto{\pgfqpoint{1.937834in}{1.889064in}}%
\pgfpathlineto{\pgfqpoint{1.982138in}{1.889064in}}%
\pgfpathlineto{\pgfqpoint{1.982138in}{1.955213in}}%
\pgfpathlineto{\pgfqpoint{2.026442in}{1.955213in}}%
\pgfpathlineto{\pgfqpoint{2.026442in}{2.021363in}}%
\pgfpathlineto{\pgfqpoint{2.085514in}{2.021363in}}%
\pgfpathlineto{\pgfqpoint{2.085514in}{2.087513in}}%
\pgfpathlineto{\pgfqpoint{2.144586in}{2.087513in}}%
\pgfpathlineto{\pgfqpoint{2.144586in}{2.153663in}}%
\pgfpathlineto{\pgfqpoint{2.204642in}{2.153663in}}%
\pgfpathlineto{\pgfqpoint{2.204642in}{2.219813in}}%
\pgfpathlineto{\pgfqpoint{2.263714in}{2.219813in}}%
\pgfpathlineto{\pgfqpoint{2.263714in}{2.285963in}}%
\pgfpathlineto{\pgfqpoint{2.322786in}{2.285963in}}%
\pgfpathlineto{\pgfqpoint{2.322786in}{2.352113in}}%
\pgfpathlineto{\pgfqpoint{2.381858in}{2.352113in}}%
\pgfpathlineto{\pgfqpoint{2.381858in}{2.418263in}}%
\pgfpathlineto{\pgfqpoint{2.440929in}{2.418263in}}%
\pgfpathlineto{\pgfqpoint{2.440929in}{2.484413in}}%
\pgfpathlineto{\pgfqpoint{2.500986in}{2.484413in}}%
\pgfpathlineto{\pgfqpoint{2.500986in}{2.550562in}}%
\pgfpathlineto{\pgfqpoint{2.560058in}{2.550562in}}%
\pgfpathlineto{\pgfqpoint{2.560058in}{2.616712in}}%
\pgfpathlineto{\pgfqpoint{2.604361in}{2.616712in}}%
\pgfpathlineto{\pgfqpoint{2.604361in}{2.682862in}}%
\pgfpathlineto{\pgfqpoint{2.648665in}{2.682862in}}%
\pgfpathlineto{\pgfqpoint{2.648665in}{2.749012in}}%
\pgfpathlineto{\pgfqpoint{3.021802in}{2.749012in}}%
\pgfpathlineto{\pgfqpoint{3.021802in}{2.815162in}}%
\pgfpathlineto{\pgfqpoint{3.379187in}{2.815162in}}%
\pgfpathlineto{\pgfqpoint{3.379187in}{2.881312in}}%
\pgfpathlineto{\pgfqpoint{3.740509in}{2.881312in}}%
\pgfpathlineto{\pgfqpoint{3.740509in}{2.947462in}}%
\pgfpathlineto{\pgfqpoint{4.168780in}{2.947462in}}%
\pgfpathlineto{\pgfqpoint{4.168780in}{3.013612in}}%
\pgfpathlineto{\pgfqpoint{4.509428in}{3.013612in}}%
\pgfpathlineto{\pgfqpoint{4.509428in}{3.079762in}}%
\pgfpathlineto{\pgfqpoint{4.906193in}{3.079762in}}%
\pgfpathlineto{\pgfqpoint{4.906193in}{3.145912in}}%
\pgfpathlineto{\pgfqpoint{5.147403in}{3.145912in}}%
\pgfpathlineto{\pgfqpoint{5.147403in}{3.212061in}}%
\pgfpathlineto{\pgfqpoint{5.291145in}{3.212061in}}%
\pgfpathlineto{\pgfqpoint{5.291145in}{3.278211in}}%
\pgfpathlineto{\pgfqpoint{5.359077in}{3.278211in}}%
\pgfpathlineto{\pgfqpoint{5.359077in}{3.344361in}}%
\pgfpathlineto{\pgfqpoint{5.425041in}{3.344361in}}%
\pgfpathlineto{\pgfqpoint{5.425041in}{3.410511in}}%
\pgfpathlineto{\pgfqpoint{5.470329in}{3.410511in}}%
\pgfpathlineto{\pgfqpoint{5.470329in}{3.476661in}}%
\pgfpathlineto{\pgfqpoint{5.514633in}{3.476661in}}%
\pgfpathlineto{\pgfqpoint{5.514633in}{3.542811in}}%
\pgfpathlineto{\pgfqpoint{5.558937in}{3.542811in}}%
\pgfpathlineto{\pgfqpoint{5.558937in}{3.608961in}}%
\pgfpathlineto{\pgfqpoint{5.603241in}{3.608961in}}%
\pgfpathlineto{\pgfqpoint{5.603241in}{3.675111in}}%
\pgfpathlineto{\pgfqpoint{5.648529in}{3.675111in}}%
\pgfpathlineto{\pgfqpoint{5.648529in}{3.741261in}}%
\pgfpathlineto{\pgfqpoint{5.692833in}{3.741261in}}%
\pgfpathlineto{\pgfqpoint{5.692833in}{3.807410in}}%
\pgfpathlineto{\pgfqpoint{5.737137in}{3.807410in}}%
\pgfpathlineto{\pgfqpoint{5.737137in}{3.873560in}}%
\pgfpathlineto{\pgfqpoint{5.781441in}{3.873560in}}%
\pgfpathlineto{\pgfqpoint{5.781441in}{3.939710in}}%
\pgfusepath{stroke}%
\end{pgfscope}%
\begin{pgfscope}%
\pgfpathrectangle{\pgfqpoint{1.230000in}{1.022500in}}{\pgfqpoint{4.807500in}{3.056125in}}%
\pgfusepath{clip}%
\pgfsetbuttcap%
\pgfsetroundjoin%
\definecolor{currentfill}{rgb}{0.298039,0.447059,0.690196}%
\pgfsetfillcolor{currentfill}%
\pgfsetlinewidth{1.003750pt}%
\definecolor{currentstroke}{rgb}{0.298039,0.447059,0.690196}%
\pgfsetstrokecolor{currentstroke}%
\pgfsetdash{}{0pt}%
\pgfsys@defobject{currentmarker}{\pgfqpoint{-0.020833in}{-0.020833in}}{\pgfqpoint{0.020833in}{0.020833in}}{%
\pgfpathmoveto{\pgfqpoint{0.000000in}{-0.020833in}}%
\pgfpathcurveto{\pgfqpoint{0.005525in}{-0.020833in}}{\pgfqpoint{0.010825in}{-0.018638in}}{\pgfqpoint{0.014731in}{-0.014731in}}%
\pgfpathcurveto{\pgfqpoint{0.018638in}{-0.010825in}}{\pgfqpoint{0.020833in}{-0.005525in}}{\pgfqpoint{0.020833in}{0.000000in}}%
\pgfpathcurveto{\pgfqpoint{0.020833in}{0.005525in}}{\pgfqpoint{0.018638in}{0.010825in}}{\pgfqpoint{0.014731in}{0.014731in}}%
\pgfpathcurveto{\pgfqpoint{0.010825in}{0.018638in}}{\pgfqpoint{0.005525in}{0.020833in}}{\pgfqpoint{0.000000in}{0.020833in}}%
\pgfpathcurveto{\pgfqpoint{-0.005525in}{0.020833in}}{\pgfqpoint{-0.010825in}{0.018638in}}{\pgfqpoint{-0.014731in}{0.014731in}}%
\pgfpathcurveto{\pgfqpoint{-0.018638in}{0.010825in}}{\pgfqpoint{-0.020833in}{0.005525in}}{\pgfqpoint{-0.020833in}{0.000000in}}%
\pgfpathcurveto{\pgfqpoint{-0.020833in}{-0.005525in}}{\pgfqpoint{-0.018638in}{-0.010825in}}{\pgfqpoint{-0.014731in}{-0.014731in}}%
\pgfpathcurveto{\pgfqpoint{-0.010825in}{-0.018638in}}{\pgfqpoint{-0.005525in}{-0.020833in}}{\pgfqpoint{0.000000in}{-0.020833in}}%
\pgfpathclose%
\pgfusepath{stroke,fill}%
}%
\begin{pgfscope}%
\pgfsys@transformshift{1.448523in}{1.161415in}%
\pgfsys@useobject{currentmarker}{}%
\end{pgfscope}%
\begin{pgfscope}%
\pgfsys@transformshift{5.781441in}{3.939710in}%
\pgfsys@useobject{currentmarker}{}%
\end{pgfscope}%
\end{pgfscope}%
\begin{pgfscope}%
\pgfpathrectangle{\pgfqpoint{1.230000in}{1.022500in}}{\pgfqpoint{4.807500in}{3.056125in}}%
\pgfusepath{clip}%
\pgfsetbuttcap%
\pgfsetroundjoin%
\pgfsetlinewidth{1.505625pt}%
\definecolor{currentstroke}{rgb}{0.866667,0.517647,0.321569}%
\pgfsetstrokecolor{currentstroke}%
\pgfsetdash{{5.550000pt}{2.400000pt}}{0.000000pt}%
\pgfpathmoveto{\pgfqpoint{1.495780in}{1.161415in}}%
\pgfpathlineto{\pgfqpoint{1.495780in}{1.227565in}}%
\pgfpathlineto{\pgfqpoint{1.541069in}{1.227565in}}%
\pgfpathlineto{\pgfqpoint{1.541069in}{1.293715in}}%
\pgfpathlineto{\pgfqpoint{1.591280in}{1.293715in}}%
\pgfpathlineto{\pgfqpoint{1.591280in}{1.359864in}}%
\pgfpathlineto{\pgfqpoint{1.639522in}{1.359864in}}%
\pgfpathlineto{\pgfqpoint{1.639522in}{1.426014in}}%
\pgfpathlineto{\pgfqpoint{1.687764in}{1.426014in}}%
\pgfpathlineto{\pgfqpoint{1.687764in}{1.492164in}}%
\pgfpathlineto{\pgfqpoint{1.735021in}{1.492164in}}%
\pgfpathlineto{\pgfqpoint{1.735021in}{1.558314in}}%
\pgfpathlineto{\pgfqpoint{1.782279in}{1.558314in}}%
\pgfpathlineto{\pgfqpoint{1.782279in}{1.624464in}}%
\pgfpathlineto{\pgfqpoint{1.829536in}{1.624464in}}%
\pgfpathlineto{\pgfqpoint{1.829536in}{1.690614in}}%
\pgfpathlineto{\pgfqpoint{1.876793in}{1.690614in}}%
\pgfpathlineto{\pgfqpoint{1.876793in}{1.756764in}}%
\pgfpathlineto{\pgfqpoint{1.924051in}{1.756764in}}%
\pgfpathlineto{\pgfqpoint{1.924051in}{1.822914in}}%
\pgfpathlineto{\pgfqpoint{1.968355in}{1.822914in}}%
\pgfpathlineto{\pgfqpoint{1.968355in}{1.889064in}}%
\pgfpathlineto{\pgfqpoint{2.012659in}{1.889064in}}%
\pgfpathlineto{\pgfqpoint{2.012659in}{1.955213in}}%
\pgfpathlineto{\pgfqpoint{2.056963in}{1.955213in}}%
\pgfpathlineto{\pgfqpoint{2.056963in}{2.021363in}}%
\pgfpathlineto{\pgfqpoint{2.140648in}{2.021363in}}%
\pgfpathlineto{\pgfqpoint{2.140648in}{2.087513in}}%
\pgfpathlineto{\pgfqpoint{2.224333in}{2.087513in}}%
\pgfpathlineto{\pgfqpoint{2.224333in}{2.153663in}}%
\pgfpathlineto{\pgfqpoint{2.306049in}{2.153663in}}%
\pgfpathlineto{\pgfqpoint{2.306049in}{2.219813in}}%
\pgfpathlineto{\pgfqpoint{2.373981in}{2.219813in}}%
\pgfpathlineto{\pgfqpoint{2.373981in}{2.285963in}}%
\pgfpathlineto{\pgfqpoint{2.462589in}{2.285963in}}%
\pgfpathlineto{\pgfqpoint{2.462589in}{2.352113in}}%
\pgfpathlineto{\pgfqpoint{2.551197in}{2.352113in}}%
\pgfpathlineto{\pgfqpoint{2.551197in}{2.418263in}}%
\pgfpathlineto{\pgfqpoint{2.637835in}{2.418263in}}%
\pgfpathlineto{\pgfqpoint{2.637835in}{2.484413in}}%
\pgfpathlineto{\pgfqpoint{2.756964in}{2.484413in}}%
\pgfpathlineto{\pgfqpoint{2.756964in}{2.550562in}}%
\pgfpathlineto{\pgfqpoint{2.861324in}{2.550562in}}%
\pgfpathlineto{\pgfqpoint{2.861324in}{2.616712in}}%
\pgfpathlineto{\pgfqpoint{3.017864in}{2.616712in}}%
\pgfpathlineto{\pgfqpoint{3.017864in}{2.682862in}}%
\pgfpathlineto{\pgfqpoint{3.174404in}{2.682862in}}%
\pgfpathlineto{\pgfqpoint{3.174404in}{2.749012in}}%
\pgfpathlineto{\pgfqpoint{5.781441in}{2.749012in}}%
\pgfpathlineto{\pgfqpoint{5.781441in}{2.749012in}}%
\pgfusepath{stroke}%
\end{pgfscope}%
\begin{pgfscope}%
\pgfpathrectangle{\pgfqpoint{1.230000in}{1.022500in}}{\pgfqpoint{4.807500in}{3.056125in}}%
\pgfusepath{clip}%
\pgfsetbuttcap%
\pgfsetroundjoin%
\definecolor{currentfill}{rgb}{0.866667,0.517647,0.321569}%
\pgfsetfillcolor{currentfill}%
\pgfsetlinewidth{1.003750pt}%
\definecolor{currentstroke}{rgb}{0.866667,0.517647,0.321569}%
\pgfsetstrokecolor{currentstroke}%
\pgfsetdash{}{0pt}%
\pgfsys@defobject{currentmarker}{\pgfqpoint{-0.020833in}{-0.020833in}}{\pgfqpoint{0.020833in}{0.020833in}}{%
\pgfpathmoveto{\pgfqpoint{0.000000in}{-0.020833in}}%
\pgfpathcurveto{\pgfqpoint{0.005525in}{-0.020833in}}{\pgfqpoint{0.010825in}{-0.018638in}}{\pgfqpoint{0.014731in}{-0.014731in}}%
\pgfpathcurveto{\pgfqpoint{0.018638in}{-0.010825in}}{\pgfqpoint{0.020833in}{-0.005525in}}{\pgfqpoint{0.020833in}{0.000000in}}%
\pgfpathcurveto{\pgfqpoint{0.020833in}{0.005525in}}{\pgfqpoint{0.018638in}{0.010825in}}{\pgfqpoint{0.014731in}{0.014731in}}%
\pgfpathcurveto{\pgfqpoint{0.010825in}{0.018638in}}{\pgfqpoint{0.005525in}{0.020833in}}{\pgfqpoint{0.000000in}{0.020833in}}%
\pgfpathcurveto{\pgfqpoint{-0.005525in}{0.020833in}}{\pgfqpoint{-0.010825in}{0.018638in}}{\pgfqpoint{-0.014731in}{0.014731in}}%
\pgfpathcurveto{\pgfqpoint{-0.018638in}{0.010825in}}{\pgfqpoint{-0.020833in}{0.005525in}}{\pgfqpoint{-0.020833in}{0.000000in}}%
\pgfpathcurveto{\pgfqpoint{-0.020833in}{-0.005525in}}{\pgfqpoint{-0.018638in}{-0.010825in}}{\pgfqpoint{-0.014731in}{-0.014731in}}%
\pgfpathcurveto{\pgfqpoint{-0.010825in}{-0.018638in}}{\pgfqpoint{-0.005525in}{-0.020833in}}{\pgfqpoint{0.000000in}{-0.020833in}}%
\pgfpathclose%
\pgfusepath{stroke,fill}%
}%
\begin{pgfscope}%
\pgfsys@transformshift{5.781441in}{2.749012in}%
\pgfsys@useobject{currentmarker}{}%
\end{pgfscope}%
\end{pgfscope}%
\begin{pgfscope}%
\pgfpathrectangle{\pgfqpoint{1.230000in}{1.022500in}}{\pgfqpoint{4.807500in}{3.056125in}}%
\pgfusepath{clip}%
\pgfsetbuttcap%
\pgfsetroundjoin%
\pgfsetlinewidth{1.505625pt}%
\definecolor{currentstroke}{rgb}{0.333333,0.658824,0.407843}%
\pgfsetstrokecolor{currentstroke}%
\pgfsetdash{{9.600000pt}{2.400000pt}{1.500000pt}{2.400000pt}}{0.000000pt}%
\pgfpathmoveto{\pgfqpoint{1.448523in}{1.161415in}}%
\pgfpathlineto{\pgfqpoint{1.492827in}{1.161415in}}%
\pgfpathlineto{\pgfqpoint{1.492827in}{1.227565in}}%
\pgfpathlineto{\pgfqpoint{1.537130in}{1.227565in}}%
\pgfpathlineto{\pgfqpoint{1.537130in}{1.293715in}}%
\pgfpathlineto{\pgfqpoint{1.581434in}{1.293715in}}%
\pgfpathlineto{\pgfqpoint{1.581434in}{1.359864in}}%
\pgfpathlineto{\pgfqpoint{1.625738in}{1.359864in}}%
\pgfpathlineto{\pgfqpoint{1.625738in}{1.426014in}}%
\pgfpathlineto{\pgfqpoint{1.670042in}{1.426014in}}%
\pgfpathlineto{\pgfqpoint{1.670042in}{1.492164in}}%
\pgfpathlineto{\pgfqpoint{1.715330in}{1.492164in}}%
\pgfpathlineto{\pgfqpoint{1.715330in}{1.558314in}}%
\pgfpathlineto{\pgfqpoint{1.759634in}{1.558314in}}%
\pgfpathlineto{\pgfqpoint{1.759634in}{1.624464in}}%
\pgfpathlineto{\pgfqpoint{1.803938in}{1.624464in}}%
\pgfpathlineto{\pgfqpoint{1.803938in}{1.690614in}}%
\pgfpathlineto{\pgfqpoint{1.848242in}{1.690614in}}%
\pgfpathlineto{\pgfqpoint{1.848242in}{1.756764in}}%
\pgfpathlineto{\pgfqpoint{1.892546in}{1.756764in}}%
\pgfpathlineto{\pgfqpoint{1.892546in}{1.822914in}}%
\pgfpathlineto{\pgfqpoint{1.936850in}{1.822914in}}%
\pgfpathlineto{\pgfqpoint{1.936850in}{1.889064in}}%
\pgfpathlineto{\pgfqpoint{1.982138in}{1.889064in}}%
\pgfpathlineto{\pgfqpoint{1.982138in}{1.955213in}}%
\pgfpathlineto{\pgfqpoint{2.026442in}{1.955213in}}%
\pgfpathlineto{\pgfqpoint{2.026442in}{2.021363in}}%
\pgfpathlineto{\pgfqpoint{2.155416in}{2.021363in}}%
\pgfpathlineto{\pgfqpoint{2.155416in}{2.087513in}}%
\pgfpathlineto{\pgfqpoint{2.245008in}{2.087513in}}%
\pgfpathlineto{\pgfqpoint{2.245008in}{2.153663in}}%
\pgfpathlineto{\pgfqpoint{2.339523in}{2.153663in}}%
\pgfpathlineto{\pgfqpoint{2.339523in}{2.219813in}}%
\pgfpathlineto{\pgfqpoint{2.450775in}{2.219813in}}%
\pgfpathlineto{\pgfqpoint{2.450775in}{2.285963in}}%
\pgfpathlineto{\pgfqpoint{2.562027in}{2.285963in}}%
\pgfpathlineto{\pgfqpoint{2.562027in}{2.352113in}}%
\pgfpathlineto{\pgfqpoint{2.659495in}{2.352113in}}%
\pgfpathlineto{\pgfqpoint{2.659495in}{2.418263in}}%
\pgfpathlineto{\pgfqpoint{2.790438in}{2.418263in}}%
\pgfpathlineto{\pgfqpoint{2.790438in}{2.484413in}}%
\pgfpathlineto{\pgfqpoint{2.931225in}{2.484413in}}%
\pgfpathlineto{\pgfqpoint{2.931225in}{2.550562in}}%
\pgfpathlineto{\pgfqpoint{3.079890in}{2.550562in}}%
\pgfpathlineto{\pgfqpoint{3.079890in}{2.616712in}}%
\pgfpathlineto{\pgfqpoint{5.781441in}{2.616712in}}%
\pgfpathlineto{\pgfqpoint{5.781441in}{2.616712in}}%
\pgfusepath{stroke}%
\end{pgfscope}%
\begin{pgfscope}%
\pgfpathrectangle{\pgfqpoint{1.230000in}{1.022500in}}{\pgfqpoint{4.807500in}{3.056125in}}%
\pgfusepath{clip}%
\pgfsetbuttcap%
\pgfsetroundjoin%
\definecolor{currentfill}{rgb}{0.333333,0.658824,0.407843}%
\pgfsetfillcolor{currentfill}%
\pgfsetlinewidth{1.003750pt}%
\definecolor{currentstroke}{rgb}{0.333333,0.658824,0.407843}%
\pgfsetstrokecolor{currentstroke}%
\pgfsetdash{}{0pt}%
\pgfsys@defobject{currentmarker}{\pgfqpoint{-0.020833in}{-0.020833in}}{\pgfqpoint{0.020833in}{0.020833in}}{%
\pgfpathmoveto{\pgfqpoint{0.000000in}{-0.020833in}}%
\pgfpathcurveto{\pgfqpoint{0.005525in}{-0.020833in}}{\pgfqpoint{0.010825in}{-0.018638in}}{\pgfqpoint{0.014731in}{-0.014731in}}%
\pgfpathcurveto{\pgfqpoint{0.018638in}{-0.010825in}}{\pgfqpoint{0.020833in}{-0.005525in}}{\pgfqpoint{0.020833in}{0.000000in}}%
\pgfpathcurveto{\pgfqpoint{0.020833in}{0.005525in}}{\pgfqpoint{0.018638in}{0.010825in}}{\pgfqpoint{0.014731in}{0.014731in}}%
\pgfpathcurveto{\pgfqpoint{0.010825in}{0.018638in}}{\pgfqpoint{0.005525in}{0.020833in}}{\pgfqpoint{0.000000in}{0.020833in}}%
\pgfpathcurveto{\pgfqpoint{-0.005525in}{0.020833in}}{\pgfqpoint{-0.010825in}{0.018638in}}{\pgfqpoint{-0.014731in}{0.014731in}}%
\pgfpathcurveto{\pgfqpoint{-0.018638in}{0.010825in}}{\pgfqpoint{-0.020833in}{0.005525in}}{\pgfqpoint{-0.020833in}{0.000000in}}%
\pgfpathcurveto{\pgfqpoint{-0.020833in}{-0.005525in}}{\pgfqpoint{-0.018638in}{-0.010825in}}{\pgfqpoint{-0.014731in}{-0.014731in}}%
\pgfpathcurveto{\pgfqpoint{-0.010825in}{-0.018638in}}{\pgfqpoint{-0.005525in}{-0.020833in}}{\pgfqpoint{0.000000in}{-0.020833in}}%
\pgfpathclose%
\pgfusepath{stroke,fill}%
}%
\begin{pgfscope}%
\pgfsys@transformshift{1.448523in}{1.161415in}%
\pgfsys@useobject{currentmarker}{}%
\end{pgfscope}%
\begin{pgfscope}%
\pgfsys@transformshift{5.781441in}{2.616712in}%
\pgfsys@useobject{currentmarker}{}%
\end{pgfscope}%
\end{pgfscope}%
\begin{pgfscope}%
\pgfsetrectcap%
\pgfsetmiterjoin%
\pgfsetlinewidth{1.254687pt}%
\definecolor{currentstroke}{rgb}{0.800000,0.800000,0.800000}%
\pgfsetstrokecolor{currentstroke}%
\pgfsetdash{}{0pt}%
\pgfpathmoveto{\pgfqpoint{1.230000in}{1.022500in}}%
\pgfpathlineto{\pgfqpoint{1.230000in}{4.078625in}}%
\pgfusepath{stroke}%
\end{pgfscope}%
\begin{pgfscope}%
\pgfsetrectcap%
\pgfsetmiterjoin%
\pgfsetlinewidth{1.254687pt}%
\definecolor{currentstroke}{rgb}{0.800000,0.800000,0.800000}%
\pgfsetstrokecolor{currentstroke}%
\pgfsetdash{}{0pt}%
\pgfpathmoveto{\pgfqpoint{6.037500in}{1.022500in}}%
\pgfpathlineto{\pgfqpoint{6.037500in}{4.078625in}}%
\pgfusepath{stroke}%
\end{pgfscope}%
\begin{pgfscope}%
\pgfsetrectcap%
\pgfsetmiterjoin%
\pgfsetlinewidth{1.254687pt}%
\definecolor{currentstroke}{rgb}{0.800000,0.800000,0.800000}%
\pgfsetstrokecolor{currentstroke}%
\pgfsetdash{}{0pt}%
\pgfpathmoveto{\pgfqpoint{1.230000in}{1.022500in}}%
\pgfpathlineto{\pgfqpoint{6.037500in}{1.022500in}}%
\pgfusepath{stroke}%
\end{pgfscope}%
\begin{pgfscope}%
\pgfsetrectcap%
\pgfsetmiterjoin%
\pgfsetlinewidth{1.254687pt}%
\definecolor{currentstroke}{rgb}{0.800000,0.800000,0.800000}%
\pgfsetstrokecolor{currentstroke}%
\pgfsetdash{}{0pt}%
\pgfpathmoveto{\pgfqpoint{1.230000in}{4.078625in}}%
\pgfpathlineto{\pgfqpoint{6.037500in}{4.078625in}}%
\pgfusepath{stroke}%
\end{pgfscope}%
\begin{pgfscope}%
\definecolor{textcolor}{rgb}{0.150000,0.150000,0.150000}%
\pgfsetstrokecolor{textcolor}%
\pgfsetfillcolor{textcolor}%
\pgftext[x=3.633750in,y=4.161958in,,base]{\color{textcolor}\sffamily\fontsize{21.000000}{25.200000}\selectfont Synthesis time for classic BPF to eBPF}%
\end{pgfscope}%
\begin{pgfscope}%
\pgfsetbuttcap%
\pgfsetmiterjoin%
\definecolor{currentfill}{rgb}{1.000000,1.000000,1.000000}%
\pgfsetfillcolor{currentfill}%
\pgfsetfillopacity{0.800000}%
\pgfsetlinewidth{1.003750pt}%
\definecolor{currentstroke}{rgb}{0.800000,0.800000,0.800000}%
\pgfsetstrokecolor{currentstroke}%
\pgfsetstrokeopacity{0.800000}%
\pgfsetdash{}{0pt}%
\pgfpathmoveto{\pgfqpoint{3.781025in}{1.156181in}}%
\pgfpathlineto{\pgfqpoint{5.850347in}{1.156181in}}%
\pgfpathquadraticcurveto{\pgfqpoint{5.903819in}{1.156181in}}{\pgfqpoint{5.903819in}{1.209653in}}%
\pgfpathlineto{\pgfqpoint{5.903819in}{2.352161in}}%
\pgfpathquadraticcurveto{\pgfqpoint{5.903819in}{2.405633in}}{\pgfqpoint{5.850347in}{2.405633in}}%
\pgfpathlineto{\pgfqpoint{3.781025in}{2.405633in}}%
\pgfpathquadraticcurveto{\pgfqpoint{3.727553in}{2.405633in}}{\pgfqpoint{3.727553in}{2.352161in}}%
\pgfpathlineto{\pgfqpoint{3.727553in}{1.209653in}}%
\pgfpathquadraticcurveto{\pgfqpoint{3.727553in}{1.156181in}}{\pgfqpoint{3.781025in}{1.156181in}}%
\pgfpathclose%
\pgfusepath{stroke,fill}%
\end{pgfscope}%
\begin{pgfscope}%
\pgfsetroundcap%
\pgfsetroundjoin%
\pgfsetlinewidth{1.505625pt}%
\definecolor{currentstroke}{rgb}{0.298039,0.447059,0.690196}%
\pgfsetstrokecolor{currentstroke}%
\pgfsetdash{}{0pt}%
\pgfpathmoveto{\pgfqpoint{3.834497in}{2.192227in}}%
\pgfpathlineto{\pgfqpoint{4.369220in}{2.192227in}}%
\pgfusepath{stroke}%
\end{pgfscope}%
\begin{pgfscope}%
\pgfsetbuttcap%
\pgfsetroundjoin%
\definecolor{currentfill}{rgb}{0.298039,0.447059,0.690196}%
\pgfsetfillcolor{currentfill}%
\pgfsetlinewidth{1.003750pt}%
\definecolor{currentstroke}{rgb}{0.298039,0.447059,0.690196}%
\pgfsetstrokecolor{currentstroke}%
\pgfsetdash{}{0pt}%
\pgfsys@defobject{currentmarker}{\pgfqpoint{-0.020833in}{-0.020833in}}{\pgfqpoint{0.020833in}{0.020833in}}{%
\pgfpathmoveto{\pgfqpoint{0.000000in}{-0.020833in}}%
\pgfpathcurveto{\pgfqpoint{0.005525in}{-0.020833in}}{\pgfqpoint{0.010825in}{-0.018638in}}{\pgfqpoint{0.014731in}{-0.014731in}}%
\pgfpathcurveto{\pgfqpoint{0.018638in}{-0.010825in}}{\pgfqpoint{0.020833in}{-0.005525in}}{\pgfqpoint{0.020833in}{0.000000in}}%
\pgfpathcurveto{\pgfqpoint{0.020833in}{0.005525in}}{\pgfqpoint{0.018638in}{0.010825in}}{\pgfqpoint{0.014731in}{0.014731in}}%
\pgfpathcurveto{\pgfqpoint{0.010825in}{0.018638in}}{\pgfqpoint{0.005525in}{0.020833in}}{\pgfqpoint{0.000000in}{0.020833in}}%
\pgfpathcurveto{\pgfqpoint{-0.005525in}{0.020833in}}{\pgfqpoint{-0.010825in}{0.018638in}}{\pgfqpoint{-0.014731in}{0.014731in}}%
\pgfpathcurveto{\pgfqpoint{-0.018638in}{0.010825in}}{\pgfqpoint{-0.020833in}{0.005525in}}{\pgfqpoint{-0.020833in}{0.000000in}}%
\pgfpathcurveto{\pgfqpoint{-0.020833in}{-0.005525in}}{\pgfqpoint{-0.018638in}{-0.010825in}}{\pgfqpoint{-0.014731in}{-0.014731in}}%
\pgfpathcurveto{\pgfqpoint{-0.010825in}{-0.018638in}}{\pgfqpoint{-0.005525in}{-0.020833in}}{\pgfqpoint{0.000000in}{-0.020833in}}%
\pgfpathclose%
\pgfusepath{stroke,fill}%
}%
\begin{pgfscope}%
\pgfsys@transformshift{4.101858in}{2.192227in}%
\pgfsys@useobject{currentmarker}{}%
\end{pgfscope}%
\end{pgfscope}%
\begin{pgfscope}%
\definecolor{textcolor}{rgb}{0.150000,0.150000,0.150000}%
\pgfsetstrokecolor{textcolor}%
\pgfsetfillcolor{textcolor}%
\pgftext[x=4.583108in,y=2.098651in,left,base]{\color{textcolor}\sffamily\fontsize{19.250000}{23.100000}\selectfont Pre-load}%
\end{pgfscope}%
\begin{pgfscope}%
\pgfsetbuttcap%
\pgfsetroundjoin%
\pgfsetlinewidth{1.505625pt}%
\definecolor{currentstroke}{rgb}{0.866667,0.517647,0.321569}%
\pgfsetstrokecolor{currentstroke}%
\pgfsetdash{{5.550000pt}{2.400000pt}}{0.000000pt}%
\pgfpathmoveto{\pgfqpoint{3.834497in}{1.802479in}}%
\pgfpathlineto{\pgfqpoint{4.369220in}{1.802479in}}%
\pgfusepath{stroke}%
\end{pgfscope}%
\begin{pgfscope}%
\pgfsetbuttcap%
\pgfsetroundjoin%
\definecolor{currentfill}{rgb}{0.866667,0.517647,0.321569}%
\pgfsetfillcolor{currentfill}%
\pgfsetlinewidth{1.003750pt}%
\definecolor{currentstroke}{rgb}{0.866667,0.517647,0.321569}%
\pgfsetstrokecolor{currentstroke}%
\pgfsetdash{}{0pt}%
\pgfsys@defobject{currentmarker}{\pgfqpoint{-0.020833in}{-0.020833in}}{\pgfqpoint{0.020833in}{0.020833in}}{%
\pgfpathmoveto{\pgfqpoint{0.000000in}{-0.020833in}}%
\pgfpathcurveto{\pgfqpoint{0.005525in}{-0.020833in}}{\pgfqpoint{0.010825in}{-0.018638in}}{\pgfqpoint{0.014731in}{-0.014731in}}%
\pgfpathcurveto{\pgfqpoint{0.018638in}{-0.010825in}}{\pgfqpoint{0.020833in}{-0.005525in}}{\pgfqpoint{0.020833in}{0.000000in}}%
\pgfpathcurveto{\pgfqpoint{0.020833in}{0.005525in}}{\pgfqpoint{0.018638in}{0.010825in}}{\pgfqpoint{0.014731in}{0.014731in}}%
\pgfpathcurveto{\pgfqpoint{0.010825in}{0.018638in}}{\pgfqpoint{0.005525in}{0.020833in}}{\pgfqpoint{0.000000in}{0.020833in}}%
\pgfpathcurveto{\pgfqpoint{-0.005525in}{0.020833in}}{\pgfqpoint{-0.010825in}{0.018638in}}{\pgfqpoint{-0.014731in}{0.014731in}}%
\pgfpathcurveto{\pgfqpoint{-0.018638in}{0.010825in}}{\pgfqpoint{-0.020833in}{0.005525in}}{\pgfqpoint{-0.020833in}{0.000000in}}%
\pgfpathcurveto{\pgfqpoint{-0.020833in}{-0.005525in}}{\pgfqpoint{-0.018638in}{-0.010825in}}{\pgfqpoint{-0.014731in}{-0.014731in}}%
\pgfpathcurveto{\pgfqpoint{-0.010825in}{-0.018638in}}{\pgfqpoint{-0.005525in}{-0.020833in}}{\pgfqpoint{0.000000in}{-0.020833in}}%
\pgfpathclose%
\pgfusepath{stroke,fill}%
}%
\begin{pgfscope}%
\pgfsys@transformshift{4.101858in}{1.802479in}%
\pgfsys@useobject{currentmarker}{}%
\end{pgfscope}%
\end{pgfscope}%
\begin{pgfscope}%
\definecolor{textcolor}{rgb}{0.150000,0.150000,0.150000}%
\pgfsetstrokecolor{textcolor}%
\pgfsetfillcolor{textcolor}%
\pgftext[x=4.583108in,y=1.708902in,left,base]{\color{textcolor}\sffamily\fontsize{19.250000}{23.100000}\selectfont Read-write}%
\end{pgfscope}%
\begin{pgfscope}%
\pgfsetbuttcap%
\pgfsetroundjoin%
\pgfsetlinewidth{1.505625pt}%
\definecolor{currentstroke}{rgb}{0.333333,0.658824,0.407843}%
\pgfsetstrokecolor{currentstroke}%
\pgfsetdash{{9.600000pt}{2.400000pt}{1.500000pt}{2.400000pt}}{0.000000pt}%
\pgfpathmoveto{\pgfqpoint{3.834497in}{1.412731in}}%
\pgfpathlineto{\pgfqpoint{4.369220in}{1.412731in}}%
\pgfusepath{stroke}%
\end{pgfscope}%
\begin{pgfscope}%
\pgfsetbuttcap%
\pgfsetroundjoin%
\definecolor{currentfill}{rgb}{0.333333,0.658824,0.407843}%
\pgfsetfillcolor{currentfill}%
\pgfsetlinewidth{1.003750pt}%
\definecolor{currentstroke}{rgb}{0.333333,0.658824,0.407843}%
\pgfsetstrokecolor{currentstroke}%
\pgfsetdash{}{0pt}%
\pgfsys@defobject{currentmarker}{\pgfqpoint{-0.020833in}{-0.020833in}}{\pgfqpoint{0.020833in}{0.020833in}}{%
\pgfpathmoveto{\pgfqpoint{0.000000in}{-0.020833in}}%
\pgfpathcurveto{\pgfqpoint{0.005525in}{-0.020833in}}{\pgfqpoint{0.010825in}{-0.018638in}}{\pgfqpoint{0.014731in}{-0.014731in}}%
\pgfpathcurveto{\pgfqpoint{0.018638in}{-0.010825in}}{\pgfqpoint{0.020833in}{-0.005525in}}{\pgfqpoint{0.020833in}{0.000000in}}%
\pgfpathcurveto{\pgfqpoint{0.020833in}{0.005525in}}{\pgfqpoint{0.018638in}{0.010825in}}{\pgfqpoint{0.014731in}{0.014731in}}%
\pgfpathcurveto{\pgfqpoint{0.010825in}{0.018638in}}{\pgfqpoint{0.005525in}{0.020833in}}{\pgfqpoint{0.000000in}{0.020833in}}%
\pgfpathcurveto{\pgfqpoint{-0.005525in}{0.020833in}}{\pgfqpoint{-0.010825in}{0.018638in}}{\pgfqpoint{-0.014731in}{0.014731in}}%
\pgfpathcurveto{\pgfqpoint{-0.018638in}{0.010825in}}{\pgfqpoint{-0.020833in}{0.005525in}}{\pgfqpoint{-0.020833in}{0.000000in}}%
\pgfpathcurveto{\pgfqpoint{-0.020833in}{-0.005525in}}{\pgfqpoint{-0.018638in}{-0.010825in}}{\pgfqpoint{-0.014731in}{-0.014731in}}%
\pgfpathcurveto{\pgfqpoint{-0.010825in}{-0.018638in}}{\pgfqpoint{-0.005525in}{-0.020833in}}{\pgfqpoint{0.000000in}{-0.020833in}}%
\pgfpathclose%
\pgfusepath{stroke,fill}%
}%
\begin{pgfscope}%
\pgfsys@transformshift{4.101858in}{1.412731in}%
\pgfsys@useobject{currentmarker}{}%
\end{pgfscope}%
\end{pgfscope}%
\begin{pgfscope}%
\definecolor{textcolor}{rgb}{0.150000,0.150000,0.150000}%
\pgfsetstrokecolor{textcolor}%
\pgfsetfillcolor{textcolor}%
\pgftext[x=4.583108in,y=1.319154in,left,base]{\color{textcolor}\sffamily\fontsize{19.250000}{23.100000}\selectfont Naïve}%
\end{pgfscope}%
\begin{pgfscope}%
\pgfpathrectangle{\pgfqpoint{1.230000in}{1.022500in}}{\pgfqpoint{4.807500in}{3.056125in}}%
\pgfusepath{clip}%
\pgfsetbuttcap%
\pgfsetroundjoin%
\definecolor{currentfill}{rgb}{1.000000,0.000000,0.000000}%
\pgfsetfillcolor{currentfill}%
\pgfsetlinewidth{1.505625pt}%
\definecolor{currentstroke}{rgb}{1.000000,0.000000,0.000000}%
\pgfsetstrokecolor{currentstroke}%
\pgfsetdash{}{0pt}%
\pgfpathmoveto{\pgfqpoint{5.727650in}{2.695221in}}%
\pgfpathlineto{\pgfqpoint{5.835232in}{2.802804in}}%
\pgfpathmoveto{\pgfqpoint{5.727650in}{2.802804in}}%
\pgfpathlineto{\pgfqpoint{5.835232in}{2.695221in}}%
\pgfusepath{stroke,fill}%
\end{pgfscope}%
\begin{pgfscope}%
\pgfpathrectangle{\pgfqpoint{1.230000in}{1.022500in}}{\pgfqpoint{4.807500in}{3.056125in}}%
\pgfusepath{clip}%
\pgfsetbuttcap%
\pgfsetroundjoin%
\definecolor{currentfill}{rgb}{1.000000,0.000000,0.000000}%
\pgfsetfillcolor{currentfill}%
\pgfsetlinewidth{1.505625pt}%
\definecolor{currentstroke}{rgb}{1.000000,0.000000,0.000000}%
\pgfsetstrokecolor{currentstroke}%
\pgfsetdash{}{0pt}%
\pgfpathmoveto{\pgfqpoint{5.727650in}{2.562921in}}%
\pgfpathlineto{\pgfqpoint{5.835232in}{2.670504in}}%
\pgfpathmoveto{\pgfqpoint{5.727650in}{2.670504in}}%
\pgfpathlineto{\pgfqpoint{5.835232in}{2.562921in}}%
\pgfusepath{stroke,fill}%
\end{pgfscope}%
\end{pgfpicture}%
\makeatother%
\endgroup%

  % }
  % \begin{center}
  %   \resizebox{.5\textwidth}{!}{
  %   %% Creator: Matplotlib, PGF backend
%%
%% To include the figure in your LaTeX document, write
%%   \input{<filename>.pgf}
%%
%% Make sure the required packages are loaded in your preamble
%%   \usepackage{pgf}
%%
%% Figures using additional raster images can only be included by \input if
%% they are in the same directory as the main LaTeX file. For loading figures
%% from other directories you can use the `import` package
%%   \usepackage{import}
%% and then include the figures with
%%   \import{<path to file>}{<filename>.pgf}
%%
%% Matplotlib used the following preamble
%%
\begingroup%
\makeatletter%
\begin{pgfpicture}%
\pgfpathrectangle{\pgfpointorigin}{\pgfqpoint{6.400000in}{4.800000in}}%
\pgfusepath{use as bounding box, clip}%
\begin{pgfscope}%
\pgfsetbuttcap%
\pgfsetmiterjoin%
\definecolor{currentfill}{rgb}{1.000000,1.000000,1.000000}%
\pgfsetfillcolor{currentfill}%
\pgfsetlinewidth{0.000000pt}%
\definecolor{currentstroke}{rgb}{1.000000,1.000000,1.000000}%
\pgfsetstrokecolor{currentstroke}%
\pgfsetdash{}{0pt}%
\pgfpathmoveto{\pgfqpoint{0.000000in}{0.000000in}}%
\pgfpathlineto{\pgfqpoint{6.400000in}{0.000000in}}%
\pgfpathlineto{\pgfqpoint{6.400000in}{4.800000in}}%
\pgfpathlineto{\pgfqpoint{0.000000in}{4.800000in}}%
\pgfpathclose%
\pgfusepath{fill}%
\end{pgfscope}%
\begin{pgfscope}%
\pgfsetbuttcap%
\pgfsetmiterjoin%
\definecolor{currentfill}{rgb}{1.000000,1.000000,1.000000}%
\pgfsetfillcolor{currentfill}%
\pgfsetlinewidth{0.000000pt}%
\definecolor{currentstroke}{rgb}{0.000000,0.000000,0.000000}%
\pgfsetstrokecolor{currentstroke}%
\pgfsetstrokeopacity{0.000000}%
\pgfsetdash{}{0pt}%
\pgfpathmoveto{\pgfqpoint{1.230000in}{1.022500in}}%
\pgfpathlineto{\pgfqpoint{5.591250in}{1.022500in}}%
\pgfpathlineto{\pgfqpoint{5.591250in}{4.078625in}}%
\pgfpathlineto{\pgfqpoint{1.230000in}{4.078625in}}%
\pgfpathclose%
\pgfusepath{fill}%
\end{pgfscope}%
\begin{pgfscope}%
\pgfpathrectangle{\pgfqpoint{1.230000in}{1.022500in}}{\pgfqpoint{4.361250in}{3.056125in}}%
\pgfusepath{clip}%
\pgfsetroundcap%
\pgfsetroundjoin%
\pgfsetlinewidth{1.003750pt}%
\definecolor{currentstroke}{rgb}{0.800000,0.800000,0.800000}%
\pgfsetstrokecolor{currentstroke}%
\pgfsetstrokeopacity{0.400000}%
\pgfsetdash{}{0pt}%
\pgfpathmoveto{\pgfqpoint{1.428239in}{1.022500in}}%
\pgfpathlineto{\pgfqpoint{1.428239in}{4.078625in}}%
\pgfusepath{stroke}%
\end{pgfscope}%
\begin{pgfscope}%
\definecolor{textcolor}{rgb}{0.150000,0.150000,0.150000}%
\pgfsetstrokecolor{textcolor}%
\pgfsetfillcolor{textcolor}%
\pgftext[x=1.428239in,y=0.890556in,,top]{\color{textcolor}\sffamily\fontsize{19.250000}{23.100000}\selectfont 0}%
\end{pgfscope}%
\begin{pgfscope}%
\pgfpathrectangle{\pgfqpoint{1.230000in}{1.022500in}}{\pgfqpoint{4.361250in}{3.056125in}}%
\pgfusepath{clip}%
\pgfsetroundcap%
\pgfsetroundjoin%
\pgfsetlinewidth{1.003750pt}%
\definecolor{currentstroke}{rgb}{0.800000,0.800000,0.800000}%
\pgfsetstrokecolor{currentstroke}%
\pgfsetstrokeopacity{0.400000}%
\pgfsetdash{}{0pt}%
\pgfpathmoveto{\pgfqpoint{2.789861in}{1.022500in}}%
\pgfpathlineto{\pgfqpoint{2.789861in}{4.078625in}}%
\pgfusepath{stroke}%
\end{pgfscope}%
\begin{pgfscope}%
\definecolor{textcolor}{rgb}{0.150000,0.150000,0.150000}%
\pgfsetstrokecolor{textcolor}%
\pgfsetfillcolor{textcolor}%
\pgftext[x=2.789861in,y=0.890556in,,top]{\color{textcolor}\sffamily\fontsize{19.250000}{23.100000}\selectfont 5000}%
\end{pgfscope}%
\begin{pgfscope}%
\pgfpathrectangle{\pgfqpoint{1.230000in}{1.022500in}}{\pgfqpoint{4.361250in}{3.056125in}}%
\pgfusepath{clip}%
\pgfsetroundcap%
\pgfsetroundjoin%
\pgfsetlinewidth{1.003750pt}%
\definecolor{currentstroke}{rgb}{0.800000,0.800000,0.800000}%
\pgfsetstrokecolor{currentstroke}%
\pgfsetstrokeopacity{0.400000}%
\pgfsetdash{}{0pt}%
\pgfpathmoveto{\pgfqpoint{4.151484in}{1.022500in}}%
\pgfpathlineto{\pgfqpoint{4.151484in}{4.078625in}}%
\pgfusepath{stroke}%
\end{pgfscope}%
\begin{pgfscope}%
\definecolor{textcolor}{rgb}{0.150000,0.150000,0.150000}%
\pgfsetstrokecolor{textcolor}%
\pgfsetfillcolor{textcolor}%
\pgftext[x=4.151484in,y=0.890556in,,top]{\color{textcolor}\sffamily\fontsize{19.250000}{23.100000}\selectfont 10000}%
\end{pgfscope}%
\begin{pgfscope}%
\pgfpathrectangle{\pgfqpoint{1.230000in}{1.022500in}}{\pgfqpoint{4.361250in}{3.056125in}}%
\pgfusepath{clip}%
\pgfsetroundcap%
\pgfsetroundjoin%
\pgfsetlinewidth{1.003750pt}%
\definecolor{currentstroke}{rgb}{0.800000,0.800000,0.800000}%
\pgfsetstrokecolor{currentstroke}%
\pgfsetstrokeopacity{0.400000}%
\pgfsetdash{}{0pt}%
\pgfpathmoveto{\pgfqpoint{5.513106in}{1.022500in}}%
\pgfpathlineto{\pgfqpoint{5.513106in}{4.078625in}}%
\pgfusepath{stroke}%
\end{pgfscope}%
\begin{pgfscope}%
\definecolor{textcolor}{rgb}{0.150000,0.150000,0.150000}%
\pgfsetstrokecolor{textcolor}%
\pgfsetfillcolor{textcolor}%
\pgftext[x=5.513106in,y=0.890556in,,top]{\color{textcolor}\sffamily\fontsize{19.250000}{23.100000}\selectfont 15000}%
\end{pgfscope}%
\begin{pgfscope}%
\definecolor{textcolor}{rgb}{0.150000,0.150000,0.150000}%
\pgfsetstrokecolor{textcolor}%
\pgfsetfillcolor{textcolor}%
\pgftext[x=3.410625in,y=0.578932in,,top]{\color{textcolor}\sffamily\fontsize{21.000000}{25.200000}\selectfont Time (s)}%
\end{pgfscope}%
\begin{pgfscope}%
\pgfpathrectangle{\pgfqpoint{1.230000in}{1.022500in}}{\pgfqpoint{4.361250in}{3.056125in}}%
\pgfusepath{clip}%
\pgfsetroundcap%
\pgfsetroundjoin%
\pgfsetlinewidth{1.003750pt}%
\definecolor{currentstroke}{rgb}{0.800000,0.800000,0.800000}%
\pgfsetstrokecolor{currentstroke}%
\pgfsetstrokeopacity{0.400000}%
\pgfsetdash{}{0pt}%
\pgfpathmoveto{\pgfqpoint{1.230000in}{1.161415in}}%
\pgfpathlineto{\pgfqpoint{5.591250in}{1.161415in}}%
\pgfusepath{stroke}%
\end{pgfscope}%
\begin{pgfscope}%
\definecolor{textcolor}{rgb}{0.150000,0.150000,0.150000}%
\pgfsetstrokecolor{textcolor}%
\pgfsetfillcolor{textcolor}%
\pgftext[x=0.962614in,y=1.061396in,left,base]{\color{textcolor}\sffamily\fontsize{19.250000}{23.100000}\selectfont 0}%
\end{pgfscope}%
\begin{pgfscope}%
\pgfpathrectangle{\pgfqpoint{1.230000in}{1.022500in}}{\pgfqpoint{4.361250in}{3.056125in}}%
\pgfusepath{clip}%
\pgfsetroundcap%
\pgfsetroundjoin%
\pgfsetlinewidth{1.003750pt}%
\definecolor{currentstroke}{rgb}{0.800000,0.800000,0.800000}%
\pgfsetstrokecolor{currentstroke}%
\pgfsetstrokeopacity{0.400000}%
\pgfsetdash{}{0pt}%
\pgfpathmoveto{\pgfqpoint{1.230000in}{1.855989in}}%
\pgfpathlineto{\pgfqpoint{5.591250in}{1.855989in}}%
\pgfusepath{stroke}%
\end{pgfscope}%
\begin{pgfscope}%
\definecolor{textcolor}{rgb}{0.150000,0.150000,0.150000}%
\pgfsetstrokecolor{textcolor}%
\pgfsetfillcolor{textcolor}%
\pgftext[x=0.827172in,y=1.755969in,left,base]{\color{textcolor}\sffamily\fontsize{19.250000}{23.100000}\selectfont 25}%
\end{pgfscope}%
\begin{pgfscope}%
\pgfpathrectangle{\pgfqpoint{1.230000in}{1.022500in}}{\pgfqpoint{4.361250in}{3.056125in}}%
\pgfusepath{clip}%
\pgfsetroundcap%
\pgfsetroundjoin%
\pgfsetlinewidth{1.003750pt}%
\definecolor{currentstroke}{rgb}{0.800000,0.800000,0.800000}%
\pgfsetstrokecolor{currentstroke}%
\pgfsetstrokeopacity{0.400000}%
\pgfsetdash{}{0pt}%
\pgfpathmoveto{\pgfqpoint{1.230000in}{2.550562in}}%
\pgfpathlineto{\pgfqpoint{5.591250in}{2.550562in}}%
\pgfusepath{stroke}%
\end{pgfscope}%
\begin{pgfscope}%
\definecolor{textcolor}{rgb}{0.150000,0.150000,0.150000}%
\pgfsetstrokecolor{textcolor}%
\pgfsetfillcolor{textcolor}%
\pgftext[x=0.827172in,y=2.450543in,left,base]{\color{textcolor}\sffamily\fontsize{19.250000}{23.100000}\selectfont 50}%
\end{pgfscope}%
\begin{pgfscope}%
\pgfpathrectangle{\pgfqpoint{1.230000in}{1.022500in}}{\pgfqpoint{4.361250in}{3.056125in}}%
\pgfusepath{clip}%
\pgfsetroundcap%
\pgfsetroundjoin%
\pgfsetlinewidth{1.003750pt}%
\definecolor{currentstroke}{rgb}{0.800000,0.800000,0.800000}%
\pgfsetstrokecolor{currentstroke}%
\pgfsetstrokeopacity{0.400000}%
\pgfsetdash{}{0pt}%
\pgfpathmoveto{\pgfqpoint{1.230000in}{3.245136in}}%
\pgfpathlineto{\pgfqpoint{5.591250in}{3.245136in}}%
\pgfusepath{stroke}%
\end{pgfscope}%
\begin{pgfscope}%
\definecolor{textcolor}{rgb}{0.150000,0.150000,0.150000}%
\pgfsetstrokecolor{textcolor}%
\pgfsetfillcolor{textcolor}%
\pgftext[x=0.827172in,y=3.145117in,left,base]{\color{textcolor}\sffamily\fontsize{19.250000}{23.100000}\selectfont 75}%
\end{pgfscope}%
\begin{pgfscope}%
\pgfpathrectangle{\pgfqpoint{1.230000in}{1.022500in}}{\pgfqpoint{4.361250in}{3.056125in}}%
\pgfusepath{clip}%
\pgfsetroundcap%
\pgfsetroundjoin%
\pgfsetlinewidth{1.003750pt}%
\definecolor{currentstroke}{rgb}{0.800000,0.800000,0.800000}%
\pgfsetstrokecolor{currentstroke}%
\pgfsetstrokeopacity{0.400000}%
\pgfsetdash{}{0pt}%
\pgfpathmoveto{\pgfqpoint{1.230000in}{3.939710in}}%
\pgfpathlineto{\pgfqpoint{5.591250in}{3.939710in}}%
\pgfusepath{stroke}%
\end{pgfscope}%
\begin{pgfscope}%
\definecolor{textcolor}{rgb}{0.150000,0.150000,0.150000}%
\pgfsetstrokecolor{textcolor}%
\pgfsetfillcolor{textcolor}%
\pgftext[x=0.691731in,y=3.839691in,left,base]{\color{textcolor}\sffamily\fontsize{19.250000}{23.100000}\selectfont 100}%
\end{pgfscope}%
\begin{pgfscope}%
\definecolor{textcolor}{rgb}{0.150000,0.150000,0.150000}%
\pgfsetstrokecolor{textcolor}%
\pgfsetfillcolor{textcolor}%
\pgftext[x=0.636175in,y=2.550563in,,bottom,rotate=90.000000]{\color{textcolor}\sffamily\fontsize{21.000000}{25.200000}\selectfont Percent of instructions synthesized}%
\end{pgfscope}%
\begin{pgfscope}%
\pgfpathrectangle{\pgfqpoint{1.230000in}{1.022500in}}{\pgfqpoint{4.361250in}{3.056125in}}%
\pgfusepath{clip}%
\pgfsetroundcap%
\pgfsetroundjoin%
\pgfsetlinewidth{1.505625pt}%
\definecolor{currentstroke}{rgb}{0.298039,0.447059,0.690196}%
\pgfsetstrokecolor{currentstroke}%
\pgfsetdash{}{0pt}%
\pgfpathmoveto{\pgfqpoint{1.428239in}{1.161415in}}%
\pgfpathlineto{\pgfqpoint{1.444578in}{1.161415in}}%
\pgfpathlineto{\pgfqpoint{1.444578in}{1.558314in}}%
\pgfpathlineto{\pgfqpoint{1.456833in}{1.558314in}}%
\pgfpathlineto{\pgfqpoint{1.456833in}{1.955213in}}%
\pgfpathlineto{\pgfqpoint{1.469087in}{1.955213in}}%
\pgfpathlineto{\pgfqpoint{1.469087in}{2.352113in}}%
\pgfpathlineto{\pgfqpoint{1.485427in}{2.352113in}}%
\pgfpathlineto{\pgfqpoint{1.485427in}{2.749012in}}%
\pgfpathlineto{\pgfqpoint{1.497681in}{2.749012in}}%
\pgfpathlineto{\pgfqpoint{1.497681in}{3.145912in}}%
\pgfpathlineto{\pgfqpoint{1.524369in}{3.145912in}}%
\pgfpathlineto{\pgfqpoint{1.524369in}{3.542811in}}%
\pgfpathlineto{\pgfqpoint{1.543977in}{3.542811in}}%
\pgfpathlineto{\pgfqpoint{1.543977in}{3.939710in}}%
\pgfusepath{stroke}%
\end{pgfscope}%
\begin{pgfscope}%
\pgfpathrectangle{\pgfqpoint{1.230000in}{1.022500in}}{\pgfqpoint{4.361250in}{3.056125in}}%
\pgfusepath{clip}%
\pgfsetbuttcap%
\pgfsetroundjoin%
\definecolor{currentfill}{rgb}{0.298039,0.447059,0.690196}%
\pgfsetfillcolor{currentfill}%
\pgfsetlinewidth{1.003750pt}%
\definecolor{currentstroke}{rgb}{0.298039,0.447059,0.690196}%
\pgfsetstrokecolor{currentstroke}%
\pgfsetdash{}{0pt}%
\pgfsys@defobject{currentmarker}{\pgfqpoint{-0.020833in}{-0.020833in}}{\pgfqpoint{0.020833in}{0.020833in}}{%
\pgfpathmoveto{\pgfqpoint{0.000000in}{-0.020833in}}%
\pgfpathcurveto{\pgfqpoint{0.005525in}{-0.020833in}}{\pgfqpoint{0.010825in}{-0.018638in}}{\pgfqpoint{0.014731in}{-0.014731in}}%
\pgfpathcurveto{\pgfqpoint{0.018638in}{-0.010825in}}{\pgfqpoint{0.020833in}{-0.005525in}}{\pgfqpoint{0.020833in}{0.000000in}}%
\pgfpathcurveto{\pgfqpoint{0.020833in}{0.005525in}}{\pgfqpoint{0.018638in}{0.010825in}}{\pgfqpoint{0.014731in}{0.014731in}}%
\pgfpathcurveto{\pgfqpoint{0.010825in}{0.018638in}}{\pgfqpoint{0.005525in}{0.020833in}}{\pgfqpoint{0.000000in}{0.020833in}}%
\pgfpathcurveto{\pgfqpoint{-0.005525in}{0.020833in}}{\pgfqpoint{-0.010825in}{0.018638in}}{\pgfqpoint{-0.014731in}{0.014731in}}%
\pgfpathcurveto{\pgfqpoint{-0.018638in}{0.010825in}}{\pgfqpoint{-0.020833in}{0.005525in}}{\pgfqpoint{-0.020833in}{0.000000in}}%
\pgfpathcurveto{\pgfqpoint{-0.020833in}{-0.005525in}}{\pgfqpoint{-0.018638in}{-0.010825in}}{\pgfqpoint{-0.014731in}{-0.014731in}}%
\pgfpathcurveto{\pgfqpoint{-0.010825in}{-0.018638in}}{\pgfqpoint{-0.005525in}{-0.020833in}}{\pgfqpoint{0.000000in}{-0.020833in}}%
\pgfpathclose%
\pgfusepath{stroke,fill}%
}%
\begin{pgfscope}%
\pgfsys@transformshift{1.428239in}{1.161415in}%
\pgfsys@useobject{currentmarker}{}%
\end{pgfscope}%
\begin{pgfscope}%
\pgfsys@transformshift{1.543977in}{3.939710in}%
\pgfsys@useobject{currentmarker}{}%
\end{pgfscope}%
\end{pgfscope}%
\begin{pgfscope}%
\pgfpathrectangle{\pgfqpoint{1.230000in}{1.022500in}}{\pgfqpoint{4.361250in}{3.056125in}}%
\pgfusepath{clip}%
\pgfsetbuttcap%
\pgfsetroundjoin%
\pgfsetlinewidth{1.505625pt}%
\definecolor{currentstroke}{rgb}{0.866667,0.517647,0.321569}%
\pgfsetstrokecolor{currentstroke}%
\pgfsetdash{{5.550000pt}{2.400000pt}}{0.000000pt}%
\pgfpathmoveto{\pgfqpoint{1.444578in}{1.161415in}}%
\pgfpathlineto{\pgfqpoint{1.444578in}{1.558314in}}%
\pgfpathlineto{\pgfqpoint{1.456833in}{1.558314in}}%
\pgfpathlineto{\pgfqpoint{1.456833in}{1.955213in}}%
\pgfpathlineto{\pgfqpoint{1.469087in}{1.955213in}}%
\pgfpathlineto{\pgfqpoint{1.469087in}{2.352113in}}%
\pgfpathlineto{\pgfqpoint{1.485427in}{2.352113in}}%
\pgfpathlineto{\pgfqpoint{1.485427in}{2.749012in}}%
\pgfpathlineto{\pgfqpoint{1.509936in}{2.749012in}}%
\pgfpathlineto{\pgfqpoint{1.509936in}{3.145912in}}%
\pgfpathlineto{\pgfqpoint{1.593267in}{3.145912in}}%
\pgfpathlineto{\pgfqpoint{1.593267in}{3.542811in}}%
\pgfpathlineto{\pgfqpoint{2.138189in}{3.542811in}}%
\pgfpathlineto{\pgfqpoint{2.138189in}{3.939710in}}%
\pgfusepath{stroke}%
\end{pgfscope}%
\begin{pgfscope}%
\pgfpathrectangle{\pgfqpoint{1.230000in}{1.022500in}}{\pgfqpoint{4.361250in}{3.056125in}}%
\pgfusepath{clip}%
\pgfsetbuttcap%
\pgfsetroundjoin%
\definecolor{currentfill}{rgb}{0.866667,0.517647,0.321569}%
\pgfsetfillcolor{currentfill}%
\pgfsetlinewidth{1.003750pt}%
\definecolor{currentstroke}{rgb}{0.866667,0.517647,0.321569}%
\pgfsetstrokecolor{currentstroke}%
\pgfsetdash{}{0pt}%
\pgfsys@defobject{currentmarker}{\pgfqpoint{-0.020833in}{-0.020833in}}{\pgfqpoint{0.020833in}{0.020833in}}{%
\pgfpathmoveto{\pgfqpoint{0.000000in}{-0.020833in}}%
\pgfpathcurveto{\pgfqpoint{0.005525in}{-0.020833in}}{\pgfqpoint{0.010825in}{-0.018638in}}{\pgfqpoint{0.014731in}{-0.014731in}}%
\pgfpathcurveto{\pgfqpoint{0.018638in}{-0.010825in}}{\pgfqpoint{0.020833in}{-0.005525in}}{\pgfqpoint{0.020833in}{0.000000in}}%
\pgfpathcurveto{\pgfqpoint{0.020833in}{0.005525in}}{\pgfqpoint{0.018638in}{0.010825in}}{\pgfqpoint{0.014731in}{0.014731in}}%
\pgfpathcurveto{\pgfqpoint{0.010825in}{0.018638in}}{\pgfqpoint{0.005525in}{0.020833in}}{\pgfqpoint{0.000000in}{0.020833in}}%
\pgfpathcurveto{\pgfqpoint{-0.005525in}{0.020833in}}{\pgfqpoint{-0.010825in}{0.018638in}}{\pgfqpoint{-0.014731in}{0.014731in}}%
\pgfpathcurveto{\pgfqpoint{-0.018638in}{0.010825in}}{\pgfqpoint{-0.020833in}{0.005525in}}{\pgfqpoint{-0.020833in}{0.000000in}}%
\pgfpathcurveto{\pgfqpoint{-0.020833in}{-0.005525in}}{\pgfqpoint{-0.018638in}{-0.010825in}}{\pgfqpoint{-0.014731in}{-0.014731in}}%
\pgfpathcurveto{\pgfqpoint{-0.010825in}{-0.018638in}}{\pgfqpoint{-0.005525in}{-0.020833in}}{\pgfqpoint{0.000000in}{-0.020833in}}%
\pgfpathclose%
\pgfusepath{stroke,fill}%
}%
\begin{pgfscope}%
\pgfsys@transformshift{2.138189in}{3.939710in}%
\pgfsys@useobject{currentmarker}{}%
\end{pgfscope}%
\end{pgfscope}%
\begin{pgfscope}%
\pgfpathrectangle{\pgfqpoint{1.230000in}{1.022500in}}{\pgfqpoint{4.361250in}{3.056125in}}%
\pgfusepath{clip}%
\pgfsetbuttcap%
\pgfsetroundjoin%
\pgfsetlinewidth{1.505625pt}%
\definecolor{currentstroke}{rgb}{0.333333,0.658824,0.407843}%
\pgfsetstrokecolor{currentstroke}%
\pgfsetdash{{9.600000pt}{2.400000pt}{1.500000pt}{2.400000pt}}{0.000000pt}%
\pgfpathmoveto{\pgfqpoint{1.428239in}{1.161415in}}%
\pgfpathlineto{\pgfqpoint{1.481887in}{1.161415in}}%
\pgfpathlineto{\pgfqpoint{1.481887in}{1.558314in}}%
\pgfpathlineto{\pgfqpoint{3.579058in}{1.558314in}}%
\pgfpathlineto{\pgfqpoint{3.579058in}{1.955213in}}%
\pgfpathlineto{\pgfqpoint{3.629710in}{1.955213in}}%
\pgfpathlineto{\pgfqpoint{3.629710in}{2.352113in}}%
\pgfpathlineto{\pgfqpoint{3.682813in}{2.352113in}}%
\pgfpathlineto{\pgfqpoint{3.682813in}{2.749012in}}%
\pgfpathlineto{\pgfqpoint{5.288711in}{2.749012in}}%
\pgfpathlineto{\pgfqpoint{5.288711in}{3.145912in}}%
\pgfpathlineto{\pgfqpoint{5.342631in}{3.145912in}}%
\pgfpathlineto{\pgfqpoint{5.342631in}{3.542811in}}%
\pgfpathlineto{\pgfqpoint{5.393011in}{3.542811in}}%
\pgfpathlineto{\pgfqpoint{5.393011in}{3.939710in}}%
\pgfusepath{stroke}%
\end{pgfscope}%
\begin{pgfscope}%
\pgfpathrectangle{\pgfqpoint{1.230000in}{1.022500in}}{\pgfqpoint{4.361250in}{3.056125in}}%
\pgfusepath{clip}%
\pgfsetbuttcap%
\pgfsetroundjoin%
\definecolor{currentfill}{rgb}{0.333333,0.658824,0.407843}%
\pgfsetfillcolor{currentfill}%
\pgfsetlinewidth{1.003750pt}%
\definecolor{currentstroke}{rgb}{0.333333,0.658824,0.407843}%
\pgfsetstrokecolor{currentstroke}%
\pgfsetdash{}{0pt}%
\pgfsys@defobject{currentmarker}{\pgfqpoint{-0.020833in}{-0.020833in}}{\pgfqpoint{0.020833in}{0.020833in}}{%
\pgfpathmoveto{\pgfqpoint{0.000000in}{-0.020833in}}%
\pgfpathcurveto{\pgfqpoint{0.005525in}{-0.020833in}}{\pgfqpoint{0.010825in}{-0.018638in}}{\pgfqpoint{0.014731in}{-0.014731in}}%
\pgfpathcurveto{\pgfqpoint{0.018638in}{-0.010825in}}{\pgfqpoint{0.020833in}{-0.005525in}}{\pgfqpoint{0.020833in}{0.000000in}}%
\pgfpathcurveto{\pgfqpoint{0.020833in}{0.005525in}}{\pgfqpoint{0.018638in}{0.010825in}}{\pgfqpoint{0.014731in}{0.014731in}}%
\pgfpathcurveto{\pgfqpoint{0.010825in}{0.018638in}}{\pgfqpoint{0.005525in}{0.020833in}}{\pgfqpoint{0.000000in}{0.020833in}}%
\pgfpathcurveto{\pgfqpoint{-0.005525in}{0.020833in}}{\pgfqpoint{-0.010825in}{0.018638in}}{\pgfqpoint{-0.014731in}{0.014731in}}%
\pgfpathcurveto{\pgfqpoint{-0.018638in}{0.010825in}}{\pgfqpoint{-0.020833in}{0.005525in}}{\pgfqpoint{-0.020833in}{0.000000in}}%
\pgfpathcurveto{\pgfqpoint{-0.020833in}{-0.005525in}}{\pgfqpoint{-0.018638in}{-0.010825in}}{\pgfqpoint{-0.014731in}{-0.014731in}}%
\pgfpathcurveto{\pgfqpoint{-0.010825in}{-0.018638in}}{\pgfqpoint{-0.005525in}{-0.020833in}}{\pgfqpoint{0.000000in}{-0.020833in}}%
\pgfpathclose%
\pgfusepath{stroke,fill}%
}%
\begin{pgfscope}%
\pgfsys@transformshift{1.428239in}{1.161415in}%
\pgfsys@useobject{currentmarker}{}%
\end{pgfscope}%
\begin{pgfscope}%
\pgfsys@transformshift{5.393011in}{3.939710in}%
\pgfsys@useobject{currentmarker}{}%
\end{pgfscope}%
\end{pgfscope}%
\begin{pgfscope}%
\pgfsetrectcap%
\pgfsetmiterjoin%
\pgfsetlinewidth{1.254687pt}%
\definecolor{currentstroke}{rgb}{0.800000,0.800000,0.800000}%
\pgfsetstrokecolor{currentstroke}%
\pgfsetdash{}{0pt}%
\pgfpathmoveto{\pgfqpoint{1.230000in}{1.022500in}}%
\pgfpathlineto{\pgfqpoint{1.230000in}{4.078625in}}%
\pgfusepath{stroke}%
\end{pgfscope}%
\begin{pgfscope}%
\pgfsetrectcap%
\pgfsetmiterjoin%
\pgfsetlinewidth{1.254687pt}%
\definecolor{currentstroke}{rgb}{0.800000,0.800000,0.800000}%
\pgfsetstrokecolor{currentstroke}%
\pgfsetdash{}{0pt}%
\pgfpathmoveto{\pgfqpoint{5.591250in}{1.022500in}}%
\pgfpathlineto{\pgfqpoint{5.591250in}{4.078625in}}%
\pgfusepath{stroke}%
\end{pgfscope}%
\begin{pgfscope}%
\pgfsetrectcap%
\pgfsetmiterjoin%
\pgfsetlinewidth{1.254687pt}%
\definecolor{currentstroke}{rgb}{0.800000,0.800000,0.800000}%
\pgfsetstrokecolor{currentstroke}%
\pgfsetdash{}{0pt}%
\pgfpathmoveto{\pgfqpoint{1.230000in}{1.022500in}}%
\pgfpathlineto{\pgfqpoint{5.591250in}{1.022500in}}%
\pgfusepath{stroke}%
\end{pgfscope}%
\begin{pgfscope}%
\pgfsetrectcap%
\pgfsetmiterjoin%
\pgfsetlinewidth{1.254687pt}%
\definecolor{currentstroke}{rgb}{0.800000,0.800000,0.800000}%
\pgfsetstrokecolor{currentstroke}%
\pgfsetdash{}{0pt}%
\pgfpathmoveto{\pgfqpoint{1.230000in}{4.078625in}}%
\pgfpathlineto{\pgfqpoint{5.591250in}{4.078625in}}%
\pgfusepath{stroke}%
\end{pgfscope}%
\begin{pgfscope}%
\definecolor{textcolor}{rgb}{0.150000,0.150000,0.150000}%
\pgfsetstrokecolor{textcolor}%
\pgfsetfillcolor{textcolor}%
\pgftext[x=3.410625in,y=4.161958in,,base]{\color{textcolor}\sffamily\fontsize{21.000000}{25.200000}\selectfont Synthesis time for classic libseccomp to eBPF}%
\end{pgfscope}%
\begin{pgfscope}%
\pgfsetbuttcap%
\pgfsetmiterjoin%
\definecolor{currentfill}{rgb}{1.000000,1.000000,1.000000}%
\pgfsetfillcolor{currentfill}%
\pgfsetfillopacity{0.800000}%
\pgfsetlinewidth{1.003750pt}%
\definecolor{currentstroke}{rgb}{0.800000,0.800000,0.800000}%
\pgfsetstrokecolor{currentstroke}%
\pgfsetstrokeopacity{0.800000}%
\pgfsetdash{}{0pt}%
\pgfpathmoveto{\pgfqpoint{2.375964in}{2.695492in}}%
\pgfpathlineto{\pgfqpoint{4.445286in}{2.695492in}}%
\pgfpathquadraticcurveto{\pgfqpoint{4.498758in}{2.695492in}}{\pgfqpoint{4.498758in}{2.748964in}}%
\pgfpathlineto{\pgfqpoint{4.498758in}{3.891472in}}%
\pgfpathquadraticcurveto{\pgfqpoint{4.498758in}{3.944944in}}{\pgfqpoint{4.445286in}{3.944944in}}%
\pgfpathlineto{\pgfqpoint{2.375964in}{3.944944in}}%
\pgfpathquadraticcurveto{\pgfqpoint{2.322492in}{3.944944in}}{\pgfqpoint{2.322492in}{3.891472in}}%
\pgfpathlineto{\pgfqpoint{2.322492in}{2.748964in}}%
\pgfpathquadraticcurveto{\pgfqpoint{2.322492in}{2.695492in}}{\pgfqpoint{2.375964in}{2.695492in}}%
\pgfpathclose%
\pgfusepath{stroke,fill}%
\end{pgfscope}%
\begin{pgfscope}%
\pgfsetroundcap%
\pgfsetroundjoin%
\pgfsetlinewidth{1.505625pt}%
\definecolor{currentstroke}{rgb}{0.298039,0.447059,0.690196}%
\pgfsetstrokecolor{currentstroke}%
\pgfsetdash{}{0pt}%
\pgfpathmoveto{\pgfqpoint{2.429436in}{3.731538in}}%
\pgfpathlineto{\pgfqpoint{2.964158in}{3.731538in}}%
\pgfusepath{stroke}%
\end{pgfscope}%
\begin{pgfscope}%
\pgfsetbuttcap%
\pgfsetroundjoin%
\definecolor{currentfill}{rgb}{0.298039,0.447059,0.690196}%
\pgfsetfillcolor{currentfill}%
\pgfsetlinewidth{1.003750pt}%
\definecolor{currentstroke}{rgb}{0.298039,0.447059,0.690196}%
\pgfsetstrokecolor{currentstroke}%
\pgfsetdash{}{0pt}%
\pgfsys@defobject{currentmarker}{\pgfqpoint{-0.020833in}{-0.020833in}}{\pgfqpoint{0.020833in}{0.020833in}}{%
\pgfpathmoveto{\pgfqpoint{0.000000in}{-0.020833in}}%
\pgfpathcurveto{\pgfqpoint{0.005525in}{-0.020833in}}{\pgfqpoint{0.010825in}{-0.018638in}}{\pgfqpoint{0.014731in}{-0.014731in}}%
\pgfpathcurveto{\pgfqpoint{0.018638in}{-0.010825in}}{\pgfqpoint{0.020833in}{-0.005525in}}{\pgfqpoint{0.020833in}{0.000000in}}%
\pgfpathcurveto{\pgfqpoint{0.020833in}{0.005525in}}{\pgfqpoint{0.018638in}{0.010825in}}{\pgfqpoint{0.014731in}{0.014731in}}%
\pgfpathcurveto{\pgfqpoint{0.010825in}{0.018638in}}{\pgfqpoint{0.005525in}{0.020833in}}{\pgfqpoint{0.000000in}{0.020833in}}%
\pgfpathcurveto{\pgfqpoint{-0.005525in}{0.020833in}}{\pgfqpoint{-0.010825in}{0.018638in}}{\pgfqpoint{-0.014731in}{0.014731in}}%
\pgfpathcurveto{\pgfqpoint{-0.018638in}{0.010825in}}{\pgfqpoint{-0.020833in}{0.005525in}}{\pgfqpoint{-0.020833in}{0.000000in}}%
\pgfpathcurveto{\pgfqpoint{-0.020833in}{-0.005525in}}{\pgfqpoint{-0.018638in}{-0.010825in}}{\pgfqpoint{-0.014731in}{-0.014731in}}%
\pgfpathcurveto{\pgfqpoint{-0.010825in}{-0.018638in}}{\pgfqpoint{-0.005525in}{-0.020833in}}{\pgfqpoint{0.000000in}{-0.020833in}}%
\pgfpathclose%
\pgfusepath{stroke,fill}%
}%
\begin{pgfscope}%
\pgfsys@transformshift{2.696797in}{3.731538in}%
\pgfsys@useobject{currentmarker}{}%
\end{pgfscope}%
\end{pgfscope}%
\begin{pgfscope}%
\definecolor{textcolor}{rgb}{0.150000,0.150000,0.150000}%
\pgfsetstrokecolor{textcolor}%
\pgfsetfillcolor{textcolor}%
\pgftext[x=3.178047in,y=3.637962in,left,base]{\color{textcolor}\sffamily\fontsize{19.250000}{23.100000}\selectfont Pre-load}%
\end{pgfscope}%
\begin{pgfscope}%
\pgfsetbuttcap%
\pgfsetroundjoin%
\pgfsetlinewidth{1.505625pt}%
\definecolor{currentstroke}{rgb}{0.866667,0.517647,0.321569}%
\pgfsetstrokecolor{currentstroke}%
\pgfsetdash{{5.550000pt}{2.400000pt}}{0.000000pt}%
\pgfpathmoveto{\pgfqpoint{2.429436in}{3.341790in}}%
\pgfpathlineto{\pgfqpoint{2.964158in}{3.341790in}}%
\pgfusepath{stroke}%
\end{pgfscope}%
\begin{pgfscope}%
\pgfsetbuttcap%
\pgfsetroundjoin%
\definecolor{currentfill}{rgb}{0.866667,0.517647,0.321569}%
\pgfsetfillcolor{currentfill}%
\pgfsetlinewidth{1.003750pt}%
\definecolor{currentstroke}{rgb}{0.866667,0.517647,0.321569}%
\pgfsetstrokecolor{currentstroke}%
\pgfsetdash{}{0pt}%
\pgfsys@defobject{currentmarker}{\pgfqpoint{-0.020833in}{-0.020833in}}{\pgfqpoint{0.020833in}{0.020833in}}{%
\pgfpathmoveto{\pgfqpoint{0.000000in}{-0.020833in}}%
\pgfpathcurveto{\pgfqpoint{0.005525in}{-0.020833in}}{\pgfqpoint{0.010825in}{-0.018638in}}{\pgfqpoint{0.014731in}{-0.014731in}}%
\pgfpathcurveto{\pgfqpoint{0.018638in}{-0.010825in}}{\pgfqpoint{0.020833in}{-0.005525in}}{\pgfqpoint{0.020833in}{0.000000in}}%
\pgfpathcurveto{\pgfqpoint{0.020833in}{0.005525in}}{\pgfqpoint{0.018638in}{0.010825in}}{\pgfqpoint{0.014731in}{0.014731in}}%
\pgfpathcurveto{\pgfqpoint{0.010825in}{0.018638in}}{\pgfqpoint{0.005525in}{0.020833in}}{\pgfqpoint{0.000000in}{0.020833in}}%
\pgfpathcurveto{\pgfqpoint{-0.005525in}{0.020833in}}{\pgfqpoint{-0.010825in}{0.018638in}}{\pgfqpoint{-0.014731in}{0.014731in}}%
\pgfpathcurveto{\pgfqpoint{-0.018638in}{0.010825in}}{\pgfqpoint{-0.020833in}{0.005525in}}{\pgfqpoint{-0.020833in}{0.000000in}}%
\pgfpathcurveto{\pgfqpoint{-0.020833in}{-0.005525in}}{\pgfqpoint{-0.018638in}{-0.010825in}}{\pgfqpoint{-0.014731in}{-0.014731in}}%
\pgfpathcurveto{\pgfqpoint{-0.010825in}{-0.018638in}}{\pgfqpoint{-0.005525in}{-0.020833in}}{\pgfqpoint{0.000000in}{-0.020833in}}%
\pgfpathclose%
\pgfusepath{stroke,fill}%
}%
\begin{pgfscope}%
\pgfsys@transformshift{2.696797in}{3.341790in}%
\pgfsys@useobject{currentmarker}{}%
\end{pgfscope}%
\end{pgfscope}%
\begin{pgfscope}%
\definecolor{textcolor}{rgb}{0.150000,0.150000,0.150000}%
\pgfsetstrokecolor{textcolor}%
\pgfsetfillcolor{textcolor}%
\pgftext[x=3.178047in,y=3.248213in,left,base]{\color{textcolor}\sffamily\fontsize{19.250000}{23.100000}\selectfont Read-write}%
\end{pgfscope}%
\begin{pgfscope}%
\pgfsetbuttcap%
\pgfsetroundjoin%
\pgfsetlinewidth{1.505625pt}%
\definecolor{currentstroke}{rgb}{0.333333,0.658824,0.407843}%
\pgfsetstrokecolor{currentstroke}%
\pgfsetdash{{9.600000pt}{2.400000pt}{1.500000pt}{2.400000pt}}{0.000000pt}%
\pgfpathmoveto{\pgfqpoint{2.429436in}{2.952042in}}%
\pgfpathlineto{\pgfqpoint{2.964158in}{2.952042in}}%
\pgfusepath{stroke}%
\end{pgfscope}%
\begin{pgfscope}%
\pgfsetbuttcap%
\pgfsetroundjoin%
\definecolor{currentfill}{rgb}{0.333333,0.658824,0.407843}%
\pgfsetfillcolor{currentfill}%
\pgfsetlinewidth{1.003750pt}%
\definecolor{currentstroke}{rgb}{0.333333,0.658824,0.407843}%
\pgfsetstrokecolor{currentstroke}%
\pgfsetdash{}{0pt}%
\pgfsys@defobject{currentmarker}{\pgfqpoint{-0.020833in}{-0.020833in}}{\pgfqpoint{0.020833in}{0.020833in}}{%
\pgfpathmoveto{\pgfqpoint{0.000000in}{-0.020833in}}%
\pgfpathcurveto{\pgfqpoint{0.005525in}{-0.020833in}}{\pgfqpoint{0.010825in}{-0.018638in}}{\pgfqpoint{0.014731in}{-0.014731in}}%
\pgfpathcurveto{\pgfqpoint{0.018638in}{-0.010825in}}{\pgfqpoint{0.020833in}{-0.005525in}}{\pgfqpoint{0.020833in}{0.000000in}}%
\pgfpathcurveto{\pgfqpoint{0.020833in}{0.005525in}}{\pgfqpoint{0.018638in}{0.010825in}}{\pgfqpoint{0.014731in}{0.014731in}}%
\pgfpathcurveto{\pgfqpoint{0.010825in}{0.018638in}}{\pgfqpoint{0.005525in}{0.020833in}}{\pgfqpoint{0.000000in}{0.020833in}}%
\pgfpathcurveto{\pgfqpoint{-0.005525in}{0.020833in}}{\pgfqpoint{-0.010825in}{0.018638in}}{\pgfqpoint{-0.014731in}{0.014731in}}%
\pgfpathcurveto{\pgfqpoint{-0.018638in}{0.010825in}}{\pgfqpoint{-0.020833in}{0.005525in}}{\pgfqpoint{-0.020833in}{0.000000in}}%
\pgfpathcurveto{\pgfqpoint{-0.020833in}{-0.005525in}}{\pgfqpoint{-0.018638in}{-0.010825in}}{\pgfqpoint{-0.014731in}{-0.014731in}}%
\pgfpathcurveto{\pgfqpoint{-0.010825in}{-0.018638in}}{\pgfqpoint{-0.005525in}{-0.020833in}}{\pgfqpoint{0.000000in}{-0.020833in}}%
\pgfpathclose%
\pgfusepath{stroke,fill}%
}%
\begin{pgfscope}%
\pgfsys@transformshift{2.696797in}{2.952042in}%
\pgfsys@useobject{currentmarker}{}%
\end{pgfscope}%
\end{pgfscope}%
\begin{pgfscope}%
\definecolor{textcolor}{rgb}{0.150000,0.150000,0.150000}%
\pgfsetstrokecolor{textcolor}%
\pgfsetfillcolor{textcolor}%
\pgftext[x=3.178047in,y=2.858465in,left,base]{\color{textcolor}\sffamily\fontsize{19.250000}{23.100000}\selectfont Naïve}%
\end{pgfscope}%
\end{pgfpicture}%
\makeatother%
\endgroup%

  %   }
  % \end{center}
  % \caption{Synthesis time per instruction for our three source-target pairs.
  % % TODO do things other than color-code (i.e. different types of ticks)
  % Green corresponds to \Naive sketches.
  % Orange corresponds to \RW sketches.
  % Blue corresponds to \LCS sketches.
  % A red X indicates that synthesis timed out or ran out of memory after that point.}

\begin{center}
\begin{tabular}{l|c|c|c}

\toprule
  Compiler & \Naive sketch & \RW sketch & \LCS sketch \\
\midrule
  eBPF to RISC-V & X & X & 44.4h \\
  classic BPF to eBPF & X & X & 1.2h \\
  libseccomp to eBPF & 4.0h & 43.5m & 7.1m  \\
\bottomrule
\end{tabular}
\end{center}

\caption{
Synthesis time for each source-target pair, broken
down by set of optimizations used in the sketch.
An X indicates that synthesis either timed out
or ran out of memory.}


  \label{fig:o2b-l2b-synthtime}
\end{figure}

In order to evaluate the effectiveness of the search optimizations
described in \autoref{s:algorithm}, we measured the time \jitsynth
takes to synthesize each of the three compilers with different optimizations
enabled.
%
Specifically, we run \jitsynth in three different configurations:
(1) using \Naive sketches, (2) using \RW sketches, and (3) using \LCS sketches.
%
For each configuration, we ran \jitsynth with a timeout of 48 hours (or until out of memory).
\autoref{fig:o2b-l2b-synthtime} shows the time to synthesize
each compiler under each configuration.
%
Note that these figures do not include time spent computing read and write sets,
which takes less than 11 minutes for all cases.
%
% The order instructions are synthesized in is fixed across configurations.
% %
Our results were collected using an 8-core AMD Ryzen 7 1700 CPU with 16~GB memory,
running Racket v7.4 and the Boolector~\cite{niemetz:boolector} solver v3.0.1-pre.


When synthesizing the eBPF-to-RISC-V compiler,
\jitsynth runs out of memory with \Naive sketches,
reaches the timeout with \RW sketches,
and completes synthesis with \LCS sketches.
%
For the classic-BPF-to-eBPF compiler,
\jitsynth times out with both \Naive sketches and \RW sketches.
\jitsynth only finishes synthesis with \LCS sketches.
%
For the libseccomp-to-eBPF compiler, all configurations finish,
but \jitsynth finishes synthesis about $\LibseccompSynthSpeedup\times$ times faster
with \LCS sketches than with \Naive sketches.
%
These results demonstrate that the techniques \jitsynth uses
are essential to the scalability of JIT synthesis.


\section{Related Work}\label{s:related}


\paragraph{JIT compilers for in-kernel languages.}

JIT compilers have been widely used to improve
the extensibility and performance of systems software,
such as OS kernels~\cite{chen:vmsec,engler:vcode,fleming:ebpf,myreen:vjit}.
%
One notable system is Jitk~\cite{wang:jitk}.
It builds on the CompCert compiler~\cite{leroy:compcert}
to compile classic BPF programs to machine instructions.
%
Both Jitk and CompCert are formally verified for correctness
using the Coq interactive theorem prover.  Jitk is further extended
to support eBPF~\cite{sobel:ejitk}.
%
Like Jitk,
\jitsynth provides formal correctness guarantees
of JIT compilers.
Unlike Jitk,
\jitsynth does not require developers to write either the implementation or proof
of a JIT compiler.
Instead,
it takes as input interpreters of both source and target languages
and state-mapping functions,
using automated verification and synthesis to produce a JIT compiler.

An in-kernel extension system such as eBPF
also contains a \emph{verifier},
which checks for safety and termination of input programs~\cite{gershuni:crab-ebpf,wang:jitk}.
\jitsynth assumes a well-formed input program that passes the verifier
and focuses on the correctness of JIT compilation.

\paragraph{Synthesis-aided compilers.}

There is a rich literature that explores generating and synthesizing
peephole optimizers and superoptimizers based on a given ISA or language
specification~\cite{bansal:superopt,davidson:peephole,gulwani:brahma,joshi:denali,massalin:superopt,sasnauskas:souper,schkufza:stoke}.
%
Bansal and Aiken described a PowerPC-to-x86 binary translator using
peephole superoptimization~\cite{bansal:binary}.
%
Chlorophyll~\cite{phothilimthana:chlorophyll} applied synthesis
to a number of compilation tasks for the GreenArrays GA144 architecture,
including code partitioning, layout, and generation.
%
\jitsynth bears the similarity of translation between a source-target pair
of languages and shares the challenge of scaling up synthesis.
%
Unlike existing work,
\jitsynth synthesizes a \emph{compiler} written in a host language,
and uses compiler metasketches for efficient synthesis.
%which is further translated into C for execution in an OS kernel.

\paragraph{Compiler testing.}
Compilers are complex pieces of software and are known to be difficult
to get right~\cite{marcozzi:compiler-fuzzing}.
Recent advances in compiler testing,
such as Csmith~\cite{yang:csmith} and EMI~\cite{zhang:emi},
have found hundreds of bugs in GCC and LLVM compilers.
%
Alive~\cite{lee:aliveinlean,lopes:alive} and Serval~\cite{nelson:serval}
use automated verification techniques to uncover bugs in the LLVM's peephole optimizer
and the Linux kernel's eBPF JIT compilers, respectively.
%
\jitsynth complements these tools by providing a correctness-by-construction
approach for writing JIT compilers.


\section{Conclusion}\label{jitsynth:s:conclusion}
This paper presents a new technique for synthesizing JIT compilers for in-kernel
DSLs. The technique creates per-instruction compilers, or compilers that
independently translate single source instructions to sequences of target
instructions. In order to synthesize each per-instruction compiler, we frame the
problem as search using compiler metasketches, which are optimized using both
read and write set information as well as pre-synthesized load operations. We
implement these techniques in \jitsynth and evaluate \jitsynth over three source
and target pairs from the Linux kernel. Our evaluation shows that (1) \jitsynth
can synthesize correct and performant compilers for real in-kernel languages,
and (2) the optimizations discussed in this paper make the synthesis of these
compilers tractable to \jitsynth. As future in-kernel DSLs are created,
\jitsynth can reduce both the programming and proof burden on developers writing
compilers for those DSLs. The \jitsynth source code is publicly available at
\url{https://github.com/uw-unsat/jitsynth}.


%\bibliographystyle{splncs04}
%\bibliography{n-str,jitsynth,n,n-conf}

% \begin{appendix}

\end{appendix}

