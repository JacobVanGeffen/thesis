\section{Conclusion}\label{jitsynth:s:conclusion}
This chapter presents a new technique for synthesizing JIT compilers for in-kernel
DSLs. The technique creates per-instruction compilers, or compilers that
independently translate single source instructions to sequences of target
instructions. In order to synthesize each per-instruction compiler, we frame the
problem as search using compiler metasketches, which are optimized using both
read and write set information as well as pre-synthesized load operations. We
implement these techniques in \jitsynth and evaluate \jitsynth over three source
and target pairs from the Linux kernel. Our evaluation shows that (1) \jitsynth
can synthesize correct and reasonably performant compilers for real in-kernel languages,
and (2) the optimizations discussed in this chapter make the synthesis of these
compilers tractable to \jitsynth. As future in-kernel DSLs are created,
\jitsynth can reduce both the programming and proof burden on developers writing
compilers for those DSLs. The \jitsynth source code is publicly available at
\url{https://github.com/uw-unsat/jitsynth}.
