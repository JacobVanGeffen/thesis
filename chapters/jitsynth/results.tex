\section{Evaluation}\label{s:eval}

This section evaluates \jitsynth by answering the following research questions:
\begin{enumerate}[label={},leftmargin=0em]
  \item \textbf{RQ1}: Can \jitsynth synthesize correct and performant compilers for
  real-world source and target languages?
  % \item \textbf{RQ2}: Can \jitsynth synthesize a wide variety of compilers?
  \item \textbf{RQ2}: How effective are the sketch optimizations described in \autoref{s:algorithm}?
\end{enumerate}


\subsection{Synthesizing compilers for real-world source-target pairs}

% \begin{figure}
%   \centering
%   \begin{tabular}{lccc}
\toprule
 & eBPF & RISC-V & Classic BPF \\
\midrule
 Arithmetic instructions & 46 & 33 & 10 \\
 Memory instructions & 12 & 11 & 9 \\
 Jump instructions & 24 & 7 & 11 \\
 Other instructions & 1 & 2 & 2 \\
 Number of registers & 10 & 31 & 2 \\

\bottomrule

\end{tabular}
%   \vspace{1.5em}\newline
%   \begin{tabular}{ll}
\toprule
libseccomp Instruction & Description\\
\midrule
  Arg Check & Check if rule argument passed \\
  Arg Fail & Mark when a rule argment fails \\
  Rule Check Failed & If a rule argument has failed, fall through to next rule \\
  Rule Check Pass & Check to see if the rule has passed \\
  Rule Pass & Perform specified action when rule passes \\
  Rule End & Clean registers after checking rule \\
  Default & Perform the default action \\
\bottomrule
\end{tabular}

%   \caption{Description of eBPF, RISC-V, classic BPF, and libseccomp languages}
%   \label{fig:lang}
% \end{figure}

\begin{figure}[h]
  \resizebox{\textwidth}{!}{
  %% Creator: Matplotlib, PGF backend
%%
%% To include the figure in your LaTeX document, write
%%   \input{<filename>.pgf}
%%
%% Make sure the required packages are loaded in your preamble
%%   \usepackage{pgf}
%%
%% Figures using additional raster images can only be included by \input if
%% they are in the same directory as the main LaTeX file. For loading figures
%% from other directories you can use the `import` package
%%   \usepackage{import}
%% and then include the figures with
%%   \import{<path to file>}{<filename>.pgf}
%%
%% Matplotlib used the following preamble
%%
\begingroup%
\makeatletter%
\begin{pgfpicture}%
\pgfpathrectangle{\pgfpointorigin}{\pgfqpoint{15.000000in}{5.000000in}}%
\pgfusepath{use as bounding box, clip}%
\begin{pgfscope}%
\pgfsetbuttcap%
\pgfsetmiterjoin%
\definecolor{currentfill}{rgb}{1.000000,1.000000,1.000000}%
\pgfsetfillcolor{currentfill}%
\pgfsetlinewidth{0.000000pt}%
\definecolor{currentstroke}{rgb}{1.000000,1.000000,1.000000}%
\pgfsetstrokecolor{currentstroke}%
\pgfsetdash{}{0pt}%
\pgfpathmoveto{\pgfqpoint{0.000000in}{0.000000in}}%
\pgfpathlineto{\pgfqpoint{15.000000in}{0.000000in}}%
\pgfpathlineto{\pgfqpoint{15.000000in}{5.000000in}}%
\pgfpathlineto{\pgfqpoint{0.000000in}{5.000000in}}%
\pgfpathclose%
\pgfusepath{fill}%
\end{pgfscope}%
\begin{pgfscope}%
\pgfsetbuttcap%
\pgfsetmiterjoin%
\definecolor{currentfill}{rgb}{1.000000,1.000000,1.000000}%
\pgfsetfillcolor{currentfill}%
\pgfsetlinewidth{0.000000pt}%
\definecolor{currentstroke}{rgb}{0.000000,0.000000,0.000000}%
\pgfsetstrokecolor{currentstroke}%
\pgfsetstrokeopacity{0.000000}%
\pgfsetdash{}{0pt}%
\pgfpathmoveto{\pgfqpoint{1.228750in}{1.022500in}}%
\pgfpathlineto{\pgfqpoint{14.685000in}{1.022500in}}%
\pgfpathlineto{\pgfqpoint{14.685000in}{4.685000in}}%
\pgfpathlineto{\pgfqpoint{1.228750in}{4.685000in}}%
\pgfpathclose%
\pgfusepath{fill}%
\end{pgfscope}%
\begin{pgfscope}%
\definecolor{textcolor}{rgb}{0.150000,0.150000,0.150000}%
\pgfsetstrokecolor{textcolor}%
\pgfsetfillcolor{textcolor}%
\pgftext[x=2.189911in,y=0.890556in,,top]{\color{textcolor}\sffamily\fontsize{19.250000}{23.100000}\selectfont OpenSSH}%
\end{pgfscope}%
\begin{pgfscope}%
\definecolor{textcolor}{rgb}{0.150000,0.150000,0.150000}%
\pgfsetstrokecolor{textcolor}%
\pgfsetfillcolor{textcolor}%
\pgftext[x=4.112232in,y=0.890556in,,top]{\color{textcolor}\sffamily\fontsize{19.250000}{23.100000}\selectfont NaCl}%
\end{pgfscope}%
\begin{pgfscope}%
\definecolor{textcolor}{rgb}{0.150000,0.150000,0.150000}%
\pgfsetstrokecolor{textcolor}%
\pgfsetfillcolor{textcolor}%
\pgftext[x=6.034554in,y=0.890556in,,top]{\color{textcolor}\sffamily\fontsize{19.250000}{23.100000}\selectfont QEMU}%
\end{pgfscope}%
\begin{pgfscope}%
\definecolor{textcolor}{rgb}{0.150000,0.150000,0.150000}%
\pgfsetstrokecolor{textcolor}%
\pgfsetfillcolor{textcolor}%
\pgftext[x=7.956875in,y=0.890556in,,top]{\color{textcolor}\sffamily\fontsize{19.250000}{23.100000}\selectfont Chrome}%
\end{pgfscope}%
\begin{pgfscope}%
\definecolor{textcolor}{rgb}{0.150000,0.150000,0.150000}%
\pgfsetstrokecolor{textcolor}%
\pgfsetfillcolor{textcolor}%
\pgftext[x=9.879196in,y=0.890556in,,top]{\color{textcolor}\sffamily\fontsize{19.250000}{23.100000}\selectfont Firefox}%
\end{pgfscope}%
\begin{pgfscope}%
\definecolor{textcolor}{rgb}{0.150000,0.150000,0.150000}%
\pgfsetstrokecolor{textcolor}%
\pgfsetfillcolor{textcolor}%
\pgftext[x=11.801518in,y=0.890556in,,top]{\color{textcolor}\sffamily\fontsize{19.250000}{23.100000}\selectfont vsftpd}%
\end{pgfscope}%
\begin{pgfscope}%
\definecolor{textcolor}{rgb}{0.150000,0.150000,0.150000}%
\pgfsetstrokecolor{textcolor}%
\pgfsetfillcolor{textcolor}%
\pgftext[x=13.723839in,y=0.890556in,,top]{\color{textcolor}\sffamily\fontsize{19.250000}{23.100000}\selectfont Tor}%
\end{pgfscope}%
\begin{pgfscope}%
\definecolor{textcolor}{rgb}{0.150000,0.150000,0.150000}%
\pgfsetstrokecolor{textcolor}%
\pgfsetfillcolor{textcolor}%
\pgftext[x=7.956875in,y=0.578932in,,top]{\color{textcolor}\sffamily\fontsize{21.000000}{25.200000}\selectfont Benchmark}%
\end{pgfscope}%
\begin{pgfscope}%
\pgfpathrectangle{\pgfqpoint{1.228750in}{1.022500in}}{\pgfqpoint{13.456250in}{3.662500in}}%
\pgfusepath{clip}%
\pgfsetroundcap%
\pgfsetroundjoin%
\pgfsetlinewidth{1.003750pt}%
\definecolor{currentstroke}{rgb}{0.800000,0.800000,0.800000}%
\pgfsetstrokecolor{currentstroke}%
\pgfsetdash{}{0pt}%
\pgfpathmoveto{\pgfqpoint{1.228750in}{1.022500in}}%
\pgfpathlineto{\pgfqpoint{14.685000in}{1.022500in}}%
\pgfusepath{stroke}%
\end{pgfscope}%
\begin{pgfscope}%
\definecolor{textcolor}{rgb}{0.150000,0.150000,0.150000}%
\pgfsetstrokecolor{textcolor}%
\pgfsetfillcolor{textcolor}%
\pgftext[x=0.961364in,y=0.922481in,left,base]{\color{textcolor}\sffamily\fontsize{19.250000}{23.100000}\selectfont 0}%
\end{pgfscope}%
\begin{pgfscope}%
\pgfpathrectangle{\pgfqpoint{1.228750in}{1.022500in}}{\pgfqpoint{13.456250in}{3.662500in}}%
\pgfusepath{clip}%
\pgfsetroundcap%
\pgfsetroundjoin%
\pgfsetlinewidth{1.003750pt}%
\definecolor{currentstroke}{rgb}{0.800000,0.800000,0.800000}%
\pgfsetstrokecolor{currentstroke}%
\pgfsetdash{}{0pt}%
\pgfpathmoveto{\pgfqpoint{1.228750in}{2.105759in}}%
\pgfpathlineto{\pgfqpoint{14.685000in}{2.105759in}}%
\pgfusepath{stroke}%
\end{pgfscope}%
\begin{pgfscope}%
\definecolor{textcolor}{rgb}{0.150000,0.150000,0.150000}%
\pgfsetstrokecolor{textcolor}%
\pgfsetfillcolor{textcolor}%
\pgftext[x=0.690481in,y=2.005740in,left,base]{\color{textcolor}\sffamily\fontsize{19.250000}{23.100000}\selectfont 200}%
\end{pgfscope}%
\begin{pgfscope}%
\pgfpathrectangle{\pgfqpoint{1.228750in}{1.022500in}}{\pgfqpoint{13.456250in}{3.662500in}}%
\pgfusepath{clip}%
\pgfsetroundcap%
\pgfsetroundjoin%
\pgfsetlinewidth{1.003750pt}%
\definecolor{currentstroke}{rgb}{0.800000,0.800000,0.800000}%
\pgfsetstrokecolor{currentstroke}%
\pgfsetdash{}{0pt}%
\pgfpathmoveto{\pgfqpoint{1.228750in}{3.189019in}}%
\pgfpathlineto{\pgfqpoint{14.685000in}{3.189019in}}%
\pgfusepath{stroke}%
\end{pgfscope}%
\begin{pgfscope}%
\definecolor{textcolor}{rgb}{0.150000,0.150000,0.150000}%
\pgfsetstrokecolor{textcolor}%
\pgfsetfillcolor{textcolor}%
\pgftext[x=0.690481in,y=3.089000in,left,base]{\color{textcolor}\sffamily\fontsize{19.250000}{23.100000}\selectfont 400}%
\end{pgfscope}%
\begin{pgfscope}%
\pgfpathrectangle{\pgfqpoint{1.228750in}{1.022500in}}{\pgfqpoint{13.456250in}{3.662500in}}%
\pgfusepath{clip}%
\pgfsetroundcap%
\pgfsetroundjoin%
\pgfsetlinewidth{1.003750pt}%
\definecolor{currentstroke}{rgb}{0.800000,0.800000,0.800000}%
\pgfsetstrokecolor{currentstroke}%
\pgfsetdash{}{0pt}%
\pgfpathmoveto{\pgfqpoint{1.228750in}{4.272278in}}%
\pgfpathlineto{\pgfqpoint{14.685000in}{4.272278in}}%
\pgfusepath{stroke}%
\end{pgfscope}%
\begin{pgfscope}%
\definecolor{textcolor}{rgb}{0.150000,0.150000,0.150000}%
\pgfsetstrokecolor{textcolor}%
\pgfsetfillcolor{textcolor}%
\pgftext[x=0.690481in,y=4.172259in,left,base]{\color{textcolor}\sffamily\fontsize{19.250000}{23.100000}\selectfont 600}%
\end{pgfscope}%
\begin{pgfscope}%
\definecolor{textcolor}{rgb}{0.150000,0.150000,0.150000}%
\pgfsetstrokecolor{textcolor}%
\pgfsetfillcolor{textcolor}%
\pgftext[x=0.634925in,y=2.853750in,,bottom,rotate=90.000000]{\color{textcolor}\sffamily\fontsize{21.000000}{25.200000}\selectfont Cycles}%
\end{pgfscope}%
\begin{pgfscope}%
\pgfpathrectangle{\pgfqpoint{1.228750in}{1.022500in}}{\pgfqpoint{13.456250in}{3.662500in}}%
\pgfusepath{clip}%
\pgfsetbuttcap%
\pgfsetmiterjoin%
\definecolor{currentfill}{rgb}{0.347059,0.458824,0.641176}%
\pgfsetfillcolor{currentfill}%
\pgfsetlinewidth{1.003750pt}%
\definecolor{currentstroke}{rgb}{1.000000,1.000000,1.000000}%
\pgfsetstrokecolor{currentstroke}%
\pgfsetdash{}{0pt}%
\pgfpathmoveto{\pgfqpoint{1.420982in}{1.022500in}}%
\pgfpathlineto{\pgfqpoint{1.933601in}{1.022500in}}%
\pgfpathlineto{\pgfqpoint{1.933601in}{1.650790in}}%
\pgfpathlineto{\pgfqpoint{1.420982in}{1.650790in}}%
\pgfpathclose%
\pgfusepath{stroke,fill}%
\end{pgfscope}%
\begin{pgfscope}%
\pgfsetbuttcap%
\pgfsetmiterjoin%
\definecolor{currentfill}{rgb}{0.347059,0.458824,0.641176}%
\pgfsetfillcolor{currentfill}%
\pgfsetlinewidth{1.003750pt}%
\definecolor{currentstroke}{rgb}{1.000000,1.000000,1.000000}%
\pgfsetstrokecolor{currentstroke}%
\pgfsetdash{}{0pt}%
\pgfpathrectangle{\pgfqpoint{1.228750in}{1.022500in}}{\pgfqpoint{13.456250in}{3.662500in}}%
\pgfusepath{clip}%
\pgfpathmoveto{\pgfqpoint{1.420982in}{1.022500in}}%
\pgfpathlineto{\pgfqpoint{1.933601in}{1.022500in}}%
\pgfpathlineto{\pgfqpoint{1.933601in}{1.650790in}}%
\pgfpathlineto{\pgfqpoint{1.420982in}{1.650790in}}%
\pgfpathclose%
\pgfusepath{clip}%
\pgfsys@defobject{currentpattern}{\pgfqpoint{0in}{0in}}{\pgfqpoint{1in}{1in}}{%
\begin{pgfscope}%
\pgfpathrectangle{\pgfqpoint{0in}{0in}}{\pgfqpoint{1in}{1in}}%
\pgfusepath{clip}%
\pgfpathmoveto{\pgfqpoint{-0.500000in}{0.500000in}}%
\pgfpathlineto{\pgfqpoint{0.500000in}{1.500000in}}%
\pgfpathmoveto{\pgfqpoint{-0.444444in}{0.444444in}}%
\pgfpathlineto{\pgfqpoint{0.555556in}{1.444444in}}%
\pgfpathmoveto{\pgfqpoint{-0.388889in}{0.388889in}}%
\pgfpathlineto{\pgfqpoint{0.611111in}{1.388889in}}%
\pgfpathmoveto{\pgfqpoint{-0.333333in}{0.333333in}}%
\pgfpathlineto{\pgfqpoint{0.666667in}{1.333333in}}%
\pgfpathmoveto{\pgfqpoint{-0.277778in}{0.277778in}}%
\pgfpathlineto{\pgfqpoint{0.722222in}{1.277778in}}%
\pgfpathmoveto{\pgfqpoint{-0.222222in}{0.222222in}}%
\pgfpathlineto{\pgfqpoint{0.777778in}{1.222222in}}%
\pgfpathmoveto{\pgfqpoint{-0.166667in}{0.166667in}}%
\pgfpathlineto{\pgfqpoint{0.833333in}{1.166667in}}%
\pgfpathmoveto{\pgfqpoint{-0.111111in}{0.111111in}}%
\pgfpathlineto{\pgfqpoint{0.888889in}{1.111111in}}%
\pgfpathmoveto{\pgfqpoint{-0.055556in}{0.055556in}}%
\pgfpathlineto{\pgfqpoint{0.944444in}{1.055556in}}%
\pgfpathmoveto{\pgfqpoint{0.000000in}{0.000000in}}%
\pgfpathlineto{\pgfqpoint{1.000000in}{1.000000in}}%
\pgfpathmoveto{\pgfqpoint{0.055556in}{-0.055556in}}%
\pgfpathlineto{\pgfqpoint{1.055556in}{0.944444in}}%
\pgfpathmoveto{\pgfqpoint{0.111111in}{-0.111111in}}%
\pgfpathlineto{\pgfqpoint{1.111111in}{0.888889in}}%
\pgfpathmoveto{\pgfqpoint{0.166667in}{-0.166667in}}%
\pgfpathlineto{\pgfqpoint{1.166667in}{0.833333in}}%
\pgfpathmoveto{\pgfqpoint{0.222222in}{-0.222222in}}%
\pgfpathlineto{\pgfqpoint{1.222222in}{0.777778in}}%
\pgfpathmoveto{\pgfqpoint{0.277778in}{-0.277778in}}%
\pgfpathlineto{\pgfqpoint{1.277778in}{0.722222in}}%
\pgfpathmoveto{\pgfqpoint{0.333333in}{-0.333333in}}%
\pgfpathlineto{\pgfqpoint{1.333333in}{0.666667in}}%
\pgfpathmoveto{\pgfqpoint{0.388889in}{-0.388889in}}%
\pgfpathlineto{\pgfqpoint{1.388889in}{0.611111in}}%
\pgfpathmoveto{\pgfqpoint{0.444444in}{-0.444444in}}%
\pgfpathlineto{\pgfqpoint{1.444444in}{0.555556in}}%
\pgfpathmoveto{\pgfqpoint{0.500000in}{-0.500000in}}%
\pgfpathlineto{\pgfqpoint{1.500000in}{0.500000in}}%
\pgfusepath{stroke}%
\end{pgfscope}%
}%
\pgfsys@transformshift{1.420982in}{1.022500in}%
\pgfsys@useobject{currentpattern}{}%
\pgfsys@transformshift{1in}{0in}%
\pgfsys@transformshift{-1in}{0in}%
\pgfsys@transformshift{0in}{1in}%
\end{pgfscope}%
\begin{pgfscope}%
\pgfpathrectangle{\pgfqpoint{1.228750in}{1.022500in}}{\pgfqpoint{13.456250in}{3.662500in}}%
\pgfusepath{clip}%
\pgfsetbuttcap%
\pgfsetmiterjoin%
\definecolor{currentfill}{rgb}{0.347059,0.458824,0.641176}%
\pgfsetfillcolor{currentfill}%
\pgfsetlinewidth{1.003750pt}%
\definecolor{currentstroke}{rgb}{1.000000,1.000000,1.000000}%
\pgfsetstrokecolor{currentstroke}%
\pgfsetdash{}{0pt}%
\pgfpathmoveto{\pgfqpoint{3.343304in}{1.022500in}}%
\pgfpathlineto{\pgfqpoint{3.855923in}{1.022500in}}%
\pgfpathlineto{\pgfqpoint{3.855923in}{1.607460in}}%
\pgfpathlineto{\pgfqpoint{3.343304in}{1.607460in}}%
\pgfpathclose%
\pgfusepath{stroke,fill}%
\end{pgfscope}%
\begin{pgfscope}%
\pgfsetbuttcap%
\pgfsetmiterjoin%
\definecolor{currentfill}{rgb}{0.347059,0.458824,0.641176}%
\pgfsetfillcolor{currentfill}%
\pgfsetlinewidth{1.003750pt}%
\definecolor{currentstroke}{rgb}{1.000000,1.000000,1.000000}%
\pgfsetstrokecolor{currentstroke}%
\pgfsetdash{}{0pt}%
\pgfpathrectangle{\pgfqpoint{1.228750in}{1.022500in}}{\pgfqpoint{13.456250in}{3.662500in}}%
\pgfusepath{clip}%
\pgfpathmoveto{\pgfqpoint{3.343304in}{1.022500in}}%
\pgfpathlineto{\pgfqpoint{3.855923in}{1.022500in}}%
\pgfpathlineto{\pgfqpoint{3.855923in}{1.607460in}}%
\pgfpathlineto{\pgfqpoint{3.343304in}{1.607460in}}%
\pgfpathclose%
\pgfusepath{clip}%
\pgfsys@defobject{currentpattern}{\pgfqpoint{0in}{0in}}{\pgfqpoint{1in}{1in}}{%
\begin{pgfscope}%
\pgfpathrectangle{\pgfqpoint{0in}{0in}}{\pgfqpoint{1in}{1in}}%
\pgfusepath{clip}%
\pgfpathmoveto{\pgfqpoint{-0.500000in}{0.500000in}}%
\pgfpathlineto{\pgfqpoint{0.500000in}{1.500000in}}%
\pgfpathmoveto{\pgfqpoint{-0.444444in}{0.444444in}}%
\pgfpathlineto{\pgfqpoint{0.555556in}{1.444444in}}%
\pgfpathmoveto{\pgfqpoint{-0.388889in}{0.388889in}}%
\pgfpathlineto{\pgfqpoint{0.611111in}{1.388889in}}%
\pgfpathmoveto{\pgfqpoint{-0.333333in}{0.333333in}}%
\pgfpathlineto{\pgfqpoint{0.666667in}{1.333333in}}%
\pgfpathmoveto{\pgfqpoint{-0.277778in}{0.277778in}}%
\pgfpathlineto{\pgfqpoint{0.722222in}{1.277778in}}%
\pgfpathmoveto{\pgfqpoint{-0.222222in}{0.222222in}}%
\pgfpathlineto{\pgfqpoint{0.777778in}{1.222222in}}%
\pgfpathmoveto{\pgfqpoint{-0.166667in}{0.166667in}}%
\pgfpathlineto{\pgfqpoint{0.833333in}{1.166667in}}%
\pgfpathmoveto{\pgfqpoint{-0.111111in}{0.111111in}}%
\pgfpathlineto{\pgfqpoint{0.888889in}{1.111111in}}%
\pgfpathmoveto{\pgfqpoint{-0.055556in}{0.055556in}}%
\pgfpathlineto{\pgfqpoint{0.944444in}{1.055556in}}%
\pgfpathmoveto{\pgfqpoint{0.000000in}{0.000000in}}%
\pgfpathlineto{\pgfqpoint{1.000000in}{1.000000in}}%
\pgfpathmoveto{\pgfqpoint{0.055556in}{-0.055556in}}%
\pgfpathlineto{\pgfqpoint{1.055556in}{0.944444in}}%
\pgfpathmoveto{\pgfqpoint{0.111111in}{-0.111111in}}%
\pgfpathlineto{\pgfqpoint{1.111111in}{0.888889in}}%
\pgfpathmoveto{\pgfqpoint{0.166667in}{-0.166667in}}%
\pgfpathlineto{\pgfqpoint{1.166667in}{0.833333in}}%
\pgfpathmoveto{\pgfqpoint{0.222222in}{-0.222222in}}%
\pgfpathlineto{\pgfqpoint{1.222222in}{0.777778in}}%
\pgfpathmoveto{\pgfqpoint{0.277778in}{-0.277778in}}%
\pgfpathlineto{\pgfqpoint{1.277778in}{0.722222in}}%
\pgfpathmoveto{\pgfqpoint{0.333333in}{-0.333333in}}%
\pgfpathlineto{\pgfqpoint{1.333333in}{0.666667in}}%
\pgfpathmoveto{\pgfqpoint{0.388889in}{-0.388889in}}%
\pgfpathlineto{\pgfqpoint{1.388889in}{0.611111in}}%
\pgfpathmoveto{\pgfqpoint{0.444444in}{-0.444444in}}%
\pgfpathlineto{\pgfqpoint{1.444444in}{0.555556in}}%
\pgfpathmoveto{\pgfqpoint{0.500000in}{-0.500000in}}%
\pgfpathlineto{\pgfqpoint{1.500000in}{0.500000in}}%
\pgfusepath{stroke}%
\end{pgfscope}%
}%
\pgfsys@transformshift{3.343304in}{1.022500in}%
\pgfsys@useobject{currentpattern}{}%
\pgfsys@transformshift{1in}{0in}%
\pgfsys@transformshift{-1in}{0in}%
\pgfsys@transformshift{0in}{1in}%
\end{pgfscope}%
\begin{pgfscope}%
\pgfpathrectangle{\pgfqpoint{1.228750in}{1.022500in}}{\pgfqpoint{13.456250in}{3.662500in}}%
\pgfusepath{clip}%
\pgfsetbuttcap%
\pgfsetmiterjoin%
\definecolor{currentfill}{rgb}{0.347059,0.458824,0.641176}%
\pgfsetfillcolor{currentfill}%
\pgfsetlinewidth{1.003750pt}%
\definecolor{currentstroke}{rgb}{1.000000,1.000000,1.000000}%
\pgfsetstrokecolor{currentstroke}%
\pgfsetdash{}{0pt}%
\pgfpathmoveto{\pgfqpoint{5.265625in}{1.022500in}}%
\pgfpathlineto{\pgfqpoint{5.778244in}{1.022500in}}%
\pgfpathlineto{\pgfqpoint{5.778244in}{1.672456in}}%
\pgfpathlineto{\pgfqpoint{5.265625in}{1.672456in}}%
\pgfpathclose%
\pgfusepath{stroke,fill}%
\end{pgfscope}%
\begin{pgfscope}%
\pgfsetbuttcap%
\pgfsetmiterjoin%
\definecolor{currentfill}{rgb}{0.347059,0.458824,0.641176}%
\pgfsetfillcolor{currentfill}%
\pgfsetlinewidth{1.003750pt}%
\definecolor{currentstroke}{rgb}{1.000000,1.000000,1.000000}%
\pgfsetstrokecolor{currentstroke}%
\pgfsetdash{}{0pt}%
\pgfpathrectangle{\pgfqpoint{1.228750in}{1.022500in}}{\pgfqpoint{13.456250in}{3.662500in}}%
\pgfusepath{clip}%
\pgfpathmoveto{\pgfqpoint{5.265625in}{1.022500in}}%
\pgfpathlineto{\pgfqpoint{5.778244in}{1.022500in}}%
\pgfpathlineto{\pgfqpoint{5.778244in}{1.672456in}}%
\pgfpathlineto{\pgfqpoint{5.265625in}{1.672456in}}%
\pgfpathclose%
\pgfusepath{clip}%
\pgfsys@defobject{currentpattern}{\pgfqpoint{0in}{0in}}{\pgfqpoint{1in}{1in}}{%
\begin{pgfscope}%
\pgfpathrectangle{\pgfqpoint{0in}{0in}}{\pgfqpoint{1in}{1in}}%
\pgfusepath{clip}%
\pgfpathmoveto{\pgfqpoint{-0.500000in}{0.500000in}}%
\pgfpathlineto{\pgfqpoint{0.500000in}{1.500000in}}%
\pgfpathmoveto{\pgfqpoint{-0.444444in}{0.444444in}}%
\pgfpathlineto{\pgfqpoint{0.555556in}{1.444444in}}%
\pgfpathmoveto{\pgfqpoint{-0.388889in}{0.388889in}}%
\pgfpathlineto{\pgfqpoint{0.611111in}{1.388889in}}%
\pgfpathmoveto{\pgfqpoint{-0.333333in}{0.333333in}}%
\pgfpathlineto{\pgfqpoint{0.666667in}{1.333333in}}%
\pgfpathmoveto{\pgfqpoint{-0.277778in}{0.277778in}}%
\pgfpathlineto{\pgfqpoint{0.722222in}{1.277778in}}%
\pgfpathmoveto{\pgfqpoint{-0.222222in}{0.222222in}}%
\pgfpathlineto{\pgfqpoint{0.777778in}{1.222222in}}%
\pgfpathmoveto{\pgfqpoint{-0.166667in}{0.166667in}}%
\pgfpathlineto{\pgfqpoint{0.833333in}{1.166667in}}%
\pgfpathmoveto{\pgfqpoint{-0.111111in}{0.111111in}}%
\pgfpathlineto{\pgfqpoint{0.888889in}{1.111111in}}%
\pgfpathmoveto{\pgfqpoint{-0.055556in}{0.055556in}}%
\pgfpathlineto{\pgfqpoint{0.944444in}{1.055556in}}%
\pgfpathmoveto{\pgfqpoint{0.000000in}{0.000000in}}%
\pgfpathlineto{\pgfqpoint{1.000000in}{1.000000in}}%
\pgfpathmoveto{\pgfqpoint{0.055556in}{-0.055556in}}%
\pgfpathlineto{\pgfqpoint{1.055556in}{0.944444in}}%
\pgfpathmoveto{\pgfqpoint{0.111111in}{-0.111111in}}%
\pgfpathlineto{\pgfqpoint{1.111111in}{0.888889in}}%
\pgfpathmoveto{\pgfqpoint{0.166667in}{-0.166667in}}%
\pgfpathlineto{\pgfqpoint{1.166667in}{0.833333in}}%
\pgfpathmoveto{\pgfqpoint{0.222222in}{-0.222222in}}%
\pgfpathlineto{\pgfqpoint{1.222222in}{0.777778in}}%
\pgfpathmoveto{\pgfqpoint{0.277778in}{-0.277778in}}%
\pgfpathlineto{\pgfqpoint{1.277778in}{0.722222in}}%
\pgfpathmoveto{\pgfqpoint{0.333333in}{-0.333333in}}%
\pgfpathlineto{\pgfqpoint{1.333333in}{0.666667in}}%
\pgfpathmoveto{\pgfqpoint{0.388889in}{-0.388889in}}%
\pgfpathlineto{\pgfqpoint{1.388889in}{0.611111in}}%
\pgfpathmoveto{\pgfqpoint{0.444444in}{-0.444444in}}%
\pgfpathlineto{\pgfqpoint{1.444444in}{0.555556in}}%
\pgfpathmoveto{\pgfqpoint{0.500000in}{-0.500000in}}%
\pgfpathlineto{\pgfqpoint{1.500000in}{0.500000in}}%
\pgfusepath{stroke}%
\end{pgfscope}%
}%
\pgfsys@transformshift{5.265625in}{1.022500in}%
\pgfsys@useobject{currentpattern}{}%
\pgfsys@transformshift{1in}{0in}%
\pgfsys@transformshift{-1in}{0in}%
\pgfsys@transformshift{0in}{1in}%
\end{pgfscope}%
\begin{pgfscope}%
\pgfpathrectangle{\pgfqpoint{1.228750in}{1.022500in}}{\pgfqpoint{13.456250in}{3.662500in}}%
\pgfusepath{clip}%
\pgfsetbuttcap%
\pgfsetmiterjoin%
\definecolor{currentfill}{rgb}{0.347059,0.458824,0.641176}%
\pgfsetfillcolor{currentfill}%
\pgfsetlinewidth{1.003750pt}%
\definecolor{currentstroke}{rgb}{1.000000,1.000000,1.000000}%
\pgfsetstrokecolor{currentstroke}%
\pgfsetdash{}{0pt}%
\pgfpathmoveto{\pgfqpoint{7.187946in}{1.022500in}}%
\pgfpathlineto{\pgfqpoint{7.700565in}{1.022500in}}%
\pgfpathlineto{\pgfqpoint{7.700565in}{1.721202in}}%
\pgfpathlineto{\pgfqpoint{7.187946in}{1.721202in}}%
\pgfpathclose%
\pgfusepath{stroke,fill}%
\end{pgfscope}%
\begin{pgfscope}%
\pgfsetbuttcap%
\pgfsetmiterjoin%
\definecolor{currentfill}{rgb}{0.347059,0.458824,0.641176}%
\pgfsetfillcolor{currentfill}%
\pgfsetlinewidth{1.003750pt}%
\definecolor{currentstroke}{rgb}{1.000000,1.000000,1.000000}%
\pgfsetstrokecolor{currentstroke}%
\pgfsetdash{}{0pt}%
\pgfpathrectangle{\pgfqpoint{1.228750in}{1.022500in}}{\pgfqpoint{13.456250in}{3.662500in}}%
\pgfusepath{clip}%
\pgfpathmoveto{\pgfqpoint{7.187946in}{1.022500in}}%
\pgfpathlineto{\pgfqpoint{7.700565in}{1.022500in}}%
\pgfpathlineto{\pgfqpoint{7.700565in}{1.721202in}}%
\pgfpathlineto{\pgfqpoint{7.187946in}{1.721202in}}%
\pgfpathclose%
\pgfusepath{clip}%
\pgfsys@defobject{currentpattern}{\pgfqpoint{0in}{0in}}{\pgfqpoint{1in}{1in}}{%
\begin{pgfscope}%
\pgfpathrectangle{\pgfqpoint{0in}{0in}}{\pgfqpoint{1in}{1in}}%
\pgfusepath{clip}%
\pgfpathmoveto{\pgfqpoint{-0.500000in}{0.500000in}}%
\pgfpathlineto{\pgfqpoint{0.500000in}{1.500000in}}%
\pgfpathmoveto{\pgfqpoint{-0.444444in}{0.444444in}}%
\pgfpathlineto{\pgfqpoint{0.555556in}{1.444444in}}%
\pgfpathmoveto{\pgfqpoint{-0.388889in}{0.388889in}}%
\pgfpathlineto{\pgfqpoint{0.611111in}{1.388889in}}%
\pgfpathmoveto{\pgfqpoint{-0.333333in}{0.333333in}}%
\pgfpathlineto{\pgfqpoint{0.666667in}{1.333333in}}%
\pgfpathmoveto{\pgfqpoint{-0.277778in}{0.277778in}}%
\pgfpathlineto{\pgfqpoint{0.722222in}{1.277778in}}%
\pgfpathmoveto{\pgfqpoint{-0.222222in}{0.222222in}}%
\pgfpathlineto{\pgfqpoint{0.777778in}{1.222222in}}%
\pgfpathmoveto{\pgfqpoint{-0.166667in}{0.166667in}}%
\pgfpathlineto{\pgfqpoint{0.833333in}{1.166667in}}%
\pgfpathmoveto{\pgfqpoint{-0.111111in}{0.111111in}}%
\pgfpathlineto{\pgfqpoint{0.888889in}{1.111111in}}%
\pgfpathmoveto{\pgfqpoint{-0.055556in}{0.055556in}}%
\pgfpathlineto{\pgfqpoint{0.944444in}{1.055556in}}%
\pgfpathmoveto{\pgfqpoint{0.000000in}{0.000000in}}%
\pgfpathlineto{\pgfqpoint{1.000000in}{1.000000in}}%
\pgfpathmoveto{\pgfqpoint{0.055556in}{-0.055556in}}%
\pgfpathlineto{\pgfqpoint{1.055556in}{0.944444in}}%
\pgfpathmoveto{\pgfqpoint{0.111111in}{-0.111111in}}%
\pgfpathlineto{\pgfqpoint{1.111111in}{0.888889in}}%
\pgfpathmoveto{\pgfqpoint{0.166667in}{-0.166667in}}%
\pgfpathlineto{\pgfqpoint{1.166667in}{0.833333in}}%
\pgfpathmoveto{\pgfqpoint{0.222222in}{-0.222222in}}%
\pgfpathlineto{\pgfqpoint{1.222222in}{0.777778in}}%
\pgfpathmoveto{\pgfqpoint{0.277778in}{-0.277778in}}%
\pgfpathlineto{\pgfqpoint{1.277778in}{0.722222in}}%
\pgfpathmoveto{\pgfqpoint{0.333333in}{-0.333333in}}%
\pgfpathlineto{\pgfqpoint{1.333333in}{0.666667in}}%
\pgfpathmoveto{\pgfqpoint{0.388889in}{-0.388889in}}%
\pgfpathlineto{\pgfqpoint{1.388889in}{0.611111in}}%
\pgfpathmoveto{\pgfqpoint{0.444444in}{-0.444444in}}%
\pgfpathlineto{\pgfqpoint{1.444444in}{0.555556in}}%
\pgfpathmoveto{\pgfqpoint{0.500000in}{-0.500000in}}%
\pgfpathlineto{\pgfqpoint{1.500000in}{0.500000in}}%
\pgfusepath{stroke}%
\end{pgfscope}%
}%
\pgfsys@transformshift{7.187946in}{1.022500in}%
\pgfsys@useobject{currentpattern}{}%
\pgfsys@transformshift{1in}{0in}%
\pgfsys@transformshift{-1in}{0in}%
\pgfsys@transformshift{0in}{1in}%
\end{pgfscope}%
\begin{pgfscope}%
\pgfpathrectangle{\pgfqpoint{1.228750in}{1.022500in}}{\pgfqpoint{13.456250in}{3.662500in}}%
\pgfusepath{clip}%
\pgfsetbuttcap%
\pgfsetmiterjoin%
\definecolor{currentfill}{rgb}{0.347059,0.458824,0.641176}%
\pgfsetfillcolor{currentfill}%
\pgfsetlinewidth{1.003750pt}%
\definecolor{currentstroke}{rgb}{1.000000,1.000000,1.000000}%
\pgfsetstrokecolor{currentstroke}%
\pgfsetdash{}{0pt}%
\pgfpathmoveto{\pgfqpoint{9.110268in}{1.022500in}}%
\pgfpathlineto{\pgfqpoint{9.622887in}{1.022500in}}%
\pgfpathlineto{\pgfqpoint{9.622887in}{1.607460in}}%
\pgfpathlineto{\pgfqpoint{9.110268in}{1.607460in}}%
\pgfpathclose%
\pgfusepath{stroke,fill}%
\end{pgfscope}%
\begin{pgfscope}%
\pgfsetbuttcap%
\pgfsetmiterjoin%
\definecolor{currentfill}{rgb}{0.347059,0.458824,0.641176}%
\pgfsetfillcolor{currentfill}%
\pgfsetlinewidth{1.003750pt}%
\definecolor{currentstroke}{rgb}{1.000000,1.000000,1.000000}%
\pgfsetstrokecolor{currentstroke}%
\pgfsetdash{}{0pt}%
\pgfpathrectangle{\pgfqpoint{1.228750in}{1.022500in}}{\pgfqpoint{13.456250in}{3.662500in}}%
\pgfusepath{clip}%
\pgfpathmoveto{\pgfqpoint{9.110268in}{1.022500in}}%
\pgfpathlineto{\pgfqpoint{9.622887in}{1.022500in}}%
\pgfpathlineto{\pgfqpoint{9.622887in}{1.607460in}}%
\pgfpathlineto{\pgfqpoint{9.110268in}{1.607460in}}%
\pgfpathclose%
\pgfusepath{clip}%
\pgfsys@defobject{currentpattern}{\pgfqpoint{0in}{0in}}{\pgfqpoint{1in}{1in}}{%
\begin{pgfscope}%
\pgfpathrectangle{\pgfqpoint{0in}{0in}}{\pgfqpoint{1in}{1in}}%
\pgfusepath{clip}%
\pgfpathmoveto{\pgfqpoint{-0.500000in}{0.500000in}}%
\pgfpathlineto{\pgfqpoint{0.500000in}{1.500000in}}%
\pgfpathmoveto{\pgfqpoint{-0.444444in}{0.444444in}}%
\pgfpathlineto{\pgfqpoint{0.555556in}{1.444444in}}%
\pgfpathmoveto{\pgfqpoint{-0.388889in}{0.388889in}}%
\pgfpathlineto{\pgfqpoint{0.611111in}{1.388889in}}%
\pgfpathmoveto{\pgfqpoint{-0.333333in}{0.333333in}}%
\pgfpathlineto{\pgfqpoint{0.666667in}{1.333333in}}%
\pgfpathmoveto{\pgfqpoint{-0.277778in}{0.277778in}}%
\pgfpathlineto{\pgfqpoint{0.722222in}{1.277778in}}%
\pgfpathmoveto{\pgfqpoint{-0.222222in}{0.222222in}}%
\pgfpathlineto{\pgfqpoint{0.777778in}{1.222222in}}%
\pgfpathmoveto{\pgfqpoint{-0.166667in}{0.166667in}}%
\pgfpathlineto{\pgfqpoint{0.833333in}{1.166667in}}%
\pgfpathmoveto{\pgfqpoint{-0.111111in}{0.111111in}}%
\pgfpathlineto{\pgfqpoint{0.888889in}{1.111111in}}%
\pgfpathmoveto{\pgfqpoint{-0.055556in}{0.055556in}}%
\pgfpathlineto{\pgfqpoint{0.944444in}{1.055556in}}%
\pgfpathmoveto{\pgfqpoint{0.000000in}{0.000000in}}%
\pgfpathlineto{\pgfqpoint{1.000000in}{1.000000in}}%
\pgfpathmoveto{\pgfqpoint{0.055556in}{-0.055556in}}%
\pgfpathlineto{\pgfqpoint{1.055556in}{0.944444in}}%
\pgfpathmoveto{\pgfqpoint{0.111111in}{-0.111111in}}%
\pgfpathlineto{\pgfqpoint{1.111111in}{0.888889in}}%
\pgfpathmoveto{\pgfqpoint{0.166667in}{-0.166667in}}%
\pgfpathlineto{\pgfqpoint{1.166667in}{0.833333in}}%
\pgfpathmoveto{\pgfqpoint{0.222222in}{-0.222222in}}%
\pgfpathlineto{\pgfqpoint{1.222222in}{0.777778in}}%
\pgfpathmoveto{\pgfqpoint{0.277778in}{-0.277778in}}%
\pgfpathlineto{\pgfqpoint{1.277778in}{0.722222in}}%
\pgfpathmoveto{\pgfqpoint{0.333333in}{-0.333333in}}%
\pgfpathlineto{\pgfqpoint{1.333333in}{0.666667in}}%
\pgfpathmoveto{\pgfqpoint{0.388889in}{-0.388889in}}%
\pgfpathlineto{\pgfqpoint{1.388889in}{0.611111in}}%
\pgfpathmoveto{\pgfqpoint{0.444444in}{-0.444444in}}%
\pgfpathlineto{\pgfqpoint{1.444444in}{0.555556in}}%
\pgfpathmoveto{\pgfqpoint{0.500000in}{-0.500000in}}%
\pgfpathlineto{\pgfqpoint{1.500000in}{0.500000in}}%
\pgfusepath{stroke}%
\end{pgfscope}%
}%
\pgfsys@transformshift{9.110268in}{1.022500in}%
\pgfsys@useobject{currentpattern}{}%
\pgfsys@transformshift{1in}{0in}%
\pgfsys@transformshift{-1in}{0in}%
\pgfsys@transformshift{0in}{1in}%
\end{pgfscope}%
\begin{pgfscope}%
\pgfpathrectangle{\pgfqpoint{1.228750in}{1.022500in}}{\pgfqpoint{13.456250in}{3.662500in}}%
\pgfusepath{clip}%
\pgfsetbuttcap%
\pgfsetmiterjoin%
\definecolor{currentfill}{rgb}{0.347059,0.458824,0.641176}%
\pgfsetfillcolor{currentfill}%
\pgfsetlinewidth{1.003750pt}%
\definecolor{currentstroke}{rgb}{1.000000,1.000000,1.000000}%
\pgfsetstrokecolor{currentstroke}%
\pgfsetdash{}{0pt}%
\pgfpathmoveto{\pgfqpoint{11.032589in}{1.022500in}}%
\pgfpathlineto{\pgfqpoint{11.545208in}{1.022500in}}%
\pgfpathlineto{\pgfqpoint{11.545208in}{1.434139in}}%
\pgfpathlineto{\pgfqpoint{11.032589in}{1.434139in}}%
\pgfpathclose%
\pgfusepath{stroke,fill}%
\end{pgfscope}%
\begin{pgfscope}%
\pgfsetbuttcap%
\pgfsetmiterjoin%
\definecolor{currentfill}{rgb}{0.347059,0.458824,0.641176}%
\pgfsetfillcolor{currentfill}%
\pgfsetlinewidth{1.003750pt}%
\definecolor{currentstroke}{rgb}{1.000000,1.000000,1.000000}%
\pgfsetstrokecolor{currentstroke}%
\pgfsetdash{}{0pt}%
\pgfpathrectangle{\pgfqpoint{1.228750in}{1.022500in}}{\pgfqpoint{13.456250in}{3.662500in}}%
\pgfusepath{clip}%
\pgfpathmoveto{\pgfqpoint{11.032589in}{1.022500in}}%
\pgfpathlineto{\pgfqpoint{11.545208in}{1.022500in}}%
\pgfpathlineto{\pgfqpoint{11.545208in}{1.434139in}}%
\pgfpathlineto{\pgfqpoint{11.032589in}{1.434139in}}%
\pgfpathclose%
\pgfusepath{clip}%
\pgfsys@defobject{currentpattern}{\pgfqpoint{0in}{0in}}{\pgfqpoint{1in}{1in}}{%
\begin{pgfscope}%
\pgfpathrectangle{\pgfqpoint{0in}{0in}}{\pgfqpoint{1in}{1in}}%
\pgfusepath{clip}%
\pgfpathmoveto{\pgfqpoint{-0.500000in}{0.500000in}}%
\pgfpathlineto{\pgfqpoint{0.500000in}{1.500000in}}%
\pgfpathmoveto{\pgfqpoint{-0.444444in}{0.444444in}}%
\pgfpathlineto{\pgfqpoint{0.555556in}{1.444444in}}%
\pgfpathmoveto{\pgfqpoint{-0.388889in}{0.388889in}}%
\pgfpathlineto{\pgfqpoint{0.611111in}{1.388889in}}%
\pgfpathmoveto{\pgfqpoint{-0.333333in}{0.333333in}}%
\pgfpathlineto{\pgfqpoint{0.666667in}{1.333333in}}%
\pgfpathmoveto{\pgfqpoint{-0.277778in}{0.277778in}}%
\pgfpathlineto{\pgfqpoint{0.722222in}{1.277778in}}%
\pgfpathmoveto{\pgfqpoint{-0.222222in}{0.222222in}}%
\pgfpathlineto{\pgfqpoint{0.777778in}{1.222222in}}%
\pgfpathmoveto{\pgfqpoint{-0.166667in}{0.166667in}}%
\pgfpathlineto{\pgfqpoint{0.833333in}{1.166667in}}%
\pgfpathmoveto{\pgfqpoint{-0.111111in}{0.111111in}}%
\pgfpathlineto{\pgfqpoint{0.888889in}{1.111111in}}%
\pgfpathmoveto{\pgfqpoint{-0.055556in}{0.055556in}}%
\pgfpathlineto{\pgfqpoint{0.944444in}{1.055556in}}%
\pgfpathmoveto{\pgfqpoint{0.000000in}{0.000000in}}%
\pgfpathlineto{\pgfqpoint{1.000000in}{1.000000in}}%
\pgfpathmoveto{\pgfqpoint{0.055556in}{-0.055556in}}%
\pgfpathlineto{\pgfqpoint{1.055556in}{0.944444in}}%
\pgfpathmoveto{\pgfqpoint{0.111111in}{-0.111111in}}%
\pgfpathlineto{\pgfqpoint{1.111111in}{0.888889in}}%
\pgfpathmoveto{\pgfqpoint{0.166667in}{-0.166667in}}%
\pgfpathlineto{\pgfqpoint{1.166667in}{0.833333in}}%
\pgfpathmoveto{\pgfqpoint{0.222222in}{-0.222222in}}%
\pgfpathlineto{\pgfqpoint{1.222222in}{0.777778in}}%
\pgfpathmoveto{\pgfqpoint{0.277778in}{-0.277778in}}%
\pgfpathlineto{\pgfqpoint{1.277778in}{0.722222in}}%
\pgfpathmoveto{\pgfqpoint{0.333333in}{-0.333333in}}%
\pgfpathlineto{\pgfqpoint{1.333333in}{0.666667in}}%
\pgfpathmoveto{\pgfqpoint{0.388889in}{-0.388889in}}%
\pgfpathlineto{\pgfqpoint{1.388889in}{0.611111in}}%
\pgfpathmoveto{\pgfqpoint{0.444444in}{-0.444444in}}%
\pgfpathlineto{\pgfqpoint{1.444444in}{0.555556in}}%
\pgfpathmoveto{\pgfqpoint{0.500000in}{-0.500000in}}%
\pgfpathlineto{\pgfqpoint{1.500000in}{0.500000in}}%
\pgfusepath{stroke}%
\end{pgfscope}%
}%
\pgfsys@transformshift{11.032589in}{1.022500in}%
\pgfsys@useobject{currentpattern}{}%
\pgfsys@transformshift{1in}{0in}%
\pgfsys@transformshift{-1in}{0in}%
\pgfsys@transformshift{0in}{1in}%
\end{pgfscope}%
\begin{pgfscope}%
\pgfpathrectangle{\pgfqpoint{1.228750in}{1.022500in}}{\pgfqpoint{13.456250in}{3.662500in}}%
\pgfusepath{clip}%
\pgfsetbuttcap%
\pgfsetmiterjoin%
\definecolor{currentfill}{rgb}{0.347059,0.458824,0.641176}%
\pgfsetfillcolor{currentfill}%
\pgfsetlinewidth{1.003750pt}%
\definecolor{currentstroke}{rgb}{1.000000,1.000000,1.000000}%
\pgfsetstrokecolor{currentstroke}%
\pgfsetdash{}{0pt}%
\pgfpathmoveto{\pgfqpoint{12.954911in}{1.022500in}}%
\pgfpathlineto{\pgfqpoint{13.467530in}{1.022500in}}%
\pgfpathlineto{\pgfqpoint{13.467530in}{1.434139in}}%
\pgfpathlineto{\pgfqpoint{12.954911in}{1.434139in}}%
\pgfpathclose%
\pgfusepath{stroke,fill}%
\end{pgfscope}%
\begin{pgfscope}%
\pgfsetbuttcap%
\pgfsetmiterjoin%
\definecolor{currentfill}{rgb}{0.347059,0.458824,0.641176}%
\pgfsetfillcolor{currentfill}%
\pgfsetlinewidth{1.003750pt}%
\definecolor{currentstroke}{rgb}{1.000000,1.000000,1.000000}%
\pgfsetstrokecolor{currentstroke}%
\pgfsetdash{}{0pt}%
\pgfpathrectangle{\pgfqpoint{1.228750in}{1.022500in}}{\pgfqpoint{13.456250in}{3.662500in}}%
\pgfusepath{clip}%
\pgfpathmoveto{\pgfqpoint{12.954911in}{1.022500in}}%
\pgfpathlineto{\pgfqpoint{13.467530in}{1.022500in}}%
\pgfpathlineto{\pgfqpoint{13.467530in}{1.434139in}}%
\pgfpathlineto{\pgfqpoint{12.954911in}{1.434139in}}%
\pgfpathclose%
\pgfusepath{clip}%
\pgfsys@defobject{currentpattern}{\pgfqpoint{0in}{0in}}{\pgfqpoint{1in}{1in}}{%
\begin{pgfscope}%
\pgfpathrectangle{\pgfqpoint{0in}{0in}}{\pgfqpoint{1in}{1in}}%
\pgfusepath{clip}%
\pgfpathmoveto{\pgfqpoint{-0.500000in}{0.500000in}}%
\pgfpathlineto{\pgfqpoint{0.500000in}{1.500000in}}%
\pgfpathmoveto{\pgfqpoint{-0.444444in}{0.444444in}}%
\pgfpathlineto{\pgfqpoint{0.555556in}{1.444444in}}%
\pgfpathmoveto{\pgfqpoint{-0.388889in}{0.388889in}}%
\pgfpathlineto{\pgfqpoint{0.611111in}{1.388889in}}%
\pgfpathmoveto{\pgfqpoint{-0.333333in}{0.333333in}}%
\pgfpathlineto{\pgfqpoint{0.666667in}{1.333333in}}%
\pgfpathmoveto{\pgfqpoint{-0.277778in}{0.277778in}}%
\pgfpathlineto{\pgfqpoint{0.722222in}{1.277778in}}%
\pgfpathmoveto{\pgfqpoint{-0.222222in}{0.222222in}}%
\pgfpathlineto{\pgfqpoint{0.777778in}{1.222222in}}%
\pgfpathmoveto{\pgfqpoint{-0.166667in}{0.166667in}}%
\pgfpathlineto{\pgfqpoint{0.833333in}{1.166667in}}%
\pgfpathmoveto{\pgfqpoint{-0.111111in}{0.111111in}}%
\pgfpathlineto{\pgfqpoint{0.888889in}{1.111111in}}%
\pgfpathmoveto{\pgfqpoint{-0.055556in}{0.055556in}}%
\pgfpathlineto{\pgfqpoint{0.944444in}{1.055556in}}%
\pgfpathmoveto{\pgfqpoint{0.000000in}{0.000000in}}%
\pgfpathlineto{\pgfqpoint{1.000000in}{1.000000in}}%
\pgfpathmoveto{\pgfqpoint{0.055556in}{-0.055556in}}%
\pgfpathlineto{\pgfqpoint{1.055556in}{0.944444in}}%
\pgfpathmoveto{\pgfqpoint{0.111111in}{-0.111111in}}%
\pgfpathlineto{\pgfqpoint{1.111111in}{0.888889in}}%
\pgfpathmoveto{\pgfqpoint{0.166667in}{-0.166667in}}%
\pgfpathlineto{\pgfqpoint{1.166667in}{0.833333in}}%
\pgfpathmoveto{\pgfqpoint{0.222222in}{-0.222222in}}%
\pgfpathlineto{\pgfqpoint{1.222222in}{0.777778in}}%
\pgfpathmoveto{\pgfqpoint{0.277778in}{-0.277778in}}%
\pgfpathlineto{\pgfqpoint{1.277778in}{0.722222in}}%
\pgfpathmoveto{\pgfqpoint{0.333333in}{-0.333333in}}%
\pgfpathlineto{\pgfqpoint{1.333333in}{0.666667in}}%
\pgfpathmoveto{\pgfqpoint{0.388889in}{-0.388889in}}%
\pgfpathlineto{\pgfqpoint{1.388889in}{0.611111in}}%
\pgfpathmoveto{\pgfqpoint{0.444444in}{-0.444444in}}%
\pgfpathlineto{\pgfqpoint{1.444444in}{0.555556in}}%
\pgfpathmoveto{\pgfqpoint{0.500000in}{-0.500000in}}%
\pgfpathlineto{\pgfqpoint{1.500000in}{0.500000in}}%
\pgfusepath{stroke}%
\end{pgfscope}%
}%
\pgfsys@transformshift{12.954911in}{1.022500in}%
\pgfsys@useobject{currentpattern}{}%
\pgfsys@transformshift{1in}{0in}%
\pgfsys@transformshift{-1in}{0in}%
\pgfsys@transformshift{0in}{1in}%
\end{pgfscope}%
\begin{pgfscope}%
\pgfpathrectangle{\pgfqpoint{1.228750in}{1.022500in}}{\pgfqpoint{13.456250in}{3.662500in}}%
\pgfusepath{clip}%
\pgfsetbuttcap%
\pgfsetmiterjoin%
\definecolor{currentfill}{rgb}{0.798529,0.536765,0.389706}%
\pgfsetfillcolor{currentfill}%
\pgfsetlinewidth{1.003750pt}%
\definecolor{currentstroke}{rgb}{1.000000,1.000000,1.000000}%
\pgfsetstrokecolor{currentstroke}%
\pgfsetdash{}{0pt}%
\pgfpathmoveto{\pgfqpoint{1.933601in}{1.022500in}}%
\pgfpathlineto{\pgfqpoint{2.446220in}{1.022500in}}%
\pgfpathlineto{\pgfqpoint{2.446220in}{1.342062in}}%
\pgfpathlineto{\pgfqpoint{1.933601in}{1.342062in}}%
\pgfpathclose%
\pgfusepath{stroke,fill}%
\end{pgfscope}%
\begin{pgfscope}%
\pgfsetbuttcap%
\pgfsetmiterjoin%
\definecolor{currentfill}{rgb}{0.798529,0.536765,0.389706}%
\pgfsetfillcolor{currentfill}%
\pgfsetlinewidth{1.003750pt}%
\definecolor{currentstroke}{rgb}{1.000000,1.000000,1.000000}%
\pgfsetstrokecolor{currentstroke}%
\pgfsetdash{}{0pt}%
\pgfpathrectangle{\pgfqpoint{1.228750in}{1.022500in}}{\pgfqpoint{13.456250in}{3.662500in}}%
\pgfusepath{clip}%
\pgfpathmoveto{\pgfqpoint{1.933601in}{1.022500in}}%
\pgfpathlineto{\pgfqpoint{2.446220in}{1.022500in}}%
\pgfpathlineto{\pgfqpoint{2.446220in}{1.342062in}}%
\pgfpathlineto{\pgfqpoint{1.933601in}{1.342062in}}%
\pgfpathclose%
\pgfusepath{clip}%
\pgfsys@defobject{currentpattern}{\pgfqpoint{0in}{0in}}{\pgfqpoint{1in}{1in}}{%
\begin{pgfscope}%
\pgfpathrectangle{\pgfqpoint{0in}{0in}}{\pgfqpoint{1in}{1in}}%
\pgfusepath{clip}%
\pgfpathmoveto{\pgfqpoint{-0.500000in}{0.500000in}}%
\pgfpathlineto{\pgfqpoint{0.500000in}{-0.500000in}}%
\pgfpathmoveto{\pgfqpoint{-0.444444in}{0.555556in}}%
\pgfpathlineto{\pgfqpoint{0.555556in}{-0.444444in}}%
\pgfpathmoveto{\pgfqpoint{-0.388889in}{0.611111in}}%
\pgfpathlineto{\pgfqpoint{0.611111in}{-0.388889in}}%
\pgfpathmoveto{\pgfqpoint{-0.333333in}{0.666667in}}%
\pgfpathlineto{\pgfqpoint{0.666667in}{-0.333333in}}%
\pgfpathmoveto{\pgfqpoint{-0.277778in}{0.722222in}}%
\pgfpathlineto{\pgfqpoint{0.722222in}{-0.277778in}}%
\pgfpathmoveto{\pgfqpoint{-0.222222in}{0.777778in}}%
\pgfpathlineto{\pgfqpoint{0.777778in}{-0.222222in}}%
\pgfpathmoveto{\pgfqpoint{-0.166667in}{0.833333in}}%
\pgfpathlineto{\pgfqpoint{0.833333in}{-0.166667in}}%
\pgfpathmoveto{\pgfqpoint{-0.111111in}{0.888889in}}%
\pgfpathlineto{\pgfqpoint{0.888889in}{-0.111111in}}%
\pgfpathmoveto{\pgfqpoint{-0.055556in}{0.944444in}}%
\pgfpathlineto{\pgfqpoint{0.944444in}{-0.055556in}}%
\pgfpathmoveto{\pgfqpoint{0.000000in}{1.000000in}}%
\pgfpathlineto{\pgfqpoint{1.000000in}{0.000000in}}%
\pgfpathmoveto{\pgfqpoint{0.055556in}{1.055556in}}%
\pgfpathlineto{\pgfqpoint{1.055556in}{0.055556in}}%
\pgfpathmoveto{\pgfqpoint{0.111111in}{1.111111in}}%
\pgfpathlineto{\pgfqpoint{1.111111in}{0.111111in}}%
\pgfpathmoveto{\pgfqpoint{0.166667in}{1.166667in}}%
\pgfpathlineto{\pgfqpoint{1.166667in}{0.166667in}}%
\pgfpathmoveto{\pgfqpoint{0.222222in}{1.222222in}}%
\pgfpathlineto{\pgfqpoint{1.222222in}{0.222222in}}%
\pgfpathmoveto{\pgfqpoint{0.277778in}{1.277778in}}%
\pgfpathlineto{\pgfqpoint{1.277778in}{0.277778in}}%
\pgfpathmoveto{\pgfqpoint{0.333333in}{1.333333in}}%
\pgfpathlineto{\pgfqpoint{1.333333in}{0.333333in}}%
\pgfpathmoveto{\pgfqpoint{0.388889in}{1.388889in}}%
\pgfpathlineto{\pgfqpoint{1.388889in}{0.388889in}}%
\pgfpathmoveto{\pgfqpoint{0.444444in}{1.444444in}}%
\pgfpathlineto{\pgfqpoint{1.444444in}{0.444444in}}%
\pgfpathmoveto{\pgfqpoint{0.500000in}{1.500000in}}%
\pgfpathlineto{\pgfqpoint{1.500000in}{0.500000in}}%
\pgfusepath{stroke}%
\end{pgfscope}%
}%
\pgfsys@transformshift{1.933601in}{1.022500in}%
\pgfsys@useobject{currentpattern}{}%
\pgfsys@transformshift{1in}{0in}%
\pgfsys@transformshift{-1in}{0in}%
\pgfsys@transformshift{0in}{1in}%
\end{pgfscope}%
\begin{pgfscope}%
\pgfpathrectangle{\pgfqpoint{1.228750in}{1.022500in}}{\pgfqpoint{13.456250in}{3.662500in}}%
\pgfusepath{clip}%
\pgfsetbuttcap%
\pgfsetmiterjoin%
\definecolor{currentfill}{rgb}{0.798529,0.536765,0.389706}%
\pgfsetfillcolor{currentfill}%
\pgfsetlinewidth{1.003750pt}%
\definecolor{currentstroke}{rgb}{1.000000,1.000000,1.000000}%
\pgfsetstrokecolor{currentstroke}%
\pgfsetdash{}{0pt}%
\pgfpathmoveto{\pgfqpoint{3.855923in}{1.022500in}}%
\pgfpathlineto{\pgfqpoint{4.368542in}{1.022500in}}%
\pgfpathlineto{\pgfqpoint{4.368542in}{1.331229in}}%
\pgfpathlineto{\pgfqpoint{3.855923in}{1.331229in}}%
\pgfpathclose%
\pgfusepath{stroke,fill}%
\end{pgfscope}%
\begin{pgfscope}%
\pgfsetbuttcap%
\pgfsetmiterjoin%
\definecolor{currentfill}{rgb}{0.798529,0.536765,0.389706}%
\pgfsetfillcolor{currentfill}%
\pgfsetlinewidth{1.003750pt}%
\definecolor{currentstroke}{rgb}{1.000000,1.000000,1.000000}%
\pgfsetstrokecolor{currentstroke}%
\pgfsetdash{}{0pt}%
\pgfpathrectangle{\pgfqpoint{1.228750in}{1.022500in}}{\pgfqpoint{13.456250in}{3.662500in}}%
\pgfusepath{clip}%
\pgfpathmoveto{\pgfqpoint{3.855923in}{1.022500in}}%
\pgfpathlineto{\pgfqpoint{4.368542in}{1.022500in}}%
\pgfpathlineto{\pgfqpoint{4.368542in}{1.331229in}}%
\pgfpathlineto{\pgfqpoint{3.855923in}{1.331229in}}%
\pgfpathclose%
\pgfusepath{clip}%
\pgfsys@defobject{currentpattern}{\pgfqpoint{0in}{0in}}{\pgfqpoint{1in}{1in}}{%
\begin{pgfscope}%
\pgfpathrectangle{\pgfqpoint{0in}{0in}}{\pgfqpoint{1in}{1in}}%
\pgfusepath{clip}%
\pgfpathmoveto{\pgfqpoint{-0.500000in}{0.500000in}}%
\pgfpathlineto{\pgfqpoint{0.500000in}{-0.500000in}}%
\pgfpathmoveto{\pgfqpoint{-0.444444in}{0.555556in}}%
\pgfpathlineto{\pgfqpoint{0.555556in}{-0.444444in}}%
\pgfpathmoveto{\pgfqpoint{-0.388889in}{0.611111in}}%
\pgfpathlineto{\pgfqpoint{0.611111in}{-0.388889in}}%
\pgfpathmoveto{\pgfqpoint{-0.333333in}{0.666667in}}%
\pgfpathlineto{\pgfqpoint{0.666667in}{-0.333333in}}%
\pgfpathmoveto{\pgfqpoint{-0.277778in}{0.722222in}}%
\pgfpathlineto{\pgfqpoint{0.722222in}{-0.277778in}}%
\pgfpathmoveto{\pgfqpoint{-0.222222in}{0.777778in}}%
\pgfpathlineto{\pgfqpoint{0.777778in}{-0.222222in}}%
\pgfpathmoveto{\pgfqpoint{-0.166667in}{0.833333in}}%
\pgfpathlineto{\pgfqpoint{0.833333in}{-0.166667in}}%
\pgfpathmoveto{\pgfqpoint{-0.111111in}{0.888889in}}%
\pgfpathlineto{\pgfqpoint{0.888889in}{-0.111111in}}%
\pgfpathmoveto{\pgfqpoint{-0.055556in}{0.944444in}}%
\pgfpathlineto{\pgfqpoint{0.944444in}{-0.055556in}}%
\pgfpathmoveto{\pgfqpoint{0.000000in}{1.000000in}}%
\pgfpathlineto{\pgfqpoint{1.000000in}{0.000000in}}%
\pgfpathmoveto{\pgfqpoint{0.055556in}{1.055556in}}%
\pgfpathlineto{\pgfqpoint{1.055556in}{0.055556in}}%
\pgfpathmoveto{\pgfqpoint{0.111111in}{1.111111in}}%
\pgfpathlineto{\pgfqpoint{1.111111in}{0.111111in}}%
\pgfpathmoveto{\pgfqpoint{0.166667in}{1.166667in}}%
\pgfpathlineto{\pgfqpoint{1.166667in}{0.166667in}}%
\pgfpathmoveto{\pgfqpoint{0.222222in}{1.222222in}}%
\pgfpathlineto{\pgfqpoint{1.222222in}{0.222222in}}%
\pgfpathmoveto{\pgfqpoint{0.277778in}{1.277778in}}%
\pgfpathlineto{\pgfqpoint{1.277778in}{0.277778in}}%
\pgfpathmoveto{\pgfqpoint{0.333333in}{1.333333in}}%
\pgfpathlineto{\pgfqpoint{1.333333in}{0.333333in}}%
\pgfpathmoveto{\pgfqpoint{0.388889in}{1.388889in}}%
\pgfpathlineto{\pgfqpoint{1.388889in}{0.388889in}}%
\pgfpathmoveto{\pgfqpoint{0.444444in}{1.444444in}}%
\pgfpathlineto{\pgfqpoint{1.444444in}{0.444444in}}%
\pgfpathmoveto{\pgfqpoint{0.500000in}{1.500000in}}%
\pgfpathlineto{\pgfqpoint{1.500000in}{0.500000in}}%
\pgfusepath{stroke}%
\end{pgfscope}%
}%
\pgfsys@transformshift{3.855923in}{1.022500in}%
\pgfsys@useobject{currentpattern}{}%
\pgfsys@transformshift{1in}{0in}%
\pgfsys@transformshift{-1in}{0in}%
\pgfsys@transformshift{0in}{1in}%
\end{pgfscope}%
\begin{pgfscope}%
\pgfpathrectangle{\pgfqpoint{1.228750in}{1.022500in}}{\pgfqpoint{13.456250in}{3.662500in}}%
\pgfusepath{clip}%
\pgfsetbuttcap%
\pgfsetmiterjoin%
\definecolor{currentfill}{rgb}{0.798529,0.536765,0.389706}%
\pgfsetfillcolor{currentfill}%
\pgfsetlinewidth{1.003750pt}%
\definecolor{currentstroke}{rgb}{1.000000,1.000000,1.000000}%
\pgfsetstrokecolor{currentstroke}%
\pgfsetdash{}{0pt}%
\pgfpathmoveto{\pgfqpoint{5.778244in}{1.022500in}}%
\pgfpathlineto{\pgfqpoint{6.290863in}{1.022500in}}%
\pgfpathlineto{\pgfqpoint{6.290863in}{1.374559in}}%
\pgfpathlineto{\pgfqpoint{5.778244in}{1.374559in}}%
\pgfpathclose%
\pgfusepath{stroke,fill}%
\end{pgfscope}%
\begin{pgfscope}%
\pgfsetbuttcap%
\pgfsetmiterjoin%
\definecolor{currentfill}{rgb}{0.798529,0.536765,0.389706}%
\pgfsetfillcolor{currentfill}%
\pgfsetlinewidth{1.003750pt}%
\definecolor{currentstroke}{rgb}{1.000000,1.000000,1.000000}%
\pgfsetstrokecolor{currentstroke}%
\pgfsetdash{}{0pt}%
\pgfpathrectangle{\pgfqpoint{1.228750in}{1.022500in}}{\pgfqpoint{13.456250in}{3.662500in}}%
\pgfusepath{clip}%
\pgfpathmoveto{\pgfqpoint{5.778244in}{1.022500in}}%
\pgfpathlineto{\pgfqpoint{6.290863in}{1.022500in}}%
\pgfpathlineto{\pgfqpoint{6.290863in}{1.374559in}}%
\pgfpathlineto{\pgfqpoint{5.778244in}{1.374559in}}%
\pgfpathclose%
\pgfusepath{clip}%
\pgfsys@defobject{currentpattern}{\pgfqpoint{0in}{0in}}{\pgfqpoint{1in}{1in}}{%
\begin{pgfscope}%
\pgfpathrectangle{\pgfqpoint{0in}{0in}}{\pgfqpoint{1in}{1in}}%
\pgfusepath{clip}%
\pgfpathmoveto{\pgfqpoint{-0.500000in}{0.500000in}}%
\pgfpathlineto{\pgfqpoint{0.500000in}{-0.500000in}}%
\pgfpathmoveto{\pgfqpoint{-0.444444in}{0.555556in}}%
\pgfpathlineto{\pgfqpoint{0.555556in}{-0.444444in}}%
\pgfpathmoveto{\pgfqpoint{-0.388889in}{0.611111in}}%
\pgfpathlineto{\pgfqpoint{0.611111in}{-0.388889in}}%
\pgfpathmoveto{\pgfqpoint{-0.333333in}{0.666667in}}%
\pgfpathlineto{\pgfqpoint{0.666667in}{-0.333333in}}%
\pgfpathmoveto{\pgfqpoint{-0.277778in}{0.722222in}}%
\pgfpathlineto{\pgfqpoint{0.722222in}{-0.277778in}}%
\pgfpathmoveto{\pgfqpoint{-0.222222in}{0.777778in}}%
\pgfpathlineto{\pgfqpoint{0.777778in}{-0.222222in}}%
\pgfpathmoveto{\pgfqpoint{-0.166667in}{0.833333in}}%
\pgfpathlineto{\pgfqpoint{0.833333in}{-0.166667in}}%
\pgfpathmoveto{\pgfqpoint{-0.111111in}{0.888889in}}%
\pgfpathlineto{\pgfqpoint{0.888889in}{-0.111111in}}%
\pgfpathmoveto{\pgfqpoint{-0.055556in}{0.944444in}}%
\pgfpathlineto{\pgfqpoint{0.944444in}{-0.055556in}}%
\pgfpathmoveto{\pgfqpoint{0.000000in}{1.000000in}}%
\pgfpathlineto{\pgfqpoint{1.000000in}{0.000000in}}%
\pgfpathmoveto{\pgfqpoint{0.055556in}{1.055556in}}%
\pgfpathlineto{\pgfqpoint{1.055556in}{0.055556in}}%
\pgfpathmoveto{\pgfqpoint{0.111111in}{1.111111in}}%
\pgfpathlineto{\pgfqpoint{1.111111in}{0.111111in}}%
\pgfpathmoveto{\pgfqpoint{0.166667in}{1.166667in}}%
\pgfpathlineto{\pgfqpoint{1.166667in}{0.166667in}}%
\pgfpathmoveto{\pgfqpoint{0.222222in}{1.222222in}}%
\pgfpathlineto{\pgfqpoint{1.222222in}{0.222222in}}%
\pgfpathmoveto{\pgfqpoint{0.277778in}{1.277778in}}%
\pgfpathlineto{\pgfqpoint{1.277778in}{0.277778in}}%
\pgfpathmoveto{\pgfqpoint{0.333333in}{1.333333in}}%
\pgfpathlineto{\pgfqpoint{1.333333in}{0.333333in}}%
\pgfpathmoveto{\pgfqpoint{0.388889in}{1.388889in}}%
\pgfpathlineto{\pgfqpoint{1.388889in}{0.388889in}}%
\pgfpathmoveto{\pgfqpoint{0.444444in}{1.444444in}}%
\pgfpathlineto{\pgfqpoint{1.444444in}{0.444444in}}%
\pgfpathmoveto{\pgfqpoint{0.500000in}{1.500000in}}%
\pgfpathlineto{\pgfqpoint{1.500000in}{0.500000in}}%
\pgfusepath{stroke}%
\end{pgfscope}%
}%
\pgfsys@transformshift{5.778244in}{1.022500in}%
\pgfsys@useobject{currentpattern}{}%
\pgfsys@transformshift{1in}{0in}%
\pgfsys@transformshift{-1in}{0in}%
\pgfsys@transformshift{0in}{1in}%
\end{pgfscope}%
\begin{pgfscope}%
\pgfpathrectangle{\pgfqpoint{1.228750in}{1.022500in}}{\pgfqpoint{13.456250in}{3.662500in}}%
\pgfusepath{clip}%
\pgfsetbuttcap%
\pgfsetmiterjoin%
\definecolor{currentfill}{rgb}{0.798529,0.536765,0.389706}%
\pgfsetfillcolor{currentfill}%
\pgfsetlinewidth{1.003750pt}%
\definecolor{currentstroke}{rgb}{1.000000,1.000000,1.000000}%
\pgfsetstrokecolor{currentstroke}%
\pgfsetdash{}{0pt}%
\pgfpathmoveto{\pgfqpoint{7.700565in}{1.022500in}}%
\pgfpathlineto{\pgfqpoint{8.213185in}{1.022500in}}%
\pgfpathlineto{\pgfqpoint{8.213185in}{1.369143in}}%
\pgfpathlineto{\pgfqpoint{7.700565in}{1.369143in}}%
\pgfpathclose%
\pgfusepath{stroke,fill}%
\end{pgfscope}%
\begin{pgfscope}%
\pgfsetbuttcap%
\pgfsetmiterjoin%
\definecolor{currentfill}{rgb}{0.798529,0.536765,0.389706}%
\pgfsetfillcolor{currentfill}%
\pgfsetlinewidth{1.003750pt}%
\definecolor{currentstroke}{rgb}{1.000000,1.000000,1.000000}%
\pgfsetstrokecolor{currentstroke}%
\pgfsetdash{}{0pt}%
\pgfpathrectangle{\pgfqpoint{1.228750in}{1.022500in}}{\pgfqpoint{13.456250in}{3.662500in}}%
\pgfusepath{clip}%
\pgfpathmoveto{\pgfqpoint{7.700565in}{1.022500in}}%
\pgfpathlineto{\pgfqpoint{8.213185in}{1.022500in}}%
\pgfpathlineto{\pgfqpoint{8.213185in}{1.369143in}}%
\pgfpathlineto{\pgfqpoint{7.700565in}{1.369143in}}%
\pgfpathclose%
\pgfusepath{clip}%
\pgfsys@defobject{currentpattern}{\pgfqpoint{0in}{0in}}{\pgfqpoint{1in}{1in}}{%
\begin{pgfscope}%
\pgfpathrectangle{\pgfqpoint{0in}{0in}}{\pgfqpoint{1in}{1in}}%
\pgfusepath{clip}%
\pgfpathmoveto{\pgfqpoint{-0.500000in}{0.500000in}}%
\pgfpathlineto{\pgfqpoint{0.500000in}{-0.500000in}}%
\pgfpathmoveto{\pgfqpoint{-0.444444in}{0.555556in}}%
\pgfpathlineto{\pgfqpoint{0.555556in}{-0.444444in}}%
\pgfpathmoveto{\pgfqpoint{-0.388889in}{0.611111in}}%
\pgfpathlineto{\pgfqpoint{0.611111in}{-0.388889in}}%
\pgfpathmoveto{\pgfqpoint{-0.333333in}{0.666667in}}%
\pgfpathlineto{\pgfqpoint{0.666667in}{-0.333333in}}%
\pgfpathmoveto{\pgfqpoint{-0.277778in}{0.722222in}}%
\pgfpathlineto{\pgfqpoint{0.722222in}{-0.277778in}}%
\pgfpathmoveto{\pgfqpoint{-0.222222in}{0.777778in}}%
\pgfpathlineto{\pgfqpoint{0.777778in}{-0.222222in}}%
\pgfpathmoveto{\pgfqpoint{-0.166667in}{0.833333in}}%
\pgfpathlineto{\pgfqpoint{0.833333in}{-0.166667in}}%
\pgfpathmoveto{\pgfqpoint{-0.111111in}{0.888889in}}%
\pgfpathlineto{\pgfqpoint{0.888889in}{-0.111111in}}%
\pgfpathmoveto{\pgfqpoint{-0.055556in}{0.944444in}}%
\pgfpathlineto{\pgfqpoint{0.944444in}{-0.055556in}}%
\pgfpathmoveto{\pgfqpoint{0.000000in}{1.000000in}}%
\pgfpathlineto{\pgfqpoint{1.000000in}{0.000000in}}%
\pgfpathmoveto{\pgfqpoint{0.055556in}{1.055556in}}%
\pgfpathlineto{\pgfqpoint{1.055556in}{0.055556in}}%
\pgfpathmoveto{\pgfqpoint{0.111111in}{1.111111in}}%
\pgfpathlineto{\pgfqpoint{1.111111in}{0.111111in}}%
\pgfpathmoveto{\pgfqpoint{0.166667in}{1.166667in}}%
\pgfpathlineto{\pgfqpoint{1.166667in}{0.166667in}}%
\pgfpathmoveto{\pgfqpoint{0.222222in}{1.222222in}}%
\pgfpathlineto{\pgfqpoint{1.222222in}{0.222222in}}%
\pgfpathmoveto{\pgfqpoint{0.277778in}{1.277778in}}%
\pgfpathlineto{\pgfqpoint{1.277778in}{0.277778in}}%
\pgfpathmoveto{\pgfqpoint{0.333333in}{1.333333in}}%
\pgfpathlineto{\pgfqpoint{1.333333in}{0.333333in}}%
\pgfpathmoveto{\pgfqpoint{0.388889in}{1.388889in}}%
\pgfpathlineto{\pgfqpoint{1.388889in}{0.388889in}}%
\pgfpathmoveto{\pgfqpoint{0.444444in}{1.444444in}}%
\pgfpathlineto{\pgfqpoint{1.444444in}{0.444444in}}%
\pgfpathmoveto{\pgfqpoint{0.500000in}{1.500000in}}%
\pgfpathlineto{\pgfqpoint{1.500000in}{0.500000in}}%
\pgfusepath{stroke}%
\end{pgfscope}%
}%
\pgfsys@transformshift{7.700565in}{1.022500in}%
\pgfsys@useobject{currentpattern}{}%
\pgfsys@transformshift{1in}{0in}%
\pgfsys@transformshift{-1in}{0in}%
\pgfsys@transformshift{0in}{1in}%
\end{pgfscope}%
\begin{pgfscope}%
\pgfpathrectangle{\pgfqpoint{1.228750in}{1.022500in}}{\pgfqpoint{13.456250in}{3.662500in}}%
\pgfusepath{clip}%
\pgfsetbuttcap%
\pgfsetmiterjoin%
\definecolor{currentfill}{rgb}{0.798529,0.536765,0.389706}%
\pgfsetfillcolor{currentfill}%
\pgfsetlinewidth{1.003750pt}%
\definecolor{currentstroke}{rgb}{1.000000,1.000000,1.000000}%
\pgfsetstrokecolor{currentstroke}%
\pgfsetdash{}{0pt}%
\pgfpathmoveto{\pgfqpoint{9.622887in}{1.022500in}}%
\pgfpathlineto{\pgfqpoint{10.135506in}{1.022500in}}%
\pgfpathlineto{\pgfqpoint{10.135506in}{1.336645in}}%
\pgfpathlineto{\pgfqpoint{9.622887in}{1.336645in}}%
\pgfpathclose%
\pgfusepath{stroke,fill}%
\end{pgfscope}%
\begin{pgfscope}%
\pgfsetbuttcap%
\pgfsetmiterjoin%
\definecolor{currentfill}{rgb}{0.798529,0.536765,0.389706}%
\pgfsetfillcolor{currentfill}%
\pgfsetlinewidth{1.003750pt}%
\definecolor{currentstroke}{rgb}{1.000000,1.000000,1.000000}%
\pgfsetstrokecolor{currentstroke}%
\pgfsetdash{}{0pt}%
\pgfpathrectangle{\pgfqpoint{1.228750in}{1.022500in}}{\pgfqpoint{13.456250in}{3.662500in}}%
\pgfusepath{clip}%
\pgfpathmoveto{\pgfqpoint{9.622887in}{1.022500in}}%
\pgfpathlineto{\pgfqpoint{10.135506in}{1.022500in}}%
\pgfpathlineto{\pgfqpoint{10.135506in}{1.336645in}}%
\pgfpathlineto{\pgfqpoint{9.622887in}{1.336645in}}%
\pgfpathclose%
\pgfusepath{clip}%
\pgfsys@defobject{currentpattern}{\pgfqpoint{0in}{0in}}{\pgfqpoint{1in}{1in}}{%
\begin{pgfscope}%
\pgfpathrectangle{\pgfqpoint{0in}{0in}}{\pgfqpoint{1in}{1in}}%
\pgfusepath{clip}%
\pgfpathmoveto{\pgfqpoint{-0.500000in}{0.500000in}}%
\pgfpathlineto{\pgfqpoint{0.500000in}{-0.500000in}}%
\pgfpathmoveto{\pgfqpoint{-0.444444in}{0.555556in}}%
\pgfpathlineto{\pgfqpoint{0.555556in}{-0.444444in}}%
\pgfpathmoveto{\pgfqpoint{-0.388889in}{0.611111in}}%
\pgfpathlineto{\pgfqpoint{0.611111in}{-0.388889in}}%
\pgfpathmoveto{\pgfqpoint{-0.333333in}{0.666667in}}%
\pgfpathlineto{\pgfqpoint{0.666667in}{-0.333333in}}%
\pgfpathmoveto{\pgfqpoint{-0.277778in}{0.722222in}}%
\pgfpathlineto{\pgfqpoint{0.722222in}{-0.277778in}}%
\pgfpathmoveto{\pgfqpoint{-0.222222in}{0.777778in}}%
\pgfpathlineto{\pgfqpoint{0.777778in}{-0.222222in}}%
\pgfpathmoveto{\pgfqpoint{-0.166667in}{0.833333in}}%
\pgfpathlineto{\pgfqpoint{0.833333in}{-0.166667in}}%
\pgfpathmoveto{\pgfqpoint{-0.111111in}{0.888889in}}%
\pgfpathlineto{\pgfqpoint{0.888889in}{-0.111111in}}%
\pgfpathmoveto{\pgfqpoint{-0.055556in}{0.944444in}}%
\pgfpathlineto{\pgfqpoint{0.944444in}{-0.055556in}}%
\pgfpathmoveto{\pgfqpoint{0.000000in}{1.000000in}}%
\pgfpathlineto{\pgfqpoint{1.000000in}{0.000000in}}%
\pgfpathmoveto{\pgfqpoint{0.055556in}{1.055556in}}%
\pgfpathlineto{\pgfqpoint{1.055556in}{0.055556in}}%
\pgfpathmoveto{\pgfqpoint{0.111111in}{1.111111in}}%
\pgfpathlineto{\pgfqpoint{1.111111in}{0.111111in}}%
\pgfpathmoveto{\pgfqpoint{0.166667in}{1.166667in}}%
\pgfpathlineto{\pgfqpoint{1.166667in}{0.166667in}}%
\pgfpathmoveto{\pgfqpoint{0.222222in}{1.222222in}}%
\pgfpathlineto{\pgfqpoint{1.222222in}{0.222222in}}%
\pgfpathmoveto{\pgfqpoint{0.277778in}{1.277778in}}%
\pgfpathlineto{\pgfqpoint{1.277778in}{0.277778in}}%
\pgfpathmoveto{\pgfqpoint{0.333333in}{1.333333in}}%
\pgfpathlineto{\pgfqpoint{1.333333in}{0.333333in}}%
\pgfpathmoveto{\pgfqpoint{0.388889in}{1.388889in}}%
\pgfpathlineto{\pgfqpoint{1.388889in}{0.388889in}}%
\pgfpathmoveto{\pgfqpoint{0.444444in}{1.444444in}}%
\pgfpathlineto{\pgfqpoint{1.444444in}{0.444444in}}%
\pgfpathmoveto{\pgfqpoint{0.500000in}{1.500000in}}%
\pgfpathlineto{\pgfqpoint{1.500000in}{0.500000in}}%
\pgfusepath{stroke}%
\end{pgfscope}%
}%
\pgfsys@transformshift{9.622887in}{1.022500in}%
\pgfsys@useobject{currentpattern}{}%
\pgfsys@transformshift{1in}{0in}%
\pgfsys@transformshift{-1in}{0in}%
\pgfsys@transformshift{0in}{1in}%
\end{pgfscope}%
\begin{pgfscope}%
\pgfpathrectangle{\pgfqpoint{1.228750in}{1.022500in}}{\pgfqpoint{13.456250in}{3.662500in}}%
\pgfusepath{clip}%
\pgfsetbuttcap%
\pgfsetmiterjoin%
\definecolor{currentfill}{rgb}{0.798529,0.536765,0.389706}%
\pgfsetfillcolor{currentfill}%
\pgfsetlinewidth{1.003750pt}%
\definecolor{currentstroke}{rgb}{1.000000,1.000000,1.000000}%
\pgfsetstrokecolor{currentstroke}%
\pgfsetdash{}{0pt}%
\pgfpathmoveto{\pgfqpoint{11.545208in}{1.022500in}}%
\pgfpathlineto{\pgfqpoint{12.057827in}{1.022500in}}%
\pgfpathlineto{\pgfqpoint{12.057827in}{1.282482in}}%
\pgfpathlineto{\pgfqpoint{11.545208in}{1.282482in}}%
\pgfpathclose%
\pgfusepath{stroke,fill}%
\end{pgfscope}%
\begin{pgfscope}%
\pgfsetbuttcap%
\pgfsetmiterjoin%
\definecolor{currentfill}{rgb}{0.798529,0.536765,0.389706}%
\pgfsetfillcolor{currentfill}%
\pgfsetlinewidth{1.003750pt}%
\definecolor{currentstroke}{rgb}{1.000000,1.000000,1.000000}%
\pgfsetstrokecolor{currentstroke}%
\pgfsetdash{}{0pt}%
\pgfpathrectangle{\pgfqpoint{1.228750in}{1.022500in}}{\pgfqpoint{13.456250in}{3.662500in}}%
\pgfusepath{clip}%
\pgfpathmoveto{\pgfqpoint{11.545208in}{1.022500in}}%
\pgfpathlineto{\pgfqpoint{12.057827in}{1.022500in}}%
\pgfpathlineto{\pgfqpoint{12.057827in}{1.282482in}}%
\pgfpathlineto{\pgfqpoint{11.545208in}{1.282482in}}%
\pgfpathclose%
\pgfusepath{clip}%
\pgfsys@defobject{currentpattern}{\pgfqpoint{0in}{0in}}{\pgfqpoint{1in}{1in}}{%
\begin{pgfscope}%
\pgfpathrectangle{\pgfqpoint{0in}{0in}}{\pgfqpoint{1in}{1in}}%
\pgfusepath{clip}%
\pgfpathmoveto{\pgfqpoint{-0.500000in}{0.500000in}}%
\pgfpathlineto{\pgfqpoint{0.500000in}{-0.500000in}}%
\pgfpathmoveto{\pgfqpoint{-0.444444in}{0.555556in}}%
\pgfpathlineto{\pgfqpoint{0.555556in}{-0.444444in}}%
\pgfpathmoveto{\pgfqpoint{-0.388889in}{0.611111in}}%
\pgfpathlineto{\pgfqpoint{0.611111in}{-0.388889in}}%
\pgfpathmoveto{\pgfqpoint{-0.333333in}{0.666667in}}%
\pgfpathlineto{\pgfqpoint{0.666667in}{-0.333333in}}%
\pgfpathmoveto{\pgfqpoint{-0.277778in}{0.722222in}}%
\pgfpathlineto{\pgfqpoint{0.722222in}{-0.277778in}}%
\pgfpathmoveto{\pgfqpoint{-0.222222in}{0.777778in}}%
\pgfpathlineto{\pgfqpoint{0.777778in}{-0.222222in}}%
\pgfpathmoveto{\pgfqpoint{-0.166667in}{0.833333in}}%
\pgfpathlineto{\pgfqpoint{0.833333in}{-0.166667in}}%
\pgfpathmoveto{\pgfqpoint{-0.111111in}{0.888889in}}%
\pgfpathlineto{\pgfqpoint{0.888889in}{-0.111111in}}%
\pgfpathmoveto{\pgfqpoint{-0.055556in}{0.944444in}}%
\pgfpathlineto{\pgfqpoint{0.944444in}{-0.055556in}}%
\pgfpathmoveto{\pgfqpoint{0.000000in}{1.000000in}}%
\pgfpathlineto{\pgfqpoint{1.000000in}{0.000000in}}%
\pgfpathmoveto{\pgfqpoint{0.055556in}{1.055556in}}%
\pgfpathlineto{\pgfqpoint{1.055556in}{0.055556in}}%
\pgfpathmoveto{\pgfqpoint{0.111111in}{1.111111in}}%
\pgfpathlineto{\pgfqpoint{1.111111in}{0.111111in}}%
\pgfpathmoveto{\pgfqpoint{0.166667in}{1.166667in}}%
\pgfpathlineto{\pgfqpoint{1.166667in}{0.166667in}}%
\pgfpathmoveto{\pgfqpoint{0.222222in}{1.222222in}}%
\pgfpathlineto{\pgfqpoint{1.222222in}{0.222222in}}%
\pgfpathmoveto{\pgfqpoint{0.277778in}{1.277778in}}%
\pgfpathlineto{\pgfqpoint{1.277778in}{0.277778in}}%
\pgfpathmoveto{\pgfqpoint{0.333333in}{1.333333in}}%
\pgfpathlineto{\pgfqpoint{1.333333in}{0.333333in}}%
\pgfpathmoveto{\pgfqpoint{0.388889in}{1.388889in}}%
\pgfpathlineto{\pgfqpoint{1.388889in}{0.388889in}}%
\pgfpathmoveto{\pgfqpoint{0.444444in}{1.444444in}}%
\pgfpathlineto{\pgfqpoint{1.444444in}{0.444444in}}%
\pgfpathmoveto{\pgfqpoint{0.500000in}{1.500000in}}%
\pgfpathlineto{\pgfqpoint{1.500000in}{0.500000in}}%
\pgfusepath{stroke}%
\end{pgfscope}%
}%
\pgfsys@transformshift{11.545208in}{1.022500in}%
\pgfsys@useobject{currentpattern}{}%
\pgfsys@transformshift{1in}{0in}%
\pgfsys@transformshift{-1in}{0in}%
\pgfsys@transformshift{0in}{1in}%
\end{pgfscope}%
\begin{pgfscope}%
\pgfpathrectangle{\pgfqpoint{1.228750in}{1.022500in}}{\pgfqpoint{13.456250in}{3.662500in}}%
\pgfusepath{clip}%
\pgfsetbuttcap%
\pgfsetmiterjoin%
\definecolor{currentfill}{rgb}{0.798529,0.536765,0.389706}%
\pgfsetfillcolor{currentfill}%
\pgfsetlinewidth{1.003750pt}%
\definecolor{currentstroke}{rgb}{1.000000,1.000000,1.000000}%
\pgfsetstrokecolor{currentstroke}%
\pgfsetdash{}{0pt}%
\pgfpathmoveto{\pgfqpoint{13.467530in}{1.022500in}}%
\pgfpathlineto{\pgfqpoint{13.980149in}{1.022500in}}%
\pgfpathlineto{\pgfqpoint{13.980149in}{1.282482in}}%
\pgfpathlineto{\pgfqpoint{13.467530in}{1.282482in}}%
\pgfpathclose%
\pgfusepath{stroke,fill}%
\end{pgfscope}%
\begin{pgfscope}%
\pgfsetbuttcap%
\pgfsetmiterjoin%
\definecolor{currentfill}{rgb}{0.798529,0.536765,0.389706}%
\pgfsetfillcolor{currentfill}%
\pgfsetlinewidth{1.003750pt}%
\definecolor{currentstroke}{rgb}{1.000000,1.000000,1.000000}%
\pgfsetstrokecolor{currentstroke}%
\pgfsetdash{}{0pt}%
\pgfpathrectangle{\pgfqpoint{1.228750in}{1.022500in}}{\pgfqpoint{13.456250in}{3.662500in}}%
\pgfusepath{clip}%
\pgfpathmoveto{\pgfqpoint{13.467530in}{1.022500in}}%
\pgfpathlineto{\pgfqpoint{13.980149in}{1.022500in}}%
\pgfpathlineto{\pgfqpoint{13.980149in}{1.282482in}}%
\pgfpathlineto{\pgfqpoint{13.467530in}{1.282482in}}%
\pgfpathclose%
\pgfusepath{clip}%
\pgfsys@defobject{currentpattern}{\pgfqpoint{0in}{0in}}{\pgfqpoint{1in}{1in}}{%
\begin{pgfscope}%
\pgfpathrectangle{\pgfqpoint{0in}{0in}}{\pgfqpoint{1in}{1in}}%
\pgfusepath{clip}%
\pgfpathmoveto{\pgfqpoint{-0.500000in}{0.500000in}}%
\pgfpathlineto{\pgfqpoint{0.500000in}{-0.500000in}}%
\pgfpathmoveto{\pgfqpoint{-0.444444in}{0.555556in}}%
\pgfpathlineto{\pgfqpoint{0.555556in}{-0.444444in}}%
\pgfpathmoveto{\pgfqpoint{-0.388889in}{0.611111in}}%
\pgfpathlineto{\pgfqpoint{0.611111in}{-0.388889in}}%
\pgfpathmoveto{\pgfqpoint{-0.333333in}{0.666667in}}%
\pgfpathlineto{\pgfqpoint{0.666667in}{-0.333333in}}%
\pgfpathmoveto{\pgfqpoint{-0.277778in}{0.722222in}}%
\pgfpathlineto{\pgfqpoint{0.722222in}{-0.277778in}}%
\pgfpathmoveto{\pgfqpoint{-0.222222in}{0.777778in}}%
\pgfpathlineto{\pgfqpoint{0.777778in}{-0.222222in}}%
\pgfpathmoveto{\pgfqpoint{-0.166667in}{0.833333in}}%
\pgfpathlineto{\pgfqpoint{0.833333in}{-0.166667in}}%
\pgfpathmoveto{\pgfqpoint{-0.111111in}{0.888889in}}%
\pgfpathlineto{\pgfqpoint{0.888889in}{-0.111111in}}%
\pgfpathmoveto{\pgfqpoint{-0.055556in}{0.944444in}}%
\pgfpathlineto{\pgfqpoint{0.944444in}{-0.055556in}}%
\pgfpathmoveto{\pgfqpoint{0.000000in}{1.000000in}}%
\pgfpathlineto{\pgfqpoint{1.000000in}{0.000000in}}%
\pgfpathmoveto{\pgfqpoint{0.055556in}{1.055556in}}%
\pgfpathlineto{\pgfqpoint{1.055556in}{0.055556in}}%
\pgfpathmoveto{\pgfqpoint{0.111111in}{1.111111in}}%
\pgfpathlineto{\pgfqpoint{1.111111in}{0.111111in}}%
\pgfpathmoveto{\pgfqpoint{0.166667in}{1.166667in}}%
\pgfpathlineto{\pgfqpoint{1.166667in}{0.166667in}}%
\pgfpathmoveto{\pgfqpoint{0.222222in}{1.222222in}}%
\pgfpathlineto{\pgfqpoint{1.222222in}{0.222222in}}%
\pgfpathmoveto{\pgfqpoint{0.277778in}{1.277778in}}%
\pgfpathlineto{\pgfqpoint{1.277778in}{0.277778in}}%
\pgfpathmoveto{\pgfqpoint{0.333333in}{1.333333in}}%
\pgfpathlineto{\pgfqpoint{1.333333in}{0.333333in}}%
\pgfpathmoveto{\pgfqpoint{0.388889in}{1.388889in}}%
\pgfpathlineto{\pgfqpoint{1.388889in}{0.388889in}}%
\pgfpathmoveto{\pgfqpoint{0.444444in}{1.444444in}}%
\pgfpathlineto{\pgfqpoint{1.444444in}{0.444444in}}%
\pgfpathmoveto{\pgfqpoint{0.500000in}{1.500000in}}%
\pgfpathlineto{\pgfqpoint{1.500000in}{0.500000in}}%
\pgfusepath{stroke}%
\end{pgfscope}%
}%
\pgfsys@transformshift{13.467530in}{1.022500in}%
\pgfsys@useobject{currentpattern}{}%
\pgfsys@transformshift{1in}{0in}%
\pgfsys@transformshift{-1in}{0in}%
\pgfsys@transformshift{0in}{1in}%
\end{pgfscope}%
\begin{pgfscope}%
\pgfpathrectangle{\pgfqpoint{1.228750in}{1.022500in}}{\pgfqpoint{13.456250in}{3.662500in}}%
\pgfusepath{clip}%
\pgfsetbuttcap%
\pgfsetmiterjoin%
\definecolor{currentfill}{rgb}{0.374020,0.618137,0.429902}%
\pgfsetfillcolor{currentfill}%
\pgfsetlinewidth{1.003750pt}%
\definecolor{currentstroke}{rgb}{1.000000,1.000000,1.000000}%
\pgfsetstrokecolor{currentstroke}%
\pgfsetdash{}{0pt}%
\pgfpathmoveto{\pgfqpoint{2.446220in}{1.022500in}}%
\pgfpathlineto{\pgfqpoint{2.958839in}{1.022500in}}%
\pgfpathlineto{\pgfqpoint{2.958839in}{4.228948in}}%
\pgfpathlineto{\pgfqpoint{2.446220in}{4.228948in}}%
\pgfpathclose%
\pgfusepath{stroke,fill}%
\end{pgfscope}%
\begin{pgfscope}%
\pgfsetbuttcap%
\pgfsetmiterjoin%
\definecolor{currentfill}{rgb}{0.374020,0.618137,0.429902}%
\pgfsetfillcolor{currentfill}%
\pgfsetlinewidth{1.003750pt}%
\definecolor{currentstroke}{rgb}{1.000000,1.000000,1.000000}%
\pgfsetstrokecolor{currentstroke}%
\pgfsetdash{}{0pt}%
\pgfpathrectangle{\pgfqpoint{1.228750in}{1.022500in}}{\pgfqpoint{13.456250in}{3.662500in}}%
\pgfusepath{clip}%
\pgfpathmoveto{\pgfqpoint{2.446220in}{1.022500in}}%
\pgfpathlineto{\pgfqpoint{2.958839in}{1.022500in}}%
\pgfpathlineto{\pgfqpoint{2.958839in}{4.228948in}}%
\pgfpathlineto{\pgfqpoint{2.446220in}{4.228948in}}%
\pgfpathclose%
\pgfusepath{clip}%
\pgfsys@defobject{currentpattern}{\pgfqpoint{0in}{0in}}{\pgfqpoint{1in}{1in}}{%
\begin{pgfscope}%
\pgfpathrectangle{\pgfqpoint{0in}{0in}}{\pgfqpoint{1in}{1in}}%
\pgfusepath{clip}%
\pgfusepath{stroke}%
\end{pgfscope}%
}%
\pgfsys@transformshift{2.446220in}{1.022500in}%
\pgfsys@useobject{currentpattern}{}%
\pgfsys@transformshift{1in}{0in}%
\pgfsys@transformshift{-1in}{0in}%
\pgfsys@transformshift{0in}{1in}%
\pgfsys@useobject{currentpattern}{}%
\pgfsys@transformshift{1in}{0in}%
\pgfsys@transformshift{-1in}{0in}%
\pgfsys@transformshift{0in}{1in}%
\pgfsys@useobject{currentpattern}{}%
\pgfsys@transformshift{1in}{0in}%
\pgfsys@transformshift{-1in}{0in}%
\pgfsys@transformshift{0in}{1in}%
\pgfsys@useobject{currentpattern}{}%
\pgfsys@transformshift{1in}{0in}%
\pgfsys@transformshift{-1in}{0in}%
\pgfsys@transformshift{0in}{1in}%
\end{pgfscope}%
\begin{pgfscope}%
\pgfpathrectangle{\pgfqpoint{1.228750in}{1.022500in}}{\pgfqpoint{13.456250in}{3.662500in}}%
\pgfusepath{clip}%
\pgfsetbuttcap%
\pgfsetmiterjoin%
\definecolor{currentfill}{rgb}{0.374020,0.618137,0.429902}%
\pgfsetfillcolor{currentfill}%
\pgfsetlinewidth{1.003750pt}%
\definecolor{currentstroke}{rgb}{1.000000,1.000000,1.000000}%
\pgfsetstrokecolor{currentstroke}%
\pgfsetdash{}{0pt}%
\pgfpathmoveto{\pgfqpoint{4.368542in}{1.022500in}}%
\pgfpathlineto{\pgfqpoint{4.881161in}{1.022500in}}%
\pgfpathlineto{\pgfqpoint{4.881161in}{4.147703in}}%
\pgfpathlineto{\pgfqpoint{4.368542in}{4.147703in}}%
\pgfpathclose%
\pgfusepath{stroke,fill}%
\end{pgfscope}%
\begin{pgfscope}%
\pgfsetbuttcap%
\pgfsetmiterjoin%
\definecolor{currentfill}{rgb}{0.374020,0.618137,0.429902}%
\pgfsetfillcolor{currentfill}%
\pgfsetlinewidth{1.003750pt}%
\definecolor{currentstroke}{rgb}{1.000000,1.000000,1.000000}%
\pgfsetstrokecolor{currentstroke}%
\pgfsetdash{}{0pt}%
\pgfpathrectangle{\pgfqpoint{1.228750in}{1.022500in}}{\pgfqpoint{13.456250in}{3.662500in}}%
\pgfusepath{clip}%
\pgfpathmoveto{\pgfqpoint{4.368542in}{1.022500in}}%
\pgfpathlineto{\pgfqpoint{4.881161in}{1.022500in}}%
\pgfpathlineto{\pgfqpoint{4.881161in}{4.147703in}}%
\pgfpathlineto{\pgfqpoint{4.368542in}{4.147703in}}%
\pgfpathclose%
\pgfusepath{clip}%
\pgfsys@defobject{currentpattern}{\pgfqpoint{0in}{0in}}{\pgfqpoint{1in}{1in}}{%
\begin{pgfscope}%
\pgfpathrectangle{\pgfqpoint{0in}{0in}}{\pgfqpoint{1in}{1in}}%
\pgfusepath{clip}%
\pgfusepath{stroke}%
\end{pgfscope}%
}%
\pgfsys@transformshift{4.368542in}{1.022500in}%
\pgfsys@useobject{currentpattern}{}%
\pgfsys@transformshift{1in}{0in}%
\pgfsys@transformshift{-1in}{0in}%
\pgfsys@transformshift{0in}{1in}%
\pgfsys@useobject{currentpattern}{}%
\pgfsys@transformshift{1in}{0in}%
\pgfsys@transformshift{-1in}{0in}%
\pgfsys@transformshift{0in}{1in}%
\pgfsys@useobject{currentpattern}{}%
\pgfsys@transformshift{1in}{0in}%
\pgfsys@transformshift{-1in}{0in}%
\pgfsys@transformshift{0in}{1in}%
\pgfsys@useobject{currentpattern}{}%
\pgfsys@transformshift{1in}{0in}%
\pgfsys@transformshift{-1in}{0in}%
\pgfsys@transformshift{0in}{1in}%
\end{pgfscope}%
\begin{pgfscope}%
\pgfpathrectangle{\pgfqpoint{1.228750in}{1.022500in}}{\pgfqpoint{13.456250in}{3.662500in}}%
\pgfusepath{clip}%
\pgfsetbuttcap%
\pgfsetmiterjoin%
\definecolor{currentfill}{rgb}{0.374020,0.618137,0.429902}%
\pgfsetfillcolor{currentfill}%
\pgfsetlinewidth{1.003750pt}%
\definecolor{currentstroke}{rgb}{1.000000,1.000000,1.000000}%
\pgfsetstrokecolor{currentstroke}%
\pgfsetdash{}{0pt}%
\pgfpathmoveto{\pgfqpoint{6.290863in}{1.022500in}}%
\pgfpathlineto{\pgfqpoint{6.803482in}{1.022500in}}%
\pgfpathlineto{\pgfqpoint{6.803482in}{4.510595in}}%
\pgfpathlineto{\pgfqpoint{6.290863in}{4.510595in}}%
\pgfpathclose%
\pgfusepath{stroke,fill}%
\end{pgfscope}%
\begin{pgfscope}%
\pgfsetbuttcap%
\pgfsetmiterjoin%
\definecolor{currentfill}{rgb}{0.374020,0.618137,0.429902}%
\pgfsetfillcolor{currentfill}%
\pgfsetlinewidth{1.003750pt}%
\definecolor{currentstroke}{rgb}{1.000000,1.000000,1.000000}%
\pgfsetstrokecolor{currentstroke}%
\pgfsetdash{}{0pt}%
\pgfpathrectangle{\pgfqpoint{1.228750in}{1.022500in}}{\pgfqpoint{13.456250in}{3.662500in}}%
\pgfusepath{clip}%
\pgfpathmoveto{\pgfqpoint{6.290863in}{1.022500in}}%
\pgfpathlineto{\pgfqpoint{6.803482in}{1.022500in}}%
\pgfpathlineto{\pgfqpoint{6.803482in}{4.510595in}}%
\pgfpathlineto{\pgfqpoint{6.290863in}{4.510595in}}%
\pgfpathclose%
\pgfusepath{clip}%
\pgfsys@defobject{currentpattern}{\pgfqpoint{0in}{0in}}{\pgfqpoint{1in}{1in}}{%
\begin{pgfscope}%
\pgfpathrectangle{\pgfqpoint{0in}{0in}}{\pgfqpoint{1in}{1in}}%
\pgfusepath{clip}%
\pgfusepath{stroke}%
\end{pgfscope}%
}%
\pgfsys@transformshift{6.290863in}{1.022500in}%
\pgfsys@useobject{currentpattern}{}%
\pgfsys@transformshift{1in}{0in}%
\pgfsys@transformshift{-1in}{0in}%
\pgfsys@transformshift{0in}{1in}%
\pgfsys@useobject{currentpattern}{}%
\pgfsys@transformshift{1in}{0in}%
\pgfsys@transformshift{-1in}{0in}%
\pgfsys@transformshift{0in}{1in}%
\pgfsys@useobject{currentpattern}{}%
\pgfsys@transformshift{1in}{0in}%
\pgfsys@transformshift{-1in}{0in}%
\pgfsys@transformshift{0in}{1in}%
\pgfsys@useobject{currentpattern}{}%
\pgfsys@transformshift{1in}{0in}%
\pgfsys@transformshift{-1in}{0in}%
\pgfsys@transformshift{0in}{1in}%
\end{pgfscope}%
\begin{pgfscope}%
\pgfpathrectangle{\pgfqpoint{1.228750in}{1.022500in}}{\pgfqpoint{13.456250in}{3.662500in}}%
\pgfusepath{clip}%
\pgfsetbuttcap%
\pgfsetmiterjoin%
\definecolor{currentfill}{rgb}{0.374020,0.618137,0.429902}%
\pgfsetfillcolor{currentfill}%
\pgfsetlinewidth{1.003750pt}%
\definecolor{currentstroke}{rgb}{1.000000,1.000000,1.000000}%
\pgfsetstrokecolor{currentstroke}%
\pgfsetdash{}{0pt}%
\pgfpathmoveto{\pgfqpoint{8.213185in}{1.022500in}}%
\pgfpathlineto{\pgfqpoint{8.725804in}{1.022500in}}%
\pgfpathlineto{\pgfqpoint{8.725804in}{4.429351in}}%
\pgfpathlineto{\pgfqpoint{8.213185in}{4.429351in}}%
\pgfpathclose%
\pgfusepath{stroke,fill}%
\end{pgfscope}%
\begin{pgfscope}%
\pgfsetbuttcap%
\pgfsetmiterjoin%
\definecolor{currentfill}{rgb}{0.374020,0.618137,0.429902}%
\pgfsetfillcolor{currentfill}%
\pgfsetlinewidth{1.003750pt}%
\definecolor{currentstroke}{rgb}{1.000000,1.000000,1.000000}%
\pgfsetstrokecolor{currentstroke}%
\pgfsetdash{}{0pt}%
\pgfpathrectangle{\pgfqpoint{1.228750in}{1.022500in}}{\pgfqpoint{13.456250in}{3.662500in}}%
\pgfusepath{clip}%
\pgfpathmoveto{\pgfqpoint{8.213185in}{1.022500in}}%
\pgfpathlineto{\pgfqpoint{8.725804in}{1.022500in}}%
\pgfpathlineto{\pgfqpoint{8.725804in}{4.429351in}}%
\pgfpathlineto{\pgfqpoint{8.213185in}{4.429351in}}%
\pgfpathclose%
\pgfusepath{clip}%
\pgfsys@defobject{currentpattern}{\pgfqpoint{0in}{0in}}{\pgfqpoint{1in}{1in}}{%
\begin{pgfscope}%
\pgfpathrectangle{\pgfqpoint{0in}{0in}}{\pgfqpoint{1in}{1in}}%
\pgfusepath{clip}%
\pgfusepath{stroke}%
\end{pgfscope}%
}%
\pgfsys@transformshift{8.213185in}{1.022500in}%
\pgfsys@useobject{currentpattern}{}%
\pgfsys@transformshift{1in}{0in}%
\pgfsys@transformshift{-1in}{0in}%
\pgfsys@transformshift{0in}{1in}%
\pgfsys@useobject{currentpattern}{}%
\pgfsys@transformshift{1in}{0in}%
\pgfsys@transformshift{-1in}{0in}%
\pgfsys@transformshift{0in}{1in}%
\pgfsys@useobject{currentpattern}{}%
\pgfsys@transformshift{1in}{0in}%
\pgfsys@transformshift{-1in}{0in}%
\pgfsys@transformshift{0in}{1in}%
\pgfsys@useobject{currentpattern}{}%
\pgfsys@transformshift{1in}{0in}%
\pgfsys@transformshift{-1in}{0in}%
\pgfsys@transformshift{0in}{1in}%
\end{pgfscope}%
\begin{pgfscope}%
\pgfpathrectangle{\pgfqpoint{1.228750in}{1.022500in}}{\pgfqpoint{13.456250in}{3.662500in}}%
\pgfusepath{clip}%
\pgfsetbuttcap%
\pgfsetmiterjoin%
\definecolor{currentfill}{rgb}{0.374020,0.618137,0.429902}%
\pgfsetfillcolor{currentfill}%
\pgfsetlinewidth{1.003750pt}%
\definecolor{currentstroke}{rgb}{1.000000,1.000000,1.000000}%
\pgfsetstrokecolor{currentstroke}%
\pgfsetdash{}{0pt}%
\pgfpathmoveto{\pgfqpoint{10.135506in}{1.022500in}}%
\pgfpathlineto{\pgfqpoint{10.648125in}{1.022500in}}%
\pgfpathlineto{\pgfqpoint{10.648125in}{4.033961in}}%
\pgfpathlineto{\pgfqpoint{10.135506in}{4.033961in}}%
\pgfpathclose%
\pgfusepath{stroke,fill}%
\end{pgfscope}%
\begin{pgfscope}%
\pgfsetbuttcap%
\pgfsetmiterjoin%
\definecolor{currentfill}{rgb}{0.374020,0.618137,0.429902}%
\pgfsetfillcolor{currentfill}%
\pgfsetlinewidth{1.003750pt}%
\definecolor{currentstroke}{rgb}{1.000000,1.000000,1.000000}%
\pgfsetstrokecolor{currentstroke}%
\pgfsetdash{}{0pt}%
\pgfpathrectangle{\pgfqpoint{1.228750in}{1.022500in}}{\pgfqpoint{13.456250in}{3.662500in}}%
\pgfusepath{clip}%
\pgfpathmoveto{\pgfqpoint{10.135506in}{1.022500in}}%
\pgfpathlineto{\pgfqpoint{10.648125in}{1.022500in}}%
\pgfpathlineto{\pgfqpoint{10.648125in}{4.033961in}}%
\pgfpathlineto{\pgfqpoint{10.135506in}{4.033961in}}%
\pgfpathclose%
\pgfusepath{clip}%
\pgfsys@defobject{currentpattern}{\pgfqpoint{0in}{0in}}{\pgfqpoint{1in}{1in}}{%
\begin{pgfscope}%
\pgfpathrectangle{\pgfqpoint{0in}{0in}}{\pgfqpoint{1in}{1in}}%
\pgfusepath{clip}%
\pgfusepath{stroke}%
\end{pgfscope}%
}%
\pgfsys@transformshift{10.135506in}{1.022500in}%
\pgfsys@useobject{currentpattern}{}%
\pgfsys@transformshift{1in}{0in}%
\pgfsys@transformshift{-1in}{0in}%
\pgfsys@transformshift{0in}{1in}%
\pgfsys@useobject{currentpattern}{}%
\pgfsys@transformshift{1in}{0in}%
\pgfsys@transformshift{-1in}{0in}%
\pgfsys@transformshift{0in}{1in}%
\pgfsys@useobject{currentpattern}{}%
\pgfsys@transformshift{1in}{0in}%
\pgfsys@transformshift{-1in}{0in}%
\pgfsys@transformshift{0in}{1in}%
\pgfsys@useobject{currentpattern}{}%
\pgfsys@transformshift{1in}{0in}%
\pgfsys@transformshift{-1in}{0in}%
\pgfsys@transformshift{0in}{1in}%
\end{pgfscope}%
\begin{pgfscope}%
\pgfpathrectangle{\pgfqpoint{1.228750in}{1.022500in}}{\pgfqpoint{13.456250in}{3.662500in}}%
\pgfusepath{clip}%
\pgfsetbuttcap%
\pgfsetmiterjoin%
\definecolor{currentfill}{rgb}{0.374020,0.618137,0.429902}%
\pgfsetfillcolor{currentfill}%
\pgfsetlinewidth{1.003750pt}%
\definecolor{currentstroke}{rgb}{1.000000,1.000000,1.000000}%
\pgfsetstrokecolor{currentstroke}%
\pgfsetdash{}{0pt}%
\pgfpathmoveto{\pgfqpoint{12.057827in}{1.022500in}}%
\pgfpathlineto{\pgfqpoint{12.570446in}{1.022500in}}%
\pgfpathlineto{\pgfqpoint{12.570446in}{3.243182in}}%
\pgfpathlineto{\pgfqpoint{12.057827in}{3.243182in}}%
\pgfpathclose%
\pgfusepath{stroke,fill}%
\end{pgfscope}%
\begin{pgfscope}%
\pgfsetbuttcap%
\pgfsetmiterjoin%
\definecolor{currentfill}{rgb}{0.374020,0.618137,0.429902}%
\pgfsetfillcolor{currentfill}%
\pgfsetlinewidth{1.003750pt}%
\definecolor{currentstroke}{rgb}{1.000000,1.000000,1.000000}%
\pgfsetstrokecolor{currentstroke}%
\pgfsetdash{}{0pt}%
\pgfpathrectangle{\pgfqpoint{1.228750in}{1.022500in}}{\pgfqpoint{13.456250in}{3.662500in}}%
\pgfusepath{clip}%
\pgfpathmoveto{\pgfqpoint{12.057827in}{1.022500in}}%
\pgfpathlineto{\pgfqpoint{12.570446in}{1.022500in}}%
\pgfpathlineto{\pgfqpoint{12.570446in}{3.243182in}}%
\pgfpathlineto{\pgfqpoint{12.057827in}{3.243182in}}%
\pgfpathclose%
\pgfusepath{clip}%
\pgfsys@defobject{currentpattern}{\pgfqpoint{0in}{0in}}{\pgfqpoint{1in}{1in}}{%
\begin{pgfscope}%
\pgfpathrectangle{\pgfqpoint{0in}{0in}}{\pgfqpoint{1in}{1in}}%
\pgfusepath{clip}%
\pgfusepath{stroke}%
\end{pgfscope}%
}%
\pgfsys@transformshift{12.057827in}{1.022500in}%
\pgfsys@useobject{currentpattern}{}%
\pgfsys@transformshift{1in}{0in}%
\pgfsys@transformshift{-1in}{0in}%
\pgfsys@transformshift{0in}{1in}%
\pgfsys@useobject{currentpattern}{}%
\pgfsys@transformshift{1in}{0in}%
\pgfsys@transformshift{-1in}{0in}%
\pgfsys@transformshift{0in}{1in}%
\pgfsys@useobject{currentpattern}{}%
\pgfsys@transformshift{1in}{0in}%
\pgfsys@transformshift{-1in}{0in}%
\pgfsys@transformshift{0in}{1in}%
\end{pgfscope}%
\begin{pgfscope}%
\pgfpathrectangle{\pgfqpoint{1.228750in}{1.022500in}}{\pgfqpoint{13.456250in}{3.662500in}}%
\pgfusepath{clip}%
\pgfsetbuttcap%
\pgfsetmiterjoin%
\definecolor{currentfill}{rgb}{0.374020,0.618137,0.429902}%
\pgfsetfillcolor{currentfill}%
\pgfsetlinewidth{1.003750pt}%
\definecolor{currentstroke}{rgb}{1.000000,1.000000,1.000000}%
\pgfsetstrokecolor{currentstroke}%
\pgfsetdash{}{0pt}%
\pgfpathmoveto{\pgfqpoint{13.980149in}{1.022500in}}%
\pgfpathlineto{\pgfqpoint{14.492768in}{1.022500in}}%
\pgfpathlineto{\pgfqpoint{14.492768in}{3.254014in}}%
\pgfpathlineto{\pgfqpoint{13.980149in}{3.254014in}}%
\pgfpathclose%
\pgfusepath{stroke,fill}%
\end{pgfscope}%
\begin{pgfscope}%
\pgfsetbuttcap%
\pgfsetmiterjoin%
\definecolor{currentfill}{rgb}{0.374020,0.618137,0.429902}%
\pgfsetfillcolor{currentfill}%
\pgfsetlinewidth{1.003750pt}%
\definecolor{currentstroke}{rgb}{1.000000,1.000000,1.000000}%
\pgfsetstrokecolor{currentstroke}%
\pgfsetdash{}{0pt}%
\pgfpathrectangle{\pgfqpoint{1.228750in}{1.022500in}}{\pgfqpoint{13.456250in}{3.662500in}}%
\pgfusepath{clip}%
\pgfpathmoveto{\pgfqpoint{13.980149in}{1.022500in}}%
\pgfpathlineto{\pgfqpoint{14.492768in}{1.022500in}}%
\pgfpathlineto{\pgfqpoint{14.492768in}{3.254014in}}%
\pgfpathlineto{\pgfqpoint{13.980149in}{3.254014in}}%
\pgfpathclose%
\pgfusepath{clip}%
\pgfsys@defobject{currentpattern}{\pgfqpoint{0in}{0in}}{\pgfqpoint{1in}{1in}}{%
\begin{pgfscope}%
\pgfpathrectangle{\pgfqpoint{0in}{0in}}{\pgfqpoint{1in}{1in}}%
\pgfusepath{clip}%
\pgfusepath{stroke}%
\end{pgfscope}%
}%
\pgfsys@transformshift{13.980149in}{1.022500in}%
\pgfsys@useobject{currentpattern}{}%
\pgfsys@transformshift{1in}{0in}%
\pgfsys@transformshift{-1in}{0in}%
\pgfsys@transformshift{0in}{1in}%
\pgfsys@useobject{currentpattern}{}%
\pgfsys@transformshift{1in}{0in}%
\pgfsys@transformshift{-1in}{0in}%
\pgfsys@transformshift{0in}{1in}%
\pgfsys@useobject{currentpattern}{}%
\pgfsys@transformshift{1in}{0in}%
\pgfsys@transformshift{-1in}{0in}%
\pgfsys@transformshift{0in}{1in}%
\end{pgfscope}%
\begin{pgfscope}%
\pgfpathrectangle{\pgfqpoint{1.228750in}{1.022500in}}{\pgfqpoint{13.456250in}{3.662500in}}%
\pgfusepath{clip}%
\pgfsetroundcap%
\pgfsetroundjoin%
\pgfsetlinewidth{2.710125pt}%
\definecolor{currentstroke}{rgb}{0.260000,0.260000,0.260000}%
\pgfsetstrokecolor{currentstroke}%
\pgfsetdash{}{0pt}%
\pgfusepath{stroke}%
\end{pgfscope}%
\begin{pgfscope}%
\pgfpathrectangle{\pgfqpoint{1.228750in}{1.022500in}}{\pgfqpoint{13.456250in}{3.662500in}}%
\pgfusepath{clip}%
\pgfsetroundcap%
\pgfsetroundjoin%
\pgfsetlinewidth{2.710125pt}%
\definecolor{currentstroke}{rgb}{0.260000,0.260000,0.260000}%
\pgfsetstrokecolor{currentstroke}%
\pgfsetdash{}{0pt}%
\pgfusepath{stroke}%
\end{pgfscope}%
\begin{pgfscope}%
\pgfpathrectangle{\pgfqpoint{1.228750in}{1.022500in}}{\pgfqpoint{13.456250in}{3.662500in}}%
\pgfusepath{clip}%
\pgfsetroundcap%
\pgfsetroundjoin%
\pgfsetlinewidth{2.710125pt}%
\definecolor{currentstroke}{rgb}{0.260000,0.260000,0.260000}%
\pgfsetstrokecolor{currentstroke}%
\pgfsetdash{}{0pt}%
\pgfusepath{stroke}%
\end{pgfscope}%
\begin{pgfscope}%
\pgfpathrectangle{\pgfqpoint{1.228750in}{1.022500in}}{\pgfqpoint{13.456250in}{3.662500in}}%
\pgfusepath{clip}%
\pgfsetroundcap%
\pgfsetroundjoin%
\pgfsetlinewidth{2.710125pt}%
\definecolor{currentstroke}{rgb}{0.260000,0.260000,0.260000}%
\pgfsetstrokecolor{currentstroke}%
\pgfsetdash{}{0pt}%
\pgfusepath{stroke}%
\end{pgfscope}%
\begin{pgfscope}%
\pgfpathrectangle{\pgfqpoint{1.228750in}{1.022500in}}{\pgfqpoint{13.456250in}{3.662500in}}%
\pgfusepath{clip}%
\pgfsetroundcap%
\pgfsetroundjoin%
\pgfsetlinewidth{2.710125pt}%
\definecolor{currentstroke}{rgb}{0.260000,0.260000,0.260000}%
\pgfsetstrokecolor{currentstroke}%
\pgfsetdash{}{0pt}%
\pgfusepath{stroke}%
\end{pgfscope}%
\begin{pgfscope}%
\pgfpathrectangle{\pgfqpoint{1.228750in}{1.022500in}}{\pgfqpoint{13.456250in}{3.662500in}}%
\pgfusepath{clip}%
\pgfsetroundcap%
\pgfsetroundjoin%
\pgfsetlinewidth{2.710125pt}%
\definecolor{currentstroke}{rgb}{0.260000,0.260000,0.260000}%
\pgfsetstrokecolor{currentstroke}%
\pgfsetdash{}{0pt}%
\pgfusepath{stroke}%
\end{pgfscope}%
\begin{pgfscope}%
\pgfpathrectangle{\pgfqpoint{1.228750in}{1.022500in}}{\pgfqpoint{13.456250in}{3.662500in}}%
\pgfusepath{clip}%
\pgfsetroundcap%
\pgfsetroundjoin%
\pgfsetlinewidth{2.710125pt}%
\definecolor{currentstroke}{rgb}{0.260000,0.260000,0.260000}%
\pgfsetstrokecolor{currentstroke}%
\pgfsetdash{}{0pt}%
\pgfusepath{stroke}%
\end{pgfscope}%
\begin{pgfscope}%
\pgfpathrectangle{\pgfqpoint{1.228750in}{1.022500in}}{\pgfqpoint{13.456250in}{3.662500in}}%
\pgfusepath{clip}%
\pgfsetroundcap%
\pgfsetroundjoin%
\pgfsetlinewidth{2.710125pt}%
\definecolor{currentstroke}{rgb}{0.260000,0.260000,0.260000}%
\pgfsetstrokecolor{currentstroke}%
\pgfsetdash{}{0pt}%
\pgfusepath{stroke}%
\end{pgfscope}%
\begin{pgfscope}%
\pgfpathrectangle{\pgfqpoint{1.228750in}{1.022500in}}{\pgfqpoint{13.456250in}{3.662500in}}%
\pgfusepath{clip}%
\pgfsetroundcap%
\pgfsetroundjoin%
\pgfsetlinewidth{2.710125pt}%
\definecolor{currentstroke}{rgb}{0.260000,0.260000,0.260000}%
\pgfsetstrokecolor{currentstroke}%
\pgfsetdash{}{0pt}%
\pgfusepath{stroke}%
\end{pgfscope}%
\begin{pgfscope}%
\pgfpathrectangle{\pgfqpoint{1.228750in}{1.022500in}}{\pgfqpoint{13.456250in}{3.662500in}}%
\pgfusepath{clip}%
\pgfsetroundcap%
\pgfsetroundjoin%
\pgfsetlinewidth{2.710125pt}%
\definecolor{currentstroke}{rgb}{0.260000,0.260000,0.260000}%
\pgfsetstrokecolor{currentstroke}%
\pgfsetdash{}{0pt}%
\pgfusepath{stroke}%
\end{pgfscope}%
\begin{pgfscope}%
\pgfpathrectangle{\pgfqpoint{1.228750in}{1.022500in}}{\pgfqpoint{13.456250in}{3.662500in}}%
\pgfusepath{clip}%
\pgfsetroundcap%
\pgfsetroundjoin%
\pgfsetlinewidth{2.710125pt}%
\definecolor{currentstroke}{rgb}{0.260000,0.260000,0.260000}%
\pgfsetstrokecolor{currentstroke}%
\pgfsetdash{}{0pt}%
\pgfusepath{stroke}%
\end{pgfscope}%
\begin{pgfscope}%
\pgfpathrectangle{\pgfqpoint{1.228750in}{1.022500in}}{\pgfqpoint{13.456250in}{3.662500in}}%
\pgfusepath{clip}%
\pgfsetroundcap%
\pgfsetroundjoin%
\pgfsetlinewidth{2.710125pt}%
\definecolor{currentstroke}{rgb}{0.260000,0.260000,0.260000}%
\pgfsetstrokecolor{currentstroke}%
\pgfsetdash{}{0pt}%
\pgfusepath{stroke}%
\end{pgfscope}%
\begin{pgfscope}%
\pgfpathrectangle{\pgfqpoint{1.228750in}{1.022500in}}{\pgfqpoint{13.456250in}{3.662500in}}%
\pgfusepath{clip}%
\pgfsetroundcap%
\pgfsetroundjoin%
\pgfsetlinewidth{2.710125pt}%
\definecolor{currentstroke}{rgb}{0.260000,0.260000,0.260000}%
\pgfsetstrokecolor{currentstroke}%
\pgfsetdash{}{0pt}%
\pgfusepath{stroke}%
\end{pgfscope}%
\begin{pgfscope}%
\pgfpathrectangle{\pgfqpoint{1.228750in}{1.022500in}}{\pgfqpoint{13.456250in}{3.662500in}}%
\pgfusepath{clip}%
\pgfsetroundcap%
\pgfsetroundjoin%
\pgfsetlinewidth{2.710125pt}%
\definecolor{currentstroke}{rgb}{0.260000,0.260000,0.260000}%
\pgfsetstrokecolor{currentstroke}%
\pgfsetdash{}{0pt}%
\pgfusepath{stroke}%
\end{pgfscope}%
\begin{pgfscope}%
\pgfpathrectangle{\pgfqpoint{1.228750in}{1.022500in}}{\pgfqpoint{13.456250in}{3.662500in}}%
\pgfusepath{clip}%
\pgfsetroundcap%
\pgfsetroundjoin%
\pgfsetlinewidth{2.710125pt}%
\definecolor{currentstroke}{rgb}{0.260000,0.260000,0.260000}%
\pgfsetstrokecolor{currentstroke}%
\pgfsetdash{}{0pt}%
\pgfusepath{stroke}%
\end{pgfscope}%
\begin{pgfscope}%
\pgfpathrectangle{\pgfqpoint{1.228750in}{1.022500in}}{\pgfqpoint{13.456250in}{3.662500in}}%
\pgfusepath{clip}%
\pgfsetroundcap%
\pgfsetroundjoin%
\pgfsetlinewidth{2.710125pt}%
\definecolor{currentstroke}{rgb}{0.260000,0.260000,0.260000}%
\pgfsetstrokecolor{currentstroke}%
\pgfsetdash{}{0pt}%
\pgfusepath{stroke}%
\end{pgfscope}%
\begin{pgfscope}%
\pgfpathrectangle{\pgfqpoint{1.228750in}{1.022500in}}{\pgfqpoint{13.456250in}{3.662500in}}%
\pgfusepath{clip}%
\pgfsetroundcap%
\pgfsetroundjoin%
\pgfsetlinewidth{2.710125pt}%
\definecolor{currentstroke}{rgb}{0.260000,0.260000,0.260000}%
\pgfsetstrokecolor{currentstroke}%
\pgfsetdash{}{0pt}%
\pgfusepath{stroke}%
\end{pgfscope}%
\begin{pgfscope}%
\pgfpathrectangle{\pgfqpoint{1.228750in}{1.022500in}}{\pgfqpoint{13.456250in}{3.662500in}}%
\pgfusepath{clip}%
\pgfsetroundcap%
\pgfsetroundjoin%
\pgfsetlinewidth{2.710125pt}%
\definecolor{currentstroke}{rgb}{0.260000,0.260000,0.260000}%
\pgfsetstrokecolor{currentstroke}%
\pgfsetdash{}{0pt}%
\pgfusepath{stroke}%
\end{pgfscope}%
\begin{pgfscope}%
\pgfpathrectangle{\pgfqpoint{1.228750in}{1.022500in}}{\pgfqpoint{13.456250in}{3.662500in}}%
\pgfusepath{clip}%
\pgfsetroundcap%
\pgfsetroundjoin%
\pgfsetlinewidth{2.710125pt}%
\definecolor{currentstroke}{rgb}{0.260000,0.260000,0.260000}%
\pgfsetstrokecolor{currentstroke}%
\pgfsetdash{}{0pt}%
\pgfusepath{stroke}%
\end{pgfscope}%
\begin{pgfscope}%
\pgfpathrectangle{\pgfqpoint{1.228750in}{1.022500in}}{\pgfqpoint{13.456250in}{3.662500in}}%
\pgfusepath{clip}%
\pgfsetroundcap%
\pgfsetroundjoin%
\pgfsetlinewidth{2.710125pt}%
\definecolor{currentstroke}{rgb}{0.260000,0.260000,0.260000}%
\pgfsetstrokecolor{currentstroke}%
\pgfsetdash{}{0pt}%
\pgfusepath{stroke}%
\end{pgfscope}%
\begin{pgfscope}%
\pgfpathrectangle{\pgfqpoint{1.228750in}{1.022500in}}{\pgfqpoint{13.456250in}{3.662500in}}%
\pgfusepath{clip}%
\pgfsetroundcap%
\pgfsetroundjoin%
\pgfsetlinewidth{2.710125pt}%
\definecolor{currentstroke}{rgb}{0.260000,0.260000,0.260000}%
\pgfsetstrokecolor{currentstroke}%
\pgfsetdash{}{0pt}%
\pgfusepath{stroke}%
\end{pgfscope}%
\begin{pgfscope}%
\pgfsetrectcap%
\pgfsetmiterjoin%
\pgfsetlinewidth{1.254687pt}%
\definecolor{currentstroke}{rgb}{0.800000,0.800000,0.800000}%
\pgfsetstrokecolor{currentstroke}%
\pgfsetdash{}{0pt}%
\pgfpathmoveto{\pgfqpoint{1.228750in}{1.022500in}}%
\pgfpathlineto{\pgfqpoint{14.685000in}{1.022500in}}%
\pgfusepath{stroke}%
\end{pgfscope}%
\begin{pgfscope}%
\pgfsetbuttcap%
\pgfsetmiterjoin%
\definecolor{currentfill}{rgb}{1.000000,1.000000,1.000000}%
\pgfsetfillcolor{currentfill}%
\pgfsetfillopacity{0.800000}%
\pgfsetlinewidth{1.003750pt}%
\definecolor{currentstroke}{rgb}{0.800000,0.800000,0.800000}%
\pgfsetstrokecolor{currentstroke}%
\pgfsetstrokeopacity{0.800000}%
\pgfsetdash{}{0pt}%
\pgfpathmoveto{\pgfqpoint{11.660327in}{3.301867in}}%
\pgfpathlineto{\pgfqpoint{14.497847in}{3.301867in}}%
\pgfpathquadraticcurveto{\pgfqpoint{14.551319in}{3.301867in}}{\pgfqpoint{14.551319in}{3.355339in}}%
\pgfpathlineto{\pgfqpoint{14.551319in}{4.497847in}}%
\pgfpathquadraticcurveto{\pgfqpoint{14.551319in}{4.551319in}}{\pgfqpoint{14.497847in}{4.551319in}}%
\pgfpathlineto{\pgfqpoint{11.660327in}{4.551319in}}%
\pgfpathquadraticcurveto{\pgfqpoint{11.606855in}{4.551319in}}{\pgfqpoint{11.606855in}{4.497847in}}%
\pgfpathlineto{\pgfqpoint{11.606855in}{3.355339in}}%
\pgfpathquadraticcurveto{\pgfqpoint{11.606855in}{3.301867in}}{\pgfqpoint{11.660327in}{3.301867in}}%
\pgfpathclose%
\pgfusepath{stroke,fill}%
\end{pgfscope}%
\begin{pgfscope}%
\pgfsetbuttcap%
\pgfsetmiterjoin%
\definecolor{currentfill}{rgb}{0.347059,0.458824,0.641176}%
\pgfsetfillcolor{currentfill}%
\pgfsetlinewidth{1.003750pt}%
\definecolor{currentstroke}{rgb}{1.000000,1.000000,1.000000}%
\pgfsetstrokecolor{currentstroke}%
\pgfsetdash{}{0pt}%
\pgfpathmoveto{\pgfqpoint{11.713800in}{4.244337in}}%
\pgfpathlineto{\pgfqpoint{12.248522in}{4.244337in}}%
\pgfpathlineto{\pgfqpoint{12.248522in}{4.431489in}}%
\pgfpathlineto{\pgfqpoint{11.713800in}{4.431489in}}%
\pgfpathclose%
\pgfusepath{stroke,fill}%
\end{pgfscope}%
\begin{pgfscope}%
\pgfsetbuttcap%
\pgfsetmiterjoin%
\definecolor{currentfill}{rgb}{0.347059,0.458824,0.641176}%
\pgfsetfillcolor{currentfill}%
\pgfsetlinewidth{1.003750pt}%
\definecolor{currentstroke}{rgb}{1.000000,1.000000,1.000000}%
\pgfsetstrokecolor{currentstroke}%
\pgfsetdash{}{0pt}%
\pgfpathmoveto{\pgfqpoint{11.713800in}{4.244337in}}%
\pgfpathlineto{\pgfqpoint{12.248522in}{4.244337in}}%
\pgfpathlineto{\pgfqpoint{12.248522in}{4.431489in}}%
\pgfpathlineto{\pgfqpoint{11.713800in}{4.431489in}}%
\pgfpathclose%
\pgfusepath{clip}%
\pgfsys@defobject{currentpattern}{\pgfqpoint{0in}{0in}}{\pgfqpoint{1in}{1in}}{%
\begin{pgfscope}%
\pgfpathrectangle{\pgfqpoint{0in}{0in}}{\pgfqpoint{1in}{1in}}%
\pgfusepath{clip}%
\pgfpathmoveto{\pgfqpoint{-0.500000in}{0.500000in}}%
\pgfpathlineto{\pgfqpoint{0.500000in}{1.500000in}}%
\pgfpathmoveto{\pgfqpoint{-0.444444in}{0.444444in}}%
\pgfpathlineto{\pgfqpoint{0.555556in}{1.444444in}}%
\pgfpathmoveto{\pgfqpoint{-0.388889in}{0.388889in}}%
\pgfpathlineto{\pgfqpoint{0.611111in}{1.388889in}}%
\pgfpathmoveto{\pgfqpoint{-0.333333in}{0.333333in}}%
\pgfpathlineto{\pgfqpoint{0.666667in}{1.333333in}}%
\pgfpathmoveto{\pgfqpoint{-0.277778in}{0.277778in}}%
\pgfpathlineto{\pgfqpoint{0.722222in}{1.277778in}}%
\pgfpathmoveto{\pgfqpoint{-0.222222in}{0.222222in}}%
\pgfpathlineto{\pgfqpoint{0.777778in}{1.222222in}}%
\pgfpathmoveto{\pgfqpoint{-0.166667in}{0.166667in}}%
\pgfpathlineto{\pgfqpoint{0.833333in}{1.166667in}}%
\pgfpathmoveto{\pgfqpoint{-0.111111in}{0.111111in}}%
\pgfpathlineto{\pgfqpoint{0.888889in}{1.111111in}}%
\pgfpathmoveto{\pgfqpoint{-0.055556in}{0.055556in}}%
\pgfpathlineto{\pgfqpoint{0.944444in}{1.055556in}}%
\pgfpathmoveto{\pgfqpoint{0.000000in}{0.000000in}}%
\pgfpathlineto{\pgfqpoint{1.000000in}{1.000000in}}%
\pgfpathmoveto{\pgfqpoint{0.055556in}{-0.055556in}}%
\pgfpathlineto{\pgfqpoint{1.055556in}{0.944444in}}%
\pgfpathmoveto{\pgfqpoint{0.111111in}{-0.111111in}}%
\pgfpathlineto{\pgfqpoint{1.111111in}{0.888889in}}%
\pgfpathmoveto{\pgfqpoint{0.166667in}{-0.166667in}}%
\pgfpathlineto{\pgfqpoint{1.166667in}{0.833333in}}%
\pgfpathmoveto{\pgfqpoint{0.222222in}{-0.222222in}}%
\pgfpathlineto{\pgfqpoint{1.222222in}{0.777778in}}%
\pgfpathmoveto{\pgfqpoint{0.277778in}{-0.277778in}}%
\pgfpathlineto{\pgfqpoint{1.277778in}{0.722222in}}%
\pgfpathmoveto{\pgfqpoint{0.333333in}{-0.333333in}}%
\pgfpathlineto{\pgfqpoint{1.333333in}{0.666667in}}%
\pgfpathmoveto{\pgfqpoint{0.388889in}{-0.388889in}}%
\pgfpathlineto{\pgfqpoint{1.388889in}{0.611111in}}%
\pgfpathmoveto{\pgfqpoint{0.444444in}{-0.444444in}}%
\pgfpathlineto{\pgfqpoint{1.444444in}{0.555556in}}%
\pgfpathmoveto{\pgfqpoint{0.500000in}{-0.500000in}}%
\pgfpathlineto{\pgfqpoint{1.500000in}{0.500000in}}%
\pgfusepath{stroke}%
\end{pgfscope}%
}%
\pgfsys@transformshift{11.713800in}{4.244337in}%
\pgfsys@useobject{currentpattern}{}%
\pgfsys@transformshift{1in}{0in}%
\pgfsys@transformshift{-1in}{0in}%
\pgfsys@transformshift{0in}{1in}%
\end{pgfscope}%
\begin{pgfscope}%
\definecolor{textcolor}{rgb}{0.150000,0.150000,0.150000}%
\pgfsetstrokecolor{textcolor}%
\pgfsetfillcolor{textcolor}%
\pgftext[x=12.462411in,y=4.244337in,left,base]{\color{textcolor}\sffamily\fontsize{19.250000}{23.100000}\selectfont JitSynth compiler}%
\end{pgfscope}%
\begin{pgfscope}%
\pgfsetbuttcap%
\pgfsetmiterjoin%
\definecolor{currentfill}{rgb}{0.798529,0.536765,0.389706}%
\pgfsetfillcolor{currentfill}%
\pgfsetlinewidth{1.003750pt}%
\definecolor{currentstroke}{rgb}{1.000000,1.000000,1.000000}%
\pgfsetstrokecolor{currentstroke}%
\pgfsetdash{}{0pt}%
\pgfpathmoveto{\pgfqpoint{11.713800in}{3.854588in}}%
\pgfpathlineto{\pgfqpoint{12.248522in}{3.854588in}}%
\pgfpathlineto{\pgfqpoint{12.248522in}{4.041741in}}%
\pgfpathlineto{\pgfqpoint{11.713800in}{4.041741in}}%
\pgfpathclose%
\pgfusepath{stroke,fill}%
\end{pgfscope}%
\begin{pgfscope}%
\pgfsetbuttcap%
\pgfsetmiterjoin%
\definecolor{currentfill}{rgb}{0.798529,0.536765,0.389706}%
\pgfsetfillcolor{currentfill}%
\pgfsetlinewidth{1.003750pt}%
\definecolor{currentstroke}{rgb}{1.000000,1.000000,1.000000}%
\pgfsetstrokecolor{currentstroke}%
\pgfsetdash{}{0pt}%
\pgfpathmoveto{\pgfqpoint{11.713800in}{3.854588in}}%
\pgfpathlineto{\pgfqpoint{12.248522in}{3.854588in}}%
\pgfpathlineto{\pgfqpoint{12.248522in}{4.041741in}}%
\pgfpathlineto{\pgfqpoint{11.713800in}{4.041741in}}%
\pgfpathclose%
\pgfusepath{clip}%
\pgfsys@defobject{currentpattern}{\pgfqpoint{0in}{0in}}{\pgfqpoint{1in}{1in}}{%
\begin{pgfscope}%
\pgfpathrectangle{\pgfqpoint{0in}{0in}}{\pgfqpoint{1in}{1in}}%
\pgfusepath{clip}%
\pgfpathmoveto{\pgfqpoint{-0.500000in}{0.500000in}}%
\pgfpathlineto{\pgfqpoint{0.500000in}{-0.500000in}}%
\pgfpathmoveto{\pgfqpoint{-0.444444in}{0.555556in}}%
\pgfpathlineto{\pgfqpoint{0.555556in}{-0.444444in}}%
\pgfpathmoveto{\pgfqpoint{-0.388889in}{0.611111in}}%
\pgfpathlineto{\pgfqpoint{0.611111in}{-0.388889in}}%
\pgfpathmoveto{\pgfqpoint{-0.333333in}{0.666667in}}%
\pgfpathlineto{\pgfqpoint{0.666667in}{-0.333333in}}%
\pgfpathmoveto{\pgfqpoint{-0.277778in}{0.722222in}}%
\pgfpathlineto{\pgfqpoint{0.722222in}{-0.277778in}}%
\pgfpathmoveto{\pgfqpoint{-0.222222in}{0.777778in}}%
\pgfpathlineto{\pgfqpoint{0.777778in}{-0.222222in}}%
\pgfpathmoveto{\pgfqpoint{-0.166667in}{0.833333in}}%
\pgfpathlineto{\pgfqpoint{0.833333in}{-0.166667in}}%
\pgfpathmoveto{\pgfqpoint{-0.111111in}{0.888889in}}%
\pgfpathlineto{\pgfqpoint{0.888889in}{-0.111111in}}%
\pgfpathmoveto{\pgfqpoint{-0.055556in}{0.944444in}}%
\pgfpathlineto{\pgfqpoint{0.944444in}{-0.055556in}}%
\pgfpathmoveto{\pgfqpoint{0.000000in}{1.000000in}}%
\pgfpathlineto{\pgfqpoint{1.000000in}{0.000000in}}%
\pgfpathmoveto{\pgfqpoint{0.055556in}{1.055556in}}%
\pgfpathlineto{\pgfqpoint{1.055556in}{0.055556in}}%
\pgfpathmoveto{\pgfqpoint{0.111111in}{1.111111in}}%
\pgfpathlineto{\pgfqpoint{1.111111in}{0.111111in}}%
\pgfpathmoveto{\pgfqpoint{0.166667in}{1.166667in}}%
\pgfpathlineto{\pgfqpoint{1.166667in}{0.166667in}}%
\pgfpathmoveto{\pgfqpoint{0.222222in}{1.222222in}}%
\pgfpathlineto{\pgfqpoint{1.222222in}{0.222222in}}%
\pgfpathmoveto{\pgfqpoint{0.277778in}{1.277778in}}%
\pgfpathlineto{\pgfqpoint{1.277778in}{0.277778in}}%
\pgfpathmoveto{\pgfqpoint{0.333333in}{1.333333in}}%
\pgfpathlineto{\pgfqpoint{1.333333in}{0.333333in}}%
\pgfpathmoveto{\pgfqpoint{0.388889in}{1.388889in}}%
\pgfpathlineto{\pgfqpoint{1.388889in}{0.388889in}}%
\pgfpathmoveto{\pgfqpoint{0.444444in}{1.444444in}}%
\pgfpathlineto{\pgfqpoint{1.444444in}{0.444444in}}%
\pgfpathmoveto{\pgfqpoint{0.500000in}{1.500000in}}%
\pgfpathlineto{\pgfqpoint{1.500000in}{0.500000in}}%
\pgfusepath{stroke}%
\end{pgfscope}%
}%
\pgfsys@transformshift{11.713800in}{3.854588in}%
\pgfsys@useobject{currentpattern}{}%
\pgfsys@transformshift{1in}{0in}%
\pgfsys@transformshift{-1in}{0in}%
\pgfsys@transformshift{0in}{1in}%
\end{pgfscope}%
\begin{pgfscope}%
\definecolor{textcolor}{rgb}{0.150000,0.150000,0.150000}%
\pgfsetstrokecolor{textcolor}%
\pgfsetfillcolor{textcolor}%
\pgftext[x=12.462411in,y=3.854588in,left,base]{\color{textcolor}\sffamily\fontsize{19.250000}{23.100000}\selectfont Linux compiler}%
\end{pgfscope}%
\begin{pgfscope}%
\pgfsetbuttcap%
\pgfsetmiterjoin%
\definecolor{currentfill}{rgb}{0.374020,0.618137,0.429902}%
\pgfsetfillcolor{currentfill}%
\pgfsetlinewidth{1.003750pt}%
\definecolor{currentstroke}{rgb}{1.000000,1.000000,1.000000}%
\pgfsetstrokecolor{currentstroke}%
\pgfsetdash{}{0pt}%
\pgfpathmoveto{\pgfqpoint{11.713800in}{3.464840in}}%
\pgfpathlineto{\pgfqpoint{12.248522in}{3.464840in}}%
\pgfpathlineto{\pgfqpoint{12.248522in}{3.651993in}}%
\pgfpathlineto{\pgfqpoint{11.713800in}{3.651993in}}%
\pgfpathclose%
\pgfusepath{stroke,fill}%
\end{pgfscope}%
\begin{pgfscope}%
\pgfsetbuttcap%
\pgfsetmiterjoin%
\definecolor{currentfill}{rgb}{0.374020,0.618137,0.429902}%
\pgfsetfillcolor{currentfill}%
\pgfsetlinewidth{1.003750pt}%
\definecolor{currentstroke}{rgb}{1.000000,1.000000,1.000000}%
\pgfsetstrokecolor{currentstroke}%
\pgfsetdash{}{0pt}%
\pgfpathmoveto{\pgfqpoint{11.713800in}{3.464840in}}%
\pgfpathlineto{\pgfqpoint{12.248522in}{3.464840in}}%
\pgfpathlineto{\pgfqpoint{12.248522in}{3.651993in}}%
\pgfpathlineto{\pgfqpoint{11.713800in}{3.651993in}}%
\pgfpathclose%
\pgfusepath{clip}%
\pgfsys@defobject{currentpattern}{\pgfqpoint{0in}{0in}}{\pgfqpoint{1in}{1in}}{%
\begin{pgfscope}%
\pgfpathrectangle{\pgfqpoint{0in}{0in}}{\pgfqpoint{1in}{1in}}%
\pgfusepath{clip}%
\pgfusepath{stroke}%
\end{pgfscope}%
}%
\pgfsys@transformshift{11.713800in}{3.464840in}%
\pgfsys@useobject{currentpattern}{}%
\pgfsys@transformshift{1in}{0in}%
\pgfsys@transformshift{-1in}{0in}%
\pgfsys@transformshift{0in}{1in}%
\end{pgfscope}%
\begin{pgfscope}%
\definecolor{textcolor}{rgb}{0.150000,0.150000,0.150000}%
\pgfsetstrokecolor{textcolor}%
\pgfsetfillcolor{textcolor}%
\pgftext[x=12.462411in,y=3.464840in,left,base]{\color{textcolor}\sffamily\fontsize{19.250000}{23.100000}\selectfont Linux interpreter}%
\end{pgfscope}%
\end{pgfpicture}%
\makeatother%
\endgroup%

  }
  \caption{Execution time of eBPF benchmarks on the HiFive Unleashed
  RISC-V development board, using the existing Linux eBPF to RISC-V
  compiler, the \jitsynth compiler, and the Linux eBPF interpreter.
  Measured in processor cycles.}
  \label{fig:b2r-runtime}
\end{figure}

To demonstrate the effectiveness of \jitsynth,
we applied \jitsynth to synthesize compilers for three different
source-target pairs: eBPF to 64-bit RISC-V, classic BPF to eBPF,
and libseccomp to eBPF.
%
This subsection describes our results for each of the synthesized
compilers.
%


\paragraph{eBPF to RISC-V.}

As a case study, we applied \jitsynth to synthesize
a compiler from eBPF to 64-bit RISC-V.
%
It supports 87 of the 102 eBPF instruction opcodes;
unsupported eBPF instructions include function calls,
endianness operations, and atomic instructions.
%
To validate that the synthesized compiler is correct, we ran the existing
eBPF test cases from the Linux kernel; our compiler passes all test cases
it supports. % 228 supported / 234 total ebpf
%
In addition, our compiler avoids bugs previously found in the existing
Linux eBPF-to-RISC-V compiler in Linux~\cite{nelson:bpf-riscv-add32-bug}.
%
To evaluate performance, we compared against the existing Linux compiler.
%
We used the same set of benchmarks used by Jitk~\cite{wang:jitk},
which includes system call filters from widely used applications.
%
Because these benchmarks were originally for classic BPF,
we first compile them to eBPF using the existing
Linux classic-BPF-to-eBPF compiler as a preprocessing step.
%
To run the benchmarks, we execute the generated
code on the HiFive Unleashed RISC-V development board~\cite{sifive:fu540-c000}, measuring
the number of cycles.
%
As input to the filter, we use a system call number that is allowed
by the filter to represent the common case execution.


\autoref{fig:b2r-runtime} shows the results of the performance evaluation.
%
eBPF programs compiled by \jitsynth JIT compilers show an average slowdown of $\EbpfCompilerSlowdown\times$
compared to programs compiled by the existing Linux compiler.
%
This overhead results from additional complexity in the compiled eBPF jump instructions.
%
Linux compilers avoid this complexity by leveraging
bounds on the size of eBPF jump offsets.
%
\jitsynth-compiled programs get an average speedup of $\EbpfInterpSpeedup\times$
compared to interpreting the eBPF programs.
%
This evidence shows that \jitsynth can synthesize a compiler that outperforms
the current Linux eBPF interpreter, and nears the performance of the Linux
compiler, while avoiding bugs.
%

\paragraph{Classic BPF to eBPF.}

\begin{figure}[h]
  \resizebox{\textwidth}{!}{
  \input{figs/o2b-cc.pgf}
  %% Creator: Matplotlib, PGF backend
%%
%% To include the figure in your LaTeX document, write
%%   \input{<filename>.pgf}
%%
%% Make sure the required packages are loaded in your preamble
%%   \usepackage{pgf}
%%
%% Figures using additional raster images can only be included by \input if
%% they are in the same directory as the main LaTeX file. For loading figures
%% from other directories you can use the `import` package
%%   \usepackage{import}
%% and then include the figures with
%%   \import{<path to file>}{<filename>.pgf}
%%
%% Matplotlib used the following preamble
%%
\begingroup%
\makeatletter%
\begin{pgfpicture}%
\pgfpathrectangle{\pgfpointorigin}{\pgfqpoint{8.000000in}{5.000000in}}%
\pgfusepath{use as bounding box, clip}%
\begin{pgfscope}%
\pgfsetbuttcap%
\pgfsetmiterjoin%
\definecolor{currentfill}{rgb}{1.000000,1.000000,1.000000}%
\pgfsetfillcolor{currentfill}%
\pgfsetlinewidth{0.000000pt}%
\definecolor{currentstroke}{rgb}{1.000000,1.000000,1.000000}%
\pgfsetstrokecolor{currentstroke}%
\pgfsetdash{}{0pt}%
\pgfpathmoveto{\pgfqpoint{0.000000in}{0.000000in}}%
\pgfpathlineto{\pgfqpoint{8.000000in}{0.000000in}}%
\pgfpathlineto{\pgfqpoint{8.000000in}{5.000000in}}%
\pgfpathlineto{\pgfqpoint{0.000000in}{5.000000in}}%
\pgfpathclose%
\pgfusepath{fill}%
\end{pgfscope}%
\begin{pgfscope}%
\pgfsetbuttcap%
\pgfsetmiterjoin%
\definecolor{currentfill}{rgb}{1.000000,1.000000,1.000000}%
\pgfsetfillcolor{currentfill}%
\pgfsetlinewidth{0.000000pt}%
\definecolor{currentstroke}{rgb}{0.000000,0.000000,0.000000}%
\pgfsetstrokecolor{currentstroke}%
\pgfsetstrokeopacity{0.000000}%
\pgfsetdash{}{0pt}%
\pgfpathmoveto{\pgfqpoint{1.230000in}{1.022500in}}%
\pgfpathlineto{\pgfqpoint{7.685000in}{1.022500in}}%
\pgfpathlineto{\pgfqpoint{7.685000in}{4.381667in}}%
\pgfpathlineto{\pgfqpoint{1.230000in}{4.381667in}}%
\pgfpathclose%
\pgfusepath{fill}%
\end{pgfscope}%
\begin{pgfscope}%
\definecolor{textcolor}{rgb}{0.150000,0.150000,0.150000}%
\pgfsetstrokecolor{textcolor}%
\pgfsetfillcolor{textcolor}%
\pgftext[x=1.875500in,y=0.890556in,,top]{\color{textcolor}\sffamily\fontsize{19.250000}{23.100000}\selectfont ctags}%
\end{pgfscope}%
\begin{pgfscope}%
\definecolor{textcolor}{rgb}{0.150000,0.150000,0.150000}%
\pgfsetstrokecolor{textcolor}%
\pgfsetfillcolor{textcolor}%
\pgftext[x=3.166500in,y=0.890556in,,top]{\color{textcolor}\sffamily\fontsize{19.250000}{23.100000}\selectfont lepton}%
\end{pgfscope}%
\begin{pgfscope}%
\definecolor{textcolor}{rgb}{0.150000,0.150000,0.150000}%
\pgfsetstrokecolor{textcolor}%
\pgfsetfillcolor{textcolor}%
\pgftext[x=4.457500in,y=0.890556in,,top]{\color{textcolor}\sffamily\fontsize{19.250000}{23.100000}\selectfont libreoffice}%
\end{pgfscope}%
\begin{pgfscope}%
\definecolor{textcolor}{rgb}{0.150000,0.150000,0.150000}%
\pgfsetstrokecolor{textcolor}%
\pgfsetfillcolor{textcolor}%
\pgftext[x=5.748500in,y=0.890556in,,top]{\color{textcolor}\sffamily\fontsize{19.250000}{23.100000}\selectfont openssh}%
\end{pgfscope}%
\begin{pgfscope}%
\definecolor{textcolor}{rgb}{0.150000,0.150000,0.150000}%
\pgfsetstrokecolor{textcolor}%
\pgfsetfillcolor{textcolor}%
\pgftext[x=7.039500in,y=0.890556in,,top]{\color{textcolor}\sffamily\fontsize{19.250000}{23.100000}\selectfont vsftpd}%
\end{pgfscope}%
\begin{pgfscope}%
\definecolor{textcolor}{rgb}{0.150000,0.150000,0.150000}%
\pgfsetstrokecolor{textcolor}%
\pgfsetfillcolor{textcolor}%
\pgftext[x=4.457500in,y=0.578932in,,top]{\color{textcolor}\sffamily\fontsize{21.000000}{25.200000}\selectfont Benchmark}%
\end{pgfscope}%
\begin{pgfscope}%
\pgfpathrectangle{\pgfqpoint{1.230000in}{1.022500in}}{\pgfqpoint{6.455000in}{3.359167in}}%
\pgfusepath{clip}%
\pgfsetroundcap%
\pgfsetroundjoin%
\pgfsetlinewidth{1.003750pt}%
\definecolor{currentstroke}{rgb}{0.800000,0.800000,0.800000}%
\pgfsetstrokecolor{currentstroke}%
\pgfsetdash{}{0pt}%
\pgfpathmoveto{\pgfqpoint{1.230000in}{1.022500in}}%
\pgfpathlineto{\pgfqpoint{7.685000in}{1.022500in}}%
\pgfusepath{stroke}%
\end{pgfscope}%
\begin{pgfscope}%
\definecolor{textcolor}{rgb}{0.150000,0.150000,0.150000}%
\pgfsetstrokecolor{textcolor}%
\pgfsetfillcolor{textcolor}%
\pgftext[x=0.962614in,y=0.922481in,left,base]{\color{textcolor}\sffamily\fontsize{19.250000}{23.100000}\selectfont 0}%
\end{pgfscope}%
\begin{pgfscope}%
\pgfpathrectangle{\pgfqpoint{1.230000in}{1.022500in}}{\pgfqpoint{6.455000in}{3.359167in}}%
\pgfusepath{clip}%
\pgfsetroundcap%
\pgfsetroundjoin%
\pgfsetlinewidth{1.003750pt}%
\definecolor{currentstroke}{rgb}{0.800000,0.800000,0.800000}%
\pgfsetstrokecolor{currentstroke}%
\pgfsetdash{}{0pt}%
\pgfpathmoveto{\pgfqpoint{1.230000in}{2.061203in}}%
\pgfpathlineto{\pgfqpoint{7.685000in}{2.061203in}}%
\pgfusepath{stroke}%
\end{pgfscope}%
\begin{pgfscope}%
\definecolor{textcolor}{rgb}{0.150000,0.150000,0.150000}%
\pgfsetstrokecolor{textcolor}%
\pgfsetfillcolor{textcolor}%
\pgftext[x=0.691731in,y=1.961184in,left,base]{\color{textcolor}\sffamily\fontsize{19.250000}{23.100000}\selectfont 100}%
\end{pgfscope}%
\begin{pgfscope}%
\pgfpathrectangle{\pgfqpoint{1.230000in}{1.022500in}}{\pgfqpoint{6.455000in}{3.359167in}}%
\pgfusepath{clip}%
\pgfsetroundcap%
\pgfsetroundjoin%
\pgfsetlinewidth{1.003750pt}%
\definecolor{currentstroke}{rgb}{0.800000,0.800000,0.800000}%
\pgfsetstrokecolor{currentstroke}%
\pgfsetdash{}{0pt}%
\pgfpathmoveto{\pgfqpoint{1.230000in}{3.099907in}}%
\pgfpathlineto{\pgfqpoint{7.685000in}{3.099907in}}%
\pgfusepath{stroke}%
\end{pgfscope}%
\begin{pgfscope}%
\definecolor{textcolor}{rgb}{0.150000,0.150000,0.150000}%
\pgfsetstrokecolor{textcolor}%
\pgfsetfillcolor{textcolor}%
\pgftext[x=0.691731in,y=2.999887in,left,base]{\color{textcolor}\sffamily\fontsize{19.250000}{23.100000}\selectfont 200}%
\end{pgfscope}%
\begin{pgfscope}%
\pgfpathrectangle{\pgfqpoint{1.230000in}{1.022500in}}{\pgfqpoint{6.455000in}{3.359167in}}%
\pgfusepath{clip}%
\pgfsetroundcap%
\pgfsetroundjoin%
\pgfsetlinewidth{1.003750pt}%
\definecolor{currentstroke}{rgb}{0.800000,0.800000,0.800000}%
\pgfsetstrokecolor{currentstroke}%
\pgfsetdash{}{0pt}%
\pgfpathmoveto{\pgfqpoint{1.230000in}{4.138610in}}%
\pgfpathlineto{\pgfqpoint{7.685000in}{4.138610in}}%
\pgfusepath{stroke}%
\end{pgfscope}%
\begin{pgfscope}%
\definecolor{textcolor}{rgb}{0.150000,0.150000,0.150000}%
\pgfsetstrokecolor{textcolor}%
\pgfsetfillcolor{textcolor}%
\pgftext[x=0.691731in,y=4.038591in,left,base]{\color{textcolor}\sffamily\fontsize{19.250000}{23.100000}\selectfont 300}%
\end{pgfscope}%
\begin{pgfscope}%
\definecolor{textcolor}{rgb}{0.150000,0.150000,0.150000}%
\pgfsetstrokecolor{textcolor}%
\pgfsetfillcolor{textcolor}%
\pgftext[x=0.636175in,y=2.702083in,,bottom,rotate=90.000000]{\color{textcolor}\sffamily\fontsize{21.000000}{25.200000}\selectfont Instructions executed}%
\end{pgfscope}%
\begin{pgfscope}%
\pgfpathrectangle{\pgfqpoint{1.230000in}{1.022500in}}{\pgfqpoint{6.455000in}{3.359167in}}%
\pgfusepath{clip}%
\pgfsetbuttcap%
\pgfsetmiterjoin%
\definecolor{currentfill}{rgb}{0.347059,0.458824,0.641176}%
\pgfsetfillcolor{currentfill}%
\pgfsetlinewidth{1.003750pt}%
\definecolor{currentstroke}{rgb}{1.000000,1.000000,1.000000}%
\pgfsetstrokecolor{currentstroke}%
\pgfsetdash{}{0pt}%
\pgfpathmoveto{\pgfqpoint{1.359100in}{1.022500in}}%
\pgfpathlineto{\pgfqpoint{1.875500in}{1.022500in}}%
\pgfpathlineto{\pgfqpoint{1.875500in}{1.136757in}}%
\pgfpathlineto{\pgfqpoint{1.359100in}{1.136757in}}%
\pgfpathclose%
\pgfusepath{stroke,fill}%
\end{pgfscope}%
\begin{pgfscope}%
\pgfsetbuttcap%
\pgfsetmiterjoin%
\definecolor{currentfill}{rgb}{0.347059,0.458824,0.641176}%
\pgfsetfillcolor{currentfill}%
\pgfsetlinewidth{1.003750pt}%
\definecolor{currentstroke}{rgb}{1.000000,1.000000,1.000000}%
\pgfsetstrokecolor{currentstroke}%
\pgfsetdash{}{0pt}%
\pgfpathrectangle{\pgfqpoint{1.230000in}{1.022500in}}{\pgfqpoint{6.455000in}{3.359167in}}%
\pgfusepath{clip}%
\pgfpathmoveto{\pgfqpoint{1.359100in}{1.022500in}}%
\pgfpathlineto{\pgfqpoint{1.875500in}{1.022500in}}%
\pgfpathlineto{\pgfqpoint{1.875500in}{1.136757in}}%
\pgfpathlineto{\pgfqpoint{1.359100in}{1.136757in}}%
\pgfpathclose%
\pgfusepath{clip}%
\pgfsys@defobject{currentpattern}{\pgfqpoint{0in}{0in}}{\pgfqpoint{1in}{1in}}{%
\begin{pgfscope}%
\pgfpathrectangle{\pgfqpoint{0in}{0in}}{\pgfqpoint{1in}{1in}}%
\pgfusepath{clip}%
\pgfpathmoveto{\pgfqpoint{-0.500000in}{0.500000in}}%
\pgfpathlineto{\pgfqpoint{0.500000in}{1.500000in}}%
\pgfpathmoveto{\pgfqpoint{-0.444444in}{0.444444in}}%
\pgfpathlineto{\pgfqpoint{0.555556in}{1.444444in}}%
\pgfpathmoveto{\pgfqpoint{-0.388889in}{0.388889in}}%
\pgfpathlineto{\pgfqpoint{0.611111in}{1.388889in}}%
\pgfpathmoveto{\pgfqpoint{-0.333333in}{0.333333in}}%
\pgfpathlineto{\pgfqpoint{0.666667in}{1.333333in}}%
\pgfpathmoveto{\pgfqpoint{-0.277778in}{0.277778in}}%
\pgfpathlineto{\pgfqpoint{0.722222in}{1.277778in}}%
\pgfpathmoveto{\pgfqpoint{-0.222222in}{0.222222in}}%
\pgfpathlineto{\pgfqpoint{0.777778in}{1.222222in}}%
\pgfpathmoveto{\pgfqpoint{-0.166667in}{0.166667in}}%
\pgfpathlineto{\pgfqpoint{0.833333in}{1.166667in}}%
\pgfpathmoveto{\pgfqpoint{-0.111111in}{0.111111in}}%
\pgfpathlineto{\pgfqpoint{0.888889in}{1.111111in}}%
\pgfpathmoveto{\pgfqpoint{-0.055556in}{0.055556in}}%
\pgfpathlineto{\pgfqpoint{0.944444in}{1.055556in}}%
\pgfpathmoveto{\pgfqpoint{0.000000in}{0.000000in}}%
\pgfpathlineto{\pgfqpoint{1.000000in}{1.000000in}}%
\pgfpathmoveto{\pgfqpoint{0.055556in}{-0.055556in}}%
\pgfpathlineto{\pgfqpoint{1.055556in}{0.944444in}}%
\pgfpathmoveto{\pgfqpoint{0.111111in}{-0.111111in}}%
\pgfpathlineto{\pgfqpoint{1.111111in}{0.888889in}}%
\pgfpathmoveto{\pgfqpoint{0.166667in}{-0.166667in}}%
\pgfpathlineto{\pgfqpoint{1.166667in}{0.833333in}}%
\pgfpathmoveto{\pgfqpoint{0.222222in}{-0.222222in}}%
\pgfpathlineto{\pgfqpoint{1.222222in}{0.777778in}}%
\pgfpathmoveto{\pgfqpoint{0.277778in}{-0.277778in}}%
\pgfpathlineto{\pgfqpoint{1.277778in}{0.722222in}}%
\pgfpathmoveto{\pgfqpoint{0.333333in}{-0.333333in}}%
\pgfpathlineto{\pgfqpoint{1.333333in}{0.666667in}}%
\pgfpathmoveto{\pgfqpoint{0.388889in}{-0.388889in}}%
\pgfpathlineto{\pgfqpoint{1.388889in}{0.611111in}}%
\pgfpathmoveto{\pgfqpoint{0.444444in}{-0.444444in}}%
\pgfpathlineto{\pgfqpoint{1.444444in}{0.555556in}}%
\pgfpathmoveto{\pgfqpoint{0.500000in}{-0.500000in}}%
\pgfpathlineto{\pgfqpoint{1.500000in}{0.500000in}}%
\pgfusepath{stroke}%
\end{pgfscope}%
}%
\pgfsys@transformshift{1.359100in}{1.022500in}%
\pgfsys@useobject{currentpattern}{}%
\pgfsys@transformshift{1in}{0in}%
\pgfsys@transformshift{-1in}{0in}%
\pgfsys@transformshift{0in}{1in}%
\end{pgfscope}%
\begin{pgfscope}%
\pgfpathrectangle{\pgfqpoint{1.230000in}{1.022500in}}{\pgfqpoint{6.455000in}{3.359167in}}%
\pgfusepath{clip}%
\pgfsetbuttcap%
\pgfsetmiterjoin%
\definecolor{currentfill}{rgb}{0.347059,0.458824,0.641176}%
\pgfsetfillcolor{currentfill}%
\pgfsetlinewidth{1.003750pt}%
\definecolor{currentstroke}{rgb}{1.000000,1.000000,1.000000}%
\pgfsetstrokecolor{currentstroke}%
\pgfsetdash{}{0pt}%
\pgfpathmoveto{\pgfqpoint{2.650100in}{1.022500in}}%
\pgfpathlineto{\pgfqpoint{3.166500in}{1.022500in}}%
\pgfpathlineto{\pgfqpoint{3.166500in}{1.209467in}}%
\pgfpathlineto{\pgfqpoint{2.650100in}{1.209467in}}%
\pgfpathclose%
\pgfusepath{stroke,fill}%
\end{pgfscope}%
\begin{pgfscope}%
\pgfsetbuttcap%
\pgfsetmiterjoin%
\definecolor{currentfill}{rgb}{0.347059,0.458824,0.641176}%
\pgfsetfillcolor{currentfill}%
\pgfsetlinewidth{1.003750pt}%
\definecolor{currentstroke}{rgb}{1.000000,1.000000,1.000000}%
\pgfsetstrokecolor{currentstroke}%
\pgfsetdash{}{0pt}%
\pgfpathrectangle{\pgfqpoint{1.230000in}{1.022500in}}{\pgfqpoint{6.455000in}{3.359167in}}%
\pgfusepath{clip}%
\pgfpathmoveto{\pgfqpoint{2.650100in}{1.022500in}}%
\pgfpathlineto{\pgfqpoint{3.166500in}{1.022500in}}%
\pgfpathlineto{\pgfqpoint{3.166500in}{1.209467in}}%
\pgfpathlineto{\pgfqpoint{2.650100in}{1.209467in}}%
\pgfpathclose%
\pgfusepath{clip}%
\pgfsys@defobject{currentpattern}{\pgfqpoint{0in}{0in}}{\pgfqpoint{1in}{1in}}{%
\begin{pgfscope}%
\pgfpathrectangle{\pgfqpoint{0in}{0in}}{\pgfqpoint{1in}{1in}}%
\pgfusepath{clip}%
\pgfpathmoveto{\pgfqpoint{-0.500000in}{0.500000in}}%
\pgfpathlineto{\pgfqpoint{0.500000in}{1.500000in}}%
\pgfpathmoveto{\pgfqpoint{-0.444444in}{0.444444in}}%
\pgfpathlineto{\pgfqpoint{0.555556in}{1.444444in}}%
\pgfpathmoveto{\pgfqpoint{-0.388889in}{0.388889in}}%
\pgfpathlineto{\pgfqpoint{0.611111in}{1.388889in}}%
\pgfpathmoveto{\pgfqpoint{-0.333333in}{0.333333in}}%
\pgfpathlineto{\pgfqpoint{0.666667in}{1.333333in}}%
\pgfpathmoveto{\pgfqpoint{-0.277778in}{0.277778in}}%
\pgfpathlineto{\pgfqpoint{0.722222in}{1.277778in}}%
\pgfpathmoveto{\pgfqpoint{-0.222222in}{0.222222in}}%
\pgfpathlineto{\pgfqpoint{0.777778in}{1.222222in}}%
\pgfpathmoveto{\pgfqpoint{-0.166667in}{0.166667in}}%
\pgfpathlineto{\pgfqpoint{0.833333in}{1.166667in}}%
\pgfpathmoveto{\pgfqpoint{-0.111111in}{0.111111in}}%
\pgfpathlineto{\pgfqpoint{0.888889in}{1.111111in}}%
\pgfpathmoveto{\pgfqpoint{-0.055556in}{0.055556in}}%
\pgfpathlineto{\pgfqpoint{0.944444in}{1.055556in}}%
\pgfpathmoveto{\pgfqpoint{0.000000in}{0.000000in}}%
\pgfpathlineto{\pgfqpoint{1.000000in}{1.000000in}}%
\pgfpathmoveto{\pgfqpoint{0.055556in}{-0.055556in}}%
\pgfpathlineto{\pgfqpoint{1.055556in}{0.944444in}}%
\pgfpathmoveto{\pgfqpoint{0.111111in}{-0.111111in}}%
\pgfpathlineto{\pgfqpoint{1.111111in}{0.888889in}}%
\pgfpathmoveto{\pgfqpoint{0.166667in}{-0.166667in}}%
\pgfpathlineto{\pgfqpoint{1.166667in}{0.833333in}}%
\pgfpathmoveto{\pgfqpoint{0.222222in}{-0.222222in}}%
\pgfpathlineto{\pgfqpoint{1.222222in}{0.777778in}}%
\pgfpathmoveto{\pgfqpoint{0.277778in}{-0.277778in}}%
\pgfpathlineto{\pgfqpoint{1.277778in}{0.722222in}}%
\pgfpathmoveto{\pgfqpoint{0.333333in}{-0.333333in}}%
\pgfpathlineto{\pgfqpoint{1.333333in}{0.666667in}}%
\pgfpathmoveto{\pgfqpoint{0.388889in}{-0.388889in}}%
\pgfpathlineto{\pgfqpoint{1.388889in}{0.611111in}}%
\pgfpathmoveto{\pgfqpoint{0.444444in}{-0.444444in}}%
\pgfpathlineto{\pgfqpoint{1.444444in}{0.555556in}}%
\pgfpathmoveto{\pgfqpoint{0.500000in}{-0.500000in}}%
\pgfpathlineto{\pgfqpoint{1.500000in}{0.500000in}}%
\pgfusepath{stroke}%
\end{pgfscope}%
}%
\pgfsys@transformshift{2.650100in}{1.022500in}%
\pgfsys@useobject{currentpattern}{}%
\pgfsys@transformshift{1in}{0in}%
\pgfsys@transformshift{-1in}{0in}%
\pgfsys@transformshift{0in}{1in}%
\end{pgfscope}%
\begin{pgfscope}%
\pgfpathrectangle{\pgfqpoint{1.230000in}{1.022500in}}{\pgfqpoint{6.455000in}{3.359167in}}%
\pgfusepath{clip}%
\pgfsetbuttcap%
\pgfsetmiterjoin%
\definecolor{currentfill}{rgb}{0.347059,0.458824,0.641176}%
\pgfsetfillcolor{currentfill}%
\pgfsetlinewidth{1.003750pt}%
\definecolor{currentstroke}{rgb}{1.000000,1.000000,1.000000}%
\pgfsetstrokecolor{currentstroke}%
\pgfsetdash{}{0pt}%
\pgfpathmoveto{\pgfqpoint{3.941100in}{1.022500in}}%
\pgfpathlineto{\pgfqpoint{4.457500in}{1.022500in}}%
\pgfpathlineto{\pgfqpoint{4.457500in}{2.030042in}}%
\pgfpathlineto{\pgfqpoint{3.941100in}{2.030042in}}%
\pgfpathclose%
\pgfusepath{stroke,fill}%
\end{pgfscope}%
\begin{pgfscope}%
\pgfsetbuttcap%
\pgfsetmiterjoin%
\definecolor{currentfill}{rgb}{0.347059,0.458824,0.641176}%
\pgfsetfillcolor{currentfill}%
\pgfsetlinewidth{1.003750pt}%
\definecolor{currentstroke}{rgb}{1.000000,1.000000,1.000000}%
\pgfsetstrokecolor{currentstroke}%
\pgfsetdash{}{0pt}%
\pgfpathrectangle{\pgfqpoint{1.230000in}{1.022500in}}{\pgfqpoint{6.455000in}{3.359167in}}%
\pgfusepath{clip}%
\pgfpathmoveto{\pgfqpoint{3.941100in}{1.022500in}}%
\pgfpathlineto{\pgfqpoint{4.457500in}{1.022500in}}%
\pgfpathlineto{\pgfqpoint{4.457500in}{2.030042in}}%
\pgfpathlineto{\pgfqpoint{3.941100in}{2.030042in}}%
\pgfpathclose%
\pgfusepath{clip}%
\pgfsys@defobject{currentpattern}{\pgfqpoint{0in}{0in}}{\pgfqpoint{1in}{1in}}{%
\begin{pgfscope}%
\pgfpathrectangle{\pgfqpoint{0in}{0in}}{\pgfqpoint{1in}{1in}}%
\pgfusepath{clip}%
\pgfpathmoveto{\pgfqpoint{-0.500000in}{0.500000in}}%
\pgfpathlineto{\pgfqpoint{0.500000in}{1.500000in}}%
\pgfpathmoveto{\pgfqpoint{-0.444444in}{0.444444in}}%
\pgfpathlineto{\pgfqpoint{0.555556in}{1.444444in}}%
\pgfpathmoveto{\pgfqpoint{-0.388889in}{0.388889in}}%
\pgfpathlineto{\pgfqpoint{0.611111in}{1.388889in}}%
\pgfpathmoveto{\pgfqpoint{-0.333333in}{0.333333in}}%
\pgfpathlineto{\pgfqpoint{0.666667in}{1.333333in}}%
\pgfpathmoveto{\pgfqpoint{-0.277778in}{0.277778in}}%
\pgfpathlineto{\pgfqpoint{0.722222in}{1.277778in}}%
\pgfpathmoveto{\pgfqpoint{-0.222222in}{0.222222in}}%
\pgfpathlineto{\pgfqpoint{0.777778in}{1.222222in}}%
\pgfpathmoveto{\pgfqpoint{-0.166667in}{0.166667in}}%
\pgfpathlineto{\pgfqpoint{0.833333in}{1.166667in}}%
\pgfpathmoveto{\pgfqpoint{-0.111111in}{0.111111in}}%
\pgfpathlineto{\pgfqpoint{0.888889in}{1.111111in}}%
\pgfpathmoveto{\pgfqpoint{-0.055556in}{0.055556in}}%
\pgfpathlineto{\pgfqpoint{0.944444in}{1.055556in}}%
\pgfpathmoveto{\pgfqpoint{0.000000in}{0.000000in}}%
\pgfpathlineto{\pgfqpoint{1.000000in}{1.000000in}}%
\pgfpathmoveto{\pgfqpoint{0.055556in}{-0.055556in}}%
\pgfpathlineto{\pgfqpoint{1.055556in}{0.944444in}}%
\pgfpathmoveto{\pgfqpoint{0.111111in}{-0.111111in}}%
\pgfpathlineto{\pgfqpoint{1.111111in}{0.888889in}}%
\pgfpathmoveto{\pgfqpoint{0.166667in}{-0.166667in}}%
\pgfpathlineto{\pgfqpoint{1.166667in}{0.833333in}}%
\pgfpathmoveto{\pgfqpoint{0.222222in}{-0.222222in}}%
\pgfpathlineto{\pgfqpoint{1.222222in}{0.777778in}}%
\pgfpathmoveto{\pgfqpoint{0.277778in}{-0.277778in}}%
\pgfpathlineto{\pgfqpoint{1.277778in}{0.722222in}}%
\pgfpathmoveto{\pgfqpoint{0.333333in}{-0.333333in}}%
\pgfpathlineto{\pgfqpoint{1.333333in}{0.666667in}}%
\pgfpathmoveto{\pgfqpoint{0.388889in}{-0.388889in}}%
\pgfpathlineto{\pgfqpoint{1.388889in}{0.611111in}}%
\pgfpathmoveto{\pgfqpoint{0.444444in}{-0.444444in}}%
\pgfpathlineto{\pgfqpoint{1.444444in}{0.555556in}}%
\pgfpathmoveto{\pgfqpoint{0.500000in}{-0.500000in}}%
\pgfpathlineto{\pgfqpoint{1.500000in}{0.500000in}}%
\pgfusepath{stroke}%
\end{pgfscope}%
}%
\pgfsys@transformshift{3.941100in}{1.022500in}%
\pgfsys@useobject{currentpattern}{}%
\pgfsys@transformshift{1in}{0in}%
\pgfsys@transformshift{-1in}{0in}%
\pgfsys@transformshift{0in}{1in}%
\pgfsys@useobject{currentpattern}{}%
\pgfsys@transformshift{1in}{0in}%
\pgfsys@transformshift{-1in}{0in}%
\pgfsys@transformshift{0in}{1in}%
\end{pgfscope}%
\begin{pgfscope}%
\pgfpathrectangle{\pgfqpoint{1.230000in}{1.022500in}}{\pgfqpoint{6.455000in}{3.359167in}}%
\pgfusepath{clip}%
\pgfsetbuttcap%
\pgfsetmiterjoin%
\definecolor{currentfill}{rgb}{0.347059,0.458824,0.641176}%
\pgfsetfillcolor{currentfill}%
\pgfsetlinewidth{1.003750pt}%
\definecolor{currentstroke}{rgb}{1.000000,1.000000,1.000000}%
\pgfsetstrokecolor{currentstroke}%
\pgfsetdash{}{0pt}%
\pgfpathmoveto{\pgfqpoint{5.232100in}{1.022500in}}%
\pgfpathlineto{\pgfqpoint{5.748500in}{1.022500in}}%
\pgfpathlineto{\pgfqpoint{5.748500in}{1.635335in}}%
\pgfpathlineto{\pgfqpoint{5.232100in}{1.635335in}}%
\pgfpathclose%
\pgfusepath{stroke,fill}%
\end{pgfscope}%
\begin{pgfscope}%
\pgfsetbuttcap%
\pgfsetmiterjoin%
\definecolor{currentfill}{rgb}{0.347059,0.458824,0.641176}%
\pgfsetfillcolor{currentfill}%
\pgfsetlinewidth{1.003750pt}%
\definecolor{currentstroke}{rgb}{1.000000,1.000000,1.000000}%
\pgfsetstrokecolor{currentstroke}%
\pgfsetdash{}{0pt}%
\pgfpathrectangle{\pgfqpoint{1.230000in}{1.022500in}}{\pgfqpoint{6.455000in}{3.359167in}}%
\pgfusepath{clip}%
\pgfpathmoveto{\pgfqpoint{5.232100in}{1.022500in}}%
\pgfpathlineto{\pgfqpoint{5.748500in}{1.022500in}}%
\pgfpathlineto{\pgfqpoint{5.748500in}{1.635335in}}%
\pgfpathlineto{\pgfqpoint{5.232100in}{1.635335in}}%
\pgfpathclose%
\pgfusepath{clip}%
\pgfsys@defobject{currentpattern}{\pgfqpoint{0in}{0in}}{\pgfqpoint{1in}{1in}}{%
\begin{pgfscope}%
\pgfpathrectangle{\pgfqpoint{0in}{0in}}{\pgfqpoint{1in}{1in}}%
\pgfusepath{clip}%
\pgfpathmoveto{\pgfqpoint{-0.500000in}{0.500000in}}%
\pgfpathlineto{\pgfqpoint{0.500000in}{1.500000in}}%
\pgfpathmoveto{\pgfqpoint{-0.444444in}{0.444444in}}%
\pgfpathlineto{\pgfqpoint{0.555556in}{1.444444in}}%
\pgfpathmoveto{\pgfqpoint{-0.388889in}{0.388889in}}%
\pgfpathlineto{\pgfqpoint{0.611111in}{1.388889in}}%
\pgfpathmoveto{\pgfqpoint{-0.333333in}{0.333333in}}%
\pgfpathlineto{\pgfqpoint{0.666667in}{1.333333in}}%
\pgfpathmoveto{\pgfqpoint{-0.277778in}{0.277778in}}%
\pgfpathlineto{\pgfqpoint{0.722222in}{1.277778in}}%
\pgfpathmoveto{\pgfqpoint{-0.222222in}{0.222222in}}%
\pgfpathlineto{\pgfqpoint{0.777778in}{1.222222in}}%
\pgfpathmoveto{\pgfqpoint{-0.166667in}{0.166667in}}%
\pgfpathlineto{\pgfqpoint{0.833333in}{1.166667in}}%
\pgfpathmoveto{\pgfqpoint{-0.111111in}{0.111111in}}%
\pgfpathlineto{\pgfqpoint{0.888889in}{1.111111in}}%
\pgfpathmoveto{\pgfqpoint{-0.055556in}{0.055556in}}%
\pgfpathlineto{\pgfqpoint{0.944444in}{1.055556in}}%
\pgfpathmoveto{\pgfqpoint{0.000000in}{0.000000in}}%
\pgfpathlineto{\pgfqpoint{1.000000in}{1.000000in}}%
\pgfpathmoveto{\pgfqpoint{0.055556in}{-0.055556in}}%
\pgfpathlineto{\pgfqpoint{1.055556in}{0.944444in}}%
\pgfpathmoveto{\pgfqpoint{0.111111in}{-0.111111in}}%
\pgfpathlineto{\pgfqpoint{1.111111in}{0.888889in}}%
\pgfpathmoveto{\pgfqpoint{0.166667in}{-0.166667in}}%
\pgfpathlineto{\pgfqpoint{1.166667in}{0.833333in}}%
\pgfpathmoveto{\pgfqpoint{0.222222in}{-0.222222in}}%
\pgfpathlineto{\pgfqpoint{1.222222in}{0.777778in}}%
\pgfpathmoveto{\pgfqpoint{0.277778in}{-0.277778in}}%
\pgfpathlineto{\pgfqpoint{1.277778in}{0.722222in}}%
\pgfpathmoveto{\pgfqpoint{0.333333in}{-0.333333in}}%
\pgfpathlineto{\pgfqpoint{1.333333in}{0.666667in}}%
\pgfpathmoveto{\pgfqpoint{0.388889in}{-0.388889in}}%
\pgfpathlineto{\pgfqpoint{1.388889in}{0.611111in}}%
\pgfpathmoveto{\pgfqpoint{0.444444in}{-0.444444in}}%
\pgfpathlineto{\pgfqpoint{1.444444in}{0.555556in}}%
\pgfpathmoveto{\pgfqpoint{0.500000in}{-0.500000in}}%
\pgfpathlineto{\pgfqpoint{1.500000in}{0.500000in}}%
\pgfusepath{stroke}%
\end{pgfscope}%
}%
\pgfsys@transformshift{5.232100in}{1.022500in}%
\pgfsys@useobject{currentpattern}{}%
\pgfsys@transformshift{1in}{0in}%
\pgfsys@transformshift{-1in}{0in}%
\pgfsys@transformshift{0in}{1in}%
\end{pgfscope}%
\begin{pgfscope}%
\pgfpathrectangle{\pgfqpoint{1.230000in}{1.022500in}}{\pgfqpoint{6.455000in}{3.359167in}}%
\pgfusepath{clip}%
\pgfsetbuttcap%
\pgfsetmiterjoin%
\definecolor{currentfill}{rgb}{0.347059,0.458824,0.641176}%
\pgfsetfillcolor{currentfill}%
\pgfsetlinewidth{1.003750pt}%
\definecolor{currentstroke}{rgb}{1.000000,1.000000,1.000000}%
\pgfsetstrokecolor{currentstroke}%
\pgfsetdash{}{0pt}%
\pgfpathmoveto{\pgfqpoint{6.523100in}{1.022500in}}%
\pgfpathlineto{\pgfqpoint{7.039500in}{1.022500in}}%
\pgfpathlineto{\pgfqpoint{7.039500in}{1.095209in}}%
\pgfpathlineto{\pgfqpoint{6.523100in}{1.095209in}}%
\pgfpathclose%
\pgfusepath{stroke,fill}%
\end{pgfscope}%
\begin{pgfscope}%
\pgfsetbuttcap%
\pgfsetmiterjoin%
\definecolor{currentfill}{rgb}{0.347059,0.458824,0.641176}%
\pgfsetfillcolor{currentfill}%
\pgfsetlinewidth{1.003750pt}%
\definecolor{currentstroke}{rgb}{1.000000,1.000000,1.000000}%
\pgfsetstrokecolor{currentstroke}%
\pgfsetdash{}{0pt}%
\pgfpathrectangle{\pgfqpoint{1.230000in}{1.022500in}}{\pgfqpoint{6.455000in}{3.359167in}}%
\pgfusepath{clip}%
\pgfpathmoveto{\pgfqpoint{6.523100in}{1.022500in}}%
\pgfpathlineto{\pgfqpoint{7.039500in}{1.022500in}}%
\pgfpathlineto{\pgfqpoint{7.039500in}{1.095209in}}%
\pgfpathlineto{\pgfqpoint{6.523100in}{1.095209in}}%
\pgfpathclose%
\pgfusepath{clip}%
\pgfsys@defobject{currentpattern}{\pgfqpoint{0in}{0in}}{\pgfqpoint{1in}{1in}}{%
\begin{pgfscope}%
\pgfpathrectangle{\pgfqpoint{0in}{0in}}{\pgfqpoint{1in}{1in}}%
\pgfusepath{clip}%
\pgfpathmoveto{\pgfqpoint{-0.500000in}{0.500000in}}%
\pgfpathlineto{\pgfqpoint{0.500000in}{1.500000in}}%
\pgfpathmoveto{\pgfqpoint{-0.444444in}{0.444444in}}%
\pgfpathlineto{\pgfqpoint{0.555556in}{1.444444in}}%
\pgfpathmoveto{\pgfqpoint{-0.388889in}{0.388889in}}%
\pgfpathlineto{\pgfqpoint{0.611111in}{1.388889in}}%
\pgfpathmoveto{\pgfqpoint{-0.333333in}{0.333333in}}%
\pgfpathlineto{\pgfqpoint{0.666667in}{1.333333in}}%
\pgfpathmoveto{\pgfqpoint{-0.277778in}{0.277778in}}%
\pgfpathlineto{\pgfqpoint{0.722222in}{1.277778in}}%
\pgfpathmoveto{\pgfqpoint{-0.222222in}{0.222222in}}%
\pgfpathlineto{\pgfqpoint{0.777778in}{1.222222in}}%
\pgfpathmoveto{\pgfqpoint{-0.166667in}{0.166667in}}%
\pgfpathlineto{\pgfqpoint{0.833333in}{1.166667in}}%
\pgfpathmoveto{\pgfqpoint{-0.111111in}{0.111111in}}%
\pgfpathlineto{\pgfqpoint{0.888889in}{1.111111in}}%
\pgfpathmoveto{\pgfqpoint{-0.055556in}{0.055556in}}%
\pgfpathlineto{\pgfqpoint{0.944444in}{1.055556in}}%
\pgfpathmoveto{\pgfqpoint{0.000000in}{0.000000in}}%
\pgfpathlineto{\pgfqpoint{1.000000in}{1.000000in}}%
\pgfpathmoveto{\pgfqpoint{0.055556in}{-0.055556in}}%
\pgfpathlineto{\pgfqpoint{1.055556in}{0.944444in}}%
\pgfpathmoveto{\pgfqpoint{0.111111in}{-0.111111in}}%
\pgfpathlineto{\pgfqpoint{1.111111in}{0.888889in}}%
\pgfpathmoveto{\pgfqpoint{0.166667in}{-0.166667in}}%
\pgfpathlineto{\pgfqpoint{1.166667in}{0.833333in}}%
\pgfpathmoveto{\pgfqpoint{0.222222in}{-0.222222in}}%
\pgfpathlineto{\pgfqpoint{1.222222in}{0.777778in}}%
\pgfpathmoveto{\pgfqpoint{0.277778in}{-0.277778in}}%
\pgfpathlineto{\pgfqpoint{1.277778in}{0.722222in}}%
\pgfpathmoveto{\pgfqpoint{0.333333in}{-0.333333in}}%
\pgfpathlineto{\pgfqpoint{1.333333in}{0.666667in}}%
\pgfpathmoveto{\pgfqpoint{0.388889in}{-0.388889in}}%
\pgfpathlineto{\pgfqpoint{1.388889in}{0.611111in}}%
\pgfpathmoveto{\pgfqpoint{0.444444in}{-0.444444in}}%
\pgfpathlineto{\pgfqpoint{1.444444in}{0.555556in}}%
\pgfpathmoveto{\pgfqpoint{0.500000in}{-0.500000in}}%
\pgfpathlineto{\pgfqpoint{1.500000in}{0.500000in}}%
\pgfusepath{stroke}%
\end{pgfscope}%
}%
\pgfsys@transformshift{6.523100in}{1.022500in}%
\pgfsys@useobject{currentpattern}{}%
\pgfsys@transformshift{1in}{0in}%
\pgfsys@transformshift{-1in}{0in}%
\pgfsys@transformshift{0in}{1in}%
\end{pgfscope}%
\begin{pgfscope}%
\pgfpathrectangle{\pgfqpoint{1.230000in}{1.022500in}}{\pgfqpoint{6.455000in}{3.359167in}}%
\pgfusepath{clip}%
\pgfsetbuttcap%
\pgfsetmiterjoin%
\definecolor{currentfill}{rgb}{0.798529,0.536765,0.389706}%
\pgfsetfillcolor{currentfill}%
\pgfsetlinewidth{1.003750pt}%
\definecolor{currentstroke}{rgb}{1.000000,1.000000,1.000000}%
\pgfsetstrokecolor{currentstroke}%
\pgfsetdash{}{0pt}%
\pgfpathmoveto{\pgfqpoint{1.875500in}{1.022500in}}%
\pgfpathlineto{\pgfqpoint{2.391900in}{1.022500in}}%
\pgfpathlineto{\pgfqpoint{2.391900in}{1.302950in}}%
\pgfpathlineto{\pgfqpoint{1.875500in}{1.302950in}}%
\pgfpathclose%
\pgfusepath{stroke,fill}%
\end{pgfscope}%
\begin{pgfscope}%
\pgfsetbuttcap%
\pgfsetmiterjoin%
\definecolor{currentfill}{rgb}{0.798529,0.536765,0.389706}%
\pgfsetfillcolor{currentfill}%
\pgfsetlinewidth{1.003750pt}%
\definecolor{currentstroke}{rgb}{1.000000,1.000000,1.000000}%
\pgfsetstrokecolor{currentstroke}%
\pgfsetdash{}{0pt}%
\pgfpathrectangle{\pgfqpoint{1.230000in}{1.022500in}}{\pgfqpoint{6.455000in}{3.359167in}}%
\pgfusepath{clip}%
\pgfpathmoveto{\pgfqpoint{1.875500in}{1.022500in}}%
\pgfpathlineto{\pgfqpoint{2.391900in}{1.022500in}}%
\pgfpathlineto{\pgfqpoint{2.391900in}{1.302950in}}%
\pgfpathlineto{\pgfqpoint{1.875500in}{1.302950in}}%
\pgfpathclose%
\pgfusepath{clip}%
\pgfsys@defobject{currentpattern}{\pgfqpoint{0in}{0in}}{\pgfqpoint{1in}{1in}}{%
\begin{pgfscope}%
\pgfpathrectangle{\pgfqpoint{0in}{0in}}{\pgfqpoint{1in}{1in}}%
\pgfusepath{clip}%
\pgfusepath{stroke}%
\end{pgfscope}%
}%
\pgfsys@transformshift{1.875500in}{1.022500in}%
\pgfsys@useobject{currentpattern}{}%
\pgfsys@transformshift{1in}{0in}%
\pgfsys@transformshift{-1in}{0in}%
\pgfsys@transformshift{0in}{1in}%
\end{pgfscope}%
\begin{pgfscope}%
\pgfpathrectangle{\pgfqpoint{1.230000in}{1.022500in}}{\pgfqpoint{6.455000in}{3.359167in}}%
\pgfusepath{clip}%
\pgfsetbuttcap%
\pgfsetmiterjoin%
\definecolor{currentfill}{rgb}{0.798529,0.536765,0.389706}%
\pgfsetfillcolor{currentfill}%
\pgfsetlinewidth{1.003750pt}%
\definecolor{currentstroke}{rgb}{1.000000,1.000000,1.000000}%
\pgfsetstrokecolor{currentstroke}%
\pgfsetdash{}{0pt}%
\pgfpathmoveto{\pgfqpoint{3.166500in}{1.022500in}}%
\pgfpathlineto{\pgfqpoint{3.682900in}{1.022500in}}%
\pgfpathlineto{\pgfqpoint{3.682900in}{1.666496in}}%
\pgfpathlineto{\pgfqpoint{3.166500in}{1.666496in}}%
\pgfpathclose%
\pgfusepath{stroke,fill}%
\end{pgfscope}%
\begin{pgfscope}%
\pgfsetbuttcap%
\pgfsetmiterjoin%
\definecolor{currentfill}{rgb}{0.798529,0.536765,0.389706}%
\pgfsetfillcolor{currentfill}%
\pgfsetlinewidth{1.003750pt}%
\definecolor{currentstroke}{rgb}{1.000000,1.000000,1.000000}%
\pgfsetstrokecolor{currentstroke}%
\pgfsetdash{}{0pt}%
\pgfpathrectangle{\pgfqpoint{1.230000in}{1.022500in}}{\pgfqpoint{6.455000in}{3.359167in}}%
\pgfusepath{clip}%
\pgfpathmoveto{\pgfqpoint{3.166500in}{1.022500in}}%
\pgfpathlineto{\pgfqpoint{3.682900in}{1.022500in}}%
\pgfpathlineto{\pgfqpoint{3.682900in}{1.666496in}}%
\pgfpathlineto{\pgfqpoint{3.166500in}{1.666496in}}%
\pgfpathclose%
\pgfusepath{clip}%
\pgfsys@defobject{currentpattern}{\pgfqpoint{0in}{0in}}{\pgfqpoint{1in}{1in}}{%
\begin{pgfscope}%
\pgfpathrectangle{\pgfqpoint{0in}{0in}}{\pgfqpoint{1in}{1in}}%
\pgfusepath{clip}%
\pgfusepath{stroke}%
\end{pgfscope}%
}%
\pgfsys@transformshift{3.166500in}{1.022500in}%
\pgfsys@useobject{currentpattern}{}%
\pgfsys@transformshift{1in}{0in}%
\pgfsys@transformshift{-1in}{0in}%
\pgfsys@transformshift{0in}{1in}%
\end{pgfscope}%
\begin{pgfscope}%
\pgfpathrectangle{\pgfqpoint{1.230000in}{1.022500in}}{\pgfqpoint{6.455000in}{3.359167in}}%
\pgfusepath{clip}%
\pgfsetbuttcap%
\pgfsetmiterjoin%
\definecolor{currentfill}{rgb}{0.798529,0.536765,0.389706}%
\pgfsetfillcolor{currentfill}%
\pgfsetlinewidth{1.003750pt}%
\definecolor{currentstroke}{rgb}{1.000000,1.000000,1.000000}%
\pgfsetstrokecolor{currentstroke}%
\pgfsetdash{}{0pt}%
\pgfpathmoveto{\pgfqpoint{4.457500in}{1.022500in}}%
\pgfpathlineto{\pgfqpoint{4.973900in}{1.022500in}}%
\pgfpathlineto{\pgfqpoint{4.973900in}{4.221706in}}%
\pgfpathlineto{\pgfqpoint{4.457500in}{4.221706in}}%
\pgfpathclose%
\pgfusepath{stroke,fill}%
\end{pgfscope}%
\begin{pgfscope}%
\pgfsetbuttcap%
\pgfsetmiterjoin%
\definecolor{currentfill}{rgb}{0.798529,0.536765,0.389706}%
\pgfsetfillcolor{currentfill}%
\pgfsetlinewidth{1.003750pt}%
\definecolor{currentstroke}{rgb}{1.000000,1.000000,1.000000}%
\pgfsetstrokecolor{currentstroke}%
\pgfsetdash{}{0pt}%
\pgfpathrectangle{\pgfqpoint{1.230000in}{1.022500in}}{\pgfqpoint{6.455000in}{3.359167in}}%
\pgfusepath{clip}%
\pgfpathmoveto{\pgfqpoint{4.457500in}{1.022500in}}%
\pgfpathlineto{\pgfqpoint{4.973900in}{1.022500in}}%
\pgfpathlineto{\pgfqpoint{4.973900in}{4.221706in}}%
\pgfpathlineto{\pgfqpoint{4.457500in}{4.221706in}}%
\pgfpathclose%
\pgfusepath{clip}%
\pgfsys@defobject{currentpattern}{\pgfqpoint{0in}{0in}}{\pgfqpoint{1in}{1in}}{%
\begin{pgfscope}%
\pgfpathrectangle{\pgfqpoint{0in}{0in}}{\pgfqpoint{1in}{1in}}%
\pgfusepath{clip}%
\pgfusepath{stroke}%
\end{pgfscope}%
}%
\pgfsys@transformshift{4.457500in}{1.022500in}%
\pgfsys@useobject{currentpattern}{}%
\pgfsys@transformshift{1in}{0in}%
\pgfsys@transformshift{-1in}{0in}%
\pgfsys@transformshift{0in}{1in}%
\pgfsys@useobject{currentpattern}{}%
\pgfsys@transformshift{1in}{0in}%
\pgfsys@transformshift{-1in}{0in}%
\pgfsys@transformshift{0in}{1in}%
\pgfsys@useobject{currentpattern}{}%
\pgfsys@transformshift{1in}{0in}%
\pgfsys@transformshift{-1in}{0in}%
\pgfsys@transformshift{0in}{1in}%
\pgfsys@useobject{currentpattern}{}%
\pgfsys@transformshift{1in}{0in}%
\pgfsys@transformshift{-1in}{0in}%
\pgfsys@transformshift{0in}{1in}%
\end{pgfscope}%
\begin{pgfscope}%
\pgfpathrectangle{\pgfqpoint{1.230000in}{1.022500in}}{\pgfqpoint{6.455000in}{3.359167in}}%
\pgfusepath{clip}%
\pgfsetbuttcap%
\pgfsetmiterjoin%
\definecolor{currentfill}{rgb}{0.798529,0.536765,0.389706}%
\pgfsetfillcolor{currentfill}%
\pgfsetlinewidth{1.003750pt}%
\definecolor{currentstroke}{rgb}{1.000000,1.000000,1.000000}%
\pgfsetstrokecolor{currentstroke}%
\pgfsetdash{}{0pt}%
\pgfpathmoveto{\pgfqpoint{5.748500in}{1.022500in}}%
\pgfpathlineto{\pgfqpoint{6.264900in}{1.022500in}}%
\pgfpathlineto{\pgfqpoint{6.264900in}{2.840231in}}%
\pgfpathlineto{\pgfqpoint{5.748500in}{2.840231in}}%
\pgfpathclose%
\pgfusepath{stroke,fill}%
\end{pgfscope}%
\begin{pgfscope}%
\pgfsetbuttcap%
\pgfsetmiterjoin%
\definecolor{currentfill}{rgb}{0.798529,0.536765,0.389706}%
\pgfsetfillcolor{currentfill}%
\pgfsetlinewidth{1.003750pt}%
\definecolor{currentstroke}{rgb}{1.000000,1.000000,1.000000}%
\pgfsetstrokecolor{currentstroke}%
\pgfsetdash{}{0pt}%
\pgfpathrectangle{\pgfqpoint{1.230000in}{1.022500in}}{\pgfqpoint{6.455000in}{3.359167in}}%
\pgfusepath{clip}%
\pgfpathmoveto{\pgfqpoint{5.748500in}{1.022500in}}%
\pgfpathlineto{\pgfqpoint{6.264900in}{1.022500in}}%
\pgfpathlineto{\pgfqpoint{6.264900in}{2.840231in}}%
\pgfpathlineto{\pgfqpoint{5.748500in}{2.840231in}}%
\pgfpathclose%
\pgfusepath{clip}%
\pgfsys@defobject{currentpattern}{\pgfqpoint{0in}{0in}}{\pgfqpoint{1in}{1in}}{%
\begin{pgfscope}%
\pgfpathrectangle{\pgfqpoint{0in}{0in}}{\pgfqpoint{1in}{1in}}%
\pgfusepath{clip}%
\pgfusepath{stroke}%
\end{pgfscope}%
}%
\pgfsys@transformshift{5.748500in}{1.022500in}%
\pgfsys@useobject{currentpattern}{}%
\pgfsys@transformshift{1in}{0in}%
\pgfsys@transformshift{-1in}{0in}%
\pgfsys@transformshift{0in}{1in}%
\pgfsys@useobject{currentpattern}{}%
\pgfsys@transformshift{1in}{0in}%
\pgfsys@transformshift{-1in}{0in}%
\pgfsys@transformshift{0in}{1in}%
\end{pgfscope}%
\begin{pgfscope}%
\pgfpathrectangle{\pgfqpoint{1.230000in}{1.022500in}}{\pgfqpoint{6.455000in}{3.359167in}}%
\pgfusepath{clip}%
\pgfsetbuttcap%
\pgfsetmiterjoin%
\definecolor{currentfill}{rgb}{0.798529,0.536765,0.389706}%
\pgfsetfillcolor{currentfill}%
\pgfsetlinewidth{1.003750pt}%
\definecolor{currentstroke}{rgb}{1.000000,1.000000,1.000000}%
\pgfsetstrokecolor{currentstroke}%
\pgfsetdash{}{0pt}%
\pgfpathmoveto{\pgfqpoint{7.039500in}{1.022500in}}%
\pgfpathlineto{\pgfqpoint{7.555900in}{1.022500in}}%
\pgfpathlineto{\pgfqpoint{7.555900in}{1.095209in}}%
\pgfpathlineto{\pgfqpoint{7.039500in}{1.095209in}}%
\pgfpathclose%
\pgfusepath{stroke,fill}%
\end{pgfscope}%
\begin{pgfscope}%
\pgfsetbuttcap%
\pgfsetmiterjoin%
\definecolor{currentfill}{rgb}{0.798529,0.536765,0.389706}%
\pgfsetfillcolor{currentfill}%
\pgfsetlinewidth{1.003750pt}%
\definecolor{currentstroke}{rgb}{1.000000,1.000000,1.000000}%
\pgfsetstrokecolor{currentstroke}%
\pgfsetdash{}{0pt}%
\pgfpathrectangle{\pgfqpoint{1.230000in}{1.022500in}}{\pgfqpoint{6.455000in}{3.359167in}}%
\pgfusepath{clip}%
\pgfpathmoveto{\pgfqpoint{7.039500in}{1.022500in}}%
\pgfpathlineto{\pgfqpoint{7.555900in}{1.022500in}}%
\pgfpathlineto{\pgfqpoint{7.555900in}{1.095209in}}%
\pgfpathlineto{\pgfqpoint{7.039500in}{1.095209in}}%
\pgfpathclose%
\pgfusepath{clip}%
\pgfsys@defobject{currentpattern}{\pgfqpoint{0in}{0in}}{\pgfqpoint{1in}{1in}}{%
\begin{pgfscope}%
\pgfpathrectangle{\pgfqpoint{0in}{0in}}{\pgfqpoint{1in}{1in}}%
\pgfusepath{clip}%
\pgfusepath{stroke}%
\end{pgfscope}%
}%
\pgfsys@transformshift{7.039500in}{1.022500in}%
\pgfsys@useobject{currentpattern}{}%
\pgfsys@transformshift{1in}{0in}%
\pgfsys@transformshift{-1in}{0in}%
\pgfsys@transformshift{0in}{1in}%
\end{pgfscope}%
\begin{pgfscope}%
\pgfpathrectangle{\pgfqpoint{1.230000in}{1.022500in}}{\pgfqpoint{6.455000in}{3.359167in}}%
\pgfusepath{clip}%
\pgfsetroundcap%
\pgfsetroundjoin%
\pgfsetlinewidth{2.710125pt}%
\definecolor{currentstroke}{rgb}{0.260000,0.260000,0.260000}%
\pgfsetstrokecolor{currentstroke}%
\pgfsetdash{}{0pt}%
\pgfusepath{stroke}%
\end{pgfscope}%
\begin{pgfscope}%
\pgfpathrectangle{\pgfqpoint{1.230000in}{1.022500in}}{\pgfqpoint{6.455000in}{3.359167in}}%
\pgfusepath{clip}%
\pgfsetroundcap%
\pgfsetroundjoin%
\pgfsetlinewidth{2.710125pt}%
\definecolor{currentstroke}{rgb}{0.260000,0.260000,0.260000}%
\pgfsetstrokecolor{currentstroke}%
\pgfsetdash{}{0pt}%
\pgfusepath{stroke}%
\end{pgfscope}%
\begin{pgfscope}%
\pgfpathrectangle{\pgfqpoint{1.230000in}{1.022500in}}{\pgfqpoint{6.455000in}{3.359167in}}%
\pgfusepath{clip}%
\pgfsetroundcap%
\pgfsetroundjoin%
\pgfsetlinewidth{2.710125pt}%
\definecolor{currentstroke}{rgb}{0.260000,0.260000,0.260000}%
\pgfsetstrokecolor{currentstroke}%
\pgfsetdash{}{0pt}%
\pgfusepath{stroke}%
\end{pgfscope}%
\begin{pgfscope}%
\pgfpathrectangle{\pgfqpoint{1.230000in}{1.022500in}}{\pgfqpoint{6.455000in}{3.359167in}}%
\pgfusepath{clip}%
\pgfsetroundcap%
\pgfsetroundjoin%
\pgfsetlinewidth{2.710125pt}%
\definecolor{currentstroke}{rgb}{0.260000,0.260000,0.260000}%
\pgfsetstrokecolor{currentstroke}%
\pgfsetdash{}{0pt}%
\pgfusepath{stroke}%
\end{pgfscope}%
\begin{pgfscope}%
\pgfpathrectangle{\pgfqpoint{1.230000in}{1.022500in}}{\pgfqpoint{6.455000in}{3.359167in}}%
\pgfusepath{clip}%
\pgfsetroundcap%
\pgfsetroundjoin%
\pgfsetlinewidth{2.710125pt}%
\definecolor{currentstroke}{rgb}{0.260000,0.260000,0.260000}%
\pgfsetstrokecolor{currentstroke}%
\pgfsetdash{}{0pt}%
\pgfusepath{stroke}%
\end{pgfscope}%
\begin{pgfscope}%
\pgfpathrectangle{\pgfqpoint{1.230000in}{1.022500in}}{\pgfqpoint{6.455000in}{3.359167in}}%
\pgfusepath{clip}%
\pgfsetroundcap%
\pgfsetroundjoin%
\pgfsetlinewidth{2.710125pt}%
\definecolor{currentstroke}{rgb}{0.260000,0.260000,0.260000}%
\pgfsetstrokecolor{currentstroke}%
\pgfsetdash{}{0pt}%
\pgfusepath{stroke}%
\end{pgfscope}%
\begin{pgfscope}%
\pgfpathrectangle{\pgfqpoint{1.230000in}{1.022500in}}{\pgfqpoint{6.455000in}{3.359167in}}%
\pgfusepath{clip}%
\pgfsetroundcap%
\pgfsetroundjoin%
\pgfsetlinewidth{2.710125pt}%
\definecolor{currentstroke}{rgb}{0.260000,0.260000,0.260000}%
\pgfsetstrokecolor{currentstroke}%
\pgfsetdash{}{0pt}%
\pgfusepath{stroke}%
\end{pgfscope}%
\begin{pgfscope}%
\pgfpathrectangle{\pgfqpoint{1.230000in}{1.022500in}}{\pgfqpoint{6.455000in}{3.359167in}}%
\pgfusepath{clip}%
\pgfsetroundcap%
\pgfsetroundjoin%
\pgfsetlinewidth{2.710125pt}%
\definecolor{currentstroke}{rgb}{0.260000,0.260000,0.260000}%
\pgfsetstrokecolor{currentstroke}%
\pgfsetdash{}{0pt}%
\pgfusepath{stroke}%
\end{pgfscope}%
\begin{pgfscope}%
\pgfpathrectangle{\pgfqpoint{1.230000in}{1.022500in}}{\pgfqpoint{6.455000in}{3.359167in}}%
\pgfusepath{clip}%
\pgfsetroundcap%
\pgfsetroundjoin%
\pgfsetlinewidth{2.710125pt}%
\definecolor{currentstroke}{rgb}{0.260000,0.260000,0.260000}%
\pgfsetstrokecolor{currentstroke}%
\pgfsetdash{}{0pt}%
\pgfusepath{stroke}%
\end{pgfscope}%
\begin{pgfscope}%
\pgfpathrectangle{\pgfqpoint{1.230000in}{1.022500in}}{\pgfqpoint{6.455000in}{3.359167in}}%
\pgfusepath{clip}%
\pgfsetroundcap%
\pgfsetroundjoin%
\pgfsetlinewidth{2.710125pt}%
\definecolor{currentstroke}{rgb}{0.260000,0.260000,0.260000}%
\pgfsetstrokecolor{currentstroke}%
\pgfsetdash{}{0pt}%
\pgfusepath{stroke}%
\end{pgfscope}%
\begin{pgfscope}%
\pgfsetrectcap%
\pgfsetmiterjoin%
\pgfsetlinewidth{1.254687pt}%
\definecolor{currentstroke}{rgb}{0.800000,0.800000,0.800000}%
\pgfsetstrokecolor{currentstroke}%
\pgfsetdash{}{0pt}%
\pgfpathmoveto{\pgfqpoint{1.230000in}{1.022500in}}%
\pgfpathlineto{\pgfqpoint{7.685000in}{1.022500in}}%
\pgfusepath{stroke}%
\end{pgfscope}%
\begin{pgfscope}%
\definecolor{textcolor}{rgb}{0.150000,0.150000,0.150000}%
\pgfsetstrokecolor{textcolor}%
\pgfsetfillcolor{textcolor}%
\pgftext[x=4.457500in,y=4.465000in,,base]{\color{textcolor}\sffamily\fontsize{21.000000}{25.200000}\selectfont libseccomp to eBPF benchmarks}%
\end{pgfscope}%
\begin{pgfscope}%
\pgfsetbuttcap%
\pgfsetmiterjoin%
\definecolor{currentfill}{rgb}{1.000000,1.000000,1.000000}%
\pgfsetfillcolor{currentfill}%
\pgfsetfillopacity{0.800000}%
\pgfsetlinewidth{1.003750pt}%
\definecolor{currentstroke}{rgb}{0.800000,0.800000,0.800000}%
\pgfsetstrokecolor{currentstroke}%
\pgfsetstrokeopacity{0.800000}%
\pgfsetdash{}{0pt}%
\pgfpathmoveto{\pgfqpoint{5.420190in}{3.388281in}}%
\pgfpathlineto{\pgfqpoint{7.497847in}{3.388281in}}%
\pgfpathquadraticcurveto{\pgfqpoint{7.551319in}{3.388281in}}{\pgfqpoint{7.551319in}{3.441754in}}%
\pgfpathlineto{\pgfqpoint{7.551319in}{4.194514in}}%
\pgfpathquadraticcurveto{\pgfqpoint{7.551319in}{4.247986in}}{\pgfqpoint{7.497847in}{4.247986in}}%
\pgfpathlineto{\pgfqpoint{5.420190in}{4.247986in}}%
\pgfpathquadraticcurveto{\pgfqpoint{5.366718in}{4.247986in}}{\pgfqpoint{5.366718in}{4.194514in}}%
\pgfpathlineto{\pgfqpoint{5.366718in}{3.441754in}}%
\pgfpathquadraticcurveto{\pgfqpoint{5.366718in}{3.388281in}}{\pgfqpoint{5.420190in}{3.388281in}}%
\pgfpathclose%
\pgfusepath{stroke,fill}%
\end{pgfscope}%
\begin{pgfscope}%
\pgfsetbuttcap%
\pgfsetmiterjoin%
\definecolor{currentfill}{rgb}{0.347059,0.458824,0.641176}%
\pgfsetfillcolor{currentfill}%
\pgfsetlinewidth{1.003750pt}%
\definecolor{currentstroke}{rgb}{1.000000,1.000000,1.000000}%
\pgfsetstrokecolor{currentstroke}%
\pgfsetdash{}{0pt}%
\pgfpathmoveto{\pgfqpoint{5.473662in}{3.941003in}}%
\pgfpathlineto{\pgfqpoint{6.008384in}{3.941003in}}%
\pgfpathlineto{\pgfqpoint{6.008384in}{4.128156in}}%
\pgfpathlineto{\pgfqpoint{5.473662in}{4.128156in}}%
\pgfpathclose%
\pgfusepath{stroke,fill}%
\end{pgfscope}%
\begin{pgfscope}%
\pgfsetbuttcap%
\pgfsetmiterjoin%
\definecolor{currentfill}{rgb}{0.347059,0.458824,0.641176}%
\pgfsetfillcolor{currentfill}%
\pgfsetlinewidth{1.003750pt}%
\definecolor{currentstroke}{rgb}{1.000000,1.000000,1.000000}%
\pgfsetstrokecolor{currentstroke}%
\pgfsetdash{}{0pt}%
\pgfpathmoveto{\pgfqpoint{5.473662in}{3.941003in}}%
\pgfpathlineto{\pgfqpoint{6.008384in}{3.941003in}}%
\pgfpathlineto{\pgfqpoint{6.008384in}{4.128156in}}%
\pgfpathlineto{\pgfqpoint{5.473662in}{4.128156in}}%
\pgfpathclose%
\pgfusepath{clip}%
\pgfsys@defobject{currentpattern}{\pgfqpoint{0in}{0in}}{\pgfqpoint{1in}{1in}}{%
\begin{pgfscope}%
\pgfpathrectangle{\pgfqpoint{0in}{0in}}{\pgfqpoint{1in}{1in}}%
\pgfusepath{clip}%
\pgfpathmoveto{\pgfqpoint{-0.500000in}{0.500000in}}%
\pgfpathlineto{\pgfqpoint{0.500000in}{1.500000in}}%
\pgfpathmoveto{\pgfqpoint{-0.444444in}{0.444444in}}%
\pgfpathlineto{\pgfqpoint{0.555556in}{1.444444in}}%
\pgfpathmoveto{\pgfqpoint{-0.388889in}{0.388889in}}%
\pgfpathlineto{\pgfqpoint{0.611111in}{1.388889in}}%
\pgfpathmoveto{\pgfqpoint{-0.333333in}{0.333333in}}%
\pgfpathlineto{\pgfqpoint{0.666667in}{1.333333in}}%
\pgfpathmoveto{\pgfqpoint{-0.277778in}{0.277778in}}%
\pgfpathlineto{\pgfqpoint{0.722222in}{1.277778in}}%
\pgfpathmoveto{\pgfqpoint{-0.222222in}{0.222222in}}%
\pgfpathlineto{\pgfqpoint{0.777778in}{1.222222in}}%
\pgfpathmoveto{\pgfqpoint{-0.166667in}{0.166667in}}%
\pgfpathlineto{\pgfqpoint{0.833333in}{1.166667in}}%
\pgfpathmoveto{\pgfqpoint{-0.111111in}{0.111111in}}%
\pgfpathlineto{\pgfqpoint{0.888889in}{1.111111in}}%
\pgfpathmoveto{\pgfqpoint{-0.055556in}{0.055556in}}%
\pgfpathlineto{\pgfqpoint{0.944444in}{1.055556in}}%
\pgfpathmoveto{\pgfqpoint{0.000000in}{0.000000in}}%
\pgfpathlineto{\pgfqpoint{1.000000in}{1.000000in}}%
\pgfpathmoveto{\pgfqpoint{0.055556in}{-0.055556in}}%
\pgfpathlineto{\pgfqpoint{1.055556in}{0.944444in}}%
\pgfpathmoveto{\pgfqpoint{0.111111in}{-0.111111in}}%
\pgfpathlineto{\pgfqpoint{1.111111in}{0.888889in}}%
\pgfpathmoveto{\pgfqpoint{0.166667in}{-0.166667in}}%
\pgfpathlineto{\pgfqpoint{1.166667in}{0.833333in}}%
\pgfpathmoveto{\pgfqpoint{0.222222in}{-0.222222in}}%
\pgfpathlineto{\pgfqpoint{1.222222in}{0.777778in}}%
\pgfpathmoveto{\pgfqpoint{0.277778in}{-0.277778in}}%
\pgfpathlineto{\pgfqpoint{1.277778in}{0.722222in}}%
\pgfpathmoveto{\pgfqpoint{0.333333in}{-0.333333in}}%
\pgfpathlineto{\pgfqpoint{1.333333in}{0.666667in}}%
\pgfpathmoveto{\pgfqpoint{0.388889in}{-0.388889in}}%
\pgfpathlineto{\pgfqpoint{1.388889in}{0.611111in}}%
\pgfpathmoveto{\pgfqpoint{0.444444in}{-0.444444in}}%
\pgfpathlineto{\pgfqpoint{1.444444in}{0.555556in}}%
\pgfpathmoveto{\pgfqpoint{0.500000in}{-0.500000in}}%
\pgfpathlineto{\pgfqpoint{1.500000in}{0.500000in}}%
\pgfusepath{stroke}%
\end{pgfscope}%
}%
\pgfsys@transformshift{5.473662in}{3.941003in}%
\pgfsys@useobject{currentpattern}{}%
\pgfsys@transformshift{1in}{0in}%
\pgfsys@transformshift{-1in}{0in}%
\pgfsys@transformshift{0in}{1in}%
\end{pgfscope}%
\begin{pgfscope}%
\definecolor{textcolor}{rgb}{0.150000,0.150000,0.150000}%
\pgfsetstrokecolor{textcolor}%
\pgfsetfillcolor{textcolor}%
\pgftext[x=6.222273in,y=3.941003in,left,base]{\color{textcolor}\sffamily\fontsize{19.250000}{23.100000}\selectfont libseccomp}%
\end{pgfscope}%
\begin{pgfscope}%
\pgfsetbuttcap%
\pgfsetmiterjoin%
\definecolor{currentfill}{rgb}{0.798529,0.536765,0.389706}%
\pgfsetfillcolor{currentfill}%
\pgfsetlinewidth{1.003750pt}%
\definecolor{currentstroke}{rgb}{1.000000,1.000000,1.000000}%
\pgfsetstrokecolor{currentstroke}%
\pgfsetdash{}{0pt}%
\pgfpathmoveto{\pgfqpoint{5.473662in}{3.551255in}}%
\pgfpathlineto{\pgfqpoint{6.008384in}{3.551255in}}%
\pgfpathlineto{\pgfqpoint{6.008384in}{3.738408in}}%
\pgfpathlineto{\pgfqpoint{5.473662in}{3.738408in}}%
\pgfpathclose%
\pgfusepath{stroke,fill}%
\end{pgfscope}%
\begin{pgfscope}%
\pgfsetbuttcap%
\pgfsetmiterjoin%
\definecolor{currentfill}{rgb}{0.798529,0.536765,0.389706}%
\pgfsetfillcolor{currentfill}%
\pgfsetlinewidth{1.003750pt}%
\definecolor{currentstroke}{rgb}{1.000000,1.000000,1.000000}%
\pgfsetstrokecolor{currentstroke}%
\pgfsetdash{}{0pt}%
\pgfpathmoveto{\pgfqpoint{5.473662in}{3.551255in}}%
\pgfpathlineto{\pgfqpoint{6.008384in}{3.551255in}}%
\pgfpathlineto{\pgfqpoint{6.008384in}{3.738408in}}%
\pgfpathlineto{\pgfqpoint{5.473662in}{3.738408in}}%
\pgfpathclose%
\pgfusepath{clip}%
\pgfsys@defobject{currentpattern}{\pgfqpoint{0in}{0in}}{\pgfqpoint{1in}{1in}}{%
\begin{pgfscope}%
\pgfpathrectangle{\pgfqpoint{0in}{0in}}{\pgfqpoint{1in}{1in}}%
\pgfusepath{clip}%
\pgfusepath{stroke}%
\end{pgfscope}%
}%
\pgfsys@transformshift{5.473662in}{3.551255in}%
\pgfsys@useobject{currentpattern}{}%
\pgfsys@transformshift{1in}{0in}%
\pgfsys@transformshift{-1in}{0in}%
\pgfsys@transformshift{0in}{1in}%
\end{pgfscope}%
\begin{pgfscope}%
\definecolor{textcolor}{rgb}{0.150000,0.150000,0.150000}%
\pgfsetstrokecolor{textcolor}%
\pgfsetfillcolor{textcolor}%
\pgftext[x=6.222273in,y=3.551255in,left,base]{\color{textcolor}\sffamily\fontsize{19.250000}{23.100000}\selectfont JitSynth}%
\end{pgfscope}%
\end{pgfpicture}%
\makeatother%
\endgroup%

  }
  \caption{Performance of code generated by \jitsynth compilers
  compared to existing compilers for the classic BPF to eBPF benchmarks
  (left) and the libseccomp to eBPF benchmarks (right). Measured in number
  of instructions executed.}
  \label{fig:o2b-l2b-runtime}
\end{figure}

Classic BPF is the original, simpler version of BPF used for packet filtering which
was later extended to eBPF in Linux.
%
Since many applications still use classic BPF, Linux must first compile classic BPF
to eBPF as an intermediary step before compiling to machine instructions.
%
As a second case study, we used \jitsynth to synthesize a compiler from classic BPF to eBPF.
%
Our synthesized compiler supports all classic BPF opcodes.
%
To evaluate performance, we compare against the existing
Linux classic-BPF-to-eBPF compiler.
%
Similar to the RISC-V benchmarks, we run each eBPF program
with input that is allowed by the filter.
%
Because eBPF does not run directly on hardware, we measure
the number of instructions executed instead of processor cycles.


\autoref{fig:o2b-l2b-runtime} shows the performance results.
%
Classic BPF programs generated by \jitsynth compilers execute
an average of $\CbpfSlowdown\times$ more instructions
than those compiled by Linux.


\paragraph{libseccomp to eBPF.}

libseccomp is a library used to simplify construction of BPF system call filters.
%
The existing libseccomp implementation compiles to classic BPF; we instead choose
to compile to eBPF because classic BPF has only two registers, which does not satisfy
the assumptions of \jitsynth.
%
Since libseccomp is a library and does not have distinct instructions, libseccomp itself does not meet the definition of an abstract register machine; we instead introduce
an intermediate libseccomp language which does satisfy this definition.
%
Our full libseccomp to eBPF compiler is composed of both a trusted program to translate from libseccomp to our intermediate language and a synthesized compiler from our intermediate language to eBPF.

To evaluate performance, we select a set of benchmark filters from real-world
applications that use libseccomp, and measure the number of eBPF instructions
executed for an input the filter allows.
%
Because no existing compiler exists from libseccomp to eBPF directly,
we compare against the composition of the existing libseccomp-to-classic-BPF and classic-BPF-to-eBPF compilers.


\autoref{fig:o2b-l2b-runtime} shows the performance results.
%
libseccomp programs generated by \jitsynth execute
$\LibseccompSlowdown\times$ more instructions on average
compared to the existing libseccomp-to-eBPF compiler stack.
%
However, the synthesized compiler avoids bugs previously found
in the libseccomp-to-classic-BPF compiler~\cite{lsc:bug}.
%
% This shows that \jitsynth can synthesize a compiler for a new
% source-target pair that approaches the performance of the existing
% compiler stack. % This sentence is super weird
%
% At the same time, the synthesized compiler avoids bugs previously
% found in the libseccomp-to-classic-BPF compiler~\cite{lsc:bug}.
%



% \if 0
% \autoref{fig:lang} describes the instructions in both instruction sets.
% %
% Our compiler supports a subset of
% %

% To validate that the synthesized compiler is correct, we ran the existing eBPF
% test suite in the Linux kernel.
% %
% In addition to passing these tests, our compiler also avoids bugs previously
% found in the existing eBPF to RISC-V compiler in Linux~\cite{nelson:bpf-riscv-add32-bug}.
% %
% This evidence shows that \jitsynth can synthesize reliable compilers for real-world DSLs.


% We evaluated the performance of code generated by the synthesized compiler against
% both the existing Linux eBPF to RISC-V compiler and the Linux eBPF interpreter.
% %
% Our evaluation uses the same set of benchmarks used by Jitk~\cite{wang:jitk}, which includes system call filters from widely used applications.
% %OpenSSH, vsftpd, Native Client (NaCl), QEMU, Firefox, Chrome, and Tor.
% %
% Because these benchmarks were originally for classic BPF,
% we first compile them to eBPF using the existing
% Linux classic BPF to eBPF compiler as a preprocessing step.
% %
% We ran all benchmarks on the HiFive Unleashed RISC-V development board, and measured
% the total number of processor cycles executed in the common case, for each filter
% and for each execution method.



% \autoref{fig:b2r-runtime} shows the results of the performance evaluation.
% %
% eBPF programs compiled by \jitsynth JIT compilers show an average slowdown of $\EbpfCompilerSlowdown\times$
% compared to programs compiled by the existing Linux compiler.
% %
% This overhead is from \todo{XXX}.
% %
% \jitsynth-compiled programs get an average speedup of $\EbpfInterpSpeedup\times$
% compared to interpreting the eBPF programs, preserving the benefit
% of JIT compilation.
% %
% %In addition, \jitsynth compilers avoid \todo{previously found bugs in the Linux compiler}.
% %
% This is evidence that \jitsynth can synthesize a compiler that outperforms
% %
% This shows that \jitsynth can synthesize a compiler for a real-world source-target pair,
% which out-performs the existing interpreter and nears the performance of the existing compiler.
% %
% ISA-level optimizations could reduce this performance gap between the \jitsynth compiler
% and the Linux compiler, but we consider this to be outside the scope of our work.
% %
% \fi

% We also evaluated the performance of code generated by the \jitsynth eBPF-to-RISC-V compiler, comparing it to both the existing Linux compiler as well as the Linux eBPF interpreter.
% To give a more realistic comparison, we also applied a simple NOP-removal pass on top of the \jitsynth compiler.
% This pass, done by trusted code, removes NOP instructions from code generated by the \jitsynth compiler
% while preserving the semantics of the code.
% \todo{it's odd to bring up NOP removal for the first time here.
% maybe add this part to the actual JIT (rather than a patch)?}
%To further improve the speed of code generated by the \jitsynth compiler, further optimization passes could also be made.

\if 0
\subsection{Synthesizing compilers for multiple source-target pairs}


%NB: should this subsection and the previous section be merged? they
%    seem to be making similar claims

%NB: this section is very repetitive. We probably can factor
%    the libseccomp->ebpf and cbpf->ebpf portions to minimize
%    the space it takes up

To demonstrate the generality of our approach, we also apply \jitsynth to
synthesize compilers for two additional source and target pairs:
classic BPF to eBPF, and libseccomp to eBPF.
%
Classic BPF is the original, simpler version of BPF used for packet filtering which
was later extended to eBPF in Linux.
%
Since many applications still use classic BPF, Linux must first compile classic BPF
to eBPF as an intermediary before compiling to machine instructions.
%
libseccomp is a library used to simplify construction of BPF system call filters.
%
The existing libseccomp implementation compiles to classic BPF; we instead choose
to compile to eBPF as classic BPF has only two registers, which does not satisfy
the assumptions of \jitsynth.
%
Since libseccomp is a library and does not have distinct instructions, libseccomp itself does not meet the definition of an abstract register machine; we instead introduce
an intermediate libseccomp language which does satisfy this definition.
%
Our full libseccomp to eBPF compiler is composed of both a trusted program to translate from libseccomp to our intermediate language and a synthesized compiler from our intermediate language to eBPF.


To evaluate the compilers generated by \jitsynth, we compile a set of benchmarks
to eBPF and perform a worst-case execution analysis on the generated code, and
compare against the existing implementation.
%
Because no existing compiler exists from libseccomp to eBPF directly,
we compare against the composition of the libseccomp to classic BPF implementation and the
existing Linux classic BPF to eBPF compiler.
%
The classic BPF benchmarks are the same used in Jitk; the libseccomp benchmarks
are selected from a handful of real-world applications.
%
\autoref{fig:o2b-l2b-runtime} shows the results of the analysis.
%
Classic BPF program generated by \jitsynth compilers execute
an average of $\CbpfSlowdown\times$ more instructions in the worst case
than those compiled Linux.
%
Similarly, libseccomp programs generated by \jitsynth show a
$\LibseccompSlowdown\times$ slowdown compared to the existing libseccomp
to eBPF compiler stack.
%
These results show that \jitsynth can generate compilers which generate
code that approaches the performance of existing compilers, while avoiding
bugs like those previously found in the libseccomp compiler~\cite{lsc:bug}.
%


\if 0
We evaluate our synthesized classic BPF to eBPF compiler using the same benchmarks
as in Jitk.
%
To compare our compiler against the existing Linux compiler, we compile each
benchmark to eBPF, and perform a worst-case execution analysis of the generated
code, measuring the number of eBPF instructions executed.
%
\autoref{fig:o2b-l2b-runtime} shows the results of this analysis.
%
BPF programs compiled by \jitsynth compilers execute an average of
$\CbpfSlowdown\times$ more instructions in the worst case than those
compiled by Linux.
%
This shows that the \jitsynth-synthesized compiler generates code
near the performance of that generated by Linux.
%
% The results for both source-target pairs are shown in \autoref{fig:o2b-l2b-runtime}.

% Classic BPF is the original version of BPF used for packet filtering, later expanded into eBPF.
% Since many applications still use classic BPF, compiling to eBPF is an essential step for running these filters.
% In particular, each of the applications discussed in the previous section emits classic BPF filters, which are compiled into eBPF in the kernel.
% In this section, we compare the compiler that \jitsynth synthesizes to the existing classic BPF to eBPF compiler in the Linux kernel using these same filters.
% For each filter, we evaluate the worst-case execution time of the compiled eBPF code in number of instructions.
% On average, filters compiled by the \jitsynth compiler suffer an average slowdown of \todo{$2.94\times$} compared to the Linux kernel compiler for classic BPF.

In addition to classic BPF, we also synthesize a compiler for libseccomp,
a library used to build classic BPF system call filters.
%
Though libseccomp itself compiles to classic BPF, we choose to generate a compiler from libseccomp to eBPF.
%
We do so because no compiler currently exists from libseccomp to eBPF, and because
classic BPF has only two registers, which does not meet our assumptions for \jitsynth.
%
Since libseccomp is a library and does not have distinct instructions, libseccomp itself does not match the assumptions of an abstract register machine.
%
However, we introduce an intermediate libseccomp language which does meet these assumptions.
%
Our full libseccomp to eBPF compiler is composed of both a trusted program to translate from libseccomp to our intermediate language and a synthesized compiler from our intermediate language to eBPF.
%
This compiler avoids bugs found in the current libseccomp implementation~\cite{lsc:bug}.

We evaluate our compiler using benchmarks taken from a handful of real-world applications
that use libseccomp, including Ctags, Lepton, LibreOffice, OpenSSH, and vsftpd.
%
To compare against the existing libseccomp compiler, we compile the libseccomp
filters to classic BPF using the existing compiler, and then into eBPF using
the Linux classic BPF to eBPF compiler.
%
We compare this result the code generated by our synthesized libseccomp to eBPF compiler,
using the same worst-case execution analysis.
%
The \jitsynth compiler has a $\LibseccompSlowdown\times$ compared to the
existing libseccomp to eBPF compiler stack.
%
% NB: What's the point to make here?
%
\fi
\fi

\subsection{Effectiveness of sketch optimizations}

\begin{figure}[h]
  % \resizebox{\textwidth}{!}{
  % %% Creator: Matplotlib, PGF backend
%%
%% To include the figure in your LaTeX document, write
%%   \input{<filename>.pgf}
%%
%% Make sure the required packages are loaded in your preamble
%%   \usepackage{pgf}
%%
%% Figures using additional raster images can only be included by \input if
%% they are in the same directory as the main LaTeX file. For loading figures
%% from other directories you can use the `import` package
%%   \usepackage{import}
%% and then include the figures with
%%   \import{<path to file>}{<filename>.pgf}
%%
%% Matplotlib used the following preamble
%%
\begingroup%
\makeatletter%
\begin{pgfpicture}%
\pgfpathrectangle{\pgfpointorigin}{\pgfqpoint{6.400000in}{4.800000in}}%
\pgfusepath{use as bounding box, clip}%
\begin{pgfscope}%
\pgfsetbuttcap%
\pgfsetmiterjoin%
\definecolor{currentfill}{rgb}{1.000000,1.000000,1.000000}%
\pgfsetfillcolor{currentfill}%
\pgfsetlinewidth{0.000000pt}%
\definecolor{currentstroke}{rgb}{1.000000,1.000000,1.000000}%
\pgfsetstrokecolor{currentstroke}%
\pgfsetdash{}{0pt}%
\pgfpathmoveto{\pgfqpoint{0.000000in}{0.000000in}}%
\pgfpathlineto{\pgfqpoint{6.400000in}{0.000000in}}%
\pgfpathlineto{\pgfqpoint{6.400000in}{4.800000in}}%
\pgfpathlineto{\pgfqpoint{0.000000in}{4.800000in}}%
\pgfpathclose%
\pgfusepath{fill}%
\end{pgfscope}%
\begin{pgfscope}%
\pgfsetbuttcap%
\pgfsetmiterjoin%
\definecolor{currentfill}{rgb}{1.000000,1.000000,1.000000}%
\pgfsetfillcolor{currentfill}%
\pgfsetlinewidth{0.000000pt}%
\definecolor{currentstroke}{rgb}{0.000000,0.000000,0.000000}%
\pgfsetstrokecolor{currentstroke}%
\pgfsetstrokeopacity{0.000000}%
\pgfsetdash{}{0pt}%
\pgfpathmoveto{\pgfqpoint{1.230000in}{1.022500in}}%
\pgfpathlineto{\pgfqpoint{6.077630in}{1.022500in}}%
\pgfpathlineto{\pgfqpoint{6.077630in}{4.078625in}}%
\pgfpathlineto{\pgfqpoint{1.230000in}{4.078625in}}%
\pgfpathclose%
\pgfusepath{fill}%
\end{pgfscope}%
\begin{pgfscope}%
\pgfpathrectangle{\pgfqpoint{1.230000in}{1.022500in}}{\pgfqpoint{4.847630in}{3.056125in}}%
\pgfusepath{clip}%
\pgfsetroundcap%
\pgfsetroundjoin%
\pgfsetlinewidth{1.003750pt}%
\definecolor{currentstroke}{rgb}{0.800000,0.800000,0.800000}%
\pgfsetstrokecolor{currentstroke}%
\pgfsetstrokeopacity{0.400000}%
\pgfsetdash{}{0pt}%
\pgfpathmoveto{\pgfqpoint{1.450347in}{1.022500in}}%
\pgfpathlineto{\pgfqpoint{1.450347in}{4.078625in}}%
\pgfusepath{stroke}%
\end{pgfscope}%
\begin{pgfscope}%
\definecolor{textcolor}{rgb}{0.150000,0.150000,0.150000}%
\pgfsetstrokecolor{textcolor}%
\pgfsetfillcolor{textcolor}%
\pgftext[x=1.450347in,y=0.890556in,,top]{\color{textcolor}\sffamily\fontsize{19.250000}{23.100000}\selectfont 0}%
\end{pgfscope}%
\begin{pgfscope}%
\pgfpathrectangle{\pgfqpoint{1.230000in}{1.022500in}}{\pgfqpoint{4.847630in}{3.056125in}}%
\pgfusepath{clip}%
\pgfsetroundcap%
\pgfsetroundjoin%
\pgfsetlinewidth{1.003750pt}%
\definecolor{currentstroke}{rgb}{0.800000,0.800000,0.800000}%
\pgfsetstrokecolor{currentstroke}%
\pgfsetstrokeopacity{0.400000}%
\pgfsetdash{}{0pt}%
\pgfpathmoveto{\pgfqpoint{2.849592in}{1.022500in}}%
\pgfpathlineto{\pgfqpoint{2.849592in}{4.078625in}}%
\pgfusepath{stroke}%
\end{pgfscope}%
\begin{pgfscope}%
\definecolor{textcolor}{rgb}{0.150000,0.150000,0.150000}%
\pgfsetstrokecolor{textcolor}%
\pgfsetfillcolor{textcolor}%
\pgftext[x=2.849592in,y=0.890556in,,top]{\color{textcolor}\sffamily\fontsize{19.250000}{23.100000}\selectfont 50000}%
\end{pgfscope}%
\begin{pgfscope}%
\pgfpathrectangle{\pgfqpoint{1.230000in}{1.022500in}}{\pgfqpoint{4.847630in}{3.056125in}}%
\pgfusepath{clip}%
\pgfsetroundcap%
\pgfsetroundjoin%
\pgfsetlinewidth{1.003750pt}%
\definecolor{currentstroke}{rgb}{0.800000,0.800000,0.800000}%
\pgfsetstrokecolor{currentstroke}%
\pgfsetstrokeopacity{0.400000}%
\pgfsetdash{}{0pt}%
\pgfpathmoveto{\pgfqpoint{4.248837in}{1.022500in}}%
\pgfpathlineto{\pgfqpoint{4.248837in}{4.078625in}}%
\pgfusepath{stroke}%
\end{pgfscope}%
\begin{pgfscope}%
\definecolor{textcolor}{rgb}{0.150000,0.150000,0.150000}%
\pgfsetstrokecolor{textcolor}%
\pgfsetfillcolor{textcolor}%
\pgftext[x=4.248837in,y=0.890556in,,top]{\color{textcolor}\sffamily\fontsize{19.250000}{23.100000}\selectfont 100000}%
\end{pgfscope}%
\begin{pgfscope}%
\pgfpathrectangle{\pgfqpoint{1.230000in}{1.022500in}}{\pgfqpoint{4.847630in}{3.056125in}}%
\pgfusepath{clip}%
\pgfsetroundcap%
\pgfsetroundjoin%
\pgfsetlinewidth{1.003750pt}%
\definecolor{currentstroke}{rgb}{0.800000,0.800000,0.800000}%
\pgfsetstrokecolor{currentstroke}%
\pgfsetstrokeopacity{0.400000}%
\pgfsetdash{}{0pt}%
\pgfpathmoveto{\pgfqpoint{5.648082in}{1.022500in}}%
\pgfpathlineto{\pgfqpoint{5.648082in}{4.078625in}}%
\pgfusepath{stroke}%
\end{pgfscope}%
\begin{pgfscope}%
\definecolor{textcolor}{rgb}{0.150000,0.150000,0.150000}%
\pgfsetstrokecolor{textcolor}%
\pgfsetfillcolor{textcolor}%
\pgftext[x=5.648082in,y=0.890556in,,top]{\color{textcolor}\sffamily\fontsize{19.250000}{23.100000}\selectfont 150000}%
\end{pgfscope}%
\begin{pgfscope}%
\definecolor{textcolor}{rgb}{0.150000,0.150000,0.150000}%
\pgfsetstrokecolor{textcolor}%
\pgfsetfillcolor{textcolor}%
\pgftext[x=3.653815in,y=0.578932in,,top]{\color{textcolor}\sffamily\fontsize{21.000000}{25.200000}\selectfont Time (s)}%
\end{pgfscope}%
\begin{pgfscope}%
\pgfpathrectangle{\pgfqpoint{1.230000in}{1.022500in}}{\pgfqpoint{4.847630in}{3.056125in}}%
\pgfusepath{clip}%
\pgfsetroundcap%
\pgfsetroundjoin%
\pgfsetlinewidth{1.003750pt}%
\definecolor{currentstroke}{rgb}{0.800000,0.800000,0.800000}%
\pgfsetstrokecolor{currentstroke}%
\pgfsetstrokeopacity{0.400000}%
\pgfsetdash{}{0pt}%
\pgfpathmoveto{\pgfqpoint{1.230000in}{1.193435in}}%
\pgfpathlineto{\pgfqpoint{6.077630in}{1.193435in}}%
\pgfusepath{stroke}%
\end{pgfscope}%
\begin{pgfscope}%
\definecolor{textcolor}{rgb}{0.150000,0.150000,0.150000}%
\pgfsetstrokecolor{textcolor}%
\pgfsetfillcolor{textcolor}%
\pgftext[x=0.962614in,y=1.093416in,left,base]{\color{textcolor}\sffamily\fontsize{19.250000}{23.100000}\selectfont 0}%
\end{pgfscope}%
\begin{pgfscope}%
\pgfpathrectangle{\pgfqpoint{1.230000in}{1.022500in}}{\pgfqpoint{4.847630in}{3.056125in}}%
\pgfusepath{clip}%
\pgfsetroundcap%
\pgfsetroundjoin%
\pgfsetlinewidth{1.003750pt}%
\definecolor{currentstroke}{rgb}{0.800000,0.800000,0.800000}%
\pgfsetstrokecolor{currentstroke}%
\pgfsetstrokeopacity{0.400000}%
\pgfsetdash{}{0pt}%
\pgfpathmoveto{\pgfqpoint{1.230000in}{1.880004in}}%
\pgfpathlineto{\pgfqpoint{6.077630in}{1.880004in}}%
\pgfusepath{stroke}%
\end{pgfscope}%
\begin{pgfscope}%
\definecolor{textcolor}{rgb}{0.150000,0.150000,0.150000}%
\pgfsetstrokecolor{textcolor}%
\pgfsetfillcolor{textcolor}%
\pgftext[x=0.827172in,y=1.779985in,left,base]{\color{textcolor}\sffamily\fontsize{19.250000}{23.100000}\selectfont 25}%
\end{pgfscope}%
\begin{pgfscope}%
\pgfpathrectangle{\pgfqpoint{1.230000in}{1.022500in}}{\pgfqpoint{4.847630in}{3.056125in}}%
\pgfusepath{clip}%
\pgfsetroundcap%
\pgfsetroundjoin%
\pgfsetlinewidth{1.003750pt}%
\definecolor{currentstroke}{rgb}{0.800000,0.800000,0.800000}%
\pgfsetstrokecolor{currentstroke}%
\pgfsetstrokeopacity{0.400000}%
\pgfsetdash{}{0pt}%
\pgfpathmoveto{\pgfqpoint{1.230000in}{2.566573in}}%
\pgfpathlineto{\pgfqpoint{6.077630in}{2.566573in}}%
\pgfusepath{stroke}%
\end{pgfscope}%
\begin{pgfscope}%
\definecolor{textcolor}{rgb}{0.150000,0.150000,0.150000}%
\pgfsetstrokecolor{textcolor}%
\pgfsetfillcolor{textcolor}%
\pgftext[x=0.827172in,y=2.466554in,left,base]{\color{textcolor}\sffamily\fontsize{19.250000}{23.100000}\selectfont 50}%
\end{pgfscope}%
\begin{pgfscope}%
\pgfpathrectangle{\pgfqpoint{1.230000in}{1.022500in}}{\pgfqpoint{4.847630in}{3.056125in}}%
\pgfusepath{clip}%
\pgfsetroundcap%
\pgfsetroundjoin%
\pgfsetlinewidth{1.003750pt}%
\definecolor{currentstroke}{rgb}{0.800000,0.800000,0.800000}%
\pgfsetstrokecolor{currentstroke}%
\pgfsetstrokeopacity{0.400000}%
\pgfsetdash{}{0pt}%
\pgfpathmoveto{\pgfqpoint{1.230000in}{3.253141in}}%
\pgfpathlineto{\pgfqpoint{6.077630in}{3.253141in}}%
\pgfusepath{stroke}%
\end{pgfscope}%
\begin{pgfscope}%
\definecolor{textcolor}{rgb}{0.150000,0.150000,0.150000}%
\pgfsetstrokecolor{textcolor}%
\pgfsetfillcolor{textcolor}%
\pgftext[x=0.827172in,y=3.153122in,left,base]{\color{textcolor}\sffamily\fontsize{19.250000}{23.100000}\selectfont 75}%
\end{pgfscope}%
\begin{pgfscope}%
\pgfpathrectangle{\pgfqpoint{1.230000in}{1.022500in}}{\pgfqpoint{4.847630in}{3.056125in}}%
\pgfusepath{clip}%
\pgfsetroundcap%
\pgfsetroundjoin%
\pgfsetlinewidth{1.003750pt}%
\definecolor{currentstroke}{rgb}{0.800000,0.800000,0.800000}%
\pgfsetstrokecolor{currentstroke}%
\pgfsetstrokeopacity{0.400000}%
\pgfsetdash{}{0pt}%
\pgfpathmoveto{\pgfqpoint{1.230000in}{3.939710in}}%
\pgfpathlineto{\pgfqpoint{6.077630in}{3.939710in}}%
\pgfusepath{stroke}%
\end{pgfscope}%
\begin{pgfscope}%
\definecolor{textcolor}{rgb}{0.150000,0.150000,0.150000}%
\pgfsetstrokecolor{textcolor}%
\pgfsetfillcolor{textcolor}%
\pgftext[x=0.691731in,y=3.839691in,left,base]{\color{textcolor}\sffamily\fontsize{19.250000}{23.100000}\selectfont 100}%
\end{pgfscope}%
\begin{pgfscope}%
\definecolor{textcolor}{rgb}{0.150000,0.150000,0.150000}%
\pgfsetstrokecolor{textcolor}%
\pgfsetfillcolor{textcolor}%
\pgftext[x=0.636175in,y=2.550563in,,bottom,rotate=90.000000]{\color{textcolor}\sffamily\fontsize{21.000000}{25.200000}\selectfont Percent of instructions synthesized}%
\end{pgfscope}%
\begin{pgfscope}%
\pgfpathrectangle{\pgfqpoint{1.230000in}{1.022500in}}{\pgfqpoint{4.847630in}{3.056125in}}%
\pgfusepath{clip}%
\pgfsetroundcap%
\pgfsetroundjoin%
\pgfsetlinewidth{1.505625pt}%
\definecolor{currentstroke}{rgb}{0.298039,0.447059,0.690196}%
\pgfsetstrokecolor{currentstroke}%
\pgfsetdash{}{0pt}%
\pgfpathmoveto{\pgfqpoint{1.450347in}{1.193435in}}%
\pgfpathlineto{\pgfqpoint{1.456252in}{1.193435in}}%
\pgfpathlineto{\pgfqpoint{1.456252in}{1.225002in}}%
\pgfpathlineto{\pgfqpoint{1.457511in}{1.225002in}}%
\pgfpathlineto{\pgfqpoint{1.457511in}{1.256568in}}%
\pgfpathlineto{\pgfqpoint{1.463220in}{1.256568in}}%
\pgfpathlineto{\pgfqpoint{1.463220in}{1.288134in}}%
\pgfpathlineto{\pgfqpoint{1.464479in}{1.288134in}}%
\pgfpathlineto{\pgfqpoint{1.464479in}{1.319701in}}%
\pgfpathlineto{\pgfqpoint{1.470860in}{1.319701in}}%
\pgfpathlineto{\pgfqpoint{1.470860in}{1.351267in}}%
\pgfpathlineto{\pgfqpoint{1.472147in}{1.351267in}}%
\pgfpathlineto{\pgfqpoint{1.472147in}{1.382834in}}%
\pgfpathlineto{\pgfqpoint{1.478388in}{1.382834in}}%
\pgfpathlineto{\pgfqpoint{1.478388in}{1.414400in}}%
\pgfpathlineto{\pgfqpoint{1.479647in}{1.414400in}}%
\pgfpathlineto{\pgfqpoint{1.479647in}{1.445966in}}%
\pgfpathlineto{\pgfqpoint{1.485608in}{1.445966in}}%
\pgfpathlineto{\pgfqpoint{1.485608in}{1.477533in}}%
\pgfpathlineto{\pgfqpoint{1.486867in}{1.477533in}}%
\pgfpathlineto{\pgfqpoint{1.486867in}{1.509099in}}%
\pgfpathlineto{\pgfqpoint{1.492716in}{1.509099in}}%
\pgfpathlineto{\pgfqpoint{1.492716in}{1.540665in}}%
\pgfpathlineto{\pgfqpoint{1.493975in}{1.540665in}}%
\pgfpathlineto{\pgfqpoint{1.493975in}{1.572232in}}%
\pgfpathlineto{\pgfqpoint{1.500020in}{1.572232in}}%
\pgfpathlineto{\pgfqpoint{1.500020in}{1.603798in}}%
\pgfpathlineto{\pgfqpoint{1.501363in}{1.603798in}}%
\pgfpathlineto{\pgfqpoint{1.501363in}{1.635365in}}%
\pgfpathlineto{\pgfqpoint{1.507380in}{1.635365in}}%
\pgfpathlineto{\pgfqpoint{1.507380in}{1.666931in}}%
\pgfpathlineto{\pgfqpoint{1.508723in}{1.666931in}}%
\pgfpathlineto{\pgfqpoint{1.508723in}{1.698497in}}%
\pgfpathlineto{\pgfqpoint{1.509983in}{1.698497in}}%
\pgfpathlineto{\pgfqpoint{1.509983in}{1.730064in}}%
\pgfpathlineto{\pgfqpoint{1.515999in}{1.730064in}}%
\pgfpathlineto{\pgfqpoint{1.515999in}{1.761630in}}%
\pgfpathlineto{\pgfqpoint{1.517259in}{1.761630in}}%
\pgfpathlineto{\pgfqpoint{1.517259in}{1.793196in}}%
\pgfpathlineto{\pgfqpoint{1.523247in}{1.793196in}}%
\pgfpathlineto{\pgfqpoint{1.523247in}{1.824763in}}%
\pgfpathlineto{\pgfqpoint{1.524507in}{1.824763in}}%
\pgfpathlineto{\pgfqpoint{1.524507in}{1.856329in}}%
\pgfpathlineto{\pgfqpoint{1.530104in}{1.856329in}}%
\pgfpathlineto{\pgfqpoint{1.530104in}{1.887896in}}%
\pgfpathlineto{\pgfqpoint{1.531363in}{1.887896in}}%
\pgfpathlineto{\pgfqpoint{1.531363in}{1.919462in}}%
\pgfpathlineto{\pgfqpoint{1.537240in}{1.919462in}}%
\pgfpathlineto{\pgfqpoint{1.537240in}{1.951028in}}%
\pgfpathlineto{\pgfqpoint{1.538611in}{1.951028in}}%
\pgfpathlineto{\pgfqpoint{1.538611in}{1.982595in}}%
\pgfpathlineto{\pgfqpoint{1.544544in}{1.982595in}}%
\pgfpathlineto{\pgfqpoint{1.544544in}{2.014161in}}%
\pgfpathlineto{\pgfqpoint{1.560327in}{2.014161in}}%
\pgfpathlineto{\pgfqpoint{1.560327in}{2.045727in}}%
\pgfpathlineto{\pgfqpoint{1.575271in}{2.045727in}}%
\pgfpathlineto{\pgfqpoint{1.575271in}{2.077294in}}%
\pgfpathlineto{\pgfqpoint{1.590523in}{2.077294in}}%
\pgfpathlineto{\pgfqpoint{1.590523in}{2.108860in}}%
\pgfpathlineto{\pgfqpoint{1.605803in}{2.108860in}}%
\pgfpathlineto{\pgfqpoint{1.605803in}{2.140427in}}%
\pgfpathlineto{\pgfqpoint{1.631913in}{2.140427in}}%
\pgfpathlineto{\pgfqpoint{1.631913in}{2.171993in}}%
\pgfpathlineto{\pgfqpoint{1.654972in}{2.171993in}}%
\pgfpathlineto{\pgfqpoint{1.654972in}{2.203559in}}%
\pgfpathlineto{\pgfqpoint{1.680187in}{2.203559in}}%
\pgfpathlineto{\pgfqpoint{1.680187in}{2.235126in}}%
\pgfpathlineto{\pgfqpoint{1.716287in}{2.235126in}}%
\pgfpathlineto{\pgfqpoint{1.716287in}{2.266692in}}%
\pgfpathlineto{\pgfqpoint{1.732015in}{2.266692in}}%
\pgfpathlineto{\pgfqpoint{1.732015in}{2.298259in}}%
\pgfpathlineto{\pgfqpoint{1.748134in}{2.298259in}}%
\pgfpathlineto{\pgfqpoint{1.748134in}{2.329825in}}%
\pgfpathlineto{\pgfqpoint{1.762882in}{2.329825in}}%
\pgfpathlineto{\pgfqpoint{1.762882in}{2.361391in}}%
\pgfpathlineto{\pgfqpoint{1.778050in}{2.361391in}}%
\pgfpathlineto{\pgfqpoint{1.778050in}{2.392958in}}%
\pgfpathlineto{\pgfqpoint{1.815886in}{2.392958in}}%
\pgfpathlineto{\pgfqpoint{1.815886in}{2.424524in}}%
\pgfpathlineto{\pgfqpoint{1.820643in}{2.424524in}}%
\pgfpathlineto{\pgfqpoint{1.820643in}{2.456090in}}%
\pgfpathlineto{\pgfqpoint{1.837714in}{2.456090in}}%
\pgfpathlineto{\pgfqpoint{1.837714in}{2.487657in}}%
\pgfpathlineto{\pgfqpoint{1.843087in}{2.487657in}}%
\pgfpathlineto{\pgfqpoint{1.843087in}{2.519223in}}%
\pgfpathlineto{\pgfqpoint{1.857443in}{2.519223in}}%
\pgfpathlineto{\pgfqpoint{1.857443in}{2.550790in}}%
\pgfpathlineto{\pgfqpoint{1.866426in}{2.550790in}}%
\pgfpathlineto{\pgfqpoint{1.866426in}{2.582356in}}%
\pgfpathlineto{\pgfqpoint{1.882406in}{2.582356in}}%
\pgfpathlineto{\pgfqpoint{1.882406in}{2.613922in}}%
\pgfpathlineto{\pgfqpoint{1.893348in}{2.613922in}}%
\pgfpathlineto{\pgfqpoint{1.893348in}{2.645489in}}%
\pgfpathlineto{\pgfqpoint{1.908432in}{2.645489in}}%
\pgfpathlineto{\pgfqpoint{1.908432in}{2.677055in}}%
\pgfpathlineto{\pgfqpoint{1.917247in}{2.677055in}}%
\pgfpathlineto{\pgfqpoint{1.917247in}{2.708621in}}%
\pgfpathlineto{\pgfqpoint{1.934374in}{2.708621in}}%
\pgfpathlineto{\pgfqpoint{1.934374in}{2.740188in}}%
\pgfpathlineto{\pgfqpoint{1.943665in}{2.740188in}}%
\pgfpathlineto{\pgfqpoint{1.943665in}{2.771754in}}%
\pgfpathlineto{\pgfqpoint{1.956062in}{2.771754in}}%
\pgfpathlineto{\pgfqpoint{1.956062in}{2.803321in}}%
\pgfpathlineto{\pgfqpoint{1.967704in}{2.803321in}}%
\pgfpathlineto{\pgfqpoint{1.967704in}{2.834887in}}%
\pgfpathlineto{\pgfqpoint{1.991491in}{2.834887in}}%
\pgfpathlineto{\pgfqpoint{1.991491in}{2.866453in}}%
\pgfpathlineto{\pgfqpoint{1.997480in}{2.866453in}}%
\pgfpathlineto{\pgfqpoint{1.997480in}{2.898020in}}%
\pgfpathlineto{\pgfqpoint{2.008925in}{2.898020in}}%
\pgfpathlineto{\pgfqpoint{2.008925in}{2.929586in}}%
\pgfpathlineto{\pgfqpoint{2.030446in}{2.929586in}}%
\pgfpathlineto{\pgfqpoint{2.030446in}{2.961152in}}%
\pgfpathlineto{\pgfqpoint{2.035819in}{2.961152in}}%
\pgfpathlineto{\pgfqpoint{2.035819in}{2.992719in}}%
\pgfpathlineto{\pgfqpoint{2.084821in}{2.992719in}}%
\pgfpathlineto{\pgfqpoint{2.084821in}{3.024285in}}%
\pgfpathlineto{\pgfqpoint{2.096630in}{3.024285in}}%
\pgfpathlineto{\pgfqpoint{2.096630in}{3.055852in}}%
\pgfpathlineto{\pgfqpoint{2.105921in}{3.055852in}}%
\pgfpathlineto{\pgfqpoint{2.105921in}{3.087418in}}%
\pgfpathlineto{\pgfqpoint{2.108216in}{3.087418in}}%
\pgfpathlineto{\pgfqpoint{2.108216in}{3.118984in}}%
\pgfpathlineto{\pgfqpoint{2.127246in}{3.118984in}}%
\pgfpathlineto{\pgfqpoint{2.127246in}{3.150551in}}%
\pgfpathlineto{\pgfqpoint{2.132731in}{3.150551in}}%
\pgfpathlineto{\pgfqpoint{2.132731in}{3.182117in}}%
\pgfpathlineto{\pgfqpoint{2.133990in}{3.182117in}}%
\pgfpathlineto{\pgfqpoint{2.133990in}{3.213684in}}%
\pgfpathlineto{\pgfqpoint{2.144568in}{3.213684in}}%
\pgfpathlineto{\pgfqpoint{2.144568in}{3.245250in}}%
\pgfpathlineto{\pgfqpoint{2.157945in}{3.245250in}}%
\pgfpathlineto{\pgfqpoint{2.157945in}{3.276816in}}%
\pgfpathlineto{\pgfqpoint{2.168915in}{3.276816in}}%
\pgfpathlineto{\pgfqpoint{2.168915in}{3.308383in}}%
\pgfpathlineto{\pgfqpoint{2.181173in}{3.308383in}}%
\pgfpathlineto{\pgfqpoint{2.181173in}{3.339949in}}%
\pgfpathlineto{\pgfqpoint{2.192031in}{3.339949in}}%
\pgfpathlineto{\pgfqpoint{2.192031in}{3.371515in}}%
\pgfpathlineto{\pgfqpoint{2.205995in}{3.371515in}}%
\pgfpathlineto{\pgfqpoint{2.205995in}{3.403082in}}%
\pgfpathlineto{\pgfqpoint{2.217973in}{3.403082in}}%
\pgfpathlineto{\pgfqpoint{2.217973in}{3.434648in}}%
\pgfpathlineto{\pgfqpoint{2.231014in}{3.434648in}}%
\pgfpathlineto{\pgfqpoint{2.231014in}{3.466215in}}%
\pgfpathlineto{\pgfqpoint{2.244362in}{3.466215in}}%
\pgfpathlineto{\pgfqpoint{2.244362in}{3.497781in}}%
\pgfpathlineto{\pgfqpoint{2.256956in}{3.497781in}}%
\pgfpathlineto{\pgfqpoint{2.256956in}{3.529347in}}%
\pgfpathlineto{\pgfqpoint{2.268849in}{3.529347in}}%
\pgfpathlineto{\pgfqpoint{2.268849in}{3.560914in}}%
\pgfpathlineto{\pgfqpoint{4.695336in}{3.560914in}}%
\pgfpathlineto{\pgfqpoint{4.695336in}{3.592480in}}%
\pgfpathlineto{\pgfqpoint{5.678949in}{3.592480in}}%
\pgfpathlineto{\pgfqpoint{5.678949in}{3.624046in}}%
\pgfpathlineto{\pgfqpoint{5.692690in}{3.624046in}}%
\pgfpathlineto{\pgfqpoint{5.692690in}{3.655613in}}%
\pgfpathlineto{\pgfqpoint{5.706627in}{3.655613in}}%
\pgfpathlineto{\pgfqpoint{5.706627in}{3.687179in}}%
\pgfpathlineto{\pgfqpoint{5.720143in}{3.687179in}}%
\pgfpathlineto{\pgfqpoint{5.720143in}{3.718746in}}%
\pgfpathlineto{\pgfqpoint{5.734723in}{3.718746in}}%
\pgfpathlineto{\pgfqpoint{5.734723in}{3.750312in}}%
\pgfpathlineto{\pgfqpoint{5.748520in}{3.750312in}}%
\pgfpathlineto{\pgfqpoint{5.748520in}{3.781878in}}%
\pgfpathlineto{\pgfqpoint{5.762288in}{3.781878in}}%
\pgfpathlineto{\pgfqpoint{5.762288in}{3.813445in}}%
\pgfpathlineto{\pgfqpoint{5.775022in}{3.813445in}}%
\pgfpathlineto{\pgfqpoint{5.775022in}{3.845011in}}%
\pgfpathlineto{\pgfqpoint{5.789210in}{3.845011in}}%
\pgfpathlineto{\pgfqpoint{5.789210in}{3.876577in}}%
\pgfpathlineto{\pgfqpoint{5.804546in}{3.876577in}}%
\pgfpathlineto{\pgfqpoint{5.804546in}{3.908144in}}%
\pgfpathlineto{\pgfqpoint{5.819434in}{3.908144in}}%
\pgfpathlineto{\pgfqpoint{5.819434in}{3.939710in}}%
\pgfpathlineto{\pgfqpoint{5.819434in}{3.939710in}}%
\pgfusepath{stroke}%
\end{pgfscope}%
\begin{pgfscope}%
\pgfpathrectangle{\pgfqpoint{1.230000in}{1.022500in}}{\pgfqpoint{4.847630in}{3.056125in}}%
\pgfusepath{clip}%
\pgfsetbuttcap%
\pgfsetroundjoin%
\definecolor{currentfill}{rgb}{0.298039,0.447059,0.690196}%
\pgfsetfillcolor{currentfill}%
\pgfsetlinewidth{1.003750pt}%
\definecolor{currentstroke}{rgb}{0.298039,0.447059,0.690196}%
\pgfsetstrokecolor{currentstroke}%
\pgfsetdash{}{0pt}%
\pgfsys@defobject{currentmarker}{\pgfqpoint{-0.020833in}{-0.020833in}}{\pgfqpoint{0.020833in}{0.020833in}}{%
\pgfpathmoveto{\pgfqpoint{0.000000in}{-0.020833in}}%
\pgfpathcurveto{\pgfqpoint{0.005525in}{-0.020833in}}{\pgfqpoint{0.010825in}{-0.018638in}}{\pgfqpoint{0.014731in}{-0.014731in}}%
\pgfpathcurveto{\pgfqpoint{0.018638in}{-0.010825in}}{\pgfqpoint{0.020833in}{-0.005525in}}{\pgfqpoint{0.020833in}{0.000000in}}%
\pgfpathcurveto{\pgfqpoint{0.020833in}{0.005525in}}{\pgfqpoint{0.018638in}{0.010825in}}{\pgfqpoint{0.014731in}{0.014731in}}%
\pgfpathcurveto{\pgfqpoint{0.010825in}{0.018638in}}{\pgfqpoint{0.005525in}{0.020833in}}{\pgfqpoint{0.000000in}{0.020833in}}%
\pgfpathcurveto{\pgfqpoint{-0.005525in}{0.020833in}}{\pgfqpoint{-0.010825in}{0.018638in}}{\pgfqpoint{-0.014731in}{0.014731in}}%
\pgfpathcurveto{\pgfqpoint{-0.018638in}{0.010825in}}{\pgfqpoint{-0.020833in}{0.005525in}}{\pgfqpoint{-0.020833in}{0.000000in}}%
\pgfpathcurveto{\pgfqpoint{-0.020833in}{-0.005525in}}{\pgfqpoint{-0.018638in}{-0.010825in}}{\pgfqpoint{-0.014731in}{-0.014731in}}%
\pgfpathcurveto{\pgfqpoint{-0.010825in}{-0.018638in}}{\pgfqpoint{-0.005525in}{-0.020833in}}{\pgfqpoint{0.000000in}{-0.020833in}}%
\pgfpathclose%
\pgfusepath{stroke,fill}%
}%
\begin{pgfscope}%
\pgfsys@transformshift{1.450347in}{1.193435in}%
\pgfsys@useobject{currentmarker}{}%
\end{pgfscope}%
\begin{pgfscope}%
\pgfsys@transformshift{5.819434in}{3.939710in}%
\pgfsys@useobject{currentmarker}{}%
\end{pgfscope}%
\end{pgfscope}%
\begin{pgfscope}%
\pgfpathrectangle{\pgfqpoint{1.230000in}{1.022500in}}{\pgfqpoint{4.847630in}{3.056125in}}%
\pgfusepath{clip}%
\pgfsetbuttcap%
\pgfsetroundjoin%
\pgfsetlinewidth{1.505625pt}%
\definecolor{currentstroke}{rgb}{0.866667,0.517647,0.321569}%
\pgfsetstrokecolor{currentstroke}%
\pgfsetdash{{5.550000pt}{2.400000pt}}{0.000000pt}%
\pgfpathmoveto{\pgfqpoint{1.450347in}{1.193435in}}%
\pgfpathlineto{\pgfqpoint{5.819434in}{1.193435in}}%
\pgfpathlineto{\pgfqpoint{5.819434in}{1.193435in}}%
\pgfusepath{stroke}%
\end{pgfscope}%
\begin{pgfscope}%
\pgfpathrectangle{\pgfqpoint{1.230000in}{1.022500in}}{\pgfqpoint{4.847630in}{3.056125in}}%
\pgfusepath{clip}%
\pgfsetbuttcap%
\pgfsetroundjoin%
\definecolor{currentfill}{rgb}{0.866667,0.517647,0.321569}%
\pgfsetfillcolor{currentfill}%
\pgfsetlinewidth{1.003750pt}%
\definecolor{currentstroke}{rgb}{0.866667,0.517647,0.321569}%
\pgfsetstrokecolor{currentstroke}%
\pgfsetdash{}{0pt}%
\pgfsys@defobject{currentmarker}{\pgfqpoint{-0.020833in}{-0.020833in}}{\pgfqpoint{0.020833in}{0.020833in}}{%
\pgfpathmoveto{\pgfqpoint{0.000000in}{-0.020833in}}%
\pgfpathcurveto{\pgfqpoint{0.005525in}{-0.020833in}}{\pgfqpoint{0.010825in}{-0.018638in}}{\pgfqpoint{0.014731in}{-0.014731in}}%
\pgfpathcurveto{\pgfqpoint{0.018638in}{-0.010825in}}{\pgfqpoint{0.020833in}{-0.005525in}}{\pgfqpoint{0.020833in}{0.000000in}}%
\pgfpathcurveto{\pgfqpoint{0.020833in}{0.005525in}}{\pgfqpoint{0.018638in}{0.010825in}}{\pgfqpoint{0.014731in}{0.014731in}}%
\pgfpathcurveto{\pgfqpoint{0.010825in}{0.018638in}}{\pgfqpoint{0.005525in}{0.020833in}}{\pgfqpoint{0.000000in}{0.020833in}}%
\pgfpathcurveto{\pgfqpoint{-0.005525in}{0.020833in}}{\pgfqpoint{-0.010825in}{0.018638in}}{\pgfqpoint{-0.014731in}{0.014731in}}%
\pgfpathcurveto{\pgfqpoint{-0.018638in}{0.010825in}}{\pgfqpoint{-0.020833in}{0.005525in}}{\pgfqpoint{-0.020833in}{0.000000in}}%
\pgfpathcurveto{\pgfqpoint{-0.020833in}{-0.005525in}}{\pgfqpoint{-0.018638in}{-0.010825in}}{\pgfqpoint{-0.014731in}{-0.014731in}}%
\pgfpathcurveto{\pgfqpoint{-0.010825in}{-0.018638in}}{\pgfqpoint{-0.005525in}{-0.020833in}}{\pgfqpoint{0.000000in}{-0.020833in}}%
\pgfpathclose%
\pgfusepath{stroke,fill}%
}%
\begin{pgfscope}%
\pgfsys@transformshift{1.450347in}{1.193435in}%
\pgfsys@useobject{currentmarker}{}%
\end{pgfscope}%
\begin{pgfscope}%
\pgfsys@transformshift{5.819434in}{1.193435in}%
\pgfsys@useobject{currentmarker}{}%
\end{pgfscope}%
\end{pgfscope}%
\begin{pgfscope}%
\pgfpathrectangle{\pgfqpoint{1.230000in}{1.022500in}}{\pgfqpoint{4.847630in}{3.056125in}}%
\pgfusepath{clip}%
\pgfsetbuttcap%
\pgfsetroundjoin%
\pgfsetlinewidth{1.505625pt}%
\definecolor{currentstroke}{rgb}{0.333333,0.658824,0.407843}%
\pgfsetstrokecolor{currentstroke}%
\pgfsetdash{{9.600000pt}{2.400000pt}{1.500000pt}{2.400000pt}}{0.000000pt}%
\pgfpathmoveto{\pgfqpoint{1.450347in}{1.193435in}}%
\pgfpathlineto{\pgfqpoint{5.819434in}{1.193435in}}%
\pgfpathlineto{\pgfqpoint{5.819434in}{1.193435in}}%
\pgfusepath{stroke}%
\end{pgfscope}%
\begin{pgfscope}%
\pgfpathrectangle{\pgfqpoint{1.230000in}{1.022500in}}{\pgfqpoint{4.847630in}{3.056125in}}%
\pgfusepath{clip}%
\pgfsetbuttcap%
\pgfsetroundjoin%
\definecolor{currentfill}{rgb}{0.333333,0.658824,0.407843}%
\pgfsetfillcolor{currentfill}%
\pgfsetlinewidth{1.003750pt}%
\definecolor{currentstroke}{rgb}{0.333333,0.658824,0.407843}%
\pgfsetstrokecolor{currentstroke}%
\pgfsetdash{}{0pt}%
\pgfsys@defobject{currentmarker}{\pgfqpoint{-0.020833in}{-0.020833in}}{\pgfqpoint{0.020833in}{0.020833in}}{%
\pgfpathmoveto{\pgfqpoint{0.000000in}{-0.020833in}}%
\pgfpathcurveto{\pgfqpoint{0.005525in}{-0.020833in}}{\pgfqpoint{0.010825in}{-0.018638in}}{\pgfqpoint{0.014731in}{-0.014731in}}%
\pgfpathcurveto{\pgfqpoint{0.018638in}{-0.010825in}}{\pgfqpoint{0.020833in}{-0.005525in}}{\pgfqpoint{0.020833in}{0.000000in}}%
\pgfpathcurveto{\pgfqpoint{0.020833in}{0.005525in}}{\pgfqpoint{0.018638in}{0.010825in}}{\pgfqpoint{0.014731in}{0.014731in}}%
\pgfpathcurveto{\pgfqpoint{0.010825in}{0.018638in}}{\pgfqpoint{0.005525in}{0.020833in}}{\pgfqpoint{0.000000in}{0.020833in}}%
\pgfpathcurveto{\pgfqpoint{-0.005525in}{0.020833in}}{\pgfqpoint{-0.010825in}{0.018638in}}{\pgfqpoint{-0.014731in}{0.014731in}}%
\pgfpathcurveto{\pgfqpoint{-0.018638in}{0.010825in}}{\pgfqpoint{-0.020833in}{0.005525in}}{\pgfqpoint{-0.020833in}{0.000000in}}%
\pgfpathcurveto{\pgfqpoint{-0.020833in}{-0.005525in}}{\pgfqpoint{-0.018638in}{-0.010825in}}{\pgfqpoint{-0.014731in}{-0.014731in}}%
\pgfpathcurveto{\pgfqpoint{-0.010825in}{-0.018638in}}{\pgfqpoint{-0.005525in}{-0.020833in}}{\pgfqpoint{0.000000in}{-0.020833in}}%
\pgfpathclose%
\pgfusepath{stroke,fill}%
}%
\begin{pgfscope}%
\pgfsys@transformshift{1.450347in}{1.193435in}%
\pgfsys@useobject{currentmarker}{}%
\end{pgfscope}%
\begin{pgfscope}%
\pgfsys@transformshift{5.819434in}{1.193435in}%
\pgfsys@useobject{currentmarker}{}%
\end{pgfscope}%
\end{pgfscope}%
\begin{pgfscope}%
\pgfsetrectcap%
\pgfsetmiterjoin%
\pgfsetlinewidth{1.254687pt}%
\definecolor{currentstroke}{rgb}{0.800000,0.800000,0.800000}%
\pgfsetstrokecolor{currentstroke}%
\pgfsetdash{}{0pt}%
\pgfpathmoveto{\pgfqpoint{1.230000in}{1.022500in}}%
\pgfpathlineto{\pgfqpoint{1.230000in}{4.078625in}}%
\pgfusepath{stroke}%
\end{pgfscope}%
\begin{pgfscope}%
\pgfsetrectcap%
\pgfsetmiterjoin%
\pgfsetlinewidth{1.254687pt}%
\definecolor{currentstroke}{rgb}{0.800000,0.800000,0.800000}%
\pgfsetstrokecolor{currentstroke}%
\pgfsetdash{}{0pt}%
\pgfpathmoveto{\pgfqpoint{6.077630in}{1.022500in}}%
\pgfpathlineto{\pgfqpoint{6.077630in}{4.078625in}}%
\pgfusepath{stroke}%
\end{pgfscope}%
\begin{pgfscope}%
\pgfsetrectcap%
\pgfsetmiterjoin%
\pgfsetlinewidth{1.254687pt}%
\definecolor{currentstroke}{rgb}{0.800000,0.800000,0.800000}%
\pgfsetstrokecolor{currentstroke}%
\pgfsetdash{}{0pt}%
\pgfpathmoveto{\pgfqpoint{1.230000in}{1.022500in}}%
\pgfpathlineto{\pgfqpoint{6.077630in}{1.022500in}}%
\pgfusepath{stroke}%
\end{pgfscope}%
\begin{pgfscope}%
\pgfsetrectcap%
\pgfsetmiterjoin%
\pgfsetlinewidth{1.254687pt}%
\definecolor{currentstroke}{rgb}{0.800000,0.800000,0.800000}%
\pgfsetstrokecolor{currentstroke}%
\pgfsetdash{}{0pt}%
\pgfpathmoveto{\pgfqpoint{1.230000in}{4.078625in}}%
\pgfpathlineto{\pgfqpoint{6.077630in}{4.078625in}}%
\pgfusepath{stroke}%
\end{pgfscope}%
\begin{pgfscope}%
\definecolor{textcolor}{rgb}{0.150000,0.150000,0.150000}%
\pgfsetstrokecolor{textcolor}%
\pgfsetfillcolor{textcolor}%
\pgftext[x=3.653815in,y=4.161958in,,base]{\color{textcolor}\sffamily\fontsize{21.000000}{25.200000}\selectfont Synthesis time for eBPF to RISC-V}%
\end{pgfscope}%
\begin{pgfscope}%
\pgfsetbuttcap%
\pgfsetmiterjoin%
\definecolor{currentfill}{rgb}{1.000000,1.000000,1.000000}%
\pgfsetfillcolor{currentfill}%
\pgfsetfillopacity{0.800000}%
\pgfsetlinewidth{1.003750pt}%
\definecolor{currentstroke}{rgb}{0.800000,0.800000,0.800000}%
\pgfsetstrokecolor{currentstroke}%
\pgfsetstrokeopacity{0.800000}%
\pgfsetdash{}{0pt}%
\pgfpathmoveto{\pgfqpoint{3.821155in}{1.925836in}}%
\pgfpathlineto{\pgfqpoint{5.890477in}{1.925836in}}%
\pgfpathquadraticcurveto{\pgfqpoint{5.943950in}{1.925836in}}{\pgfqpoint{5.943950in}{1.979308in}}%
\pgfpathlineto{\pgfqpoint{5.943950in}{3.121817in}}%
\pgfpathquadraticcurveto{\pgfqpoint{5.943950in}{3.175289in}}{\pgfqpoint{5.890477in}{3.175289in}}%
\pgfpathlineto{\pgfqpoint{3.821155in}{3.175289in}}%
\pgfpathquadraticcurveto{\pgfqpoint{3.767683in}{3.175289in}}{\pgfqpoint{3.767683in}{3.121817in}}%
\pgfpathlineto{\pgfqpoint{3.767683in}{1.979308in}}%
\pgfpathquadraticcurveto{\pgfqpoint{3.767683in}{1.925836in}}{\pgfqpoint{3.821155in}{1.925836in}}%
\pgfpathclose%
\pgfusepath{stroke,fill}%
\end{pgfscope}%
\begin{pgfscope}%
\pgfsetroundcap%
\pgfsetroundjoin%
\pgfsetlinewidth{1.505625pt}%
\definecolor{currentstroke}{rgb}{0.298039,0.447059,0.690196}%
\pgfsetstrokecolor{currentstroke}%
\pgfsetdash{}{0pt}%
\pgfpathmoveto{\pgfqpoint{3.874627in}{2.961882in}}%
\pgfpathlineto{\pgfqpoint{4.409350in}{2.961882in}}%
\pgfusepath{stroke}%
\end{pgfscope}%
\begin{pgfscope}%
\pgfsetbuttcap%
\pgfsetroundjoin%
\definecolor{currentfill}{rgb}{0.298039,0.447059,0.690196}%
\pgfsetfillcolor{currentfill}%
\pgfsetlinewidth{1.003750pt}%
\definecolor{currentstroke}{rgb}{0.298039,0.447059,0.690196}%
\pgfsetstrokecolor{currentstroke}%
\pgfsetdash{}{0pt}%
\pgfsys@defobject{currentmarker}{\pgfqpoint{-0.020833in}{-0.020833in}}{\pgfqpoint{0.020833in}{0.020833in}}{%
\pgfpathmoveto{\pgfqpoint{0.000000in}{-0.020833in}}%
\pgfpathcurveto{\pgfqpoint{0.005525in}{-0.020833in}}{\pgfqpoint{0.010825in}{-0.018638in}}{\pgfqpoint{0.014731in}{-0.014731in}}%
\pgfpathcurveto{\pgfqpoint{0.018638in}{-0.010825in}}{\pgfqpoint{0.020833in}{-0.005525in}}{\pgfqpoint{0.020833in}{0.000000in}}%
\pgfpathcurveto{\pgfqpoint{0.020833in}{0.005525in}}{\pgfqpoint{0.018638in}{0.010825in}}{\pgfqpoint{0.014731in}{0.014731in}}%
\pgfpathcurveto{\pgfqpoint{0.010825in}{0.018638in}}{\pgfqpoint{0.005525in}{0.020833in}}{\pgfqpoint{0.000000in}{0.020833in}}%
\pgfpathcurveto{\pgfqpoint{-0.005525in}{0.020833in}}{\pgfqpoint{-0.010825in}{0.018638in}}{\pgfqpoint{-0.014731in}{0.014731in}}%
\pgfpathcurveto{\pgfqpoint{-0.018638in}{0.010825in}}{\pgfqpoint{-0.020833in}{0.005525in}}{\pgfqpoint{-0.020833in}{0.000000in}}%
\pgfpathcurveto{\pgfqpoint{-0.020833in}{-0.005525in}}{\pgfqpoint{-0.018638in}{-0.010825in}}{\pgfqpoint{-0.014731in}{-0.014731in}}%
\pgfpathcurveto{\pgfqpoint{-0.010825in}{-0.018638in}}{\pgfqpoint{-0.005525in}{-0.020833in}}{\pgfqpoint{0.000000in}{-0.020833in}}%
\pgfpathclose%
\pgfusepath{stroke,fill}%
}%
\begin{pgfscope}%
\pgfsys@transformshift{4.141989in}{2.961882in}%
\pgfsys@useobject{currentmarker}{}%
\end{pgfscope}%
\end{pgfscope}%
\begin{pgfscope}%
\definecolor{textcolor}{rgb}{0.150000,0.150000,0.150000}%
\pgfsetstrokecolor{textcolor}%
\pgfsetfillcolor{textcolor}%
\pgftext[x=4.623239in,y=2.868306in,left,base]{\color{textcolor}\sffamily\fontsize{19.250000}{23.100000}\selectfont Pre-load}%
\end{pgfscope}%
\begin{pgfscope}%
\pgfsetbuttcap%
\pgfsetroundjoin%
\pgfsetlinewidth{1.505625pt}%
\definecolor{currentstroke}{rgb}{0.866667,0.517647,0.321569}%
\pgfsetstrokecolor{currentstroke}%
\pgfsetdash{{5.550000pt}{2.400000pt}}{0.000000pt}%
\pgfpathmoveto{\pgfqpoint{3.874627in}{2.572134in}}%
\pgfpathlineto{\pgfqpoint{4.409350in}{2.572134in}}%
\pgfusepath{stroke}%
\end{pgfscope}%
\begin{pgfscope}%
\pgfsetbuttcap%
\pgfsetroundjoin%
\definecolor{currentfill}{rgb}{0.866667,0.517647,0.321569}%
\pgfsetfillcolor{currentfill}%
\pgfsetlinewidth{1.003750pt}%
\definecolor{currentstroke}{rgb}{0.866667,0.517647,0.321569}%
\pgfsetstrokecolor{currentstroke}%
\pgfsetdash{}{0pt}%
\pgfsys@defobject{currentmarker}{\pgfqpoint{-0.020833in}{-0.020833in}}{\pgfqpoint{0.020833in}{0.020833in}}{%
\pgfpathmoveto{\pgfqpoint{0.000000in}{-0.020833in}}%
\pgfpathcurveto{\pgfqpoint{0.005525in}{-0.020833in}}{\pgfqpoint{0.010825in}{-0.018638in}}{\pgfqpoint{0.014731in}{-0.014731in}}%
\pgfpathcurveto{\pgfqpoint{0.018638in}{-0.010825in}}{\pgfqpoint{0.020833in}{-0.005525in}}{\pgfqpoint{0.020833in}{0.000000in}}%
\pgfpathcurveto{\pgfqpoint{0.020833in}{0.005525in}}{\pgfqpoint{0.018638in}{0.010825in}}{\pgfqpoint{0.014731in}{0.014731in}}%
\pgfpathcurveto{\pgfqpoint{0.010825in}{0.018638in}}{\pgfqpoint{0.005525in}{0.020833in}}{\pgfqpoint{0.000000in}{0.020833in}}%
\pgfpathcurveto{\pgfqpoint{-0.005525in}{0.020833in}}{\pgfqpoint{-0.010825in}{0.018638in}}{\pgfqpoint{-0.014731in}{0.014731in}}%
\pgfpathcurveto{\pgfqpoint{-0.018638in}{0.010825in}}{\pgfqpoint{-0.020833in}{0.005525in}}{\pgfqpoint{-0.020833in}{0.000000in}}%
\pgfpathcurveto{\pgfqpoint{-0.020833in}{-0.005525in}}{\pgfqpoint{-0.018638in}{-0.010825in}}{\pgfqpoint{-0.014731in}{-0.014731in}}%
\pgfpathcurveto{\pgfqpoint{-0.010825in}{-0.018638in}}{\pgfqpoint{-0.005525in}{-0.020833in}}{\pgfqpoint{0.000000in}{-0.020833in}}%
\pgfpathclose%
\pgfusepath{stroke,fill}%
}%
\begin{pgfscope}%
\pgfsys@transformshift{4.141989in}{2.572134in}%
\pgfsys@useobject{currentmarker}{}%
\end{pgfscope}%
\end{pgfscope}%
\begin{pgfscope}%
\definecolor{textcolor}{rgb}{0.150000,0.150000,0.150000}%
\pgfsetstrokecolor{textcolor}%
\pgfsetfillcolor{textcolor}%
\pgftext[x=4.623239in,y=2.478558in,left,base]{\color{textcolor}\sffamily\fontsize{19.250000}{23.100000}\selectfont Read-write}%
\end{pgfscope}%
\begin{pgfscope}%
\pgfsetbuttcap%
\pgfsetroundjoin%
\pgfsetlinewidth{1.505625pt}%
\definecolor{currentstroke}{rgb}{0.333333,0.658824,0.407843}%
\pgfsetstrokecolor{currentstroke}%
\pgfsetdash{{9.600000pt}{2.400000pt}{1.500000pt}{2.400000pt}}{0.000000pt}%
\pgfpathmoveto{\pgfqpoint{3.874627in}{2.182386in}}%
\pgfpathlineto{\pgfqpoint{4.409350in}{2.182386in}}%
\pgfusepath{stroke}%
\end{pgfscope}%
\begin{pgfscope}%
\pgfsetbuttcap%
\pgfsetroundjoin%
\definecolor{currentfill}{rgb}{0.333333,0.658824,0.407843}%
\pgfsetfillcolor{currentfill}%
\pgfsetlinewidth{1.003750pt}%
\definecolor{currentstroke}{rgb}{0.333333,0.658824,0.407843}%
\pgfsetstrokecolor{currentstroke}%
\pgfsetdash{}{0pt}%
\pgfsys@defobject{currentmarker}{\pgfqpoint{-0.020833in}{-0.020833in}}{\pgfqpoint{0.020833in}{0.020833in}}{%
\pgfpathmoveto{\pgfqpoint{0.000000in}{-0.020833in}}%
\pgfpathcurveto{\pgfqpoint{0.005525in}{-0.020833in}}{\pgfqpoint{0.010825in}{-0.018638in}}{\pgfqpoint{0.014731in}{-0.014731in}}%
\pgfpathcurveto{\pgfqpoint{0.018638in}{-0.010825in}}{\pgfqpoint{0.020833in}{-0.005525in}}{\pgfqpoint{0.020833in}{0.000000in}}%
\pgfpathcurveto{\pgfqpoint{0.020833in}{0.005525in}}{\pgfqpoint{0.018638in}{0.010825in}}{\pgfqpoint{0.014731in}{0.014731in}}%
\pgfpathcurveto{\pgfqpoint{0.010825in}{0.018638in}}{\pgfqpoint{0.005525in}{0.020833in}}{\pgfqpoint{0.000000in}{0.020833in}}%
\pgfpathcurveto{\pgfqpoint{-0.005525in}{0.020833in}}{\pgfqpoint{-0.010825in}{0.018638in}}{\pgfqpoint{-0.014731in}{0.014731in}}%
\pgfpathcurveto{\pgfqpoint{-0.018638in}{0.010825in}}{\pgfqpoint{-0.020833in}{0.005525in}}{\pgfqpoint{-0.020833in}{0.000000in}}%
\pgfpathcurveto{\pgfqpoint{-0.020833in}{-0.005525in}}{\pgfqpoint{-0.018638in}{-0.010825in}}{\pgfqpoint{-0.014731in}{-0.014731in}}%
\pgfpathcurveto{\pgfqpoint{-0.010825in}{-0.018638in}}{\pgfqpoint{-0.005525in}{-0.020833in}}{\pgfqpoint{0.000000in}{-0.020833in}}%
\pgfpathclose%
\pgfusepath{stroke,fill}%
}%
\begin{pgfscope}%
\pgfsys@transformshift{4.141989in}{2.182386in}%
\pgfsys@useobject{currentmarker}{}%
\end{pgfscope}%
\end{pgfscope}%
\begin{pgfscope}%
\definecolor{textcolor}{rgb}{0.150000,0.150000,0.150000}%
\pgfsetstrokecolor{textcolor}%
\pgfsetfillcolor{textcolor}%
\pgftext[x=4.623239in,y=2.088810in,left,base]{\color{textcolor}\sffamily\fontsize{19.250000}{23.100000}\selectfont Naïve}%
\end{pgfscope}%
\begin{pgfscope}%
\pgfpathrectangle{\pgfqpoint{1.230000in}{1.022500in}}{\pgfqpoint{4.847630in}{3.056125in}}%
\pgfusepath{clip}%
\pgfsetbuttcap%
\pgfsetroundjoin%
\definecolor{currentfill}{rgb}{1.000000,0.000000,0.000000}%
\pgfsetfillcolor{currentfill}%
\pgfsetlinewidth{1.505625pt}%
\definecolor{currentstroke}{rgb}{1.000000,0.000000,0.000000}%
\pgfsetstrokecolor{currentstroke}%
\pgfsetdash{}{0pt}%
\pgfpathmoveto{\pgfqpoint{5.765642in}{1.139644in}}%
\pgfpathlineto{\pgfqpoint{5.873225in}{1.247227in}}%
\pgfpathmoveto{\pgfqpoint{5.765642in}{1.247227in}}%
\pgfpathlineto{\pgfqpoint{5.873225in}{1.139644in}}%
\pgfusepath{stroke,fill}%
\end{pgfscope}%
\begin{pgfscope}%
\pgfpathrectangle{\pgfqpoint{1.230000in}{1.022500in}}{\pgfqpoint{4.847630in}{3.056125in}}%
\pgfusepath{clip}%
\pgfsetbuttcap%
\pgfsetroundjoin%
\definecolor{currentfill}{rgb}{1.000000,0.000000,0.000000}%
\pgfsetfillcolor{currentfill}%
\pgfsetlinewidth{1.505625pt}%
\definecolor{currentstroke}{rgb}{1.000000,0.000000,0.000000}%
\pgfsetstrokecolor{currentstroke}%
\pgfsetdash{}{0pt}%
\pgfpathmoveto{\pgfqpoint{5.765642in}{1.139644in}}%
\pgfpathlineto{\pgfqpoint{5.873225in}{1.247227in}}%
\pgfpathmoveto{\pgfqpoint{5.765642in}{1.247227in}}%
\pgfpathlineto{\pgfqpoint{5.873225in}{1.139644in}}%
\pgfusepath{stroke,fill}%
\end{pgfscope}%
\end{pgfpicture}%
\makeatother%
\endgroup%

  % %% Creator: Matplotlib, PGF backend
%%
%% To include the figure in your LaTeX document, write
%%   \input{<filename>.pgf}
%%
%% Make sure the required packages are loaded in your preamble
%%   \usepackage{pgf}
%%
%% Figures using additional raster images can only be included by \input if
%% they are in the same directory as the main LaTeX file. For loading figures
%% from other directories you can use the `import` package
%%   \usepackage{import}
%% and then include the figures with
%%   \import{<path to file>}{<filename>.pgf}
%%
%% Matplotlib used the following preamble
%%
\begingroup%
\makeatletter%
\begin{pgfpicture}%
\pgfpathrectangle{\pgfpointorigin}{\pgfqpoint{6.400000in}{4.800000in}}%
\pgfusepath{use as bounding box, clip}%
\begin{pgfscope}%
\pgfsetbuttcap%
\pgfsetmiterjoin%
\definecolor{currentfill}{rgb}{1.000000,1.000000,1.000000}%
\pgfsetfillcolor{currentfill}%
\pgfsetlinewidth{0.000000pt}%
\definecolor{currentstroke}{rgb}{1.000000,1.000000,1.000000}%
\pgfsetstrokecolor{currentstroke}%
\pgfsetdash{}{0pt}%
\pgfpathmoveto{\pgfqpoint{0.000000in}{0.000000in}}%
\pgfpathlineto{\pgfqpoint{6.400000in}{0.000000in}}%
\pgfpathlineto{\pgfqpoint{6.400000in}{4.800000in}}%
\pgfpathlineto{\pgfqpoint{0.000000in}{4.800000in}}%
\pgfpathclose%
\pgfusepath{fill}%
\end{pgfscope}%
\begin{pgfscope}%
\pgfsetbuttcap%
\pgfsetmiterjoin%
\definecolor{currentfill}{rgb}{1.000000,1.000000,1.000000}%
\pgfsetfillcolor{currentfill}%
\pgfsetlinewidth{0.000000pt}%
\definecolor{currentstroke}{rgb}{0.000000,0.000000,0.000000}%
\pgfsetstrokecolor{currentstroke}%
\pgfsetstrokeopacity{0.000000}%
\pgfsetdash{}{0pt}%
\pgfpathmoveto{\pgfqpoint{1.230000in}{1.022500in}}%
\pgfpathlineto{\pgfqpoint{6.037500in}{1.022500in}}%
\pgfpathlineto{\pgfqpoint{6.037500in}{4.078625in}}%
\pgfpathlineto{\pgfqpoint{1.230000in}{4.078625in}}%
\pgfpathclose%
\pgfusepath{fill}%
\end{pgfscope}%
\begin{pgfscope}%
\pgfpathrectangle{\pgfqpoint{1.230000in}{1.022500in}}{\pgfqpoint{4.807500in}{3.056125in}}%
\pgfusepath{clip}%
\pgfsetroundcap%
\pgfsetroundjoin%
\pgfsetlinewidth{1.003750pt}%
\definecolor{currentstroke}{rgb}{0.800000,0.800000,0.800000}%
\pgfsetstrokecolor{currentstroke}%
\pgfsetstrokeopacity{0.400000}%
\pgfsetdash{}{0pt}%
\pgfpathmoveto{\pgfqpoint{1.448523in}{1.022500in}}%
\pgfpathlineto{\pgfqpoint{1.448523in}{4.078625in}}%
\pgfusepath{stroke}%
\end{pgfscope}%
\begin{pgfscope}%
\definecolor{textcolor}{rgb}{0.150000,0.150000,0.150000}%
\pgfsetstrokecolor{textcolor}%
\pgfsetfillcolor{textcolor}%
\pgftext[x=1.448523in,y=0.890556in,,top]{\color{textcolor}\sffamily\fontsize{19.250000}{23.100000}\selectfont 0}%
\end{pgfscope}%
\begin{pgfscope}%
\pgfpathrectangle{\pgfqpoint{1.230000in}{1.022500in}}{\pgfqpoint{4.807500in}{3.056125in}}%
\pgfusepath{clip}%
\pgfsetroundcap%
\pgfsetroundjoin%
\pgfsetlinewidth{1.003750pt}%
\definecolor{currentstroke}{rgb}{0.800000,0.800000,0.800000}%
\pgfsetstrokecolor{currentstroke}%
\pgfsetstrokeopacity{0.400000}%
\pgfsetdash{}{0pt}%
\pgfpathmoveto{\pgfqpoint{2.433053in}{1.022500in}}%
\pgfpathlineto{\pgfqpoint{2.433053in}{4.078625in}}%
\pgfusepath{stroke}%
\end{pgfscope}%
\begin{pgfscope}%
\definecolor{textcolor}{rgb}{0.150000,0.150000,0.150000}%
\pgfsetstrokecolor{textcolor}%
\pgfsetfillcolor{textcolor}%
\pgftext[x=2.433053in,y=0.890556in,,top]{\color{textcolor}\sffamily\fontsize{19.250000}{23.100000}\selectfont 1000}%
\end{pgfscope}%
\begin{pgfscope}%
\pgfpathrectangle{\pgfqpoint{1.230000in}{1.022500in}}{\pgfqpoint{4.807500in}{3.056125in}}%
\pgfusepath{clip}%
\pgfsetroundcap%
\pgfsetroundjoin%
\pgfsetlinewidth{1.003750pt}%
\definecolor{currentstroke}{rgb}{0.800000,0.800000,0.800000}%
\pgfsetstrokecolor{currentstroke}%
\pgfsetstrokeopacity{0.400000}%
\pgfsetdash{}{0pt}%
\pgfpathmoveto{\pgfqpoint{3.417584in}{1.022500in}}%
\pgfpathlineto{\pgfqpoint{3.417584in}{4.078625in}}%
\pgfusepath{stroke}%
\end{pgfscope}%
\begin{pgfscope}%
\definecolor{textcolor}{rgb}{0.150000,0.150000,0.150000}%
\pgfsetstrokecolor{textcolor}%
\pgfsetfillcolor{textcolor}%
\pgftext[x=3.417584in,y=0.890556in,,top]{\color{textcolor}\sffamily\fontsize{19.250000}{23.100000}\selectfont 2000}%
\end{pgfscope}%
\begin{pgfscope}%
\pgfpathrectangle{\pgfqpoint{1.230000in}{1.022500in}}{\pgfqpoint{4.807500in}{3.056125in}}%
\pgfusepath{clip}%
\pgfsetroundcap%
\pgfsetroundjoin%
\pgfsetlinewidth{1.003750pt}%
\definecolor{currentstroke}{rgb}{0.800000,0.800000,0.800000}%
\pgfsetstrokecolor{currentstroke}%
\pgfsetstrokeopacity{0.400000}%
\pgfsetdash{}{0pt}%
\pgfpathmoveto{\pgfqpoint{4.402114in}{1.022500in}}%
\pgfpathlineto{\pgfqpoint{4.402114in}{4.078625in}}%
\pgfusepath{stroke}%
\end{pgfscope}%
\begin{pgfscope}%
\definecolor{textcolor}{rgb}{0.150000,0.150000,0.150000}%
\pgfsetstrokecolor{textcolor}%
\pgfsetfillcolor{textcolor}%
\pgftext[x=4.402114in,y=0.890556in,,top]{\color{textcolor}\sffamily\fontsize{19.250000}{23.100000}\selectfont 3000}%
\end{pgfscope}%
\begin{pgfscope}%
\pgfpathrectangle{\pgfqpoint{1.230000in}{1.022500in}}{\pgfqpoint{4.807500in}{3.056125in}}%
\pgfusepath{clip}%
\pgfsetroundcap%
\pgfsetroundjoin%
\pgfsetlinewidth{1.003750pt}%
\definecolor{currentstroke}{rgb}{0.800000,0.800000,0.800000}%
\pgfsetstrokecolor{currentstroke}%
\pgfsetstrokeopacity{0.400000}%
\pgfsetdash{}{0pt}%
\pgfpathmoveto{\pgfqpoint{5.386644in}{1.022500in}}%
\pgfpathlineto{\pgfqpoint{5.386644in}{4.078625in}}%
\pgfusepath{stroke}%
\end{pgfscope}%
\begin{pgfscope}%
\definecolor{textcolor}{rgb}{0.150000,0.150000,0.150000}%
\pgfsetstrokecolor{textcolor}%
\pgfsetfillcolor{textcolor}%
\pgftext[x=5.386644in,y=0.890556in,,top]{\color{textcolor}\sffamily\fontsize{19.250000}{23.100000}\selectfont 4000}%
\end{pgfscope}%
\begin{pgfscope}%
\definecolor{textcolor}{rgb}{0.150000,0.150000,0.150000}%
\pgfsetstrokecolor{textcolor}%
\pgfsetfillcolor{textcolor}%
\pgftext[x=3.633750in,y=0.578932in,,top]{\color{textcolor}\sffamily\fontsize{21.000000}{25.200000}\selectfont Time (s)}%
\end{pgfscope}%
\begin{pgfscope}%
\pgfpathrectangle{\pgfqpoint{1.230000in}{1.022500in}}{\pgfqpoint{4.807500in}{3.056125in}}%
\pgfusepath{clip}%
\pgfsetroundcap%
\pgfsetroundjoin%
\pgfsetlinewidth{1.003750pt}%
\definecolor{currentstroke}{rgb}{0.800000,0.800000,0.800000}%
\pgfsetstrokecolor{currentstroke}%
\pgfsetstrokeopacity{0.400000}%
\pgfsetdash{}{0pt}%
\pgfpathmoveto{\pgfqpoint{1.230000in}{1.161415in}}%
\pgfpathlineto{\pgfqpoint{6.037500in}{1.161415in}}%
\pgfusepath{stroke}%
\end{pgfscope}%
\begin{pgfscope}%
\definecolor{textcolor}{rgb}{0.150000,0.150000,0.150000}%
\pgfsetstrokecolor{textcolor}%
\pgfsetfillcolor{textcolor}%
\pgftext[x=0.962614in,y=1.061396in,left,base]{\color{textcolor}\sffamily\fontsize{19.250000}{23.100000}\selectfont 0}%
\end{pgfscope}%
\begin{pgfscope}%
\pgfpathrectangle{\pgfqpoint{1.230000in}{1.022500in}}{\pgfqpoint{4.807500in}{3.056125in}}%
\pgfusepath{clip}%
\pgfsetroundcap%
\pgfsetroundjoin%
\pgfsetlinewidth{1.003750pt}%
\definecolor{currentstroke}{rgb}{0.800000,0.800000,0.800000}%
\pgfsetstrokecolor{currentstroke}%
\pgfsetstrokeopacity{0.400000}%
\pgfsetdash{}{0pt}%
\pgfpathmoveto{\pgfqpoint{1.230000in}{1.855989in}}%
\pgfpathlineto{\pgfqpoint{6.037500in}{1.855989in}}%
\pgfusepath{stroke}%
\end{pgfscope}%
\begin{pgfscope}%
\definecolor{textcolor}{rgb}{0.150000,0.150000,0.150000}%
\pgfsetstrokecolor{textcolor}%
\pgfsetfillcolor{textcolor}%
\pgftext[x=0.827172in,y=1.755969in,left,base]{\color{textcolor}\sffamily\fontsize{19.250000}{23.100000}\selectfont 25}%
\end{pgfscope}%
\begin{pgfscope}%
\pgfpathrectangle{\pgfqpoint{1.230000in}{1.022500in}}{\pgfqpoint{4.807500in}{3.056125in}}%
\pgfusepath{clip}%
\pgfsetroundcap%
\pgfsetroundjoin%
\pgfsetlinewidth{1.003750pt}%
\definecolor{currentstroke}{rgb}{0.800000,0.800000,0.800000}%
\pgfsetstrokecolor{currentstroke}%
\pgfsetstrokeopacity{0.400000}%
\pgfsetdash{}{0pt}%
\pgfpathmoveto{\pgfqpoint{1.230000in}{2.550562in}}%
\pgfpathlineto{\pgfqpoint{6.037500in}{2.550562in}}%
\pgfusepath{stroke}%
\end{pgfscope}%
\begin{pgfscope}%
\definecolor{textcolor}{rgb}{0.150000,0.150000,0.150000}%
\pgfsetstrokecolor{textcolor}%
\pgfsetfillcolor{textcolor}%
\pgftext[x=0.827172in,y=2.450543in,left,base]{\color{textcolor}\sffamily\fontsize{19.250000}{23.100000}\selectfont 50}%
\end{pgfscope}%
\begin{pgfscope}%
\pgfpathrectangle{\pgfqpoint{1.230000in}{1.022500in}}{\pgfqpoint{4.807500in}{3.056125in}}%
\pgfusepath{clip}%
\pgfsetroundcap%
\pgfsetroundjoin%
\pgfsetlinewidth{1.003750pt}%
\definecolor{currentstroke}{rgb}{0.800000,0.800000,0.800000}%
\pgfsetstrokecolor{currentstroke}%
\pgfsetstrokeopacity{0.400000}%
\pgfsetdash{}{0pt}%
\pgfpathmoveto{\pgfqpoint{1.230000in}{3.245136in}}%
\pgfpathlineto{\pgfqpoint{6.037500in}{3.245136in}}%
\pgfusepath{stroke}%
\end{pgfscope}%
\begin{pgfscope}%
\definecolor{textcolor}{rgb}{0.150000,0.150000,0.150000}%
\pgfsetstrokecolor{textcolor}%
\pgfsetfillcolor{textcolor}%
\pgftext[x=0.827172in,y=3.145117in,left,base]{\color{textcolor}\sffamily\fontsize{19.250000}{23.100000}\selectfont 75}%
\end{pgfscope}%
\begin{pgfscope}%
\pgfpathrectangle{\pgfqpoint{1.230000in}{1.022500in}}{\pgfqpoint{4.807500in}{3.056125in}}%
\pgfusepath{clip}%
\pgfsetroundcap%
\pgfsetroundjoin%
\pgfsetlinewidth{1.003750pt}%
\definecolor{currentstroke}{rgb}{0.800000,0.800000,0.800000}%
\pgfsetstrokecolor{currentstroke}%
\pgfsetstrokeopacity{0.400000}%
\pgfsetdash{}{0pt}%
\pgfpathmoveto{\pgfqpoint{1.230000in}{3.939710in}}%
\pgfpathlineto{\pgfqpoint{6.037500in}{3.939710in}}%
\pgfusepath{stroke}%
\end{pgfscope}%
\begin{pgfscope}%
\definecolor{textcolor}{rgb}{0.150000,0.150000,0.150000}%
\pgfsetstrokecolor{textcolor}%
\pgfsetfillcolor{textcolor}%
\pgftext[x=0.691731in,y=3.839691in,left,base]{\color{textcolor}\sffamily\fontsize{19.250000}{23.100000}\selectfont 100}%
\end{pgfscope}%
\begin{pgfscope}%
\definecolor{textcolor}{rgb}{0.150000,0.150000,0.150000}%
\pgfsetstrokecolor{textcolor}%
\pgfsetfillcolor{textcolor}%
\pgftext[x=0.636175in,y=2.550563in,,bottom,rotate=90.000000]{\color{textcolor}\sffamily\fontsize{21.000000}{25.200000}\selectfont Percent of instructions synthesized}%
\end{pgfscope}%
\begin{pgfscope}%
\pgfpathrectangle{\pgfqpoint{1.230000in}{1.022500in}}{\pgfqpoint{4.807500in}{3.056125in}}%
\pgfusepath{clip}%
\pgfsetroundcap%
\pgfsetroundjoin%
\pgfsetlinewidth{1.505625pt}%
\definecolor{currentstroke}{rgb}{0.298039,0.447059,0.690196}%
\pgfsetstrokecolor{currentstroke}%
\pgfsetdash{}{0pt}%
\pgfpathmoveto{\pgfqpoint{1.448523in}{1.161415in}}%
\pgfpathlineto{\pgfqpoint{1.492827in}{1.161415in}}%
\pgfpathlineto{\pgfqpoint{1.492827in}{1.227565in}}%
\pgfpathlineto{\pgfqpoint{1.537130in}{1.227565in}}%
\pgfpathlineto{\pgfqpoint{1.537130in}{1.293715in}}%
\pgfpathlineto{\pgfqpoint{1.581434in}{1.293715in}}%
\pgfpathlineto{\pgfqpoint{1.581434in}{1.359864in}}%
\pgfpathlineto{\pgfqpoint{1.625738in}{1.359864in}}%
\pgfpathlineto{\pgfqpoint{1.625738in}{1.426014in}}%
\pgfpathlineto{\pgfqpoint{1.671027in}{1.426014in}}%
\pgfpathlineto{\pgfqpoint{1.671027in}{1.492164in}}%
\pgfpathlineto{\pgfqpoint{1.715330in}{1.492164in}}%
\pgfpathlineto{\pgfqpoint{1.715330in}{1.558314in}}%
\pgfpathlineto{\pgfqpoint{1.759634in}{1.558314in}}%
\pgfpathlineto{\pgfqpoint{1.759634in}{1.624464in}}%
\pgfpathlineto{\pgfqpoint{1.803938in}{1.624464in}}%
\pgfpathlineto{\pgfqpoint{1.803938in}{1.690614in}}%
\pgfpathlineto{\pgfqpoint{1.848242in}{1.690614in}}%
\pgfpathlineto{\pgfqpoint{1.848242in}{1.756764in}}%
\pgfpathlineto{\pgfqpoint{1.893530in}{1.756764in}}%
\pgfpathlineto{\pgfqpoint{1.893530in}{1.822914in}}%
\pgfpathlineto{\pgfqpoint{1.937834in}{1.822914in}}%
\pgfpathlineto{\pgfqpoint{1.937834in}{1.889064in}}%
\pgfpathlineto{\pgfqpoint{1.982138in}{1.889064in}}%
\pgfpathlineto{\pgfqpoint{1.982138in}{1.955213in}}%
\pgfpathlineto{\pgfqpoint{2.026442in}{1.955213in}}%
\pgfpathlineto{\pgfqpoint{2.026442in}{2.021363in}}%
\pgfpathlineto{\pgfqpoint{2.085514in}{2.021363in}}%
\pgfpathlineto{\pgfqpoint{2.085514in}{2.087513in}}%
\pgfpathlineto{\pgfqpoint{2.144586in}{2.087513in}}%
\pgfpathlineto{\pgfqpoint{2.144586in}{2.153663in}}%
\pgfpathlineto{\pgfqpoint{2.204642in}{2.153663in}}%
\pgfpathlineto{\pgfqpoint{2.204642in}{2.219813in}}%
\pgfpathlineto{\pgfqpoint{2.263714in}{2.219813in}}%
\pgfpathlineto{\pgfqpoint{2.263714in}{2.285963in}}%
\pgfpathlineto{\pgfqpoint{2.322786in}{2.285963in}}%
\pgfpathlineto{\pgfqpoint{2.322786in}{2.352113in}}%
\pgfpathlineto{\pgfqpoint{2.381858in}{2.352113in}}%
\pgfpathlineto{\pgfqpoint{2.381858in}{2.418263in}}%
\pgfpathlineto{\pgfqpoint{2.440929in}{2.418263in}}%
\pgfpathlineto{\pgfqpoint{2.440929in}{2.484413in}}%
\pgfpathlineto{\pgfqpoint{2.500986in}{2.484413in}}%
\pgfpathlineto{\pgfqpoint{2.500986in}{2.550562in}}%
\pgfpathlineto{\pgfqpoint{2.560058in}{2.550562in}}%
\pgfpathlineto{\pgfqpoint{2.560058in}{2.616712in}}%
\pgfpathlineto{\pgfqpoint{2.604361in}{2.616712in}}%
\pgfpathlineto{\pgfqpoint{2.604361in}{2.682862in}}%
\pgfpathlineto{\pgfqpoint{2.648665in}{2.682862in}}%
\pgfpathlineto{\pgfqpoint{2.648665in}{2.749012in}}%
\pgfpathlineto{\pgfqpoint{3.021802in}{2.749012in}}%
\pgfpathlineto{\pgfqpoint{3.021802in}{2.815162in}}%
\pgfpathlineto{\pgfqpoint{3.379187in}{2.815162in}}%
\pgfpathlineto{\pgfqpoint{3.379187in}{2.881312in}}%
\pgfpathlineto{\pgfqpoint{3.740509in}{2.881312in}}%
\pgfpathlineto{\pgfqpoint{3.740509in}{2.947462in}}%
\pgfpathlineto{\pgfqpoint{4.168780in}{2.947462in}}%
\pgfpathlineto{\pgfqpoint{4.168780in}{3.013612in}}%
\pgfpathlineto{\pgfqpoint{4.509428in}{3.013612in}}%
\pgfpathlineto{\pgfqpoint{4.509428in}{3.079762in}}%
\pgfpathlineto{\pgfqpoint{4.906193in}{3.079762in}}%
\pgfpathlineto{\pgfqpoint{4.906193in}{3.145912in}}%
\pgfpathlineto{\pgfqpoint{5.147403in}{3.145912in}}%
\pgfpathlineto{\pgfqpoint{5.147403in}{3.212061in}}%
\pgfpathlineto{\pgfqpoint{5.291145in}{3.212061in}}%
\pgfpathlineto{\pgfqpoint{5.291145in}{3.278211in}}%
\pgfpathlineto{\pgfqpoint{5.359077in}{3.278211in}}%
\pgfpathlineto{\pgfqpoint{5.359077in}{3.344361in}}%
\pgfpathlineto{\pgfqpoint{5.425041in}{3.344361in}}%
\pgfpathlineto{\pgfqpoint{5.425041in}{3.410511in}}%
\pgfpathlineto{\pgfqpoint{5.470329in}{3.410511in}}%
\pgfpathlineto{\pgfqpoint{5.470329in}{3.476661in}}%
\pgfpathlineto{\pgfqpoint{5.514633in}{3.476661in}}%
\pgfpathlineto{\pgfqpoint{5.514633in}{3.542811in}}%
\pgfpathlineto{\pgfqpoint{5.558937in}{3.542811in}}%
\pgfpathlineto{\pgfqpoint{5.558937in}{3.608961in}}%
\pgfpathlineto{\pgfqpoint{5.603241in}{3.608961in}}%
\pgfpathlineto{\pgfqpoint{5.603241in}{3.675111in}}%
\pgfpathlineto{\pgfqpoint{5.648529in}{3.675111in}}%
\pgfpathlineto{\pgfqpoint{5.648529in}{3.741261in}}%
\pgfpathlineto{\pgfqpoint{5.692833in}{3.741261in}}%
\pgfpathlineto{\pgfqpoint{5.692833in}{3.807410in}}%
\pgfpathlineto{\pgfqpoint{5.737137in}{3.807410in}}%
\pgfpathlineto{\pgfqpoint{5.737137in}{3.873560in}}%
\pgfpathlineto{\pgfqpoint{5.781441in}{3.873560in}}%
\pgfpathlineto{\pgfqpoint{5.781441in}{3.939710in}}%
\pgfusepath{stroke}%
\end{pgfscope}%
\begin{pgfscope}%
\pgfpathrectangle{\pgfqpoint{1.230000in}{1.022500in}}{\pgfqpoint{4.807500in}{3.056125in}}%
\pgfusepath{clip}%
\pgfsetbuttcap%
\pgfsetroundjoin%
\definecolor{currentfill}{rgb}{0.298039,0.447059,0.690196}%
\pgfsetfillcolor{currentfill}%
\pgfsetlinewidth{1.003750pt}%
\definecolor{currentstroke}{rgb}{0.298039,0.447059,0.690196}%
\pgfsetstrokecolor{currentstroke}%
\pgfsetdash{}{0pt}%
\pgfsys@defobject{currentmarker}{\pgfqpoint{-0.020833in}{-0.020833in}}{\pgfqpoint{0.020833in}{0.020833in}}{%
\pgfpathmoveto{\pgfqpoint{0.000000in}{-0.020833in}}%
\pgfpathcurveto{\pgfqpoint{0.005525in}{-0.020833in}}{\pgfqpoint{0.010825in}{-0.018638in}}{\pgfqpoint{0.014731in}{-0.014731in}}%
\pgfpathcurveto{\pgfqpoint{0.018638in}{-0.010825in}}{\pgfqpoint{0.020833in}{-0.005525in}}{\pgfqpoint{0.020833in}{0.000000in}}%
\pgfpathcurveto{\pgfqpoint{0.020833in}{0.005525in}}{\pgfqpoint{0.018638in}{0.010825in}}{\pgfqpoint{0.014731in}{0.014731in}}%
\pgfpathcurveto{\pgfqpoint{0.010825in}{0.018638in}}{\pgfqpoint{0.005525in}{0.020833in}}{\pgfqpoint{0.000000in}{0.020833in}}%
\pgfpathcurveto{\pgfqpoint{-0.005525in}{0.020833in}}{\pgfqpoint{-0.010825in}{0.018638in}}{\pgfqpoint{-0.014731in}{0.014731in}}%
\pgfpathcurveto{\pgfqpoint{-0.018638in}{0.010825in}}{\pgfqpoint{-0.020833in}{0.005525in}}{\pgfqpoint{-0.020833in}{0.000000in}}%
\pgfpathcurveto{\pgfqpoint{-0.020833in}{-0.005525in}}{\pgfqpoint{-0.018638in}{-0.010825in}}{\pgfqpoint{-0.014731in}{-0.014731in}}%
\pgfpathcurveto{\pgfqpoint{-0.010825in}{-0.018638in}}{\pgfqpoint{-0.005525in}{-0.020833in}}{\pgfqpoint{0.000000in}{-0.020833in}}%
\pgfpathclose%
\pgfusepath{stroke,fill}%
}%
\begin{pgfscope}%
\pgfsys@transformshift{1.448523in}{1.161415in}%
\pgfsys@useobject{currentmarker}{}%
\end{pgfscope}%
\begin{pgfscope}%
\pgfsys@transformshift{5.781441in}{3.939710in}%
\pgfsys@useobject{currentmarker}{}%
\end{pgfscope}%
\end{pgfscope}%
\begin{pgfscope}%
\pgfpathrectangle{\pgfqpoint{1.230000in}{1.022500in}}{\pgfqpoint{4.807500in}{3.056125in}}%
\pgfusepath{clip}%
\pgfsetbuttcap%
\pgfsetroundjoin%
\pgfsetlinewidth{1.505625pt}%
\definecolor{currentstroke}{rgb}{0.866667,0.517647,0.321569}%
\pgfsetstrokecolor{currentstroke}%
\pgfsetdash{{5.550000pt}{2.400000pt}}{0.000000pt}%
\pgfpathmoveto{\pgfqpoint{1.495780in}{1.161415in}}%
\pgfpathlineto{\pgfqpoint{1.495780in}{1.227565in}}%
\pgfpathlineto{\pgfqpoint{1.541069in}{1.227565in}}%
\pgfpathlineto{\pgfqpoint{1.541069in}{1.293715in}}%
\pgfpathlineto{\pgfqpoint{1.591280in}{1.293715in}}%
\pgfpathlineto{\pgfqpoint{1.591280in}{1.359864in}}%
\pgfpathlineto{\pgfqpoint{1.639522in}{1.359864in}}%
\pgfpathlineto{\pgfqpoint{1.639522in}{1.426014in}}%
\pgfpathlineto{\pgfqpoint{1.687764in}{1.426014in}}%
\pgfpathlineto{\pgfqpoint{1.687764in}{1.492164in}}%
\pgfpathlineto{\pgfqpoint{1.735021in}{1.492164in}}%
\pgfpathlineto{\pgfqpoint{1.735021in}{1.558314in}}%
\pgfpathlineto{\pgfqpoint{1.782279in}{1.558314in}}%
\pgfpathlineto{\pgfqpoint{1.782279in}{1.624464in}}%
\pgfpathlineto{\pgfqpoint{1.829536in}{1.624464in}}%
\pgfpathlineto{\pgfqpoint{1.829536in}{1.690614in}}%
\pgfpathlineto{\pgfqpoint{1.876793in}{1.690614in}}%
\pgfpathlineto{\pgfqpoint{1.876793in}{1.756764in}}%
\pgfpathlineto{\pgfqpoint{1.924051in}{1.756764in}}%
\pgfpathlineto{\pgfqpoint{1.924051in}{1.822914in}}%
\pgfpathlineto{\pgfqpoint{1.968355in}{1.822914in}}%
\pgfpathlineto{\pgfqpoint{1.968355in}{1.889064in}}%
\pgfpathlineto{\pgfqpoint{2.012659in}{1.889064in}}%
\pgfpathlineto{\pgfqpoint{2.012659in}{1.955213in}}%
\pgfpathlineto{\pgfqpoint{2.056963in}{1.955213in}}%
\pgfpathlineto{\pgfqpoint{2.056963in}{2.021363in}}%
\pgfpathlineto{\pgfqpoint{2.140648in}{2.021363in}}%
\pgfpathlineto{\pgfqpoint{2.140648in}{2.087513in}}%
\pgfpathlineto{\pgfqpoint{2.224333in}{2.087513in}}%
\pgfpathlineto{\pgfqpoint{2.224333in}{2.153663in}}%
\pgfpathlineto{\pgfqpoint{2.306049in}{2.153663in}}%
\pgfpathlineto{\pgfqpoint{2.306049in}{2.219813in}}%
\pgfpathlineto{\pgfqpoint{2.373981in}{2.219813in}}%
\pgfpathlineto{\pgfqpoint{2.373981in}{2.285963in}}%
\pgfpathlineto{\pgfqpoint{2.462589in}{2.285963in}}%
\pgfpathlineto{\pgfqpoint{2.462589in}{2.352113in}}%
\pgfpathlineto{\pgfqpoint{2.551197in}{2.352113in}}%
\pgfpathlineto{\pgfqpoint{2.551197in}{2.418263in}}%
\pgfpathlineto{\pgfqpoint{2.637835in}{2.418263in}}%
\pgfpathlineto{\pgfqpoint{2.637835in}{2.484413in}}%
\pgfpathlineto{\pgfqpoint{2.756964in}{2.484413in}}%
\pgfpathlineto{\pgfqpoint{2.756964in}{2.550562in}}%
\pgfpathlineto{\pgfqpoint{2.861324in}{2.550562in}}%
\pgfpathlineto{\pgfqpoint{2.861324in}{2.616712in}}%
\pgfpathlineto{\pgfqpoint{3.017864in}{2.616712in}}%
\pgfpathlineto{\pgfqpoint{3.017864in}{2.682862in}}%
\pgfpathlineto{\pgfqpoint{3.174404in}{2.682862in}}%
\pgfpathlineto{\pgfqpoint{3.174404in}{2.749012in}}%
\pgfpathlineto{\pgfqpoint{5.781441in}{2.749012in}}%
\pgfpathlineto{\pgfqpoint{5.781441in}{2.749012in}}%
\pgfusepath{stroke}%
\end{pgfscope}%
\begin{pgfscope}%
\pgfpathrectangle{\pgfqpoint{1.230000in}{1.022500in}}{\pgfqpoint{4.807500in}{3.056125in}}%
\pgfusepath{clip}%
\pgfsetbuttcap%
\pgfsetroundjoin%
\definecolor{currentfill}{rgb}{0.866667,0.517647,0.321569}%
\pgfsetfillcolor{currentfill}%
\pgfsetlinewidth{1.003750pt}%
\definecolor{currentstroke}{rgb}{0.866667,0.517647,0.321569}%
\pgfsetstrokecolor{currentstroke}%
\pgfsetdash{}{0pt}%
\pgfsys@defobject{currentmarker}{\pgfqpoint{-0.020833in}{-0.020833in}}{\pgfqpoint{0.020833in}{0.020833in}}{%
\pgfpathmoveto{\pgfqpoint{0.000000in}{-0.020833in}}%
\pgfpathcurveto{\pgfqpoint{0.005525in}{-0.020833in}}{\pgfqpoint{0.010825in}{-0.018638in}}{\pgfqpoint{0.014731in}{-0.014731in}}%
\pgfpathcurveto{\pgfqpoint{0.018638in}{-0.010825in}}{\pgfqpoint{0.020833in}{-0.005525in}}{\pgfqpoint{0.020833in}{0.000000in}}%
\pgfpathcurveto{\pgfqpoint{0.020833in}{0.005525in}}{\pgfqpoint{0.018638in}{0.010825in}}{\pgfqpoint{0.014731in}{0.014731in}}%
\pgfpathcurveto{\pgfqpoint{0.010825in}{0.018638in}}{\pgfqpoint{0.005525in}{0.020833in}}{\pgfqpoint{0.000000in}{0.020833in}}%
\pgfpathcurveto{\pgfqpoint{-0.005525in}{0.020833in}}{\pgfqpoint{-0.010825in}{0.018638in}}{\pgfqpoint{-0.014731in}{0.014731in}}%
\pgfpathcurveto{\pgfqpoint{-0.018638in}{0.010825in}}{\pgfqpoint{-0.020833in}{0.005525in}}{\pgfqpoint{-0.020833in}{0.000000in}}%
\pgfpathcurveto{\pgfqpoint{-0.020833in}{-0.005525in}}{\pgfqpoint{-0.018638in}{-0.010825in}}{\pgfqpoint{-0.014731in}{-0.014731in}}%
\pgfpathcurveto{\pgfqpoint{-0.010825in}{-0.018638in}}{\pgfqpoint{-0.005525in}{-0.020833in}}{\pgfqpoint{0.000000in}{-0.020833in}}%
\pgfpathclose%
\pgfusepath{stroke,fill}%
}%
\begin{pgfscope}%
\pgfsys@transformshift{5.781441in}{2.749012in}%
\pgfsys@useobject{currentmarker}{}%
\end{pgfscope}%
\end{pgfscope}%
\begin{pgfscope}%
\pgfpathrectangle{\pgfqpoint{1.230000in}{1.022500in}}{\pgfqpoint{4.807500in}{3.056125in}}%
\pgfusepath{clip}%
\pgfsetbuttcap%
\pgfsetroundjoin%
\pgfsetlinewidth{1.505625pt}%
\definecolor{currentstroke}{rgb}{0.333333,0.658824,0.407843}%
\pgfsetstrokecolor{currentstroke}%
\pgfsetdash{{9.600000pt}{2.400000pt}{1.500000pt}{2.400000pt}}{0.000000pt}%
\pgfpathmoveto{\pgfqpoint{1.448523in}{1.161415in}}%
\pgfpathlineto{\pgfqpoint{1.492827in}{1.161415in}}%
\pgfpathlineto{\pgfqpoint{1.492827in}{1.227565in}}%
\pgfpathlineto{\pgfqpoint{1.537130in}{1.227565in}}%
\pgfpathlineto{\pgfqpoint{1.537130in}{1.293715in}}%
\pgfpathlineto{\pgfqpoint{1.581434in}{1.293715in}}%
\pgfpathlineto{\pgfqpoint{1.581434in}{1.359864in}}%
\pgfpathlineto{\pgfqpoint{1.625738in}{1.359864in}}%
\pgfpathlineto{\pgfqpoint{1.625738in}{1.426014in}}%
\pgfpathlineto{\pgfqpoint{1.670042in}{1.426014in}}%
\pgfpathlineto{\pgfqpoint{1.670042in}{1.492164in}}%
\pgfpathlineto{\pgfqpoint{1.715330in}{1.492164in}}%
\pgfpathlineto{\pgfqpoint{1.715330in}{1.558314in}}%
\pgfpathlineto{\pgfqpoint{1.759634in}{1.558314in}}%
\pgfpathlineto{\pgfqpoint{1.759634in}{1.624464in}}%
\pgfpathlineto{\pgfqpoint{1.803938in}{1.624464in}}%
\pgfpathlineto{\pgfqpoint{1.803938in}{1.690614in}}%
\pgfpathlineto{\pgfqpoint{1.848242in}{1.690614in}}%
\pgfpathlineto{\pgfqpoint{1.848242in}{1.756764in}}%
\pgfpathlineto{\pgfqpoint{1.892546in}{1.756764in}}%
\pgfpathlineto{\pgfqpoint{1.892546in}{1.822914in}}%
\pgfpathlineto{\pgfqpoint{1.936850in}{1.822914in}}%
\pgfpathlineto{\pgfqpoint{1.936850in}{1.889064in}}%
\pgfpathlineto{\pgfqpoint{1.982138in}{1.889064in}}%
\pgfpathlineto{\pgfqpoint{1.982138in}{1.955213in}}%
\pgfpathlineto{\pgfqpoint{2.026442in}{1.955213in}}%
\pgfpathlineto{\pgfqpoint{2.026442in}{2.021363in}}%
\pgfpathlineto{\pgfqpoint{2.155416in}{2.021363in}}%
\pgfpathlineto{\pgfqpoint{2.155416in}{2.087513in}}%
\pgfpathlineto{\pgfqpoint{2.245008in}{2.087513in}}%
\pgfpathlineto{\pgfqpoint{2.245008in}{2.153663in}}%
\pgfpathlineto{\pgfqpoint{2.339523in}{2.153663in}}%
\pgfpathlineto{\pgfqpoint{2.339523in}{2.219813in}}%
\pgfpathlineto{\pgfqpoint{2.450775in}{2.219813in}}%
\pgfpathlineto{\pgfqpoint{2.450775in}{2.285963in}}%
\pgfpathlineto{\pgfqpoint{2.562027in}{2.285963in}}%
\pgfpathlineto{\pgfqpoint{2.562027in}{2.352113in}}%
\pgfpathlineto{\pgfqpoint{2.659495in}{2.352113in}}%
\pgfpathlineto{\pgfqpoint{2.659495in}{2.418263in}}%
\pgfpathlineto{\pgfqpoint{2.790438in}{2.418263in}}%
\pgfpathlineto{\pgfqpoint{2.790438in}{2.484413in}}%
\pgfpathlineto{\pgfqpoint{2.931225in}{2.484413in}}%
\pgfpathlineto{\pgfqpoint{2.931225in}{2.550562in}}%
\pgfpathlineto{\pgfqpoint{3.079890in}{2.550562in}}%
\pgfpathlineto{\pgfqpoint{3.079890in}{2.616712in}}%
\pgfpathlineto{\pgfqpoint{5.781441in}{2.616712in}}%
\pgfpathlineto{\pgfqpoint{5.781441in}{2.616712in}}%
\pgfusepath{stroke}%
\end{pgfscope}%
\begin{pgfscope}%
\pgfpathrectangle{\pgfqpoint{1.230000in}{1.022500in}}{\pgfqpoint{4.807500in}{3.056125in}}%
\pgfusepath{clip}%
\pgfsetbuttcap%
\pgfsetroundjoin%
\definecolor{currentfill}{rgb}{0.333333,0.658824,0.407843}%
\pgfsetfillcolor{currentfill}%
\pgfsetlinewidth{1.003750pt}%
\definecolor{currentstroke}{rgb}{0.333333,0.658824,0.407843}%
\pgfsetstrokecolor{currentstroke}%
\pgfsetdash{}{0pt}%
\pgfsys@defobject{currentmarker}{\pgfqpoint{-0.020833in}{-0.020833in}}{\pgfqpoint{0.020833in}{0.020833in}}{%
\pgfpathmoveto{\pgfqpoint{0.000000in}{-0.020833in}}%
\pgfpathcurveto{\pgfqpoint{0.005525in}{-0.020833in}}{\pgfqpoint{0.010825in}{-0.018638in}}{\pgfqpoint{0.014731in}{-0.014731in}}%
\pgfpathcurveto{\pgfqpoint{0.018638in}{-0.010825in}}{\pgfqpoint{0.020833in}{-0.005525in}}{\pgfqpoint{0.020833in}{0.000000in}}%
\pgfpathcurveto{\pgfqpoint{0.020833in}{0.005525in}}{\pgfqpoint{0.018638in}{0.010825in}}{\pgfqpoint{0.014731in}{0.014731in}}%
\pgfpathcurveto{\pgfqpoint{0.010825in}{0.018638in}}{\pgfqpoint{0.005525in}{0.020833in}}{\pgfqpoint{0.000000in}{0.020833in}}%
\pgfpathcurveto{\pgfqpoint{-0.005525in}{0.020833in}}{\pgfqpoint{-0.010825in}{0.018638in}}{\pgfqpoint{-0.014731in}{0.014731in}}%
\pgfpathcurveto{\pgfqpoint{-0.018638in}{0.010825in}}{\pgfqpoint{-0.020833in}{0.005525in}}{\pgfqpoint{-0.020833in}{0.000000in}}%
\pgfpathcurveto{\pgfqpoint{-0.020833in}{-0.005525in}}{\pgfqpoint{-0.018638in}{-0.010825in}}{\pgfqpoint{-0.014731in}{-0.014731in}}%
\pgfpathcurveto{\pgfqpoint{-0.010825in}{-0.018638in}}{\pgfqpoint{-0.005525in}{-0.020833in}}{\pgfqpoint{0.000000in}{-0.020833in}}%
\pgfpathclose%
\pgfusepath{stroke,fill}%
}%
\begin{pgfscope}%
\pgfsys@transformshift{1.448523in}{1.161415in}%
\pgfsys@useobject{currentmarker}{}%
\end{pgfscope}%
\begin{pgfscope}%
\pgfsys@transformshift{5.781441in}{2.616712in}%
\pgfsys@useobject{currentmarker}{}%
\end{pgfscope}%
\end{pgfscope}%
\begin{pgfscope}%
\pgfsetrectcap%
\pgfsetmiterjoin%
\pgfsetlinewidth{1.254687pt}%
\definecolor{currentstroke}{rgb}{0.800000,0.800000,0.800000}%
\pgfsetstrokecolor{currentstroke}%
\pgfsetdash{}{0pt}%
\pgfpathmoveto{\pgfqpoint{1.230000in}{1.022500in}}%
\pgfpathlineto{\pgfqpoint{1.230000in}{4.078625in}}%
\pgfusepath{stroke}%
\end{pgfscope}%
\begin{pgfscope}%
\pgfsetrectcap%
\pgfsetmiterjoin%
\pgfsetlinewidth{1.254687pt}%
\definecolor{currentstroke}{rgb}{0.800000,0.800000,0.800000}%
\pgfsetstrokecolor{currentstroke}%
\pgfsetdash{}{0pt}%
\pgfpathmoveto{\pgfqpoint{6.037500in}{1.022500in}}%
\pgfpathlineto{\pgfqpoint{6.037500in}{4.078625in}}%
\pgfusepath{stroke}%
\end{pgfscope}%
\begin{pgfscope}%
\pgfsetrectcap%
\pgfsetmiterjoin%
\pgfsetlinewidth{1.254687pt}%
\definecolor{currentstroke}{rgb}{0.800000,0.800000,0.800000}%
\pgfsetstrokecolor{currentstroke}%
\pgfsetdash{}{0pt}%
\pgfpathmoveto{\pgfqpoint{1.230000in}{1.022500in}}%
\pgfpathlineto{\pgfqpoint{6.037500in}{1.022500in}}%
\pgfusepath{stroke}%
\end{pgfscope}%
\begin{pgfscope}%
\pgfsetrectcap%
\pgfsetmiterjoin%
\pgfsetlinewidth{1.254687pt}%
\definecolor{currentstroke}{rgb}{0.800000,0.800000,0.800000}%
\pgfsetstrokecolor{currentstroke}%
\pgfsetdash{}{0pt}%
\pgfpathmoveto{\pgfqpoint{1.230000in}{4.078625in}}%
\pgfpathlineto{\pgfqpoint{6.037500in}{4.078625in}}%
\pgfusepath{stroke}%
\end{pgfscope}%
\begin{pgfscope}%
\definecolor{textcolor}{rgb}{0.150000,0.150000,0.150000}%
\pgfsetstrokecolor{textcolor}%
\pgfsetfillcolor{textcolor}%
\pgftext[x=3.633750in,y=4.161958in,,base]{\color{textcolor}\sffamily\fontsize{21.000000}{25.200000}\selectfont Synthesis time for classic BPF to eBPF}%
\end{pgfscope}%
\begin{pgfscope}%
\pgfsetbuttcap%
\pgfsetmiterjoin%
\definecolor{currentfill}{rgb}{1.000000,1.000000,1.000000}%
\pgfsetfillcolor{currentfill}%
\pgfsetfillopacity{0.800000}%
\pgfsetlinewidth{1.003750pt}%
\definecolor{currentstroke}{rgb}{0.800000,0.800000,0.800000}%
\pgfsetstrokecolor{currentstroke}%
\pgfsetstrokeopacity{0.800000}%
\pgfsetdash{}{0pt}%
\pgfpathmoveto{\pgfqpoint{3.781025in}{1.156181in}}%
\pgfpathlineto{\pgfqpoint{5.850347in}{1.156181in}}%
\pgfpathquadraticcurveto{\pgfqpoint{5.903819in}{1.156181in}}{\pgfqpoint{5.903819in}{1.209653in}}%
\pgfpathlineto{\pgfqpoint{5.903819in}{2.352161in}}%
\pgfpathquadraticcurveto{\pgfqpoint{5.903819in}{2.405633in}}{\pgfqpoint{5.850347in}{2.405633in}}%
\pgfpathlineto{\pgfqpoint{3.781025in}{2.405633in}}%
\pgfpathquadraticcurveto{\pgfqpoint{3.727553in}{2.405633in}}{\pgfqpoint{3.727553in}{2.352161in}}%
\pgfpathlineto{\pgfqpoint{3.727553in}{1.209653in}}%
\pgfpathquadraticcurveto{\pgfqpoint{3.727553in}{1.156181in}}{\pgfqpoint{3.781025in}{1.156181in}}%
\pgfpathclose%
\pgfusepath{stroke,fill}%
\end{pgfscope}%
\begin{pgfscope}%
\pgfsetroundcap%
\pgfsetroundjoin%
\pgfsetlinewidth{1.505625pt}%
\definecolor{currentstroke}{rgb}{0.298039,0.447059,0.690196}%
\pgfsetstrokecolor{currentstroke}%
\pgfsetdash{}{0pt}%
\pgfpathmoveto{\pgfqpoint{3.834497in}{2.192227in}}%
\pgfpathlineto{\pgfqpoint{4.369220in}{2.192227in}}%
\pgfusepath{stroke}%
\end{pgfscope}%
\begin{pgfscope}%
\pgfsetbuttcap%
\pgfsetroundjoin%
\definecolor{currentfill}{rgb}{0.298039,0.447059,0.690196}%
\pgfsetfillcolor{currentfill}%
\pgfsetlinewidth{1.003750pt}%
\definecolor{currentstroke}{rgb}{0.298039,0.447059,0.690196}%
\pgfsetstrokecolor{currentstroke}%
\pgfsetdash{}{0pt}%
\pgfsys@defobject{currentmarker}{\pgfqpoint{-0.020833in}{-0.020833in}}{\pgfqpoint{0.020833in}{0.020833in}}{%
\pgfpathmoveto{\pgfqpoint{0.000000in}{-0.020833in}}%
\pgfpathcurveto{\pgfqpoint{0.005525in}{-0.020833in}}{\pgfqpoint{0.010825in}{-0.018638in}}{\pgfqpoint{0.014731in}{-0.014731in}}%
\pgfpathcurveto{\pgfqpoint{0.018638in}{-0.010825in}}{\pgfqpoint{0.020833in}{-0.005525in}}{\pgfqpoint{0.020833in}{0.000000in}}%
\pgfpathcurveto{\pgfqpoint{0.020833in}{0.005525in}}{\pgfqpoint{0.018638in}{0.010825in}}{\pgfqpoint{0.014731in}{0.014731in}}%
\pgfpathcurveto{\pgfqpoint{0.010825in}{0.018638in}}{\pgfqpoint{0.005525in}{0.020833in}}{\pgfqpoint{0.000000in}{0.020833in}}%
\pgfpathcurveto{\pgfqpoint{-0.005525in}{0.020833in}}{\pgfqpoint{-0.010825in}{0.018638in}}{\pgfqpoint{-0.014731in}{0.014731in}}%
\pgfpathcurveto{\pgfqpoint{-0.018638in}{0.010825in}}{\pgfqpoint{-0.020833in}{0.005525in}}{\pgfqpoint{-0.020833in}{0.000000in}}%
\pgfpathcurveto{\pgfqpoint{-0.020833in}{-0.005525in}}{\pgfqpoint{-0.018638in}{-0.010825in}}{\pgfqpoint{-0.014731in}{-0.014731in}}%
\pgfpathcurveto{\pgfqpoint{-0.010825in}{-0.018638in}}{\pgfqpoint{-0.005525in}{-0.020833in}}{\pgfqpoint{0.000000in}{-0.020833in}}%
\pgfpathclose%
\pgfusepath{stroke,fill}%
}%
\begin{pgfscope}%
\pgfsys@transformshift{4.101858in}{2.192227in}%
\pgfsys@useobject{currentmarker}{}%
\end{pgfscope}%
\end{pgfscope}%
\begin{pgfscope}%
\definecolor{textcolor}{rgb}{0.150000,0.150000,0.150000}%
\pgfsetstrokecolor{textcolor}%
\pgfsetfillcolor{textcolor}%
\pgftext[x=4.583108in,y=2.098651in,left,base]{\color{textcolor}\sffamily\fontsize{19.250000}{23.100000}\selectfont Pre-load}%
\end{pgfscope}%
\begin{pgfscope}%
\pgfsetbuttcap%
\pgfsetroundjoin%
\pgfsetlinewidth{1.505625pt}%
\definecolor{currentstroke}{rgb}{0.866667,0.517647,0.321569}%
\pgfsetstrokecolor{currentstroke}%
\pgfsetdash{{5.550000pt}{2.400000pt}}{0.000000pt}%
\pgfpathmoveto{\pgfqpoint{3.834497in}{1.802479in}}%
\pgfpathlineto{\pgfqpoint{4.369220in}{1.802479in}}%
\pgfusepath{stroke}%
\end{pgfscope}%
\begin{pgfscope}%
\pgfsetbuttcap%
\pgfsetroundjoin%
\definecolor{currentfill}{rgb}{0.866667,0.517647,0.321569}%
\pgfsetfillcolor{currentfill}%
\pgfsetlinewidth{1.003750pt}%
\definecolor{currentstroke}{rgb}{0.866667,0.517647,0.321569}%
\pgfsetstrokecolor{currentstroke}%
\pgfsetdash{}{0pt}%
\pgfsys@defobject{currentmarker}{\pgfqpoint{-0.020833in}{-0.020833in}}{\pgfqpoint{0.020833in}{0.020833in}}{%
\pgfpathmoveto{\pgfqpoint{0.000000in}{-0.020833in}}%
\pgfpathcurveto{\pgfqpoint{0.005525in}{-0.020833in}}{\pgfqpoint{0.010825in}{-0.018638in}}{\pgfqpoint{0.014731in}{-0.014731in}}%
\pgfpathcurveto{\pgfqpoint{0.018638in}{-0.010825in}}{\pgfqpoint{0.020833in}{-0.005525in}}{\pgfqpoint{0.020833in}{0.000000in}}%
\pgfpathcurveto{\pgfqpoint{0.020833in}{0.005525in}}{\pgfqpoint{0.018638in}{0.010825in}}{\pgfqpoint{0.014731in}{0.014731in}}%
\pgfpathcurveto{\pgfqpoint{0.010825in}{0.018638in}}{\pgfqpoint{0.005525in}{0.020833in}}{\pgfqpoint{0.000000in}{0.020833in}}%
\pgfpathcurveto{\pgfqpoint{-0.005525in}{0.020833in}}{\pgfqpoint{-0.010825in}{0.018638in}}{\pgfqpoint{-0.014731in}{0.014731in}}%
\pgfpathcurveto{\pgfqpoint{-0.018638in}{0.010825in}}{\pgfqpoint{-0.020833in}{0.005525in}}{\pgfqpoint{-0.020833in}{0.000000in}}%
\pgfpathcurveto{\pgfqpoint{-0.020833in}{-0.005525in}}{\pgfqpoint{-0.018638in}{-0.010825in}}{\pgfqpoint{-0.014731in}{-0.014731in}}%
\pgfpathcurveto{\pgfqpoint{-0.010825in}{-0.018638in}}{\pgfqpoint{-0.005525in}{-0.020833in}}{\pgfqpoint{0.000000in}{-0.020833in}}%
\pgfpathclose%
\pgfusepath{stroke,fill}%
}%
\begin{pgfscope}%
\pgfsys@transformshift{4.101858in}{1.802479in}%
\pgfsys@useobject{currentmarker}{}%
\end{pgfscope}%
\end{pgfscope}%
\begin{pgfscope}%
\definecolor{textcolor}{rgb}{0.150000,0.150000,0.150000}%
\pgfsetstrokecolor{textcolor}%
\pgfsetfillcolor{textcolor}%
\pgftext[x=4.583108in,y=1.708902in,left,base]{\color{textcolor}\sffamily\fontsize{19.250000}{23.100000}\selectfont Read-write}%
\end{pgfscope}%
\begin{pgfscope}%
\pgfsetbuttcap%
\pgfsetroundjoin%
\pgfsetlinewidth{1.505625pt}%
\definecolor{currentstroke}{rgb}{0.333333,0.658824,0.407843}%
\pgfsetstrokecolor{currentstroke}%
\pgfsetdash{{9.600000pt}{2.400000pt}{1.500000pt}{2.400000pt}}{0.000000pt}%
\pgfpathmoveto{\pgfqpoint{3.834497in}{1.412731in}}%
\pgfpathlineto{\pgfqpoint{4.369220in}{1.412731in}}%
\pgfusepath{stroke}%
\end{pgfscope}%
\begin{pgfscope}%
\pgfsetbuttcap%
\pgfsetroundjoin%
\definecolor{currentfill}{rgb}{0.333333,0.658824,0.407843}%
\pgfsetfillcolor{currentfill}%
\pgfsetlinewidth{1.003750pt}%
\definecolor{currentstroke}{rgb}{0.333333,0.658824,0.407843}%
\pgfsetstrokecolor{currentstroke}%
\pgfsetdash{}{0pt}%
\pgfsys@defobject{currentmarker}{\pgfqpoint{-0.020833in}{-0.020833in}}{\pgfqpoint{0.020833in}{0.020833in}}{%
\pgfpathmoveto{\pgfqpoint{0.000000in}{-0.020833in}}%
\pgfpathcurveto{\pgfqpoint{0.005525in}{-0.020833in}}{\pgfqpoint{0.010825in}{-0.018638in}}{\pgfqpoint{0.014731in}{-0.014731in}}%
\pgfpathcurveto{\pgfqpoint{0.018638in}{-0.010825in}}{\pgfqpoint{0.020833in}{-0.005525in}}{\pgfqpoint{0.020833in}{0.000000in}}%
\pgfpathcurveto{\pgfqpoint{0.020833in}{0.005525in}}{\pgfqpoint{0.018638in}{0.010825in}}{\pgfqpoint{0.014731in}{0.014731in}}%
\pgfpathcurveto{\pgfqpoint{0.010825in}{0.018638in}}{\pgfqpoint{0.005525in}{0.020833in}}{\pgfqpoint{0.000000in}{0.020833in}}%
\pgfpathcurveto{\pgfqpoint{-0.005525in}{0.020833in}}{\pgfqpoint{-0.010825in}{0.018638in}}{\pgfqpoint{-0.014731in}{0.014731in}}%
\pgfpathcurveto{\pgfqpoint{-0.018638in}{0.010825in}}{\pgfqpoint{-0.020833in}{0.005525in}}{\pgfqpoint{-0.020833in}{0.000000in}}%
\pgfpathcurveto{\pgfqpoint{-0.020833in}{-0.005525in}}{\pgfqpoint{-0.018638in}{-0.010825in}}{\pgfqpoint{-0.014731in}{-0.014731in}}%
\pgfpathcurveto{\pgfqpoint{-0.010825in}{-0.018638in}}{\pgfqpoint{-0.005525in}{-0.020833in}}{\pgfqpoint{0.000000in}{-0.020833in}}%
\pgfpathclose%
\pgfusepath{stroke,fill}%
}%
\begin{pgfscope}%
\pgfsys@transformshift{4.101858in}{1.412731in}%
\pgfsys@useobject{currentmarker}{}%
\end{pgfscope}%
\end{pgfscope}%
\begin{pgfscope}%
\definecolor{textcolor}{rgb}{0.150000,0.150000,0.150000}%
\pgfsetstrokecolor{textcolor}%
\pgfsetfillcolor{textcolor}%
\pgftext[x=4.583108in,y=1.319154in,left,base]{\color{textcolor}\sffamily\fontsize{19.250000}{23.100000}\selectfont Naïve}%
\end{pgfscope}%
\begin{pgfscope}%
\pgfpathrectangle{\pgfqpoint{1.230000in}{1.022500in}}{\pgfqpoint{4.807500in}{3.056125in}}%
\pgfusepath{clip}%
\pgfsetbuttcap%
\pgfsetroundjoin%
\definecolor{currentfill}{rgb}{1.000000,0.000000,0.000000}%
\pgfsetfillcolor{currentfill}%
\pgfsetlinewidth{1.505625pt}%
\definecolor{currentstroke}{rgb}{1.000000,0.000000,0.000000}%
\pgfsetstrokecolor{currentstroke}%
\pgfsetdash{}{0pt}%
\pgfpathmoveto{\pgfqpoint{5.727650in}{2.695221in}}%
\pgfpathlineto{\pgfqpoint{5.835232in}{2.802804in}}%
\pgfpathmoveto{\pgfqpoint{5.727650in}{2.802804in}}%
\pgfpathlineto{\pgfqpoint{5.835232in}{2.695221in}}%
\pgfusepath{stroke,fill}%
\end{pgfscope}%
\begin{pgfscope}%
\pgfpathrectangle{\pgfqpoint{1.230000in}{1.022500in}}{\pgfqpoint{4.807500in}{3.056125in}}%
\pgfusepath{clip}%
\pgfsetbuttcap%
\pgfsetroundjoin%
\definecolor{currentfill}{rgb}{1.000000,0.000000,0.000000}%
\pgfsetfillcolor{currentfill}%
\pgfsetlinewidth{1.505625pt}%
\definecolor{currentstroke}{rgb}{1.000000,0.000000,0.000000}%
\pgfsetstrokecolor{currentstroke}%
\pgfsetdash{}{0pt}%
\pgfpathmoveto{\pgfqpoint{5.727650in}{2.562921in}}%
\pgfpathlineto{\pgfqpoint{5.835232in}{2.670504in}}%
\pgfpathmoveto{\pgfqpoint{5.727650in}{2.670504in}}%
\pgfpathlineto{\pgfqpoint{5.835232in}{2.562921in}}%
\pgfusepath{stroke,fill}%
\end{pgfscope}%
\end{pgfpicture}%
\makeatother%
\endgroup%

  % }
  % \begin{center}
  %   \resizebox{.5\textwidth}{!}{
  %   %% Creator: Matplotlib, PGF backend
%%
%% To include the figure in your LaTeX document, write
%%   \input{<filename>.pgf}
%%
%% Make sure the required packages are loaded in your preamble
%%   \usepackage{pgf}
%%
%% Figures using additional raster images can only be included by \input if
%% they are in the same directory as the main LaTeX file. For loading figures
%% from other directories you can use the `import` package
%%   \usepackage{import}
%% and then include the figures with
%%   \import{<path to file>}{<filename>.pgf}
%%
%% Matplotlib used the following preamble
%%
\begingroup%
\makeatletter%
\begin{pgfpicture}%
\pgfpathrectangle{\pgfpointorigin}{\pgfqpoint{6.400000in}{4.800000in}}%
\pgfusepath{use as bounding box, clip}%
\begin{pgfscope}%
\pgfsetbuttcap%
\pgfsetmiterjoin%
\definecolor{currentfill}{rgb}{1.000000,1.000000,1.000000}%
\pgfsetfillcolor{currentfill}%
\pgfsetlinewidth{0.000000pt}%
\definecolor{currentstroke}{rgb}{1.000000,1.000000,1.000000}%
\pgfsetstrokecolor{currentstroke}%
\pgfsetdash{}{0pt}%
\pgfpathmoveto{\pgfqpoint{0.000000in}{0.000000in}}%
\pgfpathlineto{\pgfqpoint{6.400000in}{0.000000in}}%
\pgfpathlineto{\pgfqpoint{6.400000in}{4.800000in}}%
\pgfpathlineto{\pgfqpoint{0.000000in}{4.800000in}}%
\pgfpathclose%
\pgfusepath{fill}%
\end{pgfscope}%
\begin{pgfscope}%
\pgfsetbuttcap%
\pgfsetmiterjoin%
\definecolor{currentfill}{rgb}{1.000000,1.000000,1.000000}%
\pgfsetfillcolor{currentfill}%
\pgfsetlinewidth{0.000000pt}%
\definecolor{currentstroke}{rgb}{0.000000,0.000000,0.000000}%
\pgfsetstrokecolor{currentstroke}%
\pgfsetstrokeopacity{0.000000}%
\pgfsetdash{}{0pt}%
\pgfpathmoveto{\pgfqpoint{1.230000in}{1.022500in}}%
\pgfpathlineto{\pgfqpoint{5.591250in}{1.022500in}}%
\pgfpathlineto{\pgfqpoint{5.591250in}{4.078625in}}%
\pgfpathlineto{\pgfqpoint{1.230000in}{4.078625in}}%
\pgfpathclose%
\pgfusepath{fill}%
\end{pgfscope}%
\begin{pgfscope}%
\pgfpathrectangle{\pgfqpoint{1.230000in}{1.022500in}}{\pgfqpoint{4.361250in}{3.056125in}}%
\pgfusepath{clip}%
\pgfsetroundcap%
\pgfsetroundjoin%
\pgfsetlinewidth{1.003750pt}%
\definecolor{currentstroke}{rgb}{0.800000,0.800000,0.800000}%
\pgfsetstrokecolor{currentstroke}%
\pgfsetstrokeopacity{0.400000}%
\pgfsetdash{}{0pt}%
\pgfpathmoveto{\pgfqpoint{1.428239in}{1.022500in}}%
\pgfpathlineto{\pgfqpoint{1.428239in}{4.078625in}}%
\pgfusepath{stroke}%
\end{pgfscope}%
\begin{pgfscope}%
\definecolor{textcolor}{rgb}{0.150000,0.150000,0.150000}%
\pgfsetstrokecolor{textcolor}%
\pgfsetfillcolor{textcolor}%
\pgftext[x=1.428239in,y=0.890556in,,top]{\color{textcolor}\sffamily\fontsize{19.250000}{23.100000}\selectfont 0}%
\end{pgfscope}%
\begin{pgfscope}%
\pgfpathrectangle{\pgfqpoint{1.230000in}{1.022500in}}{\pgfqpoint{4.361250in}{3.056125in}}%
\pgfusepath{clip}%
\pgfsetroundcap%
\pgfsetroundjoin%
\pgfsetlinewidth{1.003750pt}%
\definecolor{currentstroke}{rgb}{0.800000,0.800000,0.800000}%
\pgfsetstrokecolor{currentstroke}%
\pgfsetstrokeopacity{0.400000}%
\pgfsetdash{}{0pt}%
\pgfpathmoveto{\pgfqpoint{2.789861in}{1.022500in}}%
\pgfpathlineto{\pgfqpoint{2.789861in}{4.078625in}}%
\pgfusepath{stroke}%
\end{pgfscope}%
\begin{pgfscope}%
\definecolor{textcolor}{rgb}{0.150000,0.150000,0.150000}%
\pgfsetstrokecolor{textcolor}%
\pgfsetfillcolor{textcolor}%
\pgftext[x=2.789861in,y=0.890556in,,top]{\color{textcolor}\sffamily\fontsize{19.250000}{23.100000}\selectfont 5000}%
\end{pgfscope}%
\begin{pgfscope}%
\pgfpathrectangle{\pgfqpoint{1.230000in}{1.022500in}}{\pgfqpoint{4.361250in}{3.056125in}}%
\pgfusepath{clip}%
\pgfsetroundcap%
\pgfsetroundjoin%
\pgfsetlinewidth{1.003750pt}%
\definecolor{currentstroke}{rgb}{0.800000,0.800000,0.800000}%
\pgfsetstrokecolor{currentstroke}%
\pgfsetstrokeopacity{0.400000}%
\pgfsetdash{}{0pt}%
\pgfpathmoveto{\pgfqpoint{4.151484in}{1.022500in}}%
\pgfpathlineto{\pgfqpoint{4.151484in}{4.078625in}}%
\pgfusepath{stroke}%
\end{pgfscope}%
\begin{pgfscope}%
\definecolor{textcolor}{rgb}{0.150000,0.150000,0.150000}%
\pgfsetstrokecolor{textcolor}%
\pgfsetfillcolor{textcolor}%
\pgftext[x=4.151484in,y=0.890556in,,top]{\color{textcolor}\sffamily\fontsize{19.250000}{23.100000}\selectfont 10000}%
\end{pgfscope}%
\begin{pgfscope}%
\pgfpathrectangle{\pgfqpoint{1.230000in}{1.022500in}}{\pgfqpoint{4.361250in}{3.056125in}}%
\pgfusepath{clip}%
\pgfsetroundcap%
\pgfsetroundjoin%
\pgfsetlinewidth{1.003750pt}%
\definecolor{currentstroke}{rgb}{0.800000,0.800000,0.800000}%
\pgfsetstrokecolor{currentstroke}%
\pgfsetstrokeopacity{0.400000}%
\pgfsetdash{}{0pt}%
\pgfpathmoveto{\pgfqpoint{5.513106in}{1.022500in}}%
\pgfpathlineto{\pgfqpoint{5.513106in}{4.078625in}}%
\pgfusepath{stroke}%
\end{pgfscope}%
\begin{pgfscope}%
\definecolor{textcolor}{rgb}{0.150000,0.150000,0.150000}%
\pgfsetstrokecolor{textcolor}%
\pgfsetfillcolor{textcolor}%
\pgftext[x=5.513106in,y=0.890556in,,top]{\color{textcolor}\sffamily\fontsize{19.250000}{23.100000}\selectfont 15000}%
\end{pgfscope}%
\begin{pgfscope}%
\definecolor{textcolor}{rgb}{0.150000,0.150000,0.150000}%
\pgfsetstrokecolor{textcolor}%
\pgfsetfillcolor{textcolor}%
\pgftext[x=3.410625in,y=0.578932in,,top]{\color{textcolor}\sffamily\fontsize{21.000000}{25.200000}\selectfont Time (s)}%
\end{pgfscope}%
\begin{pgfscope}%
\pgfpathrectangle{\pgfqpoint{1.230000in}{1.022500in}}{\pgfqpoint{4.361250in}{3.056125in}}%
\pgfusepath{clip}%
\pgfsetroundcap%
\pgfsetroundjoin%
\pgfsetlinewidth{1.003750pt}%
\definecolor{currentstroke}{rgb}{0.800000,0.800000,0.800000}%
\pgfsetstrokecolor{currentstroke}%
\pgfsetstrokeopacity{0.400000}%
\pgfsetdash{}{0pt}%
\pgfpathmoveto{\pgfqpoint{1.230000in}{1.161415in}}%
\pgfpathlineto{\pgfqpoint{5.591250in}{1.161415in}}%
\pgfusepath{stroke}%
\end{pgfscope}%
\begin{pgfscope}%
\definecolor{textcolor}{rgb}{0.150000,0.150000,0.150000}%
\pgfsetstrokecolor{textcolor}%
\pgfsetfillcolor{textcolor}%
\pgftext[x=0.962614in,y=1.061396in,left,base]{\color{textcolor}\sffamily\fontsize{19.250000}{23.100000}\selectfont 0}%
\end{pgfscope}%
\begin{pgfscope}%
\pgfpathrectangle{\pgfqpoint{1.230000in}{1.022500in}}{\pgfqpoint{4.361250in}{3.056125in}}%
\pgfusepath{clip}%
\pgfsetroundcap%
\pgfsetroundjoin%
\pgfsetlinewidth{1.003750pt}%
\definecolor{currentstroke}{rgb}{0.800000,0.800000,0.800000}%
\pgfsetstrokecolor{currentstroke}%
\pgfsetstrokeopacity{0.400000}%
\pgfsetdash{}{0pt}%
\pgfpathmoveto{\pgfqpoint{1.230000in}{1.855989in}}%
\pgfpathlineto{\pgfqpoint{5.591250in}{1.855989in}}%
\pgfusepath{stroke}%
\end{pgfscope}%
\begin{pgfscope}%
\definecolor{textcolor}{rgb}{0.150000,0.150000,0.150000}%
\pgfsetstrokecolor{textcolor}%
\pgfsetfillcolor{textcolor}%
\pgftext[x=0.827172in,y=1.755969in,left,base]{\color{textcolor}\sffamily\fontsize{19.250000}{23.100000}\selectfont 25}%
\end{pgfscope}%
\begin{pgfscope}%
\pgfpathrectangle{\pgfqpoint{1.230000in}{1.022500in}}{\pgfqpoint{4.361250in}{3.056125in}}%
\pgfusepath{clip}%
\pgfsetroundcap%
\pgfsetroundjoin%
\pgfsetlinewidth{1.003750pt}%
\definecolor{currentstroke}{rgb}{0.800000,0.800000,0.800000}%
\pgfsetstrokecolor{currentstroke}%
\pgfsetstrokeopacity{0.400000}%
\pgfsetdash{}{0pt}%
\pgfpathmoveto{\pgfqpoint{1.230000in}{2.550562in}}%
\pgfpathlineto{\pgfqpoint{5.591250in}{2.550562in}}%
\pgfusepath{stroke}%
\end{pgfscope}%
\begin{pgfscope}%
\definecolor{textcolor}{rgb}{0.150000,0.150000,0.150000}%
\pgfsetstrokecolor{textcolor}%
\pgfsetfillcolor{textcolor}%
\pgftext[x=0.827172in,y=2.450543in,left,base]{\color{textcolor}\sffamily\fontsize{19.250000}{23.100000}\selectfont 50}%
\end{pgfscope}%
\begin{pgfscope}%
\pgfpathrectangle{\pgfqpoint{1.230000in}{1.022500in}}{\pgfqpoint{4.361250in}{3.056125in}}%
\pgfusepath{clip}%
\pgfsetroundcap%
\pgfsetroundjoin%
\pgfsetlinewidth{1.003750pt}%
\definecolor{currentstroke}{rgb}{0.800000,0.800000,0.800000}%
\pgfsetstrokecolor{currentstroke}%
\pgfsetstrokeopacity{0.400000}%
\pgfsetdash{}{0pt}%
\pgfpathmoveto{\pgfqpoint{1.230000in}{3.245136in}}%
\pgfpathlineto{\pgfqpoint{5.591250in}{3.245136in}}%
\pgfusepath{stroke}%
\end{pgfscope}%
\begin{pgfscope}%
\definecolor{textcolor}{rgb}{0.150000,0.150000,0.150000}%
\pgfsetstrokecolor{textcolor}%
\pgfsetfillcolor{textcolor}%
\pgftext[x=0.827172in,y=3.145117in,left,base]{\color{textcolor}\sffamily\fontsize{19.250000}{23.100000}\selectfont 75}%
\end{pgfscope}%
\begin{pgfscope}%
\pgfpathrectangle{\pgfqpoint{1.230000in}{1.022500in}}{\pgfqpoint{4.361250in}{3.056125in}}%
\pgfusepath{clip}%
\pgfsetroundcap%
\pgfsetroundjoin%
\pgfsetlinewidth{1.003750pt}%
\definecolor{currentstroke}{rgb}{0.800000,0.800000,0.800000}%
\pgfsetstrokecolor{currentstroke}%
\pgfsetstrokeopacity{0.400000}%
\pgfsetdash{}{0pt}%
\pgfpathmoveto{\pgfqpoint{1.230000in}{3.939710in}}%
\pgfpathlineto{\pgfqpoint{5.591250in}{3.939710in}}%
\pgfusepath{stroke}%
\end{pgfscope}%
\begin{pgfscope}%
\definecolor{textcolor}{rgb}{0.150000,0.150000,0.150000}%
\pgfsetstrokecolor{textcolor}%
\pgfsetfillcolor{textcolor}%
\pgftext[x=0.691731in,y=3.839691in,left,base]{\color{textcolor}\sffamily\fontsize{19.250000}{23.100000}\selectfont 100}%
\end{pgfscope}%
\begin{pgfscope}%
\definecolor{textcolor}{rgb}{0.150000,0.150000,0.150000}%
\pgfsetstrokecolor{textcolor}%
\pgfsetfillcolor{textcolor}%
\pgftext[x=0.636175in,y=2.550563in,,bottom,rotate=90.000000]{\color{textcolor}\sffamily\fontsize{21.000000}{25.200000}\selectfont Percent of instructions synthesized}%
\end{pgfscope}%
\begin{pgfscope}%
\pgfpathrectangle{\pgfqpoint{1.230000in}{1.022500in}}{\pgfqpoint{4.361250in}{3.056125in}}%
\pgfusepath{clip}%
\pgfsetroundcap%
\pgfsetroundjoin%
\pgfsetlinewidth{1.505625pt}%
\definecolor{currentstroke}{rgb}{0.298039,0.447059,0.690196}%
\pgfsetstrokecolor{currentstroke}%
\pgfsetdash{}{0pt}%
\pgfpathmoveto{\pgfqpoint{1.428239in}{1.161415in}}%
\pgfpathlineto{\pgfqpoint{1.444578in}{1.161415in}}%
\pgfpathlineto{\pgfqpoint{1.444578in}{1.558314in}}%
\pgfpathlineto{\pgfqpoint{1.456833in}{1.558314in}}%
\pgfpathlineto{\pgfqpoint{1.456833in}{1.955213in}}%
\pgfpathlineto{\pgfqpoint{1.469087in}{1.955213in}}%
\pgfpathlineto{\pgfqpoint{1.469087in}{2.352113in}}%
\pgfpathlineto{\pgfqpoint{1.485427in}{2.352113in}}%
\pgfpathlineto{\pgfqpoint{1.485427in}{2.749012in}}%
\pgfpathlineto{\pgfqpoint{1.497681in}{2.749012in}}%
\pgfpathlineto{\pgfqpoint{1.497681in}{3.145912in}}%
\pgfpathlineto{\pgfqpoint{1.524369in}{3.145912in}}%
\pgfpathlineto{\pgfqpoint{1.524369in}{3.542811in}}%
\pgfpathlineto{\pgfqpoint{1.543977in}{3.542811in}}%
\pgfpathlineto{\pgfqpoint{1.543977in}{3.939710in}}%
\pgfusepath{stroke}%
\end{pgfscope}%
\begin{pgfscope}%
\pgfpathrectangle{\pgfqpoint{1.230000in}{1.022500in}}{\pgfqpoint{4.361250in}{3.056125in}}%
\pgfusepath{clip}%
\pgfsetbuttcap%
\pgfsetroundjoin%
\definecolor{currentfill}{rgb}{0.298039,0.447059,0.690196}%
\pgfsetfillcolor{currentfill}%
\pgfsetlinewidth{1.003750pt}%
\definecolor{currentstroke}{rgb}{0.298039,0.447059,0.690196}%
\pgfsetstrokecolor{currentstroke}%
\pgfsetdash{}{0pt}%
\pgfsys@defobject{currentmarker}{\pgfqpoint{-0.020833in}{-0.020833in}}{\pgfqpoint{0.020833in}{0.020833in}}{%
\pgfpathmoveto{\pgfqpoint{0.000000in}{-0.020833in}}%
\pgfpathcurveto{\pgfqpoint{0.005525in}{-0.020833in}}{\pgfqpoint{0.010825in}{-0.018638in}}{\pgfqpoint{0.014731in}{-0.014731in}}%
\pgfpathcurveto{\pgfqpoint{0.018638in}{-0.010825in}}{\pgfqpoint{0.020833in}{-0.005525in}}{\pgfqpoint{0.020833in}{0.000000in}}%
\pgfpathcurveto{\pgfqpoint{0.020833in}{0.005525in}}{\pgfqpoint{0.018638in}{0.010825in}}{\pgfqpoint{0.014731in}{0.014731in}}%
\pgfpathcurveto{\pgfqpoint{0.010825in}{0.018638in}}{\pgfqpoint{0.005525in}{0.020833in}}{\pgfqpoint{0.000000in}{0.020833in}}%
\pgfpathcurveto{\pgfqpoint{-0.005525in}{0.020833in}}{\pgfqpoint{-0.010825in}{0.018638in}}{\pgfqpoint{-0.014731in}{0.014731in}}%
\pgfpathcurveto{\pgfqpoint{-0.018638in}{0.010825in}}{\pgfqpoint{-0.020833in}{0.005525in}}{\pgfqpoint{-0.020833in}{0.000000in}}%
\pgfpathcurveto{\pgfqpoint{-0.020833in}{-0.005525in}}{\pgfqpoint{-0.018638in}{-0.010825in}}{\pgfqpoint{-0.014731in}{-0.014731in}}%
\pgfpathcurveto{\pgfqpoint{-0.010825in}{-0.018638in}}{\pgfqpoint{-0.005525in}{-0.020833in}}{\pgfqpoint{0.000000in}{-0.020833in}}%
\pgfpathclose%
\pgfusepath{stroke,fill}%
}%
\begin{pgfscope}%
\pgfsys@transformshift{1.428239in}{1.161415in}%
\pgfsys@useobject{currentmarker}{}%
\end{pgfscope}%
\begin{pgfscope}%
\pgfsys@transformshift{1.543977in}{3.939710in}%
\pgfsys@useobject{currentmarker}{}%
\end{pgfscope}%
\end{pgfscope}%
\begin{pgfscope}%
\pgfpathrectangle{\pgfqpoint{1.230000in}{1.022500in}}{\pgfqpoint{4.361250in}{3.056125in}}%
\pgfusepath{clip}%
\pgfsetbuttcap%
\pgfsetroundjoin%
\pgfsetlinewidth{1.505625pt}%
\definecolor{currentstroke}{rgb}{0.866667,0.517647,0.321569}%
\pgfsetstrokecolor{currentstroke}%
\pgfsetdash{{5.550000pt}{2.400000pt}}{0.000000pt}%
\pgfpathmoveto{\pgfqpoint{1.444578in}{1.161415in}}%
\pgfpathlineto{\pgfqpoint{1.444578in}{1.558314in}}%
\pgfpathlineto{\pgfqpoint{1.456833in}{1.558314in}}%
\pgfpathlineto{\pgfqpoint{1.456833in}{1.955213in}}%
\pgfpathlineto{\pgfqpoint{1.469087in}{1.955213in}}%
\pgfpathlineto{\pgfqpoint{1.469087in}{2.352113in}}%
\pgfpathlineto{\pgfqpoint{1.485427in}{2.352113in}}%
\pgfpathlineto{\pgfqpoint{1.485427in}{2.749012in}}%
\pgfpathlineto{\pgfqpoint{1.509936in}{2.749012in}}%
\pgfpathlineto{\pgfqpoint{1.509936in}{3.145912in}}%
\pgfpathlineto{\pgfqpoint{1.593267in}{3.145912in}}%
\pgfpathlineto{\pgfqpoint{1.593267in}{3.542811in}}%
\pgfpathlineto{\pgfqpoint{2.138189in}{3.542811in}}%
\pgfpathlineto{\pgfqpoint{2.138189in}{3.939710in}}%
\pgfusepath{stroke}%
\end{pgfscope}%
\begin{pgfscope}%
\pgfpathrectangle{\pgfqpoint{1.230000in}{1.022500in}}{\pgfqpoint{4.361250in}{3.056125in}}%
\pgfusepath{clip}%
\pgfsetbuttcap%
\pgfsetroundjoin%
\definecolor{currentfill}{rgb}{0.866667,0.517647,0.321569}%
\pgfsetfillcolor{currentfill}%
\pgfsetlinewidth{1.003750pt}%
\definecolor{currentstroke}{rgb}{0.866667,0.517647,0.321569}%
\pgfsetstrokecolor{currentstroke}%
\pgfsetdash{}{0pt}%
\pgfsys@defobject{currentmarker}{\pgfqpoint{-0.020833in}{-0.020833in}}{\pgfqpoint{0.020833in}{0.020833in}}{%
\pgfpathmoveto{\pgfqpoint{0.000000in}{-0.020833in}}%
\pgfpathcurveto{\pgfqpoint{0.005525in}{-0.020833in}}{\pgfqpoint{0.010825in}{-0.018638in}}{\pgfqpoint{0.014731in}{-0.014731in}}%
\pgfpathcurveto{\pgfqpoint{0.018638in}{-0.010825in}}{\pgfqpoint{0.020833in}{-0.005525in}}{\pgfqpoint{0.020833in}{0.000000in}}%
\pgfpathcurveto{\pgfqpoint{0.020833in}{0.005525in}}{\pgfqpoint{0.018638in}{0.010825in}}{\pgfqpoint{0.014731in}{0.014731in}}%
\pgfpathcurveto{\pgfqpoint{0.010825in}{0.018638in}}{\pgfqpoint{0.005525in}{0.020833in}}{\pgfqpoint{0.000000in}{0.020833in}}%
\pgfpathcurveto{\pgfqpoint{-0.005525in}{0.020833in}}{\pgfqpoint{-0.010825in}{0.018638in}}{\pgfqpoint{-0.014731in}{0.014731in}}%
\pgfpathcurveto{\pgfqpoint{-0.018638in}{0.010825in}}{\pgfqpoint{-0.020833in}{0.005525in}}{\pgfqpoint{-0.020833in}{0.000000in}}%
\pgfpathcurveto{\pgfqpoint{-0.020833in}{-0.005525in}}{\pgfqpoint{-0.018638in}{-0.010825in}}{\pgfqpoint{-0.014731in}{-0.014731in}}%
\pgfpathcurveto{\pgfqpoint{-0.010825in}{-0.018638in}}{\pgfqpoint{-0.005525in}{-0.020833in}}{\pgfqpoint{0.000000in}{-0.020833in}}%
\pgfpathclose%
\pgfusepath{stroke,fill}%
}%
\begin{pgfscope}%
\pgfsys@transformshift{2.138189in}{3.939710in}%
\pgfsys@useobject{currentmarker}{}%
\end{pgfscope}%
\end{pgfscope}%
\begin{pgfscope}%
\pgfpathrectangle{\pgfqpoint{1.230000in}{1.022500in}}{\pgfqpoint{4.361250in}{3.056125in}}%
\pgfusepath{clip}%
\pgfsetbuttcap%
\pgfsetroundjoin%
\pgfsetlinewidth{1.505625pt}%
\definecolor{currentstroke}{rgb}{0.333333,0.658824,0.407843}%
\pgfsetstrokecolor{currentstroke}%
\pgfsetdash{{9.600000pt}{2.400000pt}{1.500000pt}{2.400000pt}}{0.000000pt}%
\pgfpathmoveto{\pgfqpoint{1.428239in}{1.161415in}}%
\pgfpathlineto{\pgfqpoint{1.481887in}{1.161415in}}%
\pgfpathlineto{\pgfqpoint{1.481887in}{1.558314in}}%
\pgfpathlineto{\pgfqpoint{3.579058in}{1.558314in}}%
\pgfpathlineto{\pgfqpoint{3.579058in}{1.955213in}}%
\pgfpathlineto{\pgfqpoint{3.629710in}{1.955213in}}%
\pgfpathlineto{\pgfqpoint{3.629710in}{2.352113in}}%
\pgfpathlineto{\pgfqpoint{3.682813in}{2.352113in}}%
\pgfpathlineto{\pgfqpoint{3.682813in}{2.749012in}}%
\pgfpathlineto{\pgfqpoint{5.288711in}{2.749012in}}%
\pgfpathlineto{\pgfqpoint{5.288711in}{3.145912in}}%
\pgfpathlineto{\pgfqpoint{5.342631in}{3.145912in}}%
\pgfpathlineto{\pgfqpoint{5.342631in}{3.542811in}}%
\pgfpathlineto{\pgfqpoint{5.393011in}{3.542811in}}%
\pgfpathlineto{\pgfqpoint{5.393011in}{3.939710in}}%
\pgfusepath{stroke}%
\end{pgfscope}%
\begin{pgfscope}%
\pgfpathrectangle{\pgfqpoint{1.230000in}{1.022500in}}{\pgfqpoint{4.361250in}{3.056125in}}%
\pgfusepath{clip}%
\pgfsetbuttcap%
\pgfsetroundjoin%
\definecolor{currentfill}{rgb}{0.333333,0.658824,0.407843}%
\pgfsetfillcolor{currentfill}%
\pgfsetlinewidth{1.003750pt}%
\definecolor{currentstroke}{rgb}{0.333333,0.658824,0.407843}%
\pgfsetstrokecolor{currentstroke}%
\pgfsetdash{}{0pt}%
\pgfsys@defobject{currentmarker}{\pgfqpoint{-0.020833in}{-0.020833in}}{\pgfqpoint{0.020833in}{0.020833in}}{%
\pgfpathmoveto{\pgfqpoint{0.000000in}{-0.020833in}}%
\pgfpathcurveto{\pgfqpoint{0.005525in}{-0.020833in}}{\pgfqpoint{0.010825in}{-0.018638in}}{\pgfqpoint{0.014731in}{-0.014731in}}%
\pgfpathcurveto{\pgfqpoint{0.018638in}{-0.010825in}}{\pgfqpoint{0.020833in}{-0.005525in}}{\pgfqpoint{0.020833in}{0.000000in}}%
\pgfpathcurveto{\pgfqpoint{0.020833in}{0.005525in}}{\pgfqpoint{0.018638in}{0.010825in}}{\pgfqpoint{0.014731in}{0.014731in}}%
\pgfpathcurveto{\pgfqpoint{0.010825in}{0.018638in}}{\pgfqpoint{0.005525in}{0.020833in}}{\pgfqpoint{0.000000in}{0.020833in}}%
\pgfpathcurveto{\pgfqpoint{-0.005525in}{0.020833in}}{\pgfqpoint{-0.010825in}{0.018638in}}{\pgfqpoint{-0.014731in}{0.014731in}}%
\pgfpathcurveto{\pgfqpoint{-0.018638in}{0.010825in}}{\pgfqpoint{-0.020833in}{0.005525in}}{\pgfqpoint{-0.020833in}{0.000000in}}%
\pgfpathcurveto{\pgfqpoint{-0.020833in}{-0.005525in}}{\pgfqpoint{-0.018638in}{-0.010825in}}{\pgfqpoint{-0.014731in}{-0.014731in}}%
\pgfpathcurveto{\pgfqpoint{-0.010825in}{-0.018638in}}{\pgfqpoint{-0.005525in}{-0.020833in}}{\pgfqpoint{0.000000in}{-0.020833in}}%
\pgfpathclose%
\pgfusepath{stroke,fill}%
}%
\begin{pgfscope}%
\pgfsys@transformshift{1.428239in}{1.161415in}%
\pgfsys@useobject{currentmarker}{}%
\end{pgfscope}%
\begin{pgfscope}%
\pgfsys@transformshift{5.393011in}{3.939710in}%
\pgfsys@useobject{currentmarker}{}%
\end{pgfscope}%
\end{pgfscope}%
\begin{pgfscope}%
\pgfsetrectcap%
\pgfsetmiterjoin%
\pgfsetlinewidth{1.254687pt}%
\definecolor{currentstroke}{rgb}{0.800000,0.800000,0.800000}%
\pgfsetstrokecolor{currentstroke}%
\pgfsetdash{}{0pt}%
\pgfpathmoveto{\pgfqpoint{1.230000in}{1.022500in}}%
\pgfpathlineto{\pgfqpoint{1.230000in}{4.078625in}}%
\pgfusepath{stroke}%
\end{pgfscope}%
\begin{pgfscope}%
\pgfsetrectcap%
\pgfsetmiterjoin%
\pgfsetlinewidth{1.254687pt}%
\definecolor{currentstroke}{rgb}{0.800000,0.800000,0.800000}%
\pgfsetstrokecolor{currentstroke}%
\pgfsetdash{}{0pt}%
\pgfpathmoveto{\pgfqpoint{5.591250in}{1.022500in}}%
\pgfpathlineto{\pgfqpoint{5.591250in}{4.078625in}}%
\pgfusepath{stroke}%
\end{pgfscope}%
\begin{pgfscope}%
\pgfsetrectcap%
\pgfsetmiterjoin%
\pgfsetlinewidth{1.254687pt}%
\definecolor{currentstroke}{rgb}{0.800000,0.800000,0.800000}%
\pgfsetstrokecolor{currentstroke}%
\pgfsetdash{}{0pt}%
\pgfpathmoveto{\pgfqpoint{1.230000in}{1.022500in}}%
\pgfpathlineto{\pgfqpoint{5.591250in}{1.022500in}}%
\pgfusepath{stroke}%
\end{pgfscope}%
\begin{pgfscope}%
\pgfsetrectcap%
\pgfsetmiterjoin%
\pgfsetlinewidth{1.254687pt}%
\definecolor{currentstroke}{rgb}{0.800000,0.800000,0.800000}%
\pgfsetstrokecolor{currentstroke}%
\pgfsetdash{}{0pt}%
\pgfpathmoveto{\pgfqpoint{1.230000in}{4.078625in}}%
\pgfpathlineto{\pgfqpoint{5.591250in}{4.078625in}}%
\pgfusepath{stroke}%
\end{pgfscope}%
\begin{pgfscope}%
\definecolor{textcolor}{rgb}{0.150000,0.150000,0.150000}%
\pgfsetstrokecolor{textcolor}%
\pgfsetfillcolor{textcolor}%
\pgftext[x=3.410625in,y=4.161958in,,base]{\color{textcolor}\sffamily\fontsize{21.000000}{25.200000}\selectfont Synthesis time for classic libseccomp to eBPF}%
\end{pgfscope}%
\begin{pgfscope}%
\pgfsetbuttcap%
\pgfsetmiterjoin%
\definecolor{currentfill}{rgb}{1.000000,1.000000,1.000000}%
\pgfsetfillcolor{currentfill}%
\pgfsetfillopacity{0.800000}%
\pgfsetlinewidth{1.003750pt}%
\definecolor{currentstroke}{rgb}{0.800000,0.800000,0.800000}%
\pgfsetstrokecolor{currentstroke}%
\pgfsetstrokeopacity{0.800000}%
\pgfsetdash{}{0pt}%
\pgfpathmoveto{\pgfqpoint{2.375964in}{2.695492in}}%
\pgfpathlineto{\pgfqpoint{4.445286in}{2.695492in}}%
\pgfpathquadraticcurveto{\pgfqpoint{4.498758in}{2.695492in}}{\pgfqpoint{4.498758in}{2.748964in}}%
\pgfpathlineto{\pgfqpoint{4.498758in}{3.891472in}}%
\pgfpathquadraticcurveto{\pgfqpoint{4.498758in}{3.944944in}}{\pgfqpoint{4.445286in}{3.944944in}}%
\pgfpathlineto{\pgfqpoint{2.375964in}{3.944944in}}%
\pgfpathquadraticcurveto{\pgfqpoint{2.322492in}{3.944944in}}{\pgfqpoint{2.322492in}{3.891472in}}%
\pgfpathlineto{\pgfqpoint{2.322492in}{2.748964in}}%
\pgfpathquadraticcurveto{\pgfqpoint{2.322492in}{2.695492in}}{\pgfqpoint{2.375964in}{2.695492in}}%
\pgfpathclose%
\pgfusepath{stroke,fill}%
\end{pgfscope}%
\begin{pgfscope}%
\pgfsetroundcap%
\pgfsetroundjoin%
\pgfsetlinewidth{1.505625pt}%
\definecolor{currentstroke}{rgb}{0.298039,0.447059,0.690196}%
\pgfsetstrokecolor{currentstroke}%
\pgfsetdash{}{0pt}%
\pgfpathmoveto{\pgfqpoint{2.429436in}{3.731538in}}%
\pgfpathlineto{\pgfqpoint{2.964158in}{3.731538in}}%
\pgfusepath{stroke}%
\end{pgfscope}%
\begin{pgfscope}%
\pgfsetbuttcap%
\pgfsetroundjoin%
\definecolor{currentfill}{rgb}{0.298039,0.447059,0.690196}%
\pgfsetfillcolor{currentfill}%
\pgfsetlinewidth{1.003750pt}%
\definecolor{currentstroke}{rgb}{0.298039,0.447059,0.690196}%
\pgfsetstrokecolor{currentstroke}%
\pgfsetdash{}{0pt}%
\pgfsys@defobject{currentmarker}{\pgfqpoint{-0.020833in}{-0.020833in}}{\pgfqpoint{0.020833in}{0.020833in}}{%
\pgfpathmoveto{\pgfqpoint{0.000000in}{-0.020833in}}%
\pgfpathcurveto{\pgfqpoint{0.005525in}{-0.020833in}}{\pgfqpoint{0.010825in}{-0.018638in}}{\pgfqpoint{0.014731in}{-0.014731in}}%
\pgfpathcurveto{\pgfqpoint{0.018638in}{-0.010825in}}{\pgfqpoint{0.020833in}{-0.005525in}}{\pgfqpoint{0.020833in}{0.000000in}}%
\pgfpathcurveto{\pgfqpoint{0.020833in}{0.005525in}}{\pgfqpoint{0.018638in}{0.010825in}}{\pgfqpoint{0.014731in}{0.014731in}}%
\pgfpathcurveto{\pgfqpoint{0.010825in}{0.018638in}}{\pgfqpoint{0.005525in}{0.020833in}}{\pgfqpoint{0.000000in}{0.020833in}}%
\pgfpathcurveto{\pgfqpoint{-0.005525in}{0.020833in}}{\pgfqpoint{-0.010825in}{0.018638in}}{\pgfqpoint{-0.014731in}{0.014731in}}%
\pgfpathcurveto{\pgfqpoint{-0.018638in}{0.010825in}}{\pgfqpoint{-0.020833in}{0.005525in}}{\pgfqpoint{-0.020833in}{0.000000in}}%
\pgfpathcurveto{\pgfqpoint{-0.020833in}{-0.005525in}}{\pgfqpoint{-0.018638in}{-0.010825in}}{\pgfqpoint{-0.014731in}{-0.014731in}}%
\pgfpathcurveto{\pgfqpoint{-0.010825in}{-0.018638in}}{\pgfqpoint{-0.005525in}{-0.020833in}}{\pgfqpoint{0.000000in}{-0.020833in}}%
\pgfpathclose%
\pgfusepath{stroke,fill}%
}%
\begin{pgfscope}%
\pgfsys@transformshift{2.696797in}{3.731538in}%
\pgfsys@useobject{currentmarker}{}%
\end{pgfscope}%
\end{pgfscope}%
\begin{pgfscope}%
\definecolor{textcolor}{rgb}{0.150000,0.150000,0.150000}%
\pgfsetstrokecolor{textcolor}%
\pgfsetfillcolor{textcolor}%
\pgftext[x=3.178047in,y=3.637962in,left,base]{\color{textcolor}\sffamily\fontsize{19.250000}{23.100000}\selectfont Pre-load}%
\end{pgfscope}%
\begin{pgfscope}%
\pgfsetbuttcap%
\pgfsetroundjoin%
\pgfsetlinewidth{1.505625pt}%
\definecolor{currentstroke}{rgb}{0.866667,0.517647,0.321569}%
\pgfsetstrokecolor{currentstroke}%
\pgfsetdash{{5.550000pt}{2.400000pt}}{0.000000pt}%
\pgfpathmoveto{\pgfqpoint{2.429436in}{3.341790in}}%
\pgfpathlineto{\pgfqpoint{2.964158in}{3.341790in}}%
\pgfusepath{stroke}%
\end{pgfscope}%
\begin{pgfscope}%
\pgfsetbuttcap%
\pgfsetroundjoin%
\definecolor{currentfill}{rgb}{0.866667,0.517647,0.321569}%
\pgfsetfillcolor{currentfill}%
\pgfsetlinewidth{1.003750pt}%
\definecolor{currentstroke}{rgb}{0.866667,0.517647,0.321569}%
\pgfsetstrokecolor{currentstroke}%
\pgfsetdash{}{0pt}%
\pgfsys@defobject{currentmarker}{\pgfqpoint{-0.020833in}{-0.020833in}}{\pgfqpoint{0.020833in}{0.020833in}}{%
\pgfpathmoveto{\pgfqpoint{0.000000in}{-0.020833in}}%
\pgfpathcurveto{\pgfqpoint{0.005525in}{-0.020833in}}{\pgfqpoint{0.010825in}{-0.018638in}}{\pgfqpoint{0.014731in}{-0.014731in}}%
\pgfpathcurveto{\pgfqpoint{0.018638in}{-0.010825in}}{\pgfqpoint{0.020833in}{-0.005525in}}{\pgfqpoint{0.020833in}{0.000000in}}%
\pgfpathcurveto{\pgfqpoint{0.020833in}{0.005525in}}{\pgfqpoint{0.018638in}{0.010825in}}{\pgfqpoint{0.014731in}{0.014731in}}%
\pgfpathcurveto{\pgfqpoint{0.010825in}{0.018638in}}{\pgfqpoint{0.005525in}{0.020833in}}{\pgfqpoint{0.000000in}{0.020833in}}%
\pgfpathcurveto{\pgfqpoint{-0.005525in}{0.020833in}}{\pgfqpoint{-0.010825in}{0.018638in}}{\pgfqpoint{-0.014731in}{0.014731in}}%
\pgfpathcurveto{\pgfqpoint{-0.018638in}{0.010825in}}{\pgfqpoint{-0.020833in}{0.005525in}}{\pgfqpoint{-0.020833in}{0.000000in}}%
\pgfpathcurveto{\pgfqpoint{-0.020833in}{-0.005525in}}{\pgfqpoint{-0.018638in}{-0.010825in}}{\pgfqpoint{-0.014731in}{-0.014731in}}%
\pgfpathcurveto{\pgfqpoint{-0.010825in}{-0.018638in}}{\pgfqpoint{-0.005525in}{-0.020833in}}{\pgfqpoint{0.000000in}{-0.020833in}}%
\pgfpathclose%
\pgfusepath{stroke,fill}%
}%
\begin{pgfscope}%
\pgfsys@transformshift{2.696797in}{3.341790in}%
\pgfsys@useobject{currentmarker}{}%
\end{pgfscope}%
\end{pgfscope}%
\begin{pgfscope}%
\definecolor{textcolor}{rgb}{0.150000,0.150000,0.150000}%
\pgfsetstrokecolor{textcolor}%
\pgfsetfillcolor{textcolor}%
\pgftext[x=3.178047in,y=3.248213in,left,base]{\color{textcolor}\sffamily\fontsize{19.250000}{23.100000}\selectfont Read-write}%
\end{pgfscope}%
\begin{pgfscope}%
\pgfsetbuttcap%
\pgfsetroundjoin%
\pgfsetlinewidth{1.505625pt}%
\definecolor{currentstroke}{rgb}{0.333333,0.658824,0.407843}%
\pgfsetstrokecolor{currentstroke}%
\pgfsetdash{{9.600000pt}{2.400000pt}{1.500000pt}{2.400000pt}}{0.000000pt}%
\pgfpathmoveto{\pgfqpoint{2.429436in}{2.952042in}}%
\pgfpathlineto{\pgfqpoint{2.964158in}{2.952042in}}%
\pgfusepath{stroke}%
\end{pgfscope}%
\begin{pgfscope}%
\pgfsetbuttcap%
\pgfsetroundjoin%
\definecolor{currentfill}{rgb}{0.333333,0.658824,0.407843}%
\pgfsetfillcolor{currentfill}%
\pgfsetlinewidth{1.003750pt}%
\definecolor{currentstroke}{rgb}{0.333333,0.658824,0.407843}%
\pgfsetstrokecolor{currentstroke}%
\pgfsetdash{}{0pt}%
\pgfsys@defobject{currentmarker}{\pgfqpoint{-0.020833in}{-0.020833in}}{\pgfqpoint{0.020833in}{0.020833in}}{%
\pgfpathmoveto{\pgfqpoint{0.000000in}{-0.020833in}}%
\pgfpathcurveto{\pgfqpoint{0.005525in}{-0.020833in}}{\pgfqpoint{0.010825in}{-0.018638in}}{\pgfqpoint{0.014731in}{-0.014731in}}%
\pgfpathcurveto{\pgfqpoint{0.018638in}{-0.010825in}}{\pgfqpoint{0.020833in}{-0.005525in}}{\pgfqpoint{0.020833in}{0.000000in}}%
\pgfpathcurveto{\pgfqpoint{0.020833in}{0.005525in}}{\pgfqpoint{0.018638in}{0.010825in}}{\pgfqpoint{0.014731in}{0.014731in}}%
\pgfpathcurveto{\pgfqpoint{0.010825in}{0.018638in}}{\pgfqpoint{0.005525in}{0.020833in}}{\pgfqpoint{0.000000in}{0.020833in}}%
\pgfpathcurveto{\pgfqpoint{-0.005525in}{0.020833in}}{\pgfqpoint{-0.010825in}{0.018638in}}{\pgfqpoint{-0.014731in}{0.014731in}}%
\pgfpathcurveto{\pgfqpoint{-0.018638in}{0.010825in}}{\pgfqpoint{-0.020833in}{0.005525in}}{\pgfqpoint{-0.020833in}{0.000000in}}%
\pgfpathcurveto{\pgfqpoint{-0.020833in}{-0.005525in}}{\pgfqpoint{-0.018638in}{-0.010825in}}{\pgfqpoint{-0.014731in}{-0.014731in}}%
\pgfpathcurveto{\pgfqpoint{-0.010825in}{-0.018638in}}{\pgfqpoint{-0.005525in}{-0.020833in}}{\pgfqpoint{0.000000in}{-0.020833in}}%
\pgfpathclose%
\pgfusepath{stroke,fill}%
}%
\begin{pgfscope}%
\pgfsys@transformshift{2.696797in}{2.952042in}%
\pgfsys@useobject{currentmarker}{}%
\end{pgfscope}%
\end{pgfscope}%
\begin{pgfscope}%
\definecolor{textcolor}{rgb}{0.150000,0.150000,0.150000}%
\pgfsetstrokecolor{textcolor}%
\pgfsetfillcolor{textcolor}%
\pgftext[x=3.178047in,y=2.858465in,left,base]{\color{textcolor}\sffamily\fontsize{19.250000}{23.100000}\selectfont Naïve}%
\end{pgfscope}%
\end{pgfpicture}%
\makeatother%
\endgroup%

  %   }
  % \end{center}
  % \caption{Synthesis time per instruction for our three source-target pairs.
  % % TODO do things other than color-code (i.e. different types of ticks)
  % Green corresponds to \Naive sketches.
  % Orange corresponds to \RW sketches.
  % Blue corresponds to \LCS sketches.
  % A red X indicates that synthesis timed out or ran out of memory after that point.}

\begin{center}
\begin{tabular}{l|c|c|c}

\toprule
  Compiler & \Naive sketch & \RW sketch & \LCS sketch \\
\midrule
  eBPF to RISC-V & X & X & 44.4h \\
  classic BPF to eBPF & X & X & 1.2h \\
  libseccomp to eBPF & 4.0h & 43.5m & 7.1m  \\
\bottomrule
\end{tabular}
\end{center}

\caption{
Synthesis time for each source-target pair, broken
down by set of optimizations used in the sketch.
An X indicates that synthesis either timed out
or ran out of memory.}


  \label{fig:o2b-l2b-synthtime}
\end{figure}

In order to evaluate the effectiveness of the search optimizations
described in \autoref{s:algorithm}, we measured the time \jitsynth
takes to synthesize each of the three compilers with different optimizations
enabled.
%
Specifically, we run \jitsynth in three different configurations:
(1) using \Naive sketches, (2) using \RW sketches, and (3) using \LCS sketches.
%
For each configuration, we ran \jitsynth with a timeout of 48 hours (or until out of memory).
\autoref{fig:o2b-l2b-synthtime} shows the time to synthesize
each compiler under each configuration.
%
Note that these figures do not include time spent computing read and write sets,
which takes less than 11 minutes for all cases.
%
% The order instructions are synthesized in is fixed across configurations.
% %
Our results were collected using an 8-core AMD Ryzen 7 1700 CPU with 16~GB memory,
running Racket v7.4 and the Boolector~\cite{niemetz:boolector} solver v3.0.1-pre.


When synthesizing the eBPF-to-RISC-V compiler,
\jitsynth runs out of memory with \Naive sketches,
reaches the timeout with \RW sketches,
and completes synthesis with \LCS sketches.
%
For the classic-BPF-to-eBPF compiler,
\jitsynth times out with both \Naive sketches and \RW sketches.
\jitsynth only finishes synthesis with \LCS sketches.
%
For the libseccomp-to-eBPF compiler, all configurations finish,
but \jitsynth finishes synthesis about $\LibseccompSynthSpeedup\times$ times faster
with \LCS sketches than with \Naive sketches.
%
These results demonstrate that the techniques \jitsynth uses
are essential to the scalability of JIT synthesis.
