\section{\jitsynth in a nutshell}% TODO bad name
\label{jitsynth:s:overview}

\begin{figure}[h]
  \centering
  \resizebox{\linewidth}{!}{
    \begin{tabular}{llll}

\toprule
\multicolumn{2}{l}{instruction} & description & semantics
\\

\midrule

\multicolumn{2}{l}{eBPF (subset):}

\\
& \cc{addi32}\ $\dst, \mathit{imm32}$
& 32-bit add (high 32 bits cleared)
& $\reg{\dst} \leftarrow 0^{32} \concat (\extract{31}{0}{\reg{\dst}} + \mathit{imm32})$

\\

\midrule

\multicolumn{2}{l}{RISC-V (subset):}
\\
& \cc{lui}\ $\rd, \mathit{imm20}$
& load upper immediate
& $\reg{\rd} \leftarrow \sextll(\mathit{imm20} \concat 0^{12})$

\\

& \cc{addiw}\ $\rd, \rs, \mathit{imm12}$
& 32-bit register-immediate add
& $\reg{\rd} \leftarrow \sextll(\extract{31}{0}{\reg{\rs}} + \sexti(\mathit{imm12}))$

\\

& \cc{add}\ $\rd, \mathit{rs1}, \mathit{rs2}$
& register-register add
& $\reg{\rd} \leftarrow \reg{\mathit{rs1}} + \reg{\mathit{rs2}}$

\\

& \cc{slli}\ $\rd, \rs, \mathit{imm6}$
& register-immediate left shift
& $\reg{\rd} \leftarrow \rs\ \texttt{<<}\ (0^{58} \concat \mathit{imm6})$

\\

& \cc{srli}\ $\rd, \rs, \mathit{imm6}$
& register-immediate logical right shift
& $\reg{\rd} \leftarrow \rs\ \texttt{>>}\ (0^{58} \concat \mathit{imm6})$

\\

& \cc{lb}\ $\rd, \rs, \mathit{imm12}$
& load byte from memory
& $\reg{\rd} \leftarrow \sextll(M[\reg{\rs} + \sextll(\mathit{imm12})])$

\\

& \cc{sb}\ $\mathit{rs1}, \mathit{rs2}, \mathit{imm12}$
& store byte to memory
& $M[\reg{\mathit{rs1}} + \sextll(\mathit{imm12})] \leftarrow \extract{7}{0}{\reg{\mathit{rs2}}} $

\\

\bottomrule

\end{tabular}


  }
  \vspace{-.5em}
  \caption{Subsets of eBPF and RISC-V
used as source and target languages, respectively,
in our running example:
$R[r]$ denotes the value of register~$r$;
$M[a]$ denotes the value at memory address~$a$;
$\concat$ denotes concatenation of bitvectors;
superscripts (e.g., $0^{32}$) denote repetition of bits;
$\sexti(x)$ and $\sextll(x)$ sign-extend $x$ to 32 and 64 bits, respectively;
and $\extractop(i, j, x)$ produces a subrange of bits of $x$ from index $i$ down to $j$.\looseness=-1}
%$r$ denotes a BPF register;
%$\rd$, $\mathit{rs1}$, and $\mathit{rs2}$ denote RISC-V registers;
\label{fig:lang-instrs}
\end{figure}

This section provides an overview of \jitsynth by illustrating how it
synthesizes a toy JIT compiler (\autoref{fig:lang-instrs}). The source language
of the JIT is a tiny subset of eBPF~\cite{fleming:ebpf} consisting of one
instruction, and the target language is a subset of 64-bit
RISC-V~\cite{riscv:isa} consisting of seven instructions. Despite the simplicity
of our languages, the Linux kernel JIT used to produce incorrect code for this
eBPF instruction~\cite{nelson:bpf-riscv-add32-bug}; such miscompilation bugs not
only lead to correctness issues, but also enable adversaries to compromise the
OS kernel by crafting malicious eBPF programs~\cite{wang:jitk}. This section
shows how \jitsynth can be used to synthesize a JIT that is verified with
respect to the semantics of the source and target languages.\tighten

\paragraph{In-kernel languages.} 

\jitsynth expects the source and target languages to be a set of instructions
for manipulating the state of an \emph{abstract register machine}
(\autoref{s:problem}). This state consists of a program counter (\pc), a finite
sequence of general-purpose registers (\regs), and a finite sequence of memory
locations (\mem), all of which store bitvectors (i.e., finite precision
integers). The length of these bitvectors is defined by the language; for
example, both eBPF and RISC-V store 64-bit values in their registers. An
instruction consists of an \emph{opcode} and a finite set of \emph{fields},
which are bitvectors representing either register identifiers or immediate
(constant) values. For instance, the \cc{addi32} instruction in eBPF
has two fields: $\dst$ is a 4-bit value representing the index of the
output register, and $\mathid{imm32}$ is a 32-bit immediate.
(eBPF instructions may have two additional fields $\src$ and $\off$,
which are not shown here as they are not used by \cc{addi32}.)
An abstract
register machine for a language gives meaning to its instructions: the machine
consumes an instruction and a state, and produces a state that is the result of
executing that instruction. \autoref{fig:lang-instrs} shows a high-level
description of the abstract register machines for our languages.\tighten

\paragraph{\jitsynth interface.}

To synthesize a compiler from one language to another, \jitsynth takes as input
their syntax, semantics, and a mapping from source to target states. All three
inputs are given as a program in a \emph{solver-aided host
language}~\cite{torlak:rosette}. \jitsynth uses Rosette as its host, but the
host can be any language with a symbolic
evaluation engine that can reduce the semantics of host programs to SMT
constraints (e.g.,~\cite{solar-lezama:sketch}).
\autoref{fig:toy-inputs} shows the interpreters for the source and
target languages (i.e., emulators for their abstract register machines), as well
as the state-mapping functions \cc{regST}, \cc{pcST}, and \cc{memST} that
\jitsynth uses to determine whether a source state $\sigma_S$ is equivalent to a
target state $\sigma_T$. In particular, \jitsynth deems these states equivalent,
denoted by $\sigma_S \cong \sigma_T$, whenever 
%
$\regs(\sigma_T)[\cc{regST}(r)] = \regs(\sigma_S)[r]$, $\pc(\sigma_T) =
\cc{pcST}(\pc(\sigma_S))$, and $\mem(\sigma_T)[\cc{memST}(a)] =
\mem(\sigma_S)[a]$ 
%
for all registers $r$ and memory addresses $a$.\tighten

\begin{figure}[h]
\centering
\resizebox{\linewidth}{!}{%
\begin{minipage}{1.03\linewidth}
\lstinputlisting[language=rosette,xleftmargin=-.5em,firstline=1]{code/toy-inputs.rkt}%multicols=2,numbers=left,numbersep=.75em,
\end{minipage}}
\vspace{-.5em}
  \caption{Snippets of inputs to \jitsynth: the interpreters for the source (eBPF) and and target (RISC-V) languages and state-mapping functions.}
  \label{fig:toy-inputs}
  \end{figure}

\paragraph{Decomposition into per-instruction compilers.} 

Given these inputs, \jitsynth generates a \emph{per-instruction compiler} from
the source to the target language. To ensure that the resulting compiler is
correct (\autoref{thm:end-to-end-soundness}), and that one will be found if it
exists (\autoref{thm:synthesis-soundness-and-completeness}), \jitsynth puts two
restrictions on its inputs. First, the inputs must be
self-finitizing~\cite{torlak:rosette}, meaning that both the interpreters and
the mapping functions must have a finite symbolic execution tree when applied to
symbolic inputs.
%such as an instruction with symbolic fields or a state with symbolic
%contents. 
Second, the target machine must have at least as many registers and memory
locations as the source machine; these storage cells must be as wide as those of
the source machine; and the state-mapping functions (\cc{pcST}, \cc{regST}, and
\cc{memST}) must be injective. Our toy inputs satisfy these restrictions, as do
the real in-kernel languages evaluated in \autoref{s:eval}.

\paragraph{Synthesis workflow.} 

\jitsynth generates a per-instruction compiler for a given source and target
pair in two stages. The first stage uses an optimized \emph{compiler metasketch}
to synthesize a \minicompiler from every instruction in the source language to a
sequence of instructions in the target language (\autoref{s:algorithm}). The
second stage then simply stitches these mini compilers into a full C compiler
using a trusted outer loop and a switch statement. The first stage is a core
technical contribution of this paper, and we illustrate it next on our toy
example.\tighten


\paragraph{Metasketches.} 

To understand how \jitsynth works, consider the basic problem of determining if
every \cc{addi32} instruction can be emulated by a sequence of $k$ instructions
in toy RISC-V\@. In particular, we are interested in finding a program
$C_\cc{addi32}$ in our host language (which \jitsynth translates to C) that
takes as input a source instruction $s = \cc{addi32}\ \dst, \mathid{imm32}$ and
outputs a semantically equivalent RISC-V program $t = [t_1,\ldots,t_k]$. That
is, for all $\dst, \mathid{imm32}$, and for all equivalent states $\sigma_S \cong
\sigma_T$, we have $\mathid{run}(s, \sigma_S,
\cc{ebpf-interpret})\cong\mathid{run}(t, \sigma_T, \cc{rv-interpret})$, where
$\mathid{run}(e, \sigma, f)$ executes the instruction interpreter $f$ on the
sequence of instructions $e$, starting from the state $\sigma$
(\autoref{def:arm}).\tighten

We can solve this problem by asking the host synthesizer to search for
$C_\cc{addi32}$ in a space of candidate \minicompilers of length $k$. We
describe this space with a syntactic template, or a \emph{sketch}, as shown in \autoref{fig:example-sketch}.

In this sketch, \cc{(??insn dst imm)} stands for a missing expression---a hole---that the
synthesizer needs to fill with an instruction from the toy RISC-V language. To
fill an instruction hole, the synthesizer must find an expression that computes
the value of the target instruction's fields. \jitsynth limits this expression
language to bitvector expressions (of any depth) over the fields of the source
instruction and arbitrary bitvector constants.\tighten

Given this sketch, and our correctness specification for $C_\cc{addi32}$, the
synthesizer will search the space defined by the sketch for a program that
satisfies the specification.
%
Below is an example of the resulting toy compiler from eBPF to
RISC-V, synthesized and translated to C by \jitsynth (without the
outer loop):\tighten

\begin{minipage}{\linewidth}
\lstinputlisting[language=C,xleftmargin=2em,firstline=1]{code/toy-compiler.c}%multicols=2,numbers=left,numbersep=.75em,
\end{minipage}

\begin{figure}[h]
%\begin{minipage}{\linewidth}
\begin{lstlisting}[language=rosette,xleftmargin=0em,mathescape=true]
(define (compile-addi32 s)       ; Returns a list of k instruction holes, to be 
  (define dst (ebpf-insn-dst s)) ; filled with toy RISC-V instructions. Each    
  (define imm (ebpf-insn-imm s)) ; hole represents a set of choices, defined 
  (list (??insn dst imm) ...))   ; by the ??insn procedure. 

(define (??insn . sf)            ; Takes as input source instruction fields and
  (define rd  (??reg sf))        ; uses them to construct target field holes. 
  (define rs1 (??reg sf))        ; ??reg and ??imm field holes are bitvector 
  (define rs2 (??reg sf))        ; expressions over sf and arbitrary constants.
  (choose*                       ; Returns an expression that chooses among  
    (rv-insn lui rd rs1 rs2 (??imm 20 sf))   ; lui, addiw,
    ...                                      ; ..., and
    (rv-insn sb  rd rs1 rs2 (??imm 12 sf)))) ; sb instructions.
\end{lstlisting}
%\end{minipage}
\caption{A fixed-length sketch describing the space of candidate \minicompilers for
         the eBPF \cc{addi32} instruction. \cc{(??insn dst imm)} describes a missing
         expression that the synthesizer must fill with a RISC-V instruction.}
\label{fig:example-sketch}
\end{figure}

Once we know how to synthesize a compiler of length $k$, we can easily extend
this solution into a naive method for synthesizing a compiler of any length. We
simply enumerate sketches of increasing lengths, $k = 1, 2, 3, \ldots$, invoke
the synthesizer on each generated sketch, and stop as soon as a solution is
found (if ever). The resulting ordered set of sketches forms a
metasketch~\cite{bornholt:synapse}---i.e., a search space and a strategy for
exploring it---that contains all candidate mini compilers (in a subset of the
host language) from the source to the target language. This naive metasketch can
be used to find a mini compiler for our toy example in {493 minutes}. However,
it fails to scale to real in-kernel DSLs (\autoref{s:eval}), motivating the need
for \jitsynth's optimized compiler metasketches.\tighten

\paragraph{Compiler metasketches.} \jitsynth optimizes the naive metasketch by
extending it with two kinds of more tightly constrained sketches, which are
explored first. A constrained sketch of size $k$ usually contains a correct
solution of a given size if one exists, but if not, \jitsynth will eventually
explore the naive sketch of the same length, to maintain completeness. We give
the intuition behind the two optimizations here, and present them in detail in
\autoref{s:algorithm}. 

First, we observe that practical source and target languages include similar
kinds of instructions. For example, both eBPF and RISC-V include instructions
for adding immediate values to registers. This similarity often makes it
possible to emulate a source instruction with a sequence of target instructions
that access the same part of the state (the program counter, registers, or
memory) as the source instruction. For example, \cc{addi32} reads and writes
only registers, not memory, and it can be emulated with RISC-V instructions that
also access only registers. To exploit this observation, we introduce
\emph{read-write sets}, which summarize, soundly and precisely, how an
instruction accesses state. \jitsynth uses these sets to define \emph{read-write
sketches} for a given source instruction, including only target instructions
that access the state in the same way as the source instruction. For instance, a
read-write sketch for \cc{addi32} excludes both \cc{lb} and \cc{sb} instructions
because they read and write memory as well as registers.\tighten

Second, we observe that hand-written JITs use pseudoinstructions to simplify
their implementation of \minicompilers. These are simply subroutines or macros
for generating target sequences that implement common functionality. For
example, the Linux JIT from eBPF to RISC-V includes a pseudoinstruction for
loading 32-bit immediates into registers. \jitsynth mimics the way hand-written
JITs use pseudoinstructions with the help of \emph{pre-load sketches}. These
sketches first use a synthesized pseudoinstruction to create a sequence of
concrete target instructions that load source immediates into scratch registers;
then, they include a compute sequence comprised of read-write instruction holes.
Applying these optimizations to our toy example, \jitsynth finds a \minicompiler
for \cc{addi32} in {5 seconds}---a {roughly $6000\times$} speedup over the naive
metasketch.\tighten
