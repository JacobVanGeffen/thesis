\section{Problem Statement}\label{s:problem}

This section formalizes the compiler synthesis problem for in-kernel DSLs. We
focus on JIT compilers, which, for our purposes, means one-pass
compilers~\cite{engler:vcode}. To start, we define \emph{abstract register
machines} as a way to specify the syntax and semantics of in-kernel languages.
Next, we formulate our compiler synthesis problem as one of synthesizing a set
of sound \emph{\minicompilers} from a single source instruction to a sequence of
target instructions. Finally, we show that these \minicompilers compose into a
sound JIT compiler, which translates every source program into a semantically
equivalent target program.\tighten 


\paragraph{Abstract register machines.}

An abstract register machine (ARM) provides a simple interface for specifying
the syntax and semantics of an in-kernel language. The syntax is given as a set
of {abstract instructions}, and the semantics is given as a {transition
function} over instructions and machine {states}. 

An \emph{abstract instruction} (\autoref{def:instruction}) defines the name
(\mathid{op}) and type signature ($\mathcal{F}$) of an operation in the
underlying language. For example, the abstract instruction $(\mathid{addi32}, r
\mapsto \Reg, imm32 \mapsto \BV{32})$ specifies the name and signature of the
\texttt{addi32} operation from the eBPF language
(\autoref{fig:lang-instrs}). Each abstract instruction represents the (finite)
set of all \emph{concrete instructions} that instantiate the abstract
instruction's parameters with values of the right type. For example,
$\cc{addi32}\, 0, 5$ is a concrete instantiation of the abstract instruction for
\cc{addi32}. In the rest of this chapter, we will write ``instruction'' to mean a
concrete instruction.\tighten

\begin{definition}[Abstract and Concrete Instructions]\label{def:instruction}
  An \textup{abstract instruction} $\iota$ is a pair $(\mathid{op}, \mathcal{F})$ where
  \mathid{op} is an opcode and $\mathcal{F}$ is a mapping from \emph{fields}
  to their \emph{types}. Field types include \Reg, denoting register names, and
  \BV{k}, denoting $k$-bit bitvector values. The abstract instruction $\iota$
  represents all \textup{concrete instructions} $p = (\mathid{op}, F)$ with the
  opcode \mathid{op} that bind each field $f\in \mathid{dom}(\mathcal{F})$ to a
  value $F(f)$ of type $\mathcal{F}(f)$. We write $\prog{\iota}$ to denote the
  set of all concrete instructions for $\iota$, and we extend this notation to
  sets of abstract instructions in the usual way, i.e.,
  $\prog{\mathcal{I}}=\bigcup_{\iota\in\mathcal{I}}\prog{\iota}$ for the set
  $\mathcal{I}$.\tighten
\end{definition}

Instructions operate on machine \emph{states} (\autoref{def:state}), and their
semantics are given by the machine's \emph{transition function}
(\autoref{def:arm}). A machine state  consists of a program counter, a map from
register names to register values, and a map from memory addresses to memory
values. Each state component is either a bitvector or a map over bitvectors,
making the set of all states of an ARM finite. The transition function of an ARM
defines an interpreter for the ARM's language by specifying how to compute the
output state for a given instruction and input state. We can apply this
interpreter, together with the ARM's \emph{fuel function}, to define an
\emph{execution} of the machine on a program and an initial state. The fuel
function takes as input a sequence of instructions and returns a natural number
that bounds the number of steps (i.e., state transitions) the machine can make
to execute the given sequence. The inclusion of fuel models the requirement of
in-kernel languages for all program executions to terminate~\cite{wang:jitk}. It
also enables us to use symbolic execution to soundly reduce the semantics of
these languages to SMT constraints, in order to formulate the synthesis queries
in \autoref{s:algorithm:solving}.\tighten 


\begin{definition}[State]\label{def:state}
 A \textup{state} $\sigma$ is a tuple $(\pc, \regs, \mem)$ where $\pc$ is a
 value, \regs is a function from register names to values, and \mem is a
 function from memory addresses to values. Register names, memory addresses, and
 all values are finite-precision integers, or bitvectors. We write
 $\size{\sigma}$ to denote the \emph{size} of the state $\sigma$. The size
 $\size{\sigma}$ is defined to be the tuple $(r, m, k_\pc, k_\regs, k_\mem)$,
 where $r$ is the number of registers in $\sigma$, $m$ is the number of memory
 addresses, and $k_\pc$, $k_\regs$, and $k_\mem$ are the width of the bitvector
 values stored in the \pc, \regs, and \mem, respectively. Two states have the
 same size if $\size{\sigma_i} = \size{\sigma_j}$; one state is smaller than
 another, $\size{\sigma_i} \leq \size{\sigma_j}$, if each element of $\size{\sigma_i}$ 
 is less than or equal to the corresponding element of $\size{\sigma_j}$.\tighten
 \end{definition}


\begin{definition}[Abstract Register Machines and Executions]\label{def:arm}
  An \textup{abstract register machine} $\mathcal{A}$ is a tuple $(\mathcal{I},
  \Sigma, \mathcal{T}, \Phi)$  where $\mathcal{I}$ is a set of abstract
  instructions, $\Sigma$ is a set of states of the same size, $\mathcal{T} :
  \prog{\mathcal{I}} \rightarrow \Sigma \rightarrow \Sigma$ is a
  \textup{transition function} from instructions and states to states, and
  $\Phi: \mathid{List}(\prog{\mathcal{I}}) \rightarrow \mathbb{N}$ is a
  \emph{fuel function} from sequences of instructions to natural numbers. Given
  a state $\sigma_0\in\Sigma$ and a sequence of instructions $\vec{p}$ drawn
  from $\prog{\mathcal{I}}$, we define the \textup{execution} of $\mathcal{A}$
  on $\vec{p}$ and $\sigma_0$ to be the result of applying $\mathcal{T}$ to
  $\vec{p}$ at most $\Phi(\vec{p})$ times. That is, $\arm(\vec{p}, \sigma_0) =
  \mathid{run}(\vec{p}, \sigma_0, \mathcal{T}, \Phi(\vec{p}))$, where  
  \[
    \mathid{run}(\vec{p}, \sigma, \mathcal{T}, k) = 
    \begin{cases}
        \sigma,& \text{if } k = 0 \text{ or } \pc(\sigma) \not\in [0, \size{\vec{p}})\\
        \mathid{run}(\vec{p}, \mathcal{T}(\vec{p}[\pc(\sigma)], \sigma), \mathcal{T}, k-1),   & \text{otherwise.}
    \end{cases}
  \]
\end{definition}

\paragraph{Synthesizing JIT compilers for ARMs.} Given a source and target ARM,
our goal is to synthesize a one-pass JIT compiler that translates source
programs to semantically equivalent target programs. To make synthesis
tractable, we fix the structure of the JIT to consist of an outer loop and a
switch statement that dispatches compilation tasks to a set of
\emph{\minicompilers} (\autoref{def:mini-compiler}). Our synthesis
problem is therefore to find a sound \minicompiler for each abstract
instruction in the source machine (\autoref{def:synthesis-problem}).\tighten

\begin{definition}[\MiniCompiler]\label{def:mini-compiler}
Let $\arm_S = (\mathcal{I}_S, \Sigma_S, \mathcal{T}_S, \Phi_S)$ and
$\arm_T = (\mathcal{I}_T, \Sigma_T, \\ \mathcal{T}_T, \Phi_T)$ be two abstract register
machines, $\cong$ an equivalence relation on their states $\Sigma_S$ and
$\Sigma_T$, and $C: \prog{\iota} \rightarrow
\mathid{List}(\prog{\mathcal{I}_T})$ a function for some
$\iota\in\mathcal{I}_S$. We say that $C$ is a \textup{sound \minicompiler} for
$\iota$ with respect to $\cong$ iff
\[
    \forall \sigma_S\in\Sigma_S,\ \sigma_T\in\Sigma_T,\ p\in\prog{\iota}. \ 
            \sigma_S \cong \sigma_T \Rightarrow
                 \arm_S(p, \sigma_S) \cong \arm_T(C(p), \sigma_T)
\]
\end{definition} 

\begin{definition}[\MiniCompiler Synthesis]\label{def:synthesis-problem}  
Given two abstract register machines $\arm_S = (\mathcal{I}_S, \Sigma_S,
\mathcal{T}_S, \Phi_S)$ and $\arm_T=(\mathcal{I}_T, \Sigma_T,  \mathcal{T}_T, \Phi_T)$, as well
as an equivalence relation $\cong$ on their states, the \textup{\minicompiler
synthesis problem} is to generate a sound \minicompiler $C_\iota$ for each
$\iota\in\mathcal{I}_S$ with respect to $\cong$.\tighten
\end{definition} 

The general version of our synthesis problem, defined above, uses an arbitrary
equivalence relation $\cong$ between the states of the source and target
machines to determine if a source and target program are semantically
equivalent. \jitsynth can, in principle, solve this problem with the naive
metasketch described in \autoref{jitsynth:s:overview}. In practice, however, the naive
metasketch scales poorly, even on small languages such as toy eBPF and RISC-V\@.
So, in this work, we focus on source and target ARMs that satisfy an additional
assumption on their state equivalence relation: it can be expressed in terms of
injective mappings from source to target states (\autoref{def:state-equiv}).
%When we write $\cong$ from now on, we mean this restricted form of state equivalence. 
This restriction enables \jitsynth to employ optimizations (such as pre-load
sketches described in \autoref{s:algorithm:pld}) that are crucial to scaling
synthesis to real in-kernel languages.\tighten


\begin{definition}[Injective State Equivalence Relation]\label{def:state-equiv}
Let $\arm_S$ and $\arm_T$ be abstract register machines with states $\Sigma_S$
and $\Sigma_T$ such that $\size{\sigma_S}\leq\size{\sigma_T}$ for all
$\sigma_S\in\Sigma_S$ and $\sigma_T\in\Sigma_T$. Let $\mathcal{M}$ be a
\emph{state mapping} $(\mathcal{M}_\pc, \mathcal{M}_\regs, \mathcal{M}_\mem)$
from $\Sigma_S$ and $\Sigma_T$,  where $\mathcal{M}_\pc$ multiplies the program
counter of the states in $\Sigma_S$ by a constant factor, $\mathcal{M}_\regs$ is
an injective map from register names in $\Sigma_S$ to those in $\Sigma_T$, and
$\mathcal{M}_\mem$ is an injective map from memory addresses in $\Sigma_S$ to
those in $\Sigma_T$. We say that two states  $\sigma_S\in\Sigma_S$ and
$\sigma_T\in\Sigma_T$ are equivalent according to $\mathcal{M}$, written
$\sigma_S \cong_\mathcal{M} \sigma_T$, iff $\mathcal{M}_\pc(\pc(\sigma_S)) =
\pc(\sigma_T)$, $\regs(\sigma_S)[r] = \regs(\sigma_T)[\mathcal{M_\regs}(r)]$ for
all register names $r\in\emph{dom}(\regs(\sigma_S))$, and $\mem(\sigma_S)[a] =
\mem(\sigma_T)[\mathcal{M_\mem}(a)]$ for all memory addresses
$a\in\emph{dom}(\mem(\sigma_S))$. The binary relation $\cong_\mathcal{M}$ is
called an \textup{injective state equivalence relation} on $\arm_S$ and
$\arm_T$.
\end{definition}

\paragraph{Soundness of JIT compilers for ARMs.} Finally, we note that a JIT
compiler composed from the synthesized \minicompilers correctly
translates every source program to an equivalent target program.
We formulate and prove this theorem using the Lean theorem prover~\cite{moura:lean}.

\begin{theorem}[Soundness of JIT compilers]\label{thm:end-to-end-soundness}
Let $\arm_S = (\mathcal{I}_S, \Sigma_S, \mathcal{T}_S, \Phi_S)$ and
$\arm_T=(\mathcal{I}_T, \Sigma_T,  \mathcal{T}_T, \Phi_T)$ be abstract register
machines, $\cong_\mathcal{M}$ an injective state equivalence relation on their
states such that $M_\pc(\pc(\sigma_S)) = N_\pc\pc(\sigma_S)$,
and $\{C_1,\ldots,C_{|\mathcal{I}_S|}\}$ a solution to the \minicompiler
synthesis problem for $\arm_S$, $\arm_T$, and $\cong_\mathcal{M}$
where $\forall s\in\prog{\iota}.\ |C_i(s)| = N_\pc$. Let
$\mathcal{C} : \prog{\mathcal{I}_S} \rightarrow
\mathid{List}(\prog{\mathcal{I}_T})$ be a function that maps concrete
instructions $s\in\prog{\iota}$ to the compiler output $C_\iota(s)$ for
$\iota\in\mathcal{I}_S$. If $\vec{s} = s_1, \ldots, s_n$ is a sequence of
concrete instructions drawn from $\mathcal{I}_S$, and $\vec{t} =
\mathcal{C}(s_1)\cdot\ldots\cdot\mathcal{C}(s_n)$ where $\cdot$ stands for
sequence concatenation, then $\forall \sigma_S\in\Sigma_S, \sigma_T\in\Sigma_T.\
\sigma_S \cong_\mathcal{M} \sigma_T \Rightarrow \arm_S(\vec{s}, \sigma_S)
\cong_\mathcal{M} \arm_T(\vec{t}, \sigma_T)$.\tighten
\end{theorem}
