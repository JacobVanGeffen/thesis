Modern operating systems (OSes) can be customized with user-specified programs
that implement functionality like system call whitelisting, performance
profiling, and power management~\cite{engler:vcode,fleming:ebpf,mccanne:bpf}.
For portability and safety, these programs are written in restricted
domain-specific languages (DSLs), and the kernel executes them via
interpretation and, for better performance, just-in-time (JIT) compilation. The
correctness of in-kernel interpreters and JITs is crucial for the reliability
and security of the kernel, and bugs in their implementations have led to
numerous critical issues and patches~\cite{gpz:1454,paul:cve-2020-8835}.
More broadly, embedded DSLs are
also used to customize---and compromise~\cite{blazakis:jit-spraying,kocher:spectre}---other low-level software, such as font
rendering and anti-virus engines~\cite{chen:vmsec}. Providing formal guarantees
of correctness for in-kernel DSLs is thus a pressing practical and research
problem with applications to a wide range of systems software.\tighten

Prior work has tackled this problem through interactive theorem proving. For
example, the Jitk framework~\cite{wang:jitk} uses the Coq interactive theorem
prover~\cite{coq} to implement and verify the correctness of a JIT compiler for
the classic Berkeley Packet Filter (BPF) language~\cite{mccanne:bpf} in the
Linux kernel. But such an approach presents two key challenges. First, Jitk
imposes a significant burden on DSL developers, requiring them to implement both
the interpreter and the JIT compiler in Coq, and then manually prove the
correctness of the JIT compiler with respect to the interpreter. Second, the
resulting JIT implementation is extracted from Coq into OCaml and cannot be run
in the kernel; rather, it must be run in user space, sacrificing performance and
enlarging the trusted computing base (TCB) by relying on the OCaml runtime as
part of the TCB\@.\tighten

This paper addresses these challenges with \jitsynth, the first tool for
synthesizing verified JIT compilers for in-kernel DSLs. \jitsynth takes as input
interpreters for the source DSL and the target instruction set architecture
(ISA), and it synthesizes a JIT compiler that is guaranteed to transform each
source program into a semantically equivalent target program. Using \jitsynth,
DSL developers write no proofs or compilers.
Instead, they write the semantics of the source and target
languages in the form of interpreters and a mapping from source to target states,
which \jitsynth trusts to be correct. The
synthesized JIT compiler is implemented in C; thus, it can run directly in the
kernel.\tighten

At first glance, synthesizing a JIT compiler seems intractable. Even the simplest
compiler contains thousands of instructions, whereas existing synthesis
techniques scale to tens of instructions. To tackle this problem in our setting,
we observe that in-kernel DSLs are similar to ISAs: both take the form of
bytecode instructions for an \emph{abstract register machine}, a simple virtual
machine with a program counter, a few registers, and limited memory
store~\cite{wang:jitk}. We also observe that in practice, the target machine has
at least as many resources (registers and memory) as the source machine;
and that JIT compilers for such abstract register machines
perform register allocation statically at compile time.
%
Our main insight is that we can exploit these properties to make synthesis
tractable through \emph{decomposition} and \emph{prioritization}, while
preserving soundness and completeness.\tighten

\jitsynth works by decomposing the JIT synthesis problem into the problem of
synthesizing individual \emph{\minicompilers} for every instruction in the
source language. Each \minicompiler is synthesized by generating a
\emph{compiler metasketch}~\cite{bornholt:synapse}, a set of ordered sketches
that collectively represent \emph{all} instruction sequences in the target
ISA\@. These sketches are then solved by an off-the-shelf synthesis tool based
on reduction to SMT~\cite{torlak:rosette}. The synthesis tool ensures that the
target instruction sequence is semantically equivalent to the source
instruction, according to the input interpreters.  The order in which the
sketches are explored is key to making this search practical, and \jitsynth
contributes two techniques for biasing the search towards tightly constrained,
and therefore tractable, sketches that are likely to contain a correct program. 

First, we observe that source instructions can often be implemented with target
instructions that access the same parts of the state (e.g., only registers).
Based on this observation, we develop \emph{read-write sketches}, which restrict
the synthesis search space to a subset of the target instructions, based on a
sound and precise summary of their semantics. Second, we observe that
hand-written JITs rely on pseudoinstructions to generate common target
sequences, such as loading immediate (constant) values into registers. We use
this observation to develop \emph{pre-load sketches}, which employ synthesized
pseudoinstructions to eliminate the need to repeatedly search for common target
instruction subsequences.   

We have implemented \jitsynth in Rosette~\cite{torlak:rosette} and used it to
synthesize JIT compilers for three widely used in-kernel DSLs. As our main case
study, we used \jitsynth to synthesize a RISC-V~\cite{riscv:isa} compiler for
extended BPF~(eBPF)~\cite{fleming:ebpf}, an extension of classic
BPF~\cite{mccanne:bpf}, used by the Linux kernel. Concurrently with our work,
Linux developers manually built a JIT compiler for the same source and target
pair, and a team of researchers found nine correctness bugs in that compiler
shortly after its release~\cite{nelson:serval}. In contrast, our JIT compiler is
verified by construction; it supports 87 out of 102 eBPF instructions and passes
all the Linux kernel tests within this subset, including the regression tests
for these nine bugs. Our synthesized compiler generates code that is
$\EbpfInterpSpeedup\times$ faster than interpreted code and
$\EbpfCompilerSlowdown\times$ times slower than the code generated by the Linux
JIT\@. We also used  \jitsynth to synthesize a JIT from
libseccomp~\cite{edge:libseccomp}, a policy language for system call
whitelisting, to eBPF, and a JIT from classic BPF to eBPF\@. The synthesized
JITs avoid previously found bugs in the existing generators for these source
target pairs, while incurring, on average, a
$\CbpfSlowdown$--$\LibseccompSlowdown\times$ slowdown in the performance of the
generated code.\tighten


To summarize, this paper makes the following contributions:
\begin{enumerate}
    \item \jitsynth, the first tool for synthesizing verified JIT compilers for
    in-kernel DSLs, given the semantics of the source and target languages as
    interpreters.\tighten
    \item A novel formulation of the JIT synthesis problem as one of
    synthesizing a per-instruction compiler for \emph{abstract register
    machines}.\tighten
    \item A novel \emph{compiler metasketch} that enables \jitsynth to solve the JIT
    synthesis problem with an off-the-shelf synthesis engine.
    \item An evaluation of \jitsynth's effectiveness, showing that it can
    synthesize verified JIT compilers for three widely used in-kernel DSLs.
\end{enumerate}

The rest of this paper is organized as follows.
%
\autoref{s:overview} illustrates \jitsynth on a small example.
%
\autoref{s:problem} formalizes the JIT synthesis problem for in-kernel DSLs.
%
\autoref{s:algorithm} presents the \jitsynth algorithm for generating and
solving compiler metasketches.
%
\autoref{s:impl} provides implementation details.
%
\autoref{s:eval} evaluates \jitsynth.
%
\autoref{s:related} discusses related work.
\autoref{s:conclusion} concludes.\tighten
