\section{Implementation}\label{s:impl}
We implemented \jitsynth as described in \autoref{s:overview} using
Rosette~\cite{torlak:rosette} as our host language.
%Without bounding the length of the compiler, the space explored by \jitsynth is infinite, so we use a timeout
%of 48 hours to bound execution.
Since the search spaces for different compiler lengths are disjoint,
the \jitsynth implementation searches these spaces in parallel~\cite{bornholt:synapse}.
%In addition, \jitsynth implements several optimizations
%in order to synthesize real-world JIT compilers, as follows.
%
\if 0
\paragraph{Synthesizing Common Sequences.}
In addition to separately synthesizing pre-loads for parameters,
\jitsynth also synthesizes a few other small sequences used commonly in compiled programs.

One such class of sequences are register extensions.
The target abstract register machine may have larger register values than the source machine.
In these cases, the compiler may often need to either sign-extend the target register values,
filling the upper bits of the register with the sign bit,
or zero-extend, filling the upper bits of the register with 0.
\jitsynth synthesizes a sequence for each of these extension types.
When $L_\regs\in\Read\iota$ of source abstract instruction $\iota$,
\jitsynth will try sketches that extend used registers in both ways.

Additionally, \jitsynth synthesizes sequences
that both load the PC into a temporary register
and write to the PC from a register value.
These operations are needed for compiling jumps,
and are reused in many \minicompilers.

\paragraph{PC Equivalence.}
To determine the equivalence function between the source and target PC,
\jitsynth considers both the user-input PC multiple $m_{\PC}$,
as well as the length of each \minicompiler $L$.
Specifically, we say that source $\PC_S$ and target $\PC_T$
are equivalent when $\PC_S \times L \times m_{\PC} = \PC_T$.
To avoid arithmetic overflow,
\jitsynth bounds the maximum value of the $\PC_S$ to
$\lfloor\frac{2^B}{L \times m_{\PC}}\rfloor$,
where $B$ is the bitlength of the source PC.

\paragraph{NOP Padding and Removal.}
\jitsynth synthesizes \minicompilers of equal length to ensure that
the target PC for jumps can be computed as a multiple of the source PC.
%
To do so, \jitsynth pads instruction sequences with NOPs to match the length
of the largest \minicompiler.
%
To mitigate the performance impact this incurs, we implemented a trusted compiler
pass that removes the NOP instructions while preserving the correctness of
the compiled code.
\fi
%
%\paragraph{ARM Fuel Function.}
%For simplicity,
We use $\Phi(\vec{p}) = \texttt{length}(\vec{p})$ as the fuel function for all
languages studied in this paper. This provides sufficient fuel for evaluating
programs in these languages that are accepted by the OS kernel. For example, the
Linux kernel requires eBPF programs to be loop-free, and it enforces this
restriction with a conservative static check; programs that fail the check are
not passed to the JIT~\cite{gershuni:crab-ebpf}.
\tighten
